\documentclass[a4paper]{article}
\usepackage{amsmath, amssymb, amsfonts, amsthm}
\usepackage[UTF8]{ctex}
\usepackage{graphicx}
\usepackage{color}
\usepackage{xcolor}
\usepackage{tcolorbox}
\tcbuselibrary{skins, breakable, theorems}
\usepackage{physics}
\usepackage{booktabs}   
\usepackage{geometry} 
\usepackage{chngcntr}
\usepackage{fancyhdr}
\usepackage{mathrsfs}
\usepackage{makecell}
\usepackage{enumitem}
\usepackage{bm}           % 加粗数学符号
 \newenvironment{proof*}
   {\par\noindent\textbf{证明:}\quad}
   {\hfill$\square$}
\numberwithin{equation}{section}% 设置公式编号按section编号
\newcommand{\highlight}[1]{\colorbox{yellow}{#1}}
\usepackage[unicode=true,
pdfencoding=auto,
bookmarks=true,
bookmarksnumbered=true,
bookmarksopen=true,
pdfstartview=FitH,
colorlinks=false,
pdfborder={0 0 0}]{hyperref}

% 定义问题盒子
\newtcolorbox[auto counter, 
number within=section]{problembox}[2][]{
	enhanced,        % 增强模式
	breakable,       % 允许跨页分断
	colframe = blue!50!black,
	colback = blue!3,
	colbacktitle = blue!10,
	fonttitle = \bfseries\color{black},
	title = \thetcbcounter: #2,
	boxrule = 1pt,
	before skip = 10pt,
	after skip = 10pt,
	#1
}

% 定义证明盒子
\newtcolorbox{proofbox}{
	colframe = green!50!black,
	colback = green!3,
	colbacktitle = green!10,
	fonttitle = \bfseries\color{black},  % 明确设置标题为黑色
	title = 证明,
	boxrule = 1pt,
	before skip = 8pt,
	after skip = 8pt
}

\begin{document}
	\pdfbookmark[0]{目录}{contents}
	\tableofcontents
	\newpage
	
	\part{基本公设}
	\section*{[1] 根据量子力学基本公设}
	
	量子系统的每一个状态,都可以由希尔伯特空间中的一个非零向量来描述,这个向量叫做系统的量子态。Thanks dirac, we could write it as $ \ket{\psi} $
	
	解释一点东西:
	
	希尔伯特空间是由定义在物理系统位形空间上平方可积复值函数的全体构成的,这个空间也称为系统的状态空间。
	
	一点讨论:注意!一个物理上可实现的量子态,其波函数必须是平方可积的。(这只要求积分是有限的,波函数是可以发散的,但是必须是在孤立点或边界上,只要这种发散足够 “弱”, 使得其模平方的积分仍然有限)
	
	\section*{[2] 根据量子力学公设:}
	
	物理系统的每一个力学量 $\hat{A}$,都可以描述成系统状态空间上的一个厄米算子。$|\psi_{n}\rangle$ 是厄米算子 $\hat{A}$ 对应于本征值 $\lambda_n$ 的本征态,$\hat{A}|\psi_{n}\rangle=\lambda_n|\psi_{n}\rangle$。在任意波函数 $|\psi\rangle=\sum C_n|\psi_{n}\rangle$ 上测量 $\hat{A}$,$P_n=\left|C_n\right|^2$ 代表在 $|\psi\rangle$ 态上测得 $\lambda_n$ 的概率。
	
	解释一点东西:
	
	定义厄米算子 $\hat{A}^{\dagger}:\bra{\varphi}\ket{\hat{A}^{\dagger}\psi}=\bra{\hat{A}\varphi}\ket{\psi}$。
	
	给定 \(\hat{A}\), 如果对于任意的 \(\ket{\psi},\ket{\varphi}\) 有 $\bra{\varphi}\ket{\hat{A}\psi}=\bra{\hat{A}\varphi}\ket{\psi}$, 则称\(\hat{A}\) 是厄米的,即 $\hat{A}=\hat{A}^{\dagger}$ 此时满足
	\[\bra{\psi}A=\bra{\psi}A^{\dagger}=\bra{A\psi}\]
	
	厄米算子的本征方程:
	\[\hat{A}|\psi_{n}\rangle=\lambda_n|\psi_{n}\rangle\]
	给出归一化的本征矢 $|\psi_{n}\rangle$ 和对应的本征值。$\{|\psi_{n}\rangle\}$ 完备地张成了一个希尔伯特空间 V,即对于一个态矢 $|\psi\rangle\in V$,都有 $|\psi\rangle=\sum C_n|\psi_{n}\rangle$。
	
	一点讨论:写到这里,我发现我忽略了“算子”的重要性,此处是明确说明了,厄米算子的本征方程给出了我们“喜欢的”本征矢。
	
	\section*{[3] 根据量子力学公设}
	
	微观系统演化满足薛定谔方程 $i\hbar\partial_t|\psi\rangle=\hat{H}|\psi\rangle$。
	
	\section*{[4] 根据量子力学公设:}
	
	全同粒子假设。
	\section*{[5]关于坍缩的一点讨论}
	\textbf{哥本哈根学派认为:对态函数$\left| \psi  \right\rangle $进行测量,它会变为它的一个本征态$\left| n  \right\rangle $.}\par
	另外有一些人反对他们的观点,诸如薛定谔,Weinberg,孙昌璞\par
	\begin{tcolorbox}[
		colback=white, % 背景色
		colframe=black, % 边框颜色
		arc=0pt, % 边框直角
		boxsep=5pt, % 内边距
		left=10pt, % 左内边距
		right=10pt, % 右内边距
		top=10pt, % 上内边距
		bottom=10pt, % 下内边距
		width=\textwidth % 宽度与文本相同
		]
		1979年诺奖得主、物理学标准模型的奠基者之一史蒂文·温伯格(Steven Weinberg)在《爱因斯坦的错误》一文中,很具体、很直接地批评了哥本哈根诠释的倡导者玻尔对于测量过程的不当处理:\par
		
		“量子经典诠释的玻尔版本有很大的瑕疵,其原因并非爱因斯坦所想象的。哥本哈根诠释试图描述观测(量子系统)所发生的状况,\underline{却经典地处理观察者与测量的过程}。这种处理方法肯定不对:观察者与他们的仪器也得遵守同样的量子力学规则,正如宇宙的每一个量子系统都必须遵守量子力学规则。”\par
		
		“哥本哈根诠释明显地可以解释量子系统的量子行为,但它并没有达成解释的任务,那就是应用波函数演化确定性方程(薛定谔方程)于观察者和他们的仪器。”\par
		\hfill————《量子力学诠释问题》孙昌璞
	\end{tcolorbox}
	
	\section*{[6]数学形式的表示}
	\begin{tabular}{|l|c|c|}
		\hline
		特征 & 离散谱 & 连续谱 \\
		\hline
		本征值方程 & $\hat{O} | o_n \rangle = o_n | o_n \rangle$ & $\hat{O} | o \rangle = o | o \rangle$ \\
		\hline
		完备性关系 & $\sum_n | o_n \rangle\langle o_n | = \hat{I}$ & $\int do\, | o \rangle\langle o | = \hat{I}$ \\
		\hline
		正交归一性 & $\langle o_m | o_n \rangle = \delta_{mn}$ & $\langle o | o' \rangle = \delta(o - o')$ \\
		\hline
		态展开 & $ |\psi\rangle = \sum_n c_n | o_n \rangle, \quad c_n = \langle o_n | \psi \rangle $ & $ |\psi\rangle = \int do\, c(o) | o \rangle, \quad c(o) = \langle o | \psi \rangle $ \\
		\hline
		概率(测量) & 测得 $o_n$ 的概率:$ P(o_n) = |c_n|^2 $ & 
		\makecell[l]{测得结果在 $o$ 到 $o+do$ 内的概率密度: \\ $\rho(o) = |c(o)|^2$} \\
		\hline
		归一化条件 & $\sum_n |c_n|^2 = 1$ & $\int do\, |c(o)|^2 = 1$ \\
		\hline
	\end{tabular}
	
	\begin{itemize}
		\item 对于连续谱,本征态本身不是希尔伯特空间中的矢量(因为模无穷大),而是“广义本征函数”,物理态必须表示为这些本征态的叠加(即积分)
	\end{itemize}
	
	\newpage
	
	
	\part{概念梳理}
	\section{我们习惯的薛定谔方程是什么?为什么?}
	为了表示一个量子态,我们首先需要找到的是希尔伯特空间,空间是由基矢张成的,那么我们应该先确定基矢。我们通过厄米算子 $\hat{A}$ 找到了一组归一化的本征基矢$|\psi_{n}\rangle$,这个时候我们就说我们是在 $\hat{A}$ 表象($\hat{A}$ represention)下。
	
	现在我来说明一下为什么我们喜欢用“坐标表象下的波函数”。
	
	in position space, we know
	\begin{equation}
		\hat{H}=\frac{\hat{p}^2}{2 m}+\hat{V}
	\end{equation}
	
		now you should notice that in the position space, $\hat{p}=-i\hbar\nabla$ and we only consider one dimension, and you could write down
		\begin{equation}
			\hat{H}=-\frac{\hbar^2}{2 m}\frac{\partial^2}{\partial x^2}+V(x)
		\end{equation}
		why? because you have to know $\hat{V}$, in position space, it's easy for you to express potential.
		
		Then solve the Schrödinger equation:
		\begin{equation}
			i\hbar\partial_t|\psi\rangle=\hat{H}|\psi\rangle
		\end{equation}
		we have to do
		\begin{equation}
			\left\langle {x}
			\mathrel{\left | {\vphantom {x \psi }}
				\right. \kern-\nulldelimiterspace}
			{\psi } \right\rangle  = \psi (x,t)
		\end{equation}
		then we get
		\begin{equation}
			i\hbar\frac{\partial\psi}{\partial t}=\hat{H}\psi
		\end{equation}
		now you just need to solve the partial equation.
		
		of course, if you are in the situation you must use momentum space, there is a approach for you.
		
		Firstly, you know
		\begin{equation}
			\hat{P}|p\rangle=p|p\rangle,\left\langle p^{\prime}\mid p^{\prime\prime}\right\rangle=\delta\left(p^{\prime}-p^{\prime\prime}\right)
		\end{equation}
		
		In fact, we could derive the latter by the former.
		
		Then, we know in the position space
		\begin{equation}
			\langle x|P| p\rangle=-i\hbar\frac{\partial}{\partial x}\langle x\mid p\rangle
		\end{equation}
		in momentum space
		\begin{equation}
			\langle x|P| p\rangle=p\langle x\mid p\rangle
		\end{equation}
		$\langle x|P| p\rangle$ should remain the same whether any space, because inner product can't change the value.
		
		Then we get
		\begin{equation}
			-i\hbar\frac{\partial}{\partial x}\langle x\mid p\rangle=p\langle x\mid p\rangle
		\end{equation}
		This is a partial equation.
		
		Then we get
		\begin{equation}
			\langle x\mid p\rangle=\exp\left(\frac{i p x}{\hbar}\right)
		\end{equation}
		
		Take the equation
		\begin{align}
			\langle p\mid\psi\rangle&=\int d x\langle p\mid x\rangle\langle x\mid\psi\rangle\\
			&=\int d x\exp\left(\frac{i p x}{\hbar}\right)\psi(x, t)
		\end{align}
		you will get the $\langle p\mid\psi\rangle=\psi(p, t)$.
		
		
		
		\newpage
		\section{矩阵与狄拉克符号}
	首先,考虑算符 $X$。我们需要构造一个矩阵来等效描述它,众所周知,矩阵的元素必须是标量(数)而非矢量,因此我们可以将单位算符作用于 $X$:
\begin{equation}
	X = \sum_{a'} \sum_{a''} \ket{a'} \bra{a'} X \ket{a''} \bra{a''}
\end{equation}

接下来我们构造标量 $\bra{a'} X \ket{a''}$,只要能为态矢量构造出与 $\bra{a'} X \ket{a''}$ 匹配的恰当矩阵形式,就可以将该标量作为矩阵的元素,形式如下:
\begin{equation}
	X \doteq \begin{pmatrix}
		\bra{a_{(1)}} X \ket{a_{(1)}} & \bra{a_{(1)}} X \ket{a_{(2)}} & \cdots \\
		\bra{a_{(2)}} X \ket{a_{(1)}} & \bra{a_{(2)}} X \ket{a_{(2)}} & \cdots \\
		\vdots & \vdots & \ddots
	\end{pmatrix}
\end{equation}

为了让态矢量与算符 $X$ 的形式匹配,我们回顾算符 $X$ 的定义——态矢量空间中的基本方程:
\begin{equation}
	\ket{\beta} = X \ket{\alpha}
\end{equation}

将单位算符作用于该式两侧:
\begin{equation}
	\bra{a_l} \ket{\beta} = \sum_m \bra{a_l} X \ket{a_m} \bra{a_m} \ket{\alpha}
\end{equation}

若我们将 $\bra{a_l} \ket{\beta}$ 和 $\bra{a_m} \ket{\alpha}$ 排列为列矩阵,并定义 $\ket{\beta}$ 和 $\ket{\alpha}$ 的新表示形式:
\begin{equation}
	|\beta\rangle \doteq \begin{pmatrix}
		\langle a_1 | \beta \rangle \\
		\langle a_2 | \beta \rangle \\
		\vdots
	\end{pmatrix}, \quad
	|\alpha\rangle \doteq \begin{pmatrix}
		\langle a_1 | \alpha \rangle \\
		\langle a_2 | \alpha \rangle \\
		\vdots
	\end{pmatrix}
\end{equation}

此时我们会惊奇地发现,上式恰好对应矩阵乘法运算。因此,若我们按上述方式定义 $X$、$|\alpha\rangle$ 和 $|\beta\rangle$ 的矩阵表示,它们将完全契合狄拉克态矢量理论。

类似地,我们可以推导出左矢的恰当表示形式:
\begin{equation}
	\langle \beta| = \langle \alpha| X
\end{equation}
\begin{equation}
	\langle \beta| a'\rangle = \sum_m \langle \alpha| a_m \rangle \langle a_m | X | a' \rangle
\end{equation}
\begin{equation}
	\langle \beta| \doteq \left( \langle a_1 | \beta \rangle \ \langle a_2 | \beta \rangle \ \cdots \right), \quad
	\langle \alpha| \doteq \left( \langle a_1 | \alpha \rangle \ \langle a_2 | \alpha \rangle \ \cdots \right)
\end{equation}

由此我们可以大胆得出结论:若将态矢量和算符替换为对应的矩阵形式,同时保持标量不变,那么态矢量理论中的所有方程形式都将保持不变。接下来我们可以通过内积验证该理论的正确性:
\begin{equation}
	\langle \beta | \alpha \rangle = \sum_i \langle \beta | a_i \rangle \langle a_i | \alpha \rangle
\end{equation}

在矩阵表示下:
\begin{equation}
	\langle \beta | \alpha \rangle \doteq \left( \langle a_1 | \beta \rangle \ \langle a_2 | \beta \rangle \ \cdots \right) \begin{pmatrix}
		\langle a_1 | \alpha \rangle \\
		\langle a_2 | \alpha \rangle \\
		\vdots
	\end{pmatrix} = \sum_a \langle \beta | a \rangle \langle a | \alpha \rangle
\end{equation}

我们发现,无论采用哪种表示形式,$\langle \beta | \alpha \rangle$ 的结果(标量)始终一致,这证明了矩阵表示的正确性。

\subsection{基变换}
\subsubsection{变换关系}

在量子力学中,我们可以选择不同的基右矢,以 $|a\rangle$ 和 $|b\rangle$ 为例。因此,要研究基变换问题,首先需要分析变换算符 $U$:
\begin{equation}
	|b_i \rangle = U |a_i \rangle
\end{equation}
若这些基右矢满足正交归一性,则可推得:
\begin{equation}
	U = \sum_{l} \ket{b_{l}} \bra{a_{l}}
\end{equation}
\begin{equation}
	UU^{\dagger} = 1
\end{equation}

我们同样可以写出 $U$ 的矩阵表示:
\begin{equation}
	\bra{a_{l}} U \ket{a_{m}} = \braket{a_{l}}{b_{m}}
\end{equation}

接下来分析算符 $X$ 在新基下的变换形式:

首先,在矩阵表示下:
\begin{align}
	\bra{b_{l}} X \ket{b_{m}} &= \sum_{p} \sum_{q} \braket{b_{l}}{a_{p}} \bra{a_{p}} X \ket{a_{q}} \braket{a_{q}}{b_{m}} \\
	&= \sum_{p} \sum_{q} \bra{a_{l}} U^{\dagger} \ket{a_{p}} \bra{a_{p}} X \ket{a_{q}} \bra{a_{q}} U \ket{a_{m}}
\end{align}

这一形式恰好对应矩阵乘法,因此我们得到算符 $X$ 的变换关系:
\begin{equation}
	X' = U^{\dagger} X U
\end{equation}

\subsubsection{对角化}

借助矩阵表示,我们可以利用线性代数中的方法求解量子力学中的本征方程。考虑算符 $B$ 的本征方程:
\begin{equation}
	B \ket{b_{l}} = b_{l} \ket{b_{l}}
\end{equation}

已知 $B$ 在基 $\ket{a}$ 下的矩阵表示,将其代入本征方程可得:
\begin{equation}
	\sum_{k} \bra{a_{m}} B \ket{a_{k}} \braket{a_{k}}{b_{l}} = b_{l} \braket{a_{m}}{b_{l}}
\end{equation}

若定义 $C_{m l} = \braket{a_{m}}{b_{l}}$,则上式可写为:
\begin{equation}
	\begin{pmatrix}
		B_{11} & B_{12} & \cdots \\
		B_{21} & B_{22} & \cdots \\
		\vdots & \vdots & \ddots
	\end{pmatrix}
	\begin{pmatrix}
		C_{1 l} \\
		C_{2 l} \\
		\vdots
	\end{pmatrix}
	= b_{l}
	\begin{pmatrix}
		C_{1 l} \\
		C_{2 l} \\
		\vdots
	\end{pmatrix}
\end{equation}

由此,我们可以利用线性代数中的方法(求解 $\det(B - \lambda I) = 0$)得到本征值 $b_{l}$ 和列向量 $\begin{pmatrix} C_{1 l} \\ C_{2 l} \\ \vdots \end{pmatrix}$。注意到 $C_{m l} = \braket{a_{m}}{b_{l}}$ 意味着该列向量正是本征右矢 $\ket{b_{l}}$ 在基 $\ket{a}$ 下的矩阵表示。至此,我们通过矩阵表示得到了本征值 $b_{l}$ 和本征右矢 $\ket{b_{l}}$.

接下来分析如何将基 $\ket{a}$ 下的矩阵 $B$ 对角化:显然,在 $B$ 的本征基 $\ket{b}$ 下,$B$ 是对角矩阵:
\begin{equation}
	\langle b_m | B | b_l \rangle = b_l \delta_{ml}
\end{equation}

因此,我们只需通过基变换即可实现对角化:
\begin{equation}
	B_b = U^\dagger B_a U \quad B_a = U B_b U^{-1}
\end{equation}

那么矩阵 $U$ 的具体形式是什么?幸运的是,$C_{m l} = \langle a_m | b_l \rangle$ 恰好是矩阵 $U$ 第 $m$ 行第 $l$ 列的元素——将所有列向量 $\begin{pmatrix} C_{1 1} \\ C_{2 1} \\ \vdots \end{pmatrix}$ 按列排列为方阵 $\begin{pmatrix} C_{1 1} & C_{1 2} & \cdots \\ C_{2 1} & C_{2 2} & \cdots \\ \vdots & \vdots & \ddots \end{pmatrix}$,该方阵即为 $U$。

简言之,矩阵对角化的步骤为:首先求解本征方程得到本征值和本征右矢;然后将所有本征右矢的矩阵表示按列排列为方阵 

		\subsection{基本对易关系}
		坐标与动量的正则对易关系
		\begin{equation}
		[x_i, p_j] = i\hbar \delta_{ij}
		\end{equation}
		
		角动量分量的对易关系
		\begin{equation}
		[L_i, L_j] = i\hbar \epsilon_{ijk} L_k
		\end{equation}
		
		角动量平方与分量的对易关系
		\begin{equation}
		[L^2, L_i] = 0
		\end{equation}
		
		角动量与位置算符
		\begin{equation}
		[L_i, x_j] = i\hbar \epsilon_{ijk} x_k
		\end{equation}
		
		角动量与动量算符
		\begin{equation}
		[L_i, p_j] = i\hbar \epsilon_{ijk} p_k
		\end{equation}
		
		Pauli 矩阵的对易关系
		\begin{equation}
		[\sigma_i, \sigma_j] = 2i \epsilon_{ijk} \sigma_k
		\end{equation}
		
		对任意 \(i,j \in \{1,2,3\}\), Pauli 矩阵的乘积可写为
		\[
		\sigma_i \sigma_j = \delta_{ij} I + i\sum_{k=1}^3 \epsilon_{ijk}\sigma_k
		\]
		
		自旋角动量分量的对易关系
		\begin{equation}
		[S_i, S_j] = i\hbar \epsilon_{ijk} S_k
		\end{equation}
		
		对易关系的基本性质
		\begin{equation}
			[A, B + C] = [A, B] + [A, C], \quad [A + B, C] = [A, C] + [B, C] 
		\end{equation}
		\begin{equation}
			[A, BC] = [A, B]C + B[A, C], \quad [AB, C] = A[B, C] + [A, C]B
		\end{equation}
		\begin{equation}
			[A, [B, C]] + [B, [C, A]] + [C, [A, B]] = 0
		\end{equation}
		\newpage
			\section{基本算符在坐标表象的表示}
			\subsection{坐标算符}
			在量子力学中,坐标表象是最直观的表象之一。坐标算符 $\hat{X}$ 具有连续谱,其本征态 $\ket{x}$ 满足本征方程:
			\[
			\hat{X} \ket{x} = x \ket{x},
			\]
			利用 \(\bra{x'}\) 作内积有:
			\[
			 \bra{x'} \hat X\ket{x} = x\bra{x'} \ket{x}
			\]
			利用 \(\hat{x}\) 是厄米算符(左乘 \(\bra{x'}\)),有:
			
			\[
			x' \bra{x'} \ket{x} = x \langle x'
			x \rangle
			\]
			即:
			
			\[
			(x' - x) \langle x' | x \rangle = 0
			\]
			
			这意味着当 \(x' \neq x \)时,\(\bra{x'} \ket{x} = 0\)。只有当 \(x' = x\) 时,\(\bra{x'} \ket{x}\) 才非零。因此,\(\bra{x'}\ket{x}\) 是一个在\( x' = x \)处奇异的函数,且 (连续谱) 满足\textbf{狄拉克归一化条件}
			
			\[
			\int_{-\infty}^{\infty} \langle x' | x \rangle \, dx' = 1
			\]
			这个函数正是狄拉克δ函数:
			
			\[
			\boxed{\langle x' | x \rangle = \delta(x' - x)}
			\]
			以及完备性关系:
			\[
			\int_{-\infty}^{\infty} \ket{x} \bra{x}  \dd x = \hat{I}.
			\]
			任意量子态 $\ket{\psi}$ 在坐标表象下的表示即为波函数:
			\[
			\psi(x) = \langle x | \psi \rangle.
			\]
			
			\subsection{动量算符在坐标表象下的表示推导}
			
			\subsubsection*{推导一:基于平移生成元}
			
			\begin{enumerate}
				\item \textbf{定义无穷小平移算符}:
				考虑系统沿 $x$ 方向平移一个无穷小量 $a$。设存在一个幺正算符 $\hat{T}(a)$ 实现此平移,其作用于位置本征态 $|x\rangle$ 满足:
				\begin{equation}
					\hat{T}(a) |x\rangle = |x + a\rangle.
				\end{equation}
				考虑平移算符的幺正性有
				\[\hat T{\hat T^\dag } = I\quad \Rightarrow \quad {\hat T^\dag(a) }\ket{x} = \ket{x-a}\]
				量子力学中,连续对称变换的生成元是对应的守恒量算符。对于空间平移,生成元是动量算符 $\hat{p}$。因此,无穷小平移算符可表示为:
				\begin{equation}\label{2a}
					\hat{T}(a) = \exp\left(-\frac{i}{\hbar} a \hat{p}\right) \approx 1 - \frac{i}{\hbar} a \hat{p} + \mathcal{O}(a^2),
				\end{equation}
				其中展开式在 $a \to 0$ 时成立。
				
				\item \textbf{在坐标表象中考察平移效果}:
				对任意态 $|\psi\rangle$,其坐标表象波函数为 $\psi(x) = \langle x | \psi \rangle$。平移后的态为 $\hat{T}(a) |\psi\rangle$,其在坐标表象的波函数为:
				\begin{equation}\label{3a}
					\langle x | \hat{T}(a) | \psi \rangle = \langle x - a | \psi \rangle = \psi(x - a).
				\end{equation}
				
				\item \textbf{无穷小展开与比较}:
				将式 (\ref{2a}) 代入式 (\ref{3a}) 并展开:
				\begin{equation}\label{4a}
					\langle x | \hat{T}(a) | \psi \rangle = \left\langle x \left| 1 - \frac{i}{\hbar} a \hat{p} \right| \psi \right\rangle = \psi(x) - \frac{i}{\hbar} a \langle x | \hat{p} | \psi \rangle + \mathcal{O}(a^2).
				\end{equation}
				同时,将式 (3) 右边 $\psi(x - a)$ 作泰勒展开:
				\begin{equation}\label{5a}
					\psi(x - a) = \psi(x) - a \frac{\partial \psi}{\partial x} + \mathcal{O}(a^2).
				\end{equation}
				比较式 (\ref{4a}) 和式 (\ref{5a}),并忽略高阶小量,可得:
				\begin{equation}
					- \frac{i}{\hbar} a \langle x | \hat{p} | \psi \rangle = - a \frac{\partial \psi}{\partial x}.
				\end{equation}
				由于 $a$ 是任意的无穷小量,我们得到动量算符在坐标表象下的作用为:
				\begin{equation}
					\boxed{\langle x | \hat{p} | \psi \rangle = -i\hbar \frac{\partial}{\partial x} \psi(x)}.
				\end{equation}
			\end{enumerate}
			
			\subsubsection*{推导二:基于基本对易关系}
			\label{sec:com}
			
			此推导从量子力学的基本假设——正则对易关系出发,更具公理化色彩。优化点在于更严谨地处理微分算符的域和形式推导。
			
			\begin{enumerate}
				\item \textbf{基本假设}:
				位置算符 $\hat{x}$ 与动量算符 $\hat{p}$ 满足海森堡对易关系:
				\begin{equation}
					[\hat{x}, \hat{p}] = i\hbar.
				\end{equation}
				
				\item \textbf{坐标表象下的表示}:
				在坐标表象 $\{|x\rangle\}$ 中,位置算符的作用是乘法:
				\begin{equation}
					\langle x | \hat{x} | \psi \rangle = x \psi(x).
				\end{equation}
				设动量算符在坐标表象下为一个线性微分算符 $\hat{p} = D$,其具体形式待定,即 $\langle x | \hat{p} | \psi \rangle = D \psi(x)$。
				
				\item \textbf{将对易关系作用于任意态}:
				计算对易子 $[\hat{x}, \hat{p}]$ 在坐标表象下的矩阵元:
				\begin{align}
					\langle x | [\hat{x}, \hat{p}] | \psi \rangle &= \langle x | \hat{x}\hat{p} | \psi \rangle - \langle x | \hat{p}\hat{x} | \psi \rangle \nonumber \\
					&= x \cdot D\psi(x) - D\left( x \psi(x) \right).
				\end{align}
				根据对易关系 (8),上式应等于 $i\hbar \langle x | \psi \rangle = i\hbar \psi(x)$。因此有:
				\begin{equation}
					x D\psi - D(x\psi) = i\hbar \psi.
					\label{eq:com_rel}
				\end{equation}
				
				\item \textbf{求解微分算符 $D$}:
				假设 $D$ 是至少一阶的微分算符(因为动量与变化率相关),并假设其满足莱布尼茨积法则。将 $D(x\psi)$ 展开:
				\begin{equation}
					D(x\psi) = (D x) \psi + x (D \psi).
				\end{equation}
				注意,$D$ 对函数 $x$ 的作用结果为 $D x$,它是一个新的函数。将上式代入式 \eqref{eq:com_rel}:
				\begin{equation}
					x D\psi - \left[ (D x) \psi + x D\psi \right] = i\hbar \psi \quad \Rightarrow \quad -(D x) \psi = i\hbar \psi.
				\end{equation}
				由于 $\psi$ 是任意波函数,可得:
				\begin{equation}
					D x = -i\hbar.
				\end{equation}
				这意味着算符 $D$ 对函数 $f(x)=x$ 的作用结果是常数 $-i\hbar$。满足此条件的最简单的微分算符是 $\displaystyle D = -i\hbar \frac{\partial}{\partial x}$,因为
				\begin{equation}
					\left( -i\hbar \frac{\partial}{\partial x} \right) x = -i\hbar \left( \frac{\partial x}{\partial x} + x \frac{\partial}{\partial x} \right) = -i\hbar (1 + 0) = -i\hbar.
				\end{equation}
				将此形式代入 $D(x\psi)$ 验证莱布尼茨律:
				\begin{equation}
					-i\hbar \frac{\partial}{\partial x} (x\psi) = -i\hbar \left( \psi + x \frac{\partial \psi}{\partial x} \right) = (-i\hbar) \psi + x \left( -i\hbar \frac{\partial \psi}{\partial x} \right),
				\end{equation}
				确实满足。因此,我们得到:
				\begin{equation}
					\boxed{\hat{p} = -i\hbar \frac{\partial}{\partial x} \quad \text{在坐标表象下}}.
				\end{equation}
			\end{enumerate}
			
			\begin{itemize}
				
				\item \textbf{广义坐标警告}:此简单形式仅对笛卡尔坐标成立。在广义坐标(如球坐标)下,共轭动量的算符形式需考虑度规因子,例如径向动量 $p_r \neq -i\hbar \frac{\partial}{\partial r}$。
				
			\end{itemize}
			\subsection{动量表象与傅里叶变换}
			动量表象与坐标表象通过傅里叶变换相联系:
			\[
			\phi(p) = \langle p | \psi \rangle = \frac{1}{\sqrt{2\pi\hbar}} \int_{-\infty}^{\infty} e^{-ipx/\hbar} \psi(x)  \dd x.
			\]
			在动量表象中,动量算符为乘法算符, 坐标算符为微分算符, 动量表象和坐标表象对偶。
			\subsection{算符的谱分解形式}
				\begin{problembox}{算符的谱分解与算符函数定义}
				算符 $\hat{Q}$ 具有一组完备的正交归一本征矢:
				\[
				\hat{Q} \ket{e_{n}} = q_{n} \ket{e_{n}} \quad (n=1,2,3,\cdots).
				\]
				
				\begin{enumerate}[label=(\arabic*), leftmargin=*]
					\item 证明:$\hat{Q}$ 可以写成它的谱分解形式
					\[
					\hat{Q} = \sum_{n} q_{n} \ket{e_{n}}\bra{e_{n}}.
					\]
					\item 定义 $\hat{Q}$ 的函数的另一种方法是通过谱分解:
					\[
					f(\hat{Q}) = \sum_{n} f(q_{n}) \ket{e_{n}}\bra{e_{n}}.
					\]
					证明:对于 $e^{\hat{Q}}$ 的情况, $e^{\hat{Q}} = \sum_{n} e^{q_{n}} \ket{e_{n}}\bra{e_{n}}$
				\end{enumerate}
			\end{problembox}
			
			\begin{proof}
				\begin{enumerate}[label=(\arabic*), leftmargin=*]
					\item 设 $\ket{\alpha} = \sum_{n} c_{n} \ket{e_{n}}$,其中 $c_{n} = \braket{e_{n}|\alpha}$。则
					\[
					\hat{Q} \ket{\alpha} = \sum_{n} c_{n} q_{n} \ket{e_{n}}
					= \left( \sum_{n} q_{n} \ket{e_{n}}\bra{e_{n}} \right) \ket{\alpha}.
					\]
					故 $\hat{Q} = \sum_{n} q_{n} \ket{e_{n}}\bra{e_{n}}$。
					\item 由 $\hat{Q}^{k} \ket{e_{n}} = q_{n}^{k} \ket{e_{n}}$ 得
					\[
					e^{\hat{Q}} \ket{e_{n}} = \sum_{k=0}^{\infty} \frac{1}{k!} q_{n}^{k} \ket{e_{n}}
					= e^{q_{n}} \ket{e_{n}}.
					\]
					对任意 $\ket{\alpha} = \sum_{n} c_{n} \ket{e_{n}}$,
					\[
					e^{\hat{Q}} \ket{\alpha} = \sum_{n} c_{n} e^{q_{n}} \ket{e_{n}}
					= \left( \sum_{n} e^{q_{n}} \ket{e_{n}}\bra{e_{n}} \right) \ket{\alpha}.
					\]
					故 $e^{\hat{Q}} = \sum_{n} e^{q_{n}} \ket{e_{n}}\bra{e_{n}}$,与谱分解定义一致。
				\end{enumerate}
			\end{proof}
		\newpage
		
		\section{坐标表象(连续谱)下期望值的表示}
		\subsection{期望值的一般形式与坐标表象展开}
		在量子力学中,一个可观测量算符 $\hat{Q}$ 在态 $|\psi\rangle$ 下的期望值定义为
		\[
		\langle Q \rangle = \langle \psi | \hat{Q} | \psi \rangle .
		\]
		这是一个在抽象希尔伯特空间中的内积表达式。为了将其在坐标表象(位置表象)中具体表示出来,我们需要插入坐标本征态 $\{|x\rangle\}$ 的完备性关系
		\[
		\int \dd x \, |x\rangle\langle x| = \hat{I}.
		\]
		
		在 $\langle \psi |$ 与 $\hat{Q}$ 之间、以及 $\hat{Q}$ 与 $|\psi\rangle$ 之间分别插入单位算符,得到
		\begin{align}
			\langle Q \rangle &= \langle \psi | \hat{Q} | \psi \rangle \nonumber \\
			&= \langle \psi | \left( \int \dd x \, |x\rangle\langle x| \right) \hat{Q} \left( \int \dd x' \, |x'\rangle\langle x'| \right) |\psi \rangle \nonumber \\
			&= \iint \dd x \dd x' \, \langle \psi | x \rangle \langle x | \hat{Q} | x' \rangle \langle x' | \psi \rangle .
			\label{eq:double_int}
		\end{align}
		这里 $\langle x | \psi \rangle = \psi(x)$ 是态 $|\psi\rangle$ 在坐标表象下的波函数,而 $\langle \psi | x \rangle = \psi^*(x)$ 是其复共轭。$\langle x | \hat{Q} | x' \rangle$ 是算符 $\hat{Q}$ 在坐标表象下的矩阵元。于是期望值可写为
		\[
		\langle Q \rangle = \iint \dd x \dd x' \, \psi^*(x) \, \langle x | \hat{Q} | x' \rangle \, \psi(x').
		\]
		这是一个双重积分的形式,适用于任意算符 $\hat{Q}$。
		
		\subsection{化简为单积分:物理常见算符的情形}
		对于物理中常见的算符,矩阵元 $\langle x | \hat{Q} | x' \rangle$ 具有特殊结构,使得双重积分可以化为单积分。化简的关键在于利用矩阵元中所含的 Dirac $\delta$ 函数(及其导数)的性质。
		
		\subsubsection{坐标函数算符(乘法算符)}
		若 $\hat{Q} = f(\hat{x})$,例如势能 $V(\hat{x})$,则有
		\[
		\langle x | f(\hat{x}) | x' \rangle = f(x) \, \delta(x - x').
		\]
		代入式~\eqref{eq:double_int} 并对 $x'$ 积分,利用 $\delta$ 函数的筛选性质,得到
		\[
		\langle Q \rangle = \int \dd x \, \psi^*(x) f(x) \psi(x) = \int \dd x \, \psi^*(x) \bigl[ f(x) \psi(x) \bigr].
		\]
		
		\subsubsection{动量算符及其函数(微分算符)}
		以动量算符 $\hat{p}_x$ 为例,其坐标表象矩阵元为
		\[
		\langle x | \hat{p}_x | x' \rangle = -i\hbar \pdv{x} \delta(x - x').
		\]
		代入式~\eqref{eq:double_int} 得
		\[
		\langle Q \rangle = \iint \dd x \dd x' \, \psi^*(x) \left[ -i\hbar \pdv{x} \delta(x - x') \right] \psi(x').
		\]
		对 $x'$ 积分时,将导数从 $\delta$ 函数上转移到 $\psi(x')$ 上(分部积分):
		\[
		\int \dd x' \, \left[ -i\hbar \pdv{x} \delta(x - x') \right] \psi(x') = i\hbar \int \dd x' \, \delta(x - x') \pdv{x'} \psi(x') = i\hbar \pdv{x} \psi(x).
		\]
		注意这里偏导变成了对 $x$ 的导数,且由于 $\delta$ 函数的对称性,实际上得到
		\[
		\langle Q \rangle = \int \dd x \, \psi^*(x) \left( -i\hbar \pdv{x} \right) \psi(x).
		\]
		类似地,对于动能算符 $\hat{T} = \hat{p}_x^2/(2m)$,经过两次分部积分可得
		\[
		\langle T \rangle = \int \dd x \, \psi^*(x) \left( -\frac{\hbar^2}{2m} \pdv[2]{x} \right) \psi(x).
		\]
		
		\subsection{最一般的情况}
		这里需要强调:(海森堡代数) 任何关于 \(\hat x,\hat p\)的实函数(经适当排序)可给出一个厄米算符 \(\hat Q\), 且该系统的大多数物理可观测量(如哈密顿量、角动量等)可表示为 \(\hat x,\hat p\) 的函数.利用对易关系,通常可以通过换序将算符重排为幂级数形式
		
		以常见的哈密顿量举例
		\[
		\hat{H} = \frac{\hat{p}^2}{2m} + V(\hat{x}),
		\]
		它是坐标函数与动量函数的组合。根据以上讨论,其在坐标表象下的期望值可写为
		\[
		\boxed{\; \langle H \rangle = \int \dd x \, \psi^*(x) \left[ -\frac{\hbar^2}{2m} \dv[2]{x} + V(x) \right] \psi(x) \;}
		\]
		这正是定态薛定谔方程中出现的算符的期望值形式。
		
		更一般地,对于任意由 $\hat{x}$ 和 $\hat{p}$ 构造的算符 $\hat{Q}$,经过上述化简过程,其期望值在坐标表象下均可表示为单积分
		\[
		\boxed{\; \langle Q \rangle = \int \dd x \, \psi^*(x) \, \hat{Q}_x \, \psi(x) \;}
		\]
		其中 $\hat{Q}_x$ 是抽象算符 $\hat{Q}$ 在坐标表象下的具体微分算符形式。具体地:
		\begin{itemize}
			\item 遇到 $\hat{x}$,替换为乘法因子 $x$;
			\item 遇到 $\hat{p}_x$,替换为微分算符 $-i\hbar \pdv{x}$。
		\end{itemize}
		
		\subsection{总结}
		从抽象的期望值定义 $\langle \psi | \hat{Q} | \psi \rangle$ 出发,通过插入坐标完备性关系,得到包含矩阵元 $\langle x | \hat{Q} | x' \rangle$ 的双重积分表达式。利用物理算符矩阵元中必含 $\delta$ 函数或其导数的性质,通过 $\delta$ 函数的筛选以及适当的分部积分,最终可将双重积分化简为单积分形式。该单积分中,算符 $\hat{Q}_x$ 以微分算符(及乘法因子)的形式作用于波函数 $\psi(x)$ 上。这一化简过程构成了波函数表述下计算可观测量的理论基础,也是连接抽象 Dirac 符号与具体波函数动力学的关键桥梁。
		\newpage
		\section{狄拉克符号下的基矢变换}
			在量子力学中,狄拉克符号(Bra-ket notation)不仅是一种记号,它还蕴含了希尔伯特空间的几何结构。从狄拉克符号推导幺正变换(Unitary Transformation),本质上是研究\textbf{基矢的变换}以及如何保持\textbf{内积不变}。
			
			\subsection{变换的构造:从一组基到另一组基}
			
			假设我们在希尔伯特空间中有两组完备的正交归一基:
			\begin{enumerate}
				\item 旧基矢:$\{|a_n\rangle\}$
				\item 新基矢:$\{|b_n\rangle\}$
			\end{enumerate}
			
			由于它们都是正交归一的,满足:
			\begin{equation}
				\langle a_m | a_n \rangle = \delta_{mn}, \quad \langle b_m | b_n \rangle = \delta_{mn}
			\end{equation}
			
			我们希望定义一个算符 $U$,使得它能将旧基矢映射到新基矢:
			\begin{equation}
				U |a_n\rangle = |b_n\rangle
			\end{equation}
			
			利用\textbf{完备性关系(Completeness Relation)} $\sum_n |a_n\rangle \langle a_n| = I$,我们可以显式地写出 $U$:
			\begin{equation}
				U = U \cdot I = U \sum_n |a_n\rangle \langle a_n| = \sum_n (U |a_n\rangle) \langle a_n|
			\end{equation}
			代入 $U |a_n\rangle = |b_n\rangle$,得到:
			\begin{equation}
				U = \sum_n |b_n\rangle \langle a_n|
			\end{equation}
			
			\subsection{证明该变换是“幺正”的}
			
			一个算符 $U$ 是幺正的,必须满足 $U^\dagger U = I$。我们来验证这一点。
			
			首先,写出 $U$ 的伴随算符(Adjoint Operator)$U^\dagger$:
			\begin{equation}
				U^\dagger = \left( \sum_n |b_n\rangle \langle a_n| \right)^\dagger = \sum_n |a_n\rangle \langle b_n|
			\end{equation}
			
			现在计算 $U^\dagger U$:
			\begin{align}
				U^\dagger U &= \left( \sum_m |a_m\rangle \langle b_m| \right) \left( \sum_n |b_n\rangle \langle a_n| \right) \\
				&= \sum_m \sum_n |a_m\rangle \langle b_m | b_n \rangle \langle a_n|
			\end{align}
			
			利用新基矢的正交归一性 $\langle b_m | b_n \rangle = \delta_{mn}$:
			\begin{equation}
				U^\dagger U = \sum_n |a_n\rangle \langle a_n| = I
			\end{equation}
			
			同理可以证明 $UU^\dagger = I$。因此,$U$ 是一个幺正算符。
			
			\section*{幺正变换的物理意义}
			
			\subsection{内积的不变性}
			假设有两个态 $|\psi\rangle$ 和 $|\phi\rangle$,经过幺正变换后变为 $|\psi'\rangle = U|\psi\rangle$ 和 $|\phi'\rangle = U|\phi\rangle$。
			变换后的内积为:
			\begin{equation}
				\langle \phi' | \psi' \rangle = ( \langle \phi | U^\dagger ) ( U | \psi \rangle ) = \langle \phi | U^\dagger U | \psi \rangle = \langle \phi | I | \psi \rangle = \langle \phi | \psi \rangle
			\end{equation}
			这说明幺正变换保持了概率幅不变。
			
			\subsection{算符的变换}
			如果一个算符 $A$ 作用在 $|\psi\rangle$ 上得到 $|\phi\rangle = A|\psi\rangle$,那么在新的基下,为了保持物理规律一致,应满足:
			\begin{equation}
				U|\phi\rangle = A' U|\psi\rangle \implies U(A|\psi\rangle) = A' U|\psi\rangle
			\end{equation}
			由于这对任何 $|\psi\rangle$ 都成立,我们得到算符的变换公式:
			\begin{equation}
				A' = U A U^\dagger
			\end{equation}
			
			\begin{table}[h]
				\centering
				\begin{tabular}{lll}
					\toprule
					物理量 & 变换前 & 变换后 (幺正变换 $U$) \\
					\midrule
					态矢量 & $|\psi\rangle$ & $|\psi'\rangle = U |\psi\rangle$ \\
					对偶矢量 & $\langle \psi |$ & $\langle \psi' | = \langle \psi | U^\dagger$ \\
					算符 & $A$ & $A' = U A U^\dagger$ \\
					期望值 & $\langle \psi | A | \psi \rangle$ & 保持不变 \\
					\bottomrule
				\end{tabular}
				\caption{幺正变换前后的对应关系}
			\end{table}
		\subsection{引入一个含时幺正变换}
			
			我们希望用一个时间相关的幺正算符 \(U(t)\) 来做一个“主动的视角变换”。幺正性要求 \(U^{\dagger}(t)U(t) = U(t)U^{\dagger}(t) = I\)。
			
			我们定义变换后的态矢量 \(|\tilde{\psi}(t)\rangle\) 为:
			\[
			|\tilde{\psi}(t)\rangle = U^{\dagger}(t) |\psi(t)\rangle
			\]
			相应地,原始态矢量可以表示为:
			\[
			|\psi(t)\rangle = U(t) |\tilde{\psi}(t)\rangle
			\]
			
			我们的目标:找到一个新的哈密顿量 \(\tilde{H}(t)\),使得变换后的态矢量 \(|\tilde{\psi}(t)\rangle\) 也满足一个形式完全相同的薛定谔方程:
			\[
			i\hbar \frac{\partial}{\partial t} |\tilde{\psi}(t)\rangle = \tilde{H}(t) |\tilde{\psi}(t)\rangle
			\]
			这个 \(\tilde{H}(t)\) 就是我们要寻找的、在新绘景(新视角)下有效的哈密顿量。
			
			\subsection*{ 推导 \(\tilde{H}(t)\) 的表达式}
			
			我们从新态矢量 \(|\tilde{\psi}(t)\rangle\) 的定义出发,对其时间 \(t\) 求导。这里的关键是,\(U^{\dagger}(t)\) 和 \(|\psi(t)\rangle\) 都依赖于时间,所以要用乘积求导法则:
			\[
			\begin{aligned}
				i\hbar \frac{\partial}{\partial t} |\tilde{\psi}(t)\rangle &= i\hbar \frac{\partial}{\partial t} \left[ U^{\dagger}(t) |\psi(t)\rangle \right] \\
				&= i\hbar \left[ \frac{\partial U^{\dagger}(t)}{\partial t} |\psi(t)\rangle + U^{\dagger}(t) \frac{\partial |\psi(t)\rangle}{\partial t} \right]
			\end{aligned}
			\]
			
			现在,我们将原始薛定谔方程 \(i\hbar \frac{\partial |\psi(t)\rangle}{\partial t} = H(t) |\psi(t)\rangle\) 代入上式第二项:
			\[
			\begin{aligned}
				i\hbar \frac{\partial}{\partial t} |\tilde{\psi}(t)\rangle &= i\hbar \frac{\partial U^{\dagger}(t)}{\partial t} |\psi(t)\rangle + U^{\dagger}(t) H(t) |\psi(t)\rangle
			\end{aligned}
			\]
			
			接下来,我们要将方程右边的 \(|\psi(t)\rangle\) 用新的态矢量 \(|\tilde{\psi}(t)\rangle\) 表示。因为 \(|\psi(t)\rangle = U(t) |\tilde{\psi}(t)\rangle\),代入上式:
			\[
			\begin{aligned}
				i\hbar \frac{\partial}{\partial t} |\tilde{\psi}(t)\rangle &= i\hbar \frac{\partial U^{\dagger}(t)}{\partial t} U(t) |\tilde{\psi}(t)\rangle + U^{\dagger}(t) H(t) U(t) |\tilde{\psi}(t)\rangle \\
				&= \left[ i\hbar \frac{\partial U^{\dagger}(t)}{\partial t} U(t) + U^{\dagger}(t) H(t) U(t) \right] |\tilde{\psi}(t)\rangle
			\end{aligned}
			\]
			
			为了得到形如 \(i\hbar \partial_t |\tilde{\psi}\rangle = \tilde{H} |\tilde{\psi}\rangle\) 的方程,我们令:
			\[
			\tilde{H}(t) = i\hbar \frac{\partial U^{\dagger}(t)}{\partial t} U(t) + U^{\dagger}(t) H(t) U(t)
			\]
			
			\subsection*{ 化简表达式}
			
			上述表达式是正确的,但可以写成更对称、更常见的形式。我们利用幺正算符的性质来化简第一项。
			
			由于 \(U^{\dagger}(t)U(t) = I\),对该恒等式两边求时间导数:
			\[
			\frac{\partial U^{\dagger}(t)}{\partial t} U(t) + U^{\dagger}(t) \frac{\partial U(t)}{\partial t} = 0
			\]
			因此,
			\[
			\frac{\partial U^{\dagger}(t)}{\partial t} U(t) = - U^{\dagger}(t) \frac{\partial U(t)}{\partial t}
			\]
			
			将这一关系代入我们之前得到的 \(\tilde{H}(t)\) 表达式:
			\[
			\begin{aligned}
				\tilde{H}(t) &= i\hbar \left[ - U^{\dagger}(t) \frac{\partial U(t)}{\partial t} \right] + U^{\dagger}(t) H(t) U(t) \\
				&= U^{\dagger}(t) H(t) U(t) - i\hbar\, U^{\dagger}(t) \frac{\partial U(t)}{\partial t}
			\end{aligned}
			\]
			
			于是得到:
			\[
			\boxed{\tilde{H} = U^{\dagger} H U - i\hbar\, U^{\dagger} \frac{\partial U}{\partial t}}
			\]
		\subsection{示例:一维无限深方势阱的坐标基与能量基}
		考虑一维无限深方势阱 $x \in [0, L]$。
		\begin{itemize}
			\item \textbf{坐标基}:$\{ |x\rangle \}$,连续基,满足 $\langle x | x' \rangle = \delta(x - x')$。
			\item \textbf{能量基(旧基)}:$\{ |n\rangle \}$,离散基,对应本征态 $\psi_n(x) = \langle x | n \rangle = \sqrt{\frac{2}{L}} \sin\left( \frac{n\pi x}{L} \right)$。
		\end{itemize}
		
		\textbf{问题:} 将坐标算符 $\hat{X}$ 从坐标表象变换到能量表象。
		
		\textbf{解:}
		\begin{enumerate}
			\item \textbf{明确目标:} 求算符 $\hat{X}$ 在能量基 $\{|n\rangle\}$ 下的矩阵元 $X_{mn} = \langle m | \hat{X} | n \rangle$。
			\item \textbf{利用完备性关系:} 在坐标基下,完备性关系为 $\int dx |x\rangle\langle x| = \hat{I}$。
			\begin{align*}
				X_{mn} &= \langle m | \hat{X} | n \rangle 
				= \langle m | \hat{I} \hat{X} \hat{I} | n \rangle \\
				&= \int_0^L \int_0^L 
				\langle m | x \rangle 
				\langle x | \hat{X} | x' \rangle 
				\langle x' | n \rangle \, dx\, dx' \\
				&= \int_0^L \int_0^L 
				\psi_m^*(x) \cdot \left[ x \delta(x - x') \right] \cdot \psi_n(x') \, dx\, dx' \\
				&= \int_0^L \psi_m^*(x) \, x \, \psi_n(x) \, dx.
			\end{align*}
			\item \textbf{代入波函数计算:}
			\[
			X_{mn} = \frac{2}{L} \int_0^L 
			\sin\left( \frac{m\pi x}{L} \right) \, x \, 
			\sin\left( \frac{n\pi x}{L} \right) \, dx.
			\]
			这个积分可以解析求出。当 $m=n$ 时,得到对角元(平均位置);当 $m \ne n$ 时,得到非对角元,描述了不同能级之间的“跃迁矩”。
		\end{enumerate}
		
		\textbf{这个例子展示了如何利用狄拉克符号的完备性关系,清晰地将一个抽象算符 $\langle m | \hat{X} | n \rangle$ 转化为一个具体的、可计算的积分 $\int \psi_m^*(x) x \psi_n(x) dx$。} 这正是狄拉克符号强大与优雅之处。
		
		\subsection{总结步骤}
		\begin{enumerate}
			\item 写出两组基的完备性关系。
			\item 对于态矢量:在求新基分量时,在投影算符 $\langle v_k|$ 和态矢量 $|\psi\rangle$ 之间插入旧基的完备性关系 $\sum_n |u_n\rangle\langle u_n|$。
			\item 对于算符:在求新基矩阵元 $\langle v_i | \hat{A} | v_j \rangle$ 时,在 $\langle v_i|$ 和 $\hat{A}$ 之间,以及 $\hat{A}$ 和 $|v_j\rangle$ 之间,分别插入旧基的完备性关系。
			\item 识别变换矩阵元 $S_{kn} = \langle v_k | u_n \rangle$,并用它表示结果。
			\item 记住幺正性:$S^{-1} = S^\dagger$,反向变换只需使用 $S^\dagger$。
		\end{enumerate}
		
		\begin{problembox}{Example}
			在量子力学中,当我们从标准的 $\sigma_z$ 表象(以 $\sigma_z$ 的本征态 $\{|z+\rangle, |z-\rangle\}$ 为基)切换到 $\sigma_x$ 表象时,算符的矩阵形式会发生相应的变换。
		\end{problembox}
		
		幺正变换矩阵是连接两个不同基矢组的桥梁。其核心在于将“新基矢”用“旧基矢”的线性组合来表示。设旧基矢为 $\{|e_i\rangle\}$,新基矢为 $\{|\psi_j\rangle\}$。变换矩阵 $U$ 的矩阵元素定义为旧基矢与新基矢的内积:
		\[ U_{ij} = \bra{\psi_i}( \sum_n |\psi_n\rangle \langle e_n| )\ket{\psi_j} = \langle e_i | \psi_j \rangle \]
		
		这意味着 $U$ 的每一列分别是新基矢在旧基矢表象下的坐标表示。
		
		1. \textbf{旧基矢}($\sigma_z$ 本征态):
		\[ |z+\rangle = \begin{pmatrix} 1 \\ 0 \end{pmatrix}, \quad |z-\rangle = \begin{pmatrix} 0 \\ 1 \end{pmatrix} \]
		
		2. \textbf{新基矢}($\sigma_x$ 本征态,在旧表象下表示):
		\[ |x+\rangle = \frac{1}{\sqrt{2}} \begin{pmatrix} 1 \\ 1 \end{pmatrix}, \quad |x-\rangle = \frac{1}{\sqrt{2}} \begin{pmatrix} 1 \\ -1 \end{pmatrix} \]
		
		3. \textbf{构造矩阵 $U$}:
		将新基矢按列排列:
		\[ U = \begin{pmatrix} \langle z+|x+\rangle & \langle z+|x-\rangle \\ \langle z-|x+\rangle & \langle z-|x-\rangle \end{pmatrix} = \frac{1}{\sqrt{2}} \begin{pmatrix} 1 & 1 \\ 1 & -1 \end{pmatrix} \]
		\subsection*{算符的变换公式}
		一旦得到 $U$,任何算符 $A$ 在新表象下的形式 $A'$ 可由下式给出:
		\[ A' = U^\dagger A U \]
		其中 $U^\dagger$ 是 $U$ 的共轭转置。对于上述矩阵 $U$,由于其为实对称矩阵,故 $U^\dagger = U$。
		\begin{itemize}
			\item \textbf{$\sigma_x$ 的形式}(在该表象下对角化):
			\[ \sigma_x' = \begin{pmatrix} 1 & 0 \\ 0 & -1 \end{pmatrix} \]
			
			\item \textbf{$\sigma_y$ 的形式}:
			\[ \sigma_y' = \begin{pmatrix} 0 & i \\ -i & 0 \end{pmatrix} \]
			
			\item \textbf{$\sigma_z$ 的形式}:
			\[ \sigma_z' = \begin{pmatrix} 0 & 1 \\ 1 & 0 \end{pmatrix} \]
		\end{itemize}
		\newpage
		\section{时间演化算符、幺正变换与绘景}
		\subsection{广义绘景的性质}
		从表象 $ A $ 到表象 $ B $ 的变换通过以下态矢变换方式实现:
		\begin{equation}
			|\psi_A(t)\rangle \to |\psi_B(t)\rangle = U_{BA}(t) |\psi_A(t)\rangle,\label{18}
		\end{equation}
		
		其中 $ U_{BA}(t) $ 是由\textbf{不含时的厄米算符} $ G $ 生成的幺正算符:
		\begin{equation}
			U_{BA}(t) = e^{i G t / \hbar}.
		\end{equation}
		
		由定义可知, $ U_{BA} $ 的时间导数为:
		\begin{equation}
			\dot{U}_{BA} = \frac{i}{\hbar} G U_{BA} = \frac{i}{\hbar} U_{BA} G.
		\end{equation}
		
		逆变换 $ U_{BA}^\dagger $ 满足:
		\begin{equation}
			U_{BA}^\dagger(t) = e^{-i G t / \hbar}.
		\end{equation}
		
		根据这些定义,显然有:
		\begin{equation}
			U_{BA}^\dagger(t) U_{BA}(t) = I,
		\end{equation}
		
		根据观测结果的不变性:
		\begin{equation}
			\begin{aligned}
				\langle O \rangle_B &= \left\langle \psi_B \left| O_B \right| \psi_B \right\rangle = \left\langle \psi_A \left| U_{BA}^\dagger  O_B  U_{BA} \right| \psi_A \right\rangle \\
				&= \left\langle \psi_A \left| O_A \right| \psi_A \right\rangle \\
				&= \langle O \rangle_A
			\end{aligned}
		\end{equation}
		
		得到算符变换:
		\begin{equation}
			O_A(t) \to O_B(t) = U_{BA}(t) O_A(t) U_{BA}^\dagger(t),\label{19}
		\end{equation}
		
		假设表象 $ A $ 中 \boxed{\text{态矢的运动方程}} 为:
		\begin{equation}
			\frac{d}{dt} \left|\psi_A\right\rangle = -\frac{i}{\hbar} H_{\psi,A} \left|\psi_A\right\rangle,
		\end{equation}
		
		\boxed{\text{算符的运动方程}} 为:
		\begin{equation}
			\frac{d}{dt} O_A = \frac{i}{\hbar} \left[ H_{O,A}, O_A \right].
		\end{equation}
		
		同理,表象 $ B $ 中的运动方程为:
		\begin{equation}
			\frac{d}{dt} \left|\psi_B\right\rangle = -\frac{i}{\hbar} H_{\psi,B} \left|\psi_B\right\rangle,
		\end{equation}
		
		和
		\begin{equation}
			\frac{d}{dt} O_B = \frac{i}{\hbar} \left[ H_{O,B}, O_B \right].
		\end{equation}
		
		通过直接求导,可推导两表象中 $ H_\psi $ 和 $ H_O $ 的关系:
		\begin{equation}
			\begin{aligned}
				\frac{d}{dt} \left|\psi_B\right\rangle &= \frac{d}{dt} U_{BA} \left|\psi_A\right\rangle \\
				&= -\frac{i}{\hbar} \left( U_{BA} H_{\psi,A} U_{BA}^\dagger - G \right) U_{BA} \left|\psi_A\right\rangle,
			\end{aligned}
		\end{equation}
		
		由此可得:
		\begin{equation}
			H_{\psi,B} = U_{BA} H_{\psi,A} U_{BA}^\dagger - G.
		\end{equation}
		
		同理,对 $ O_B $ 求导:
		\begin{equation}
			\begin{aligned}
				\frac{d}{dt} O_B &= \frac{d}{dt} U_{BA} O_A U_{BA}^\dagger \\
				&= \frac{i}{\hbar} \left[ U_{BA} H_{O,A} U_{BA}^\dagger + G, O_B \right],
			\end{aligned}
		\end{equation}
		
		因此:
		\begin{equation}
			H_{O,B} = U_{BA} H_{O,A} U_{BA}^\dagger + G.
		\end{equation}
		
		注意到 $ H = H_\psi + H_O $ 是不变量,即:
		\begin{equation}
			H_{\psi,B} + H_{O,B} = H_{\psi,A} + H_{O,A}.
		\end{equation}
		
		\begin{center}
			\fbox{\parbox{0.85\linewidth}{\centering 这表明,不同表象对应于将哈密顿量分解为两部分的不同方式——\\ \textbf{一部分与态关联,另一部分与算符关联}。}}
		\end{center}
	
		
		我们可在表象 $ A $ 中定义时间传播子 $ U_A(t) $ :
		\begin{equation}
			\left|\psi_A(t)\right\rangle = U_A(t) \left|\psi_A(0)\right\rangle,\label{34}
		\end{equation}
		
		同理,表象 $ B $ 中的传播子为:
		\begin{equation}
			\left|\psi_B(t)\right\rangle = U_B(t) \left|\psi_B(0)\right\rangle.\label{35}
		\end{equation}
		
		为找到两传播子的关系,从式 (\ref{18}) 出发,利用式 (\ref{34}) 替换 $ |\psi_A(t)\rangle $ ,得到:
		\begin{equation}
			\left|\psi_B(t)\right\rangle = U_{BA}(t) U_A(t) \left|\psi_A(0)\right\rangle.
		\end{equation}
		
		在 $ t=0 $ 时, $ U_{BA}(0) = U_{BA}^\dagger(0) = I $, 因此两表象中的态和算符完全相同,即 $ |\psi_A(0)\rangle = |\psi_B(0)\rangle $ 。结合式 (\ref{35}) 可得:
		\begin{equation}
			U_B(t) \left|\psi_B(0)\right\rangle = U_{BA}(t) U_A(t) \left|\psi_B(0)\right\rangle,
		\end{equation}
		因此:
		\begin{equation}
			U_B(t) = U_{BA}(t) U_A(t).
		\end{equation}
		
		需注意, $ H_\psi $, $ H_O $ 和时间传播子均不遵循式 (\ref{19}) 所示的算符变换规律——这是因为该变换具有时间依赖性,而这些算符的定义仅依赖于其在特定表象中的作用。
		
		\subsection{薛定谔绘景时间演化算符 $U(t, t_0)$}
		
		在薛定谔绘景中,态矢量随时间的变化由时间演化算符定义:
		\begin{equation}
			|\psi(t)\rangle = U(t, t_0) |\psi(t_0)\rangle
		\end{equation}
		
		\subsubsection{幺正性}
		由于量子系统的总概率 $\langle \psi(t) | \psi(t) \rangle$ 必须守恒(归一化保持),则有:
		\begin{equation}
			\langle \psi(t_0) | U^\dagger U | \psi(t_0) \rangle = \langle \psi(t_0) | \psi(t_0) \rangle
		\end{equation}
		由此推导出 $U^\dagger U = I$,即时间演化是一个\textbf{幺正变换}。
		
		\subsubsection{时间演化算符推导}
		将定义式代入薛定谔方程 $i\hbar \frac{\partial}{\partial t} |\psi(t)\rangle = H |\psi(t)\rangle$:
		\begin{equation}
			i\hbar \frac{\partial}{\partial t} U(t, t_0) |\psi(t_0)\rangle = H U(t, t_0) |\psi(t_0)\rangle
		\end{equation}
		由于这对任意 $|\psi(t_0)\rangle$ 成立,算符 $U$ 满足:
		\begin{equation}
			i\hbar \frac{\partial}{\partial t} U(t, t_0) = H U(t, t_0)
		\end{equation}
		对于不含时哈密顿量 $H$,得到时间演化算符
		\begin{equation}
			 U(t, t_0) = e^{-\frac{i}{\hbar}H(t-t_0)}
		\end{equation}
		为契合矩阵力学, 利用指数算符的泰勒展开, 时间演化算符还可以写作:
		\begin{equation}
		\hat{U}(t) = \sum_{k=0}^{\infty} \frac{1}{k!} \left( -\frac{i\hat{H}t}{\hbar} \right)^k,
		\end{equation}
		对应矩阵元:
		\begin{equation}
		U_{mn}(t) = \sum_{k=0}^{\infty} \frac{1}{k!} \left( -\frac{it}{\hbar} \right)^k \bra{m}\hat{H}^k\ket{n}.
		\end{equation}
		\subsection{海森堡绘景(Heisenberg Picture)}
		
		在海森堡绘景中,我们将演化的“角色”从态矢量完全转移到算符上。
		
		\subsubsection{算符演化的推导}
		通过幺正变换, 将任意时刻的海森堡态矢转换为为初始态矢:
		\begin{equation}
			|\psi_H(t)\rangle = |\psi_S(t_0)\rangle = U^\dagger(t, t_0) \ket{\psi_S(t)}
		\end{equation}
		
		为了保持物理观测值 $\langle A \rangle = \langle \psi(t) | A_S | \psi(t) \rangle$ 不变,观测值的等效要求:
		\begin{equation}
			\langle \psi_S(t) | A_S | \psi_S(t) \rangle = \langle \psi_H | U^\dagger A_S U | \psi_H \rangle \equiv \langle \psi_H | A_H(t) | \psi_H \rangle
		\end{equation}
		
		因此,海森堡算符定义为:$A_H(t) = U^\dagger(t, t_0) A_S U(t, t_0)$。
		
		\subsubsection{海森堡运动方程的推导}
		对 $A_H(t)$ 求时间导数:
		\begin{align}
			\frac{d}{dt} A_H(t) &= \frac{\partial U^\dagger}{\partial t} A_S U + U^\dagger A_S \frac{\partial U}{\partial t} + U^\dagger \frac{\partial A_S}{\partial t} U \\
			&= \left( \frac{i}{\hbar} H U^\dagger \right) A_S U + U^\dagger A_S \left( -\frac{i}{\hbar} H U \right) + \left( \frac{\partial A_S}{\partial t} \right)_H \\
			&= \frac{i}{\hbar} [H_H, A_H] + \left( \frac{\partial A_S}{\partial t} \right)_H
		\end{align}
		
		整理得到\textbf{海森堡方程}:
		\begin{equation}
			\frac{d}{dt} A_H(t) = \frac{1}{i\hbar} [A_H(t), H] + \left( \frac{\partial A_S}{\partial t} \right)_H
		\end{equation}	
		
		如果在薛定谔表象下\(A_S\)含时,我们需要考虑最后一项
		\subsection{相互作用绘景(Interaction Picture)的推导}
		
		相互作用绘景(也称狄拉克绘景)是处理微扰问题的核心工具。我们将总哈密顿量拆分为自由项 $H_0$ 和相互作用项 $V_I$:
		\begin{equation}
			H = H_0 + V_S(t)
		\end{equation}
		我们希望找到一个态矢 \( |\psi_I(t)\rangle \) 满足
		\begin{equation}
			\frac{d}{dt} |\psi_I(t)\rangle = -\frac{i}{\hbar} V_I(t) |\psi_I(t)\rangle,
		\end{equation}
		其中 \( V_I(t) \) 是相互作用表象下的微扰算符。
		
		寻找一个幺正算符 \( U(t) \),使得 \( |\psi_I(t)\rangle = U(t) |\psi_s(t)\rangle \),并利用两种表象的演化方程确定 \( U(t) \) 的具体形式。
		
		我们假设存在幺正算符 \( U(t) \),使得
		\begin{equation}
			|\psi_I(t)\rangle = U(t) |\psi_s(t)\rangle,
			\quad U(0) = 1 \quad (\text{即 } |\psi_I(0)\rangle = |\psi_s(0)\rangle).
		\end{equation}
		
		步骤 1:对 \( |\psi_I(t)\rangle = U(t) |\psi_s(t)\rangle \) 两边求导,并代入已知的演化方程:
		\begin{equation}
			\frac{d}{dt} |\psi_I(t)\rangle = \dot{U}(t) |\psi_s(t)\rangle + U(t) \frac{d}{dt} |\psi_s(t)\rangle.
		\end{equation}
		
		利用薛定谔方程 \(\displaystyle \frac{d}{dt} |\psi_s(t)\rangle = -\frac{i}{\hbar} H_s |\psi_s(t)\rangle \),得
		\begin{equation}
			\frac{d}{dt} |\psi_I(t)\rangle = \dot{U}(t) |\psi_s(t)\rangle - \frac{i}{\hbar} U(t) H_s |\psi_s(t)\rangle. \label{6.4.1}
		\end{equation}
		
		步骤 2:根据相互作用表象的演化方程,又有
		\begin{equation}
			\frac{d}{dt} |\psi_I(t)\rangle = -\frac{i}{\hbar} V_I(t) |\psi_I(t)\rangle = -\frac{i}{\hbar} V_I(t) U(t) |\psi_s(t)\rangle. \label{6.4.2}
		\end{equation}
		
		步骤 3:比较 (\ref{6.4.1}) 和 (\ref{6.4.2}),由于 \( |\psi_s(t)\rangle \) 任意,得到算符方程:
		\begin{equation}
			\dot{U}(t) - \frac{i}{\hbar} U(t) H_s = -\frac{i}{\hbar} V_I(t) U(t),
		\end{equation}
		即
		\begin{equation}
			\dot{U}(t) = \frac{i}{\hbar} U(t) H_s - \frac{i}{\hbar} V_I(t) U(t). \label{6.4.3}
		\end{equation}
		
		步骤 4:将 \( H_s = H_0 + V_s \) 代入 (\ref{6.4.3}),并利用 \( V_I(t) \) 与 \( V_s \) 的关系。注意,若我们取 \( U(t) = e^{iH_0 t/\hbar} \),则自然有
		\begin{equation}
			V_I(t) = U(t) V_s U^\dagger(t).
		\end{equation}
		但此时我们尚未确定 \( U(t) \),可将此关系作为合理假设(因为相互作用表象通常要求算符如此变换)。代入 (\ref{6.4.3}):
		\begin{equation}
			\dot{U}(t) = \frac{i}{\hbar} U(t) (H_0 + V_s) - \frac{i}{\hbar} \big( U(t) V_s U^\dagger(t) \big) U(t).
		\end{equation}
		由于 \( U^\dagger(t) U(t) = 1 \)(幺正性),右边第二项简化为 \( \displaystyle -\frac{i}{\hbar} U(t) V_s \),于是
		\begin{equation}
			\dot{U}(t) = \frac{i}{\hbar} U(t) H_0 + \frac{i}{\hbar} U(t) V_s - \frac{i}{\hbar} U(t) V_s = \frac{i}{\hbar} U(t) H_0. \label{6.4.4}
		\end{equation}
		
		步骤 5:解微分方程 (\ref{6.4.4})。这是一个一阶线性方程,解为
		\begin{equation}
			U(t) = e^{iH_0 t/\hbar} U(0).
		\end{equation}
		
		取初始条件 \( U(0) = 1 \),得
		\begin{equation}
			U(t) = e^{iH_0 t/\hbar}.
		\end{equation}
		
		步骤 6:代回变换假设,即得
		\begin{equation}
			\boxed{|\psi_I(t)\rangle = e^{iH_0 t/\hbar} |\psi_s(t)\rangle}.
		\end{equation}
		
		这个变换关系正是相互作用表象的定义式。它表明,相互作用表象的态矢是通过从薛定谔表象的态矢中"剔除"自由演化部分 \( e^{-iH_0 t/\hbar} \) 而得到的,因此其演化完全由微扰 \( V_I(t) \) 驱动。
		
		反过来,薛定谔表象的态矢可写为 \( |\psi_s(t)\rangle = e^{-iH_0 t/\hbar} |\psi_I(t)\rangle \)。
		
		\subsubsection{绘景的定义与变换}
		我们定义相互作用绘景下的态矢量 $\ket{\psi_I(t)}$ 为剔除了 $H_0$ 演化后的状态:
		\begin{equation}
			\ket{\psi_I(t)} = e^{\frac{i}{\hbar} H_0 (t-t_0)} \ket{\psi_S(t)} 
		\end{equation}
		
		相应地,为了保持观测物理量不变,算符 $A_I(t)$ 必须随 $H_0$ 演化:
		\begin{equation}
			A_I(t) = e^{\frac{i}{\hbar} H_0 (t-t_0)} A_S e^{-\frac{i}{\hbar} H_0 (t-t_0)}
		\end{equation}
		
		同时我们也有相互作用表象态矢与薛定谔表象态矢的联系
		\begin{equation}
			\ket{\psi_S(t)} = e^{-\frac{i}{\hbar} H_0 (t-t_0)} \ket{\psi_I(t)}
		\end{equation}
		
		\subsubsection{态矢量演化方程的推导}
		我们来推导 $\ket{\psi_I(t)}$ 随时间演化遵循的方程。对定义式两边关于时间 $t$ 求导(为简化书写,令 $t_0 = 0$):
		\begin{equation}
			i\hbar \frac{\partial}{\partial t} \ket{\psi_I(t)} = i\hbar \frac{\partial}{\partial t} \left( e^{\frac{i}{\hbar} H_0 t} \ket{\psi_S(t)} \right)
		\end{equation}
		
		利用导数的乘积法则:
		\begin{equation}
			i\hbar \frac{\partial}{\partial t} \ket{\psi_I(t)} = i\hbar \left[ \left( \frac{i}{\hbar} H_0 \right) e^{\frac{i}{\hbar} H_0 t} \ket{\psi_S(t)} + e^{\frac{i}{\hbar} H_0 t} \frac{\partial}{\partial t} \ket{\psi_S(t)} \right]
		\end{equation}
		
		代入薛定谔方程 $i\hbar \frac{\partial}{\partial t} \ket{\psi_S(t)} = (H_0 + V_S) \ket{\psi_S(t)}$:
		\begin{align}
			i\hbar \frac{\partial}{\partial t} \ket{\psi_I(t)} &= -H_0 e^{\frac{i}{\hbar} H_0 t} \ket{\psi_S(t)} + e^{\frac{i}{\hbar} H_0 t} (H_0 + V_S) \ket{\psi_S(t)} \\
			&= -H_0 \ket{\psi_I(t)} + e^{\frac{i}{\hbar} H_0 t} H_0 \ket{\psi_S(t)} + e^{\frac{i}{\hbar} H_0 t} V_S \ket{\psi_S(t)}
		\end{align}
		
		由于 $H_0$ 与 $e^{\frac{i}{\hbar} H_0 t}$ 对易,中间两项抵消:
		\begin{align}
			i\hbar \frac{\partial}{\partial t} \ket{\psi_I(t)} &= e^{\frac{i}{\hbar} H_0 t} V_S \ket{\psi_S(t)} \\
			&= e^{\frac{i}{\hbar} H_0 t} V_S \left( e^{-\frac{i}{\hbar} H_0 t} \ket{\psi_I(t)} \right) \\
			&= \left( e^{\frac{i}{\hbar} H_0 t} V_S e^{-\frac{i}{\hbar} H_0 t} \right) \ket{\psi_I(t)}
		\end{align}
		
		定义相互作用绘景下的相互作用算符 
		\begin{equation}
			V_I(t) = e^{\frac{i}{\hbar} H_0 t} V_S e^{-\frac{i}{\hbar} H_0 t}
		\end{equation}
		
		我们得到:
		\begin{equation}
			i\hbar \frac{\partial}{\partial t} \ket{\psi_I(t)} = V_I(t) \ket{\psi_I(t)}
		\end{equation}
		
		这就是著名的\textbf{朝永振一郎-施温格方程}。
		
		\subsubsection{算符变换的推导}
		为了保证物理观测值在绘景变换下是不变的,必须满足:
		\begin{equation}
			\langle \psi_S(t) | A_S | \psi_S(t) \rangle = \langle \psi_I(t) | A_I(t) | \psi_I(t) \rangle
		\end{equation}
		
		代入态矢量变换关系 $\ket{\psi_S(t)} = e^{-\frac{i}{\hbar} H_0 t} \ket{\psi_I(t)}$,得到:
		\begin{equation}
			\langle \psi_I(t) | e^{\frac{i}{\hbar} H_0 t} A_S e^{-\frac{i}{\hbar} H_0 t} | \psi_I(t) \rangle = \langle \psi_I(t) | A_I(t) | \psi_I(t) \rangle
		\end{equation}
		
		由此定义相互作用绘景下的算符为:
		\begin{equation}
			A_I(t) = e^{\frac{i}{\hbar} H_0 t} A_S e^{-\frac{i}{\hbar} H_0 t}
		\end{equation}
		
		对时间 $t$ 求导可得算符的演化方程:
		\begin{equation}
			\frac{d}{dt} A_I(t) = \frac{1}{i\hbar} [A_I(t), H_0] + \left( \frac{\partial A_S}{\partial t} \right)_I
		\end{equation}
		
		这说明算符 $A_I$ 的时间演化仅由自由项 $H_0$ 决定。
		\begin{table}[h!]
			\caption{总结:三大绘景的对比}
			\centering
			\begin{tabular}{@{}llll@{}}
				\toprule
				绘景 & 态矢量演化 & 算符演化 & 核心方程 \\ \midrule
				薛定谔 (S) & $i\hbar \partial_t |\psi_S\rangle = H |\psi_S\rangle$ & $A_S$ 固定 & 薛定谔方程 \\
				海森堡 (H) & 固定不变 & $\frac{d}{dt}A_H = \frac{1}{i\hbar}[A_H, H]$ & 海森堡方程 \\
				相互作用 (I) & $i\hbar \partial_t |\psi_I\rangle = V_I |\psi_I\rangle$ & $\frac{d}{dt}A_I = \frac{1}{i\hbar}[A_I, H_0]$ & 朝永振一郎-施温格方程 \\ \bottomrule
			\end{tabular}
		\end{table}
		\newpage
		\section{量子态的时间演化}
		\section*{离散谱的时间演化}
		在量子力学中,对于哈密顿量不显含时间且能谱离散的系统,波函数随时间演化可通过以下步骤展开:
		
		\subsection{求解定态薛定谔方程}
		求出哈密顿量 \(\hat{H}\) 的本征态 \(\psi_n\) 和本征值 \(E_n\):
		\[
		\hat{H} \psi_n = E_n \psi_n,
		\]
		其中 \(\{\psi_n\}\) 构成一组完备正交归一基。
		
		\subsection{展开初始波函数}
		将初始波函数 \(\Psi(x,0)\) 按能量本征态展开:
		\[
		\Psi(x,0) = \sum_n c_n \psi_n(x), \quad c_n = \int \psi_n^*(x) \Psi(x,0)  \dd{x}.
		\]
		
		\subsection{加入时间演化因子}
		任意时刻 \(t\) 的波函数为:
		\[
		\Psi(x,t) = \sum_n c_n e^{-i E_n t / \hbar} \psi_n(x).
		\]
		
		该展开式是含时薛定谔方程的解,描述了波函数随时间演化的一般形式。
		
		\section*{连续谱的时间演化}
		
		对于具有\textbf{连续谱}的量子系统(例如自由粒子、电离态、散射态),波函数的时间演化在数学上更精细,但物理思想与离散谱完全相同。
		
		核心思想仍然是:\textbf{将任意初始波函数按能量本征态展开,然后每个本征态乘以其时间演化相位因子}。关键在于,连续谱的“展开”是一个积分,本征函数具有不同的归一化方式。
		
		\subsection{核心步骤与公式(以自由粒子一维运动为例)}
		
		考虑哈密顿量 \(\displaystyle \hat{H} = \frac{\hat{p}^2}{2m}\),其能量本征值 \(\displaystyle E = \frac{\hbar^2 k^2}{2m}\) 是连续的(\(E \ge 0\)),对应的本征函数是平面波。
		
		\subsubsection{求解“定态薛定谔方程”}
		方程的解是平面波,但\textbf{不可归一化}(在无穷空间积分发散)。我们采用“狄拉克δ函数归一化”:
		\[
		\psi_k(x) = \frac{1}{\sqrt{2\pi}} e^{ikx}, \quad E_k = \frac{\hbar^2 k^2}{2m}
		\]
		其正交归一性为:
		\[
		\int_{-\infty}^{\infty} \psi_{k'}^*(x) \psi_k(x) \dd{x} = \delta(k - k').
		\]
		
		\subsubsection{ 展开初始波函数}
		由于 \(k\) 是连续变量,展开变为积分。初始波函数 \(\Psi(x, 0)\) 可以表示为所有动量本征态的叠加:
		\[
		\Psi(x, 0) = \int_{-\infty}^{\infty} \phi(k) \psi_k(x) \dd{k} = \frac{1}{\sqrt{2\pi}} \int_{-\infty}^{\infty} \phi(k) e^{ikx} \dd{k}.
		\]
		这里,展开“系数” \(\phi(k)\) 本身是一个关于 \(k\) 的连续函数,称为\textbf{动量空间的波函数}。它由初始波函数通过傅里叶变换得到:
		\[
		\phi(k) = \int_{-\infty}^{\infty} \psi_k^*(x) \Psi(x, 0) \dd{x} = \frac{1}{\sqrt{2\pi}} \int_{-\infty}^{\infty} \Psi(x, 0) e^{-ikx} \dd{x}.
		\]
		
		\subsubsection{加入时间演化因子}
		每个具有确定波数 \(k\) 的平面波本征态随时间演化的相位因子是 \(e^{-i E_k t / \hbar} = e^{-i (\hbar k^2 / 2m) t}\)。因此,任意时刻的波函数为:
		\[
		\boxed{\Psi(x, t) = \frac{1}{\sqrt{2\pi}} \int_{-\infty}^{\infty} \phi(k) e^{i k x} e^{-i \frac{\hbar k^2}{2m} t} \dd{k}}.
		\]
		这是连续谱时间演化的一般公式。其物理图像是:\textbf{无数个不同动量、不同相位的平面波在时空各点发生干涉,其干涉图样 \(|\Psi(x, t)|^2\) 随时间的演化,就构成了波包的扩散与传播。}
		
		\begin{table}[h!]
			\centering
			\caption{离散谱与连续谱的对比}
			\begin{tabular}{p{0.2\linewidth} p{0.4\linewidth} p{0.41\linewidth}}
				\toprule
				\textbf{特性} & \textbf{离散谱} & \textbf{连续谱} \\
				\midrule
				\textbf{本征值} & 分立,可数: \(E_n\) & 连续,不可数: \(E_k\) 或 \(E\) \\
				\textbf{本征函数} & 通常可归一化: \(\int \|\psi_n\|^2 \dd{x} = 1\) & 通常\textbf{不可归一化},采用δ函数归一化 \\
				\textbf{完备性关系} & \(\sum_n \psi_n^*(x') \psi_n(x) = \delta(x-x')\) & \(\int \dd{k}  \psi_k^*(x') \psi_k(x) = \delta(x-x')\) \\
				\textbf{展开形式} & 求和: \(\Psi = \sum_n c_n \psi_n\) & 积分: \(\Psi = \int \dd{k}  \phi(k) \psi_k\) \\
				\textbf{展开系数} & 离散数列: \(c_n = \int \psi_n^* \Psi \dd{x}\) & 连续函数: \(\phi(k) = \int \psi_k^* \Psi \dd{x}\) \\
				\textbf{时间演化} & \(\Psi(t) = \sum_n c_n e^{-iE_n t/\hbar} \psi_n\) & \(\Psi(t) = \int \phi(k) e^{-iE_k t/\hbar} \psi_k \dd{k}\) \\
				\bottomrule
			\end{tabular}
		\end{table}
			
		\subsection{更一般的情况}
		对于任意势场(如有限深方势阱,其可能同时具有离散的束缚态谱和连续的散射态谱),一个完备的波函数展开应包含\textbf{离散求和与连续积分两部分}:
		\[
		\Psi(x, t) = \sum_{n(\text{离散})} c_n e^{-iE_n t/\hbar} \psi_n(x) + \int_{E_{\text{cont.}}} c(E) e^{-iE t/\hbar} \psi_E(x) \dd{E},
		\]
		其中第一项对应束缚态,第二项对应非束缚的连续散射态。
		\subsection{物理图像与典型例子}
		
		\subsubsection{例子:高斯波包的扩散}
		一个典型的应用是初始为高斯波包的自由粒子演化:
		\[
		\Psi(x, 0) = \left( \frac{1}{2\pi \sigma_x^2} \right)^{1/4} e^{-x^2/(4\sigma_x^2)} e^{ik_0 x}.
		\]
		\begin{enumerate}
			\item \textbf{展开}: 其动量空间波函数 \(\phi(k)\) 也是一个高斯分布(中心在 \(k_0\))。
			\item \textbf{演化}: 代入上述积分公式,可以解析地求出 \(\Psi(x, t)\)。
			\item \textbf{结果}: 波包的中心以群速度 \(v_g = \hbar k_0 / m\) 运动,同时波包的宽度 \(\sigma_x(t)\) 随时间增加而展宽(扩散)。这正是不同动量分量以不同速度(相速度)运动导致干涉图样变化的结果。
		\end{enumerate}
		
		
		
		\newpage
		\section{同时对角化}
		定理:对于每一个厄米算符$\Omega$,(至少)存在一个由其正交本征矢组成的基。它在这个本征基下是对角的,并且它的本征值就作为它的对角元。\par
		定理:如果$\Omega ,\Lambda $是两个对易的厄米算符,则(至少)存在一个共同的本征态能把它们二者同时对角化\par
		一个小补充:所谓对易,用物理语言来说就是可以同时观测到\par
		在这里我们应该就可以看出来了,如果我们选定了由算符$\Omega$生成的基,如果另一个算符$\Lambda $是与其对易的,那么我们就可以得到一个本征态的两个信息量(请注意,我们始终在原来的空间中,即基不变)\par
		既然我们可以得到一个本征态的两个信息量,很自然地,我们就可以对这个本征态进行标记(如角动量,用轨道角动量 \textit{l} 和磁量子数 \textit{m} 来表示一个态)
		
	\newpage
	
\section{守恒量}
\subsection{Derivation of the Ehrenfest Theorem}
The Ehrenfest theorem describes how quantum mechanical expectation values evolve in time according to the laws of classical mechanics. Here's the detailed derivation:

1. Starting Point: Time-Dependent Schrödinger Equation

The time evolution of a quantum state is governed by:
\begin{equation}
	i\hbar\frac{\partial}{\partial t}|\psi(t)\rangle = \hat{H}|\psi(t)\rangle
\end{equation}
and its adjoint:
\begin{equation}
	-i\hbar\frac{\partial}{\partial t}\langle\psi(t)| = \langle\psi(t)|\hat{H}
\end{equation}

2. Definition of Expectation Value

The expectation value of an operator $\hat{A}$ is:
\begin{equation}
	\langle\hat{A}\rangle = \langle\psi(t)|\hat{A}|\psi(t)\rangle
\end{equation}

3. Time Derivative of the Expectation Value

Using the product rule:
\begin{equation}
	\frac{d}{dt}\langle\hat{A}\rangle = \frac{d}{dt}\left(\langle\psi(t)|\hat{A}|\psi(t)\rangle\right)
	= \left(\frac{\partial}{\partial t}\langle\psi(t)|\right)\hat{A}|\psi(t)\rangle + \langle\psi(t)|\frac{\partial\hat{A}}{\partial t}|\psi(t)\rangle + \langle\psi(t)|\hat{A}\left(\frac{\partial}{\partial t}|\psi(t)\rangle\right)
\end{equation}

4. Substitute Schrödinger Equation

Using the time-dependent Schrödinger equation and its adjoint:
\begin{equation}
	\frac{d}{dt}\langle\hat{A}\rangle = \left(\frac{1}{i\hbar}\langle\psi(t)|\hat{H}\right)\hat{A}|\psi(t)\rangle + \langle\psi(t)|\frac{\partial\hat{A}}{\partial t}|\psi(t)\rangle + \langle\psi(t)|\hat{A}\left(\frac{1}{i\hbar}\hat{H}|\psi(t)\rangle\right)
\end{equation}

5. Simplify the Expression
\begin{equation}
	\frac{d}{dt}\langle\hat{A}\rangle = \frac{1}{i\hbar}\langle\psi(t)|\hat{H}\hat{A}|\psi(t)\rangle + \langle\psi(t)|\frac{\partial\hat{A}}{\partial t}|\psi(t)\rangle - \frac{1}{i\hbar}\langle\psi(t)|\hat{A}\hat{H}|\psi(t)\rangle
\end{equation}

6. Final Form of Ehrenfest Theorem

Combining terms using the commutator $[\hat{A},\hat{H}] = \hat{A}\hat{H} - \hat{H}\hat{A}$:
\begin{equation}
	\frac{d}{dt}\langle\hat{A}\rangle = \frac{1}{i\hbar}\langle\psi(t)|[\hat{A},\hat{H}]|\psi(t)\rangle + \langle\psi(t)|\frac{\partial\hat{A}}{\partial t}|\psi(t)\rangle
\end{equation}

Which gives us the \textbf{Ehrenfest theorem}:
\begin{equation}\label{E-F-T}
	\boxed{
	\frac{d}{dt}\langle\hat{A}\rangle = \frac{1}{i\hbar}\langle[\hat{A},\hat{H}]\rangle + 
	\left\langle\frac{\partial\hat{A}}{\partial t}\right\rangle
}
\end{equation}

For time-independent operators ($\frac{\partial\hat{A}}{\partial t} = 0$):
\begin{equation}
	\frac{d}{dt}\langle\hat{A}\rangle = \frac{1}{i\hbar}\langle[\hat{A},\hat{H}]\rangle
\end{equation}

For conserved quantities ($[\hat{A},\hat{H}] = 0$ and $\frac{\partial\hat{A}}{\partial t} = 0$):
\begin{equation}
	\frac{d}{dt}\langle\hat{A}\rangle = 0
\end{equation}

8. Physical Interpretation

The Ehrenfest theorem shows that:

- Quantum expectation values follow classical equations of motion.

- The commutator $[\hat{A},\hat{H}]$ generates the time evolution of $\langle\hat{A}\rangle$.

- When an operator commutes with the Hamiltonian and is time-independent, its expectation value is conserved.

This theorem provides a crucial bridge between quantum mechanics and classical physics, demonstrating how classical behavior emerges from quantum mechanics for expectation values.

\subsection{定态}
封闭系统(或处于恒定外场中的系统即势能\textit{V}不含时)的哈密顿量不可能显含时间,这是因为对这样的系统来讲,所有时间都是等价的。另一方面,由于任一算符和它自己当然是对易的,我们就得到这样一个结论,对不在可变外场中的一个系统来讲,它的哈密顿函数是一个守恒量。大家知道,守恒的哈密顿函数称为能量。量子力学中能量守恒定律的意义就在于,如果所给态的能量具有定值,则此值将不随时间变化。

能量为定值的态称为该系统的定态。描述定态的波函数$\Psi_{n}$是哈密顿算符的本征函数,满足方程$\hat{H}\Psi_{n}=E_{n}\Psi_{n}$。其中的$E_{n}$为能量的本征值。对函数$\Psi_{n}$讲薛定谔方程成为
\begin{equation}
	i\hbar\frac{\partial\Psi_{n}}{\partial t}=\hat{H}\Psi_{n}=E_{n}\Psi_{n}
\end{equation}
此式可对时间直接积分,并得
\begin{equation}
	\Psi_{n}=e^{-i(E_{n}/\hbar)t}\psi_{n}
\end{equation}

式中$\psi_{n}$与时间无关。上式确定了定态波函数对时间的依赖关系。

\subsection{概率动力学}
根据量子力学的基本原理,我们可以从薛定谔方程出发推导出概率守恒定律。不过这要求势能必须是实数\par
概率密度定义为:
\begin{equation}
	\rho(\mathbf{r}, t) = |\psi(\mathbf{r}, t)|^2 = \psi^*(\mathbf{r}, t)\psi(\mathbf{r}, t)
\end{equation}

从时间演化方程程出发:
\begin{equation}
	i\hbar\frac{\partial\psi}{\partial t} = \hat{H}\psi = -\frac{\hbar^2}{2m}\nabla^2\psi + V\psi
\end{equation}

及其复共轭:
\begin{equation}
	-i\hbar\frac{\partial\psi^*}{\partial t} = -\frac{\hbar^2}{2m}\nabla^2\psi^* + V^*\psi^*
\end{equation}

对概率密度求时间偏导:
\begin{align}
	\frac{\partial\rho}{\partial t} &= \frac{\partial}{\partial t}(\psi^*\psi) \\
	&= \psi^*\frac{\partial\psi}{\partial t} + \psi\frac{\partial\psi^*}{\partial t}
\end{align}

将薛定谔方程代入上式:
\begin{align}
	\frac{\partial\rho}{\partial t} &= \psi^*\left(\frac{1}{i\hbar}\left[-\frac{\hbar^2}{2m}\nabla^2\psi + V\psi\right]\right) + \psi\left(-\frac{1}{i\hbar}\left[-\frac{\hbar^2}{2m}\nabla^2\psi^* + V^*\psi^*\right]\right) \\	
\end{align}	

如果$V = {V^*}$,那么
\begin{align}
	\frac{\partial\rho}{\partial t} = \frac{i\hbar}{2m}(\psi^*\nabla^2\psi - \psi\nabla^2\psi^*)
\end{align}	

引入概率流密度矢量:
\begin{equation}
	\mathbf{j} = -\frac{i\hbar}{2m}(\psi^*\nabla\psi - \psi\nabla\psi^*)
\end{equation}

可以验证:
\begin{align}
	\nabla\cdot\mathbf{j} &= -\frac{i\hbar}{2m}\nabla\cdot(\psi^*\nabla\psi - \psi\nabla\psi^*) \\
	&= -\frac{i\hbar}{2m}(\psi^*\nabla^2\psi - \psi\nabla^2\psi^*)
\end{align}

比较可得连续性方程:
\begin{equation}
	\frac{\partial\rho}{\partial t} + \nabla\cdot\mathbf{j} = 0
\end{equation}

\subsubsection{定态情况下的概率守恒}

对于定态波函数:
\begin{equation}
	\Psi_n(\mathbf{r}, t) = e^{-(i/\hbar)E_nt}\psi_n(\mathbf{r})
\end{equation}

概率密度为:
\begin{align}
	\rho(\mathbf{r}, t) &= |\Psi_n(\mathbf{r}, t)|^2 \\
	&= |e^{-(i/\hbar)E_nt}\psi_n(\mathbf{r})|^2 \\
	&= |\psi_n(\mathbf{r})|^2
\end{align}

这表明在定态下,概率密度不随时间变化,体现了概率守恒。

\subsection{位力定理}
考虑单粒子量子系统,哈密顿量为:
\[
\hat{H} = \frac{\hat{\mathbf{p}}^2}{2m} + V(\hat{\mathbf{r}})
\]


定义维里算符 (Virial Operator) 为:
\begin{equation}
	G = \mathbf{r} \cdot \mathbf{p} = \sum_{i} r_i p_i
\end{equation}

根据 Ehrenfest theorem (式\ref{E-F-T}), 位力期望值的时间导数为:
\begin{equation}
\frac{d}{dt}\langle G \rangle = \frac{i}{\hbar} \langle [\hat{H}, G] \rangle +\left\langle {\frac{{\partial \left( {{\bf{r}} \cdot {\bf{p}}} \right)}}{{\partial t}}} \right\rangle
\end{equation}

由于系统处于\textbf{定态},任何不显含时间的算符的期望值随时间的变化率必为 0:
\begin{equation}
	\frac{d}{dt}\langle G \rangle = 0 \implies \langle [H, G] \rangle = 0
\end{equation}

\paragraph{1. 计算 $[H, G]$: }

已知 $H = T + V = \frac{\mathbf{p}^2}{2m} + V(\mathbf{r})$。根据对易子分配律:
\begin{equation}
	[H, G] = [T, \mathbf{r} \cdot \mathbf{p}] + [V, \mathbf{r} \cdot \mathbf{p}]
\end{equation}

\paragraph{2. 计算 $[T, \mathbf{r} \cdot \mathbf{p}]$:}
利用正则对易关系 $[r_i, p_j] = i\hbar \delta_{ij}$:
\begin{align}
	[p_j^2, r_i p_i] &= p_j [p_j, r_i] p_i + [p_j, r_i] p_j p_i \notag \\
	&= p_j (-i\hbar \delta_{ij}) p_i + (-i\hbar \delta_{ij}) p_j p_i \notag \\
	&= -2i\hbar p_i p_j \delta_{ij}
\end{align}
因此:
\begin{equation}
	[T, \mathbf{r} \cdot \mathbf{p}] = \frac{1}{2m} \sum_{i,j} (-2i\hbar p_i p_j \delta_{ij}) = -2i\hbar \left( \frac{\mathbf{p}^2}{2m} \right) = -2i\hbar T
\end{equation}

\paragraph{3. 计算 $[V, \mathbf{r} \cdot \mathbf{p}]$:}
由于 $V$ 仅是位置的函数,$[V, r_i] = 0$,且 $\displaystyle [V, p_i] = i\hbar \frac{\partial V}{\partial r_i}$:
\begin{equation}
	[V, \mathbf{r} \cdot \mathbf{p}] = \sum_i r_i [V, p_i] = \sum_i r_i \left( i\hbar \frac{\partial V}{\partial r_i} \right) = i\hbar (\mathbf{r} \cdot \nabla V)
\end{equation}

\subsection*{结论}
将上述结果代入期望值公式:
\begin{equation}
	\langle -2i\hbar T + i\hbar (\mathbf{r} \cdot \nabla V) \rangle = 0
\end{equation}

消去 $i\hbar$,得到:
\begin{equation}\boxed{
	2\langle T \rangle = \langle \mathbf{r} \cdot \nabla V \rangle
}
\end{equation}
\paragraph{齐次势能特例}
若 $V(k \mathbf{r}) = k^n V(\mathbf{r})$($n$ 次齐次函数),则欧拉定理给出:
\[
\mathbf{r} \cdot \nabla V = n V
\]
代入位力定理:
\begin{equation}
\boxed{2\langle T \rangle = n \langle V \rangle}
\end{equation}

\paragraph{典型势能应用}
在量子系统中存在
\[\left\langle T \right\rangle  + \left\langle V \right\rangle  = {E_n}\]
\begin{itemize}
	\item \textbf{库仑势} ($V \propto -1/r$, $n = -1$):
	\[
	2\langle T \rangle = -\langle V \rangle \implies \langle T \rangle = -E_n, \quad \langle V \rangle = 2E_n
	\]
	
	\item \textbf{谐振子势} ($V \propto r^2$, $n = 2$):
	\[
	2\langle T \rangle = 2\langle V \rangle \implies \langle T \rangle = \langle V \rangle = \frac{E_n}{2}
	\]
\end{itemize}

\newpage

\section{子空间和块对角化}
定义1:给定一个矢量空间$\mathbb{V}$,其元素的子集形成的矢量空间称为子空间。我们将维数为$n_i$的特定子空间i用$\mathbb{V}_i^{n_i}$表示\par
定义1补充:空间的维数是空间中相互正交矢量的最大数目。\par
定义2:给定两个子空间 $\mathbb{V}_i^{n_i}$ 和 $\mathbb{V}_j^{m_j}$,我们定义其和 $\mathbb{V}_i^{n_i} \oplus \mathbb{V}_j^{m_j} = \mathbb{V}_k^{m_k}$ 为这样的一个集合,它包含:
\begin{enumerate}
	\item[(1)] $\mathbb{V}_i^{n_i}$ 的所有元素;
	\item[(2)] $\mathbb{V}_j^{m_j}$ 的所有元素;
	\item[(3)] 上述两个子空间中所有元素的所有可能的线性组合。
\end{enumerate}
如果缺少(3),那么就会丢失封闭性。\par
正如Homework 9中in the basis $\{\ket{ \uparrow \uparrow }, \ket{\uparrow \downarrow}, \ket{\downarrow \uparrow}, \ket{\downarrow \downarrow}\}$:
\begin{equation}
	{S_z} = \hbar \left[ {\begin{array}{*{20}{c}}
			1&0&0&0\\
			0&0&0&0\\
			0&0&0&0\\
			0&0&0&{ - 1}
	\end{array}} \right]
	\qquad{{\bf{S}}^2} = {\hbar ^2}\left[ {\begin{array}{*{20}{c}}
			2&0&0&0\\
			0&1&1&0\\
			0&1&1&0\\
			0&0&0&2
	\end{array}} \right]
\end{equation}
我们该如何去理解这个矩阵呢?(I'm sorry I don't want to show you the details about tensor product,I believe you had already konwn it!)\par
首先,我们先确认一下这个矩阵到底是怎么作用的?矩阵的作用方式是一行作用于一列上,(以上文为例)对于矩阵而言,每一行有四个元素,这四个元素分别是四个基矢的coefficient,作用的结果是"coefficient * 对应本征矢"的线性组合。对应于块对角矩阵,我们专注于"块(block)",我们发现block所在的两行,作用后的结果是两个本征矢量的线性组合,既然是两个正交基矢的组合,我们非常理所当然地认为它们处于$\ket{\uparrow \downarrow}, \ket{\downarrow \uparrow}$形成的子空间中,这就是(我认为的)块对角化的带来的"delicious idea".\par
那么现在,我们可以很好地理解块对角矩阵:\par
一个块对角矩阵看起来像这样:
\begin{equation}
	A = \begin{pmatrix}
		A_1 & 0 & \cdots & 0 \\
		0 & A_2 & \cdots & 0 \\
		\vdots & \vdots & \ddots & \vdots \\
		0 & 0 & \cdots & A_m
	\end{pmatrix}
\end{equation}
其中, $A_1, A_2, ..., A_m$  是较小的方块矩阵。我们可以将这个矩阵A所作用的向量空间$\mathbb{V}$分解为若干个子空间的直和:

\begin{equation}
	\mathbb{V} =\oplus \mathbb{V}_1 \oplus \mathbb{V}_2 \ldots \oplus \mathbb{V}_m
\end{equation}

这里的  $\mathbb{V}_i$  正是由那些只与块  $A_i $ 对应的基向量所张成的子空间。\par
\subsection{一点解释}
\subsubsection{非对角块为零的含义:不变子空间}

\textbf{关键点:零非对角块意味着"不变性"。}

当矩阵 \( A \) 以块对角形式出现时,它揭示了一个非常重要的性质:

\begin{itemize}
	\item \textbf{作用封闭性}:对于任意一个属于子空间 \( V_i \) 的向量 \( v \in V_i \),经过线性变换 \( A \) 作用后得到的向量 \( Av \),\textbf{仍然完全属于同一个子空间 \( V_i \)}。
	\item \textbf{无相互作用}:变换 \( A \) 不会将 \( V_i \) 中的向量"搅和"到其他子空间 \( V_j (j \neq i) \) 中去。同样,\( V_j \) 中的向量也不会被"搅和"到 \( V_i \) 中。
\end{itemize}
\subsubsection{如果非对角块不是零呢?}
假设矩阵是:
\begin{equation}
	B = \begin{pmatrix}
		B_1 & B_3 \\
		B_2 & B_4
	\end{pmatrix}
\end{equation}
那么作用的结果是:
\begin{equation}
	Bx = \begin{pmatrix}
		B_1 & B_3 \\
		B_2 & B_4
	\end{pmatrix} \begin{pmatrix}
		x_1 \\
		x_2
	\end{pmatrix} = \begin{pmatrix}
		B_1 x_1 + B_3 x_2 \\
		B_2 x_1 + B_4 x_2
	\end{pmatrix}
\end{equation}

现在,结果的第一个分量不仅依赖于 \( V_1 \) 中的 \( x_1 \),还依赖于 \( V_2 \) 中的 \( x_2 \)(通过 \( B_3 \))。这意味着变换 \( B \) 把 \( V_2 \) 中的信息"混合"进了 \( V_1 \)。同样,\( V_1 \) 中的信息也被 \( B_2 \) "混合"进了 \( V_2 \)。这时,\( V_1 \) 和 \( V_2 \) 就不再是\textbf{不变子空间}了,它们被变换"耦合"在了一起。

\subsubsection{耦合与解耦}

这个概念在物理学和工程中极其重要,通常被称为\textbf{解耦}。

\begin{itemize}
	\item \textbf{振动系统}:想象一个由几个弹簧和重物组成的复杂系统。如果你能找到一个坐标系(即一组简正模),使得系统的运动方程矩阵是块对角的,那么每一个块就对应一种独立的振动模式(简正模)。非对角元素为零意味着这些振动模式是相互独立的,一种模式的激发不会影响另一种模式。\textbf{非对角元素如果非零,则代表模式之间存在"耦合",能量会在不同模式间传递。}
	\item \textbf{量子力学}:系统的哈密顿量(能量算符)如果能被块对角化,意味着整个希尔伯特空间可以分解为几个互不演化的子空间。系统处于某一个子空间的状态,不会随时间演化到另一个子空间去。这通常与某种守恒量(如角动量、宇称)相关。非对角元素则代表了不同子空间之间的"量子跃迁"概率。
\end{itemize}

\subsection{总结}

\textbf{块对角化矩阵的含义是:}

\begin{enumerate}
	\item \textbf{数学上}:它标识了向量空间分解成了一系列\textbf{不变子空间}的直和。线性变换被限制在每个子空间内部独立作用,子空间之间没有"交叉"或"混合"。
	\item \textbf{物理上}:它意味着一个复杂的耦合系统被分解成了多个\textbf{独立的、解耦的子系统}。每个子系统可以单独进行分析,大大简化了问题。
\end{enumerate}

因此,寻找一个矩阵的块对角化过程,本质上就是在寻找这个矩阵变换下保持不变的子空间结构,从而将一个复杂问题分解为几个更简单的、互不相关的子问题。非对角元素的存在与否,是判断这种"独立性"的关键标志。

\newpage
\section{宇称,宇称算符与哈密顿量}
简单来说,\textbf{如果势能函数是偶函数(即 $ V(x) = V(-x) $),那么系统具有空间反演对称性,这直接导致了体系的"宇称"是守恒量,能量本征态具有确定的奇偶性(偶函数或奇函数)。}

\subsection{从对称性到守恒量:诺特定理的思想}
在物理中,一个连续对称性对应一个守恒量(如时间平移对称性对应能量守恒)。对于\textbf{离散对称性}(如空间反演),虽然不产生连续的守恒流,但也会导致一个\textbf{守恒的量子数},在这里就是宇称。

\subsection{关键角色:宇称算符 $ \hat{P} $}
宇称算符 $ \hat{P} $ 的作用是进行空间反演(主动变换视角):
\[
\hat{P} \, \psi(x) = \psi(-x)
\]
它的本征值只有 \textbf{+1} 和 \textbf{-1}:
\begin{itemize}
	\item 本征值 \textbf{+1} 的态称为\textbf{偶宇称态}:$ \hat{P}\psi(x) = +\psi(x) $,即 $ \psi(-x) = \psi(x) $。
	\item 本征值 \textbf{-1} 的态称为\textbf{奇宇称态}:$ \hat{P}\psi(x) = -\psi(x) $,即 $ \psi(-x) = -\psi(x) $。
\end{itemize}

\subsection{势能偶函数与宇称守恒的关联}
系统的动力学由哈密顿算符 $ \hat{H} $ 决定:
\[
\hat{H} = -\frac{\hbar^2}{2m} \frac{d^2}{dx^2} + V(x)
\]

\begin{itemize}
	\item \textbf{动能项}(导数项)本身是偶运算:$ \frac{d^2}{d(-x)^2} = \frac{d^2}{dx^2} $,所以它对空间反演是对称的。
	\item \textbf{关键在势能项 $ V(x) $}:
	\begin{itemize}
		\item 如果 $ V(x) $ 是偶函数,即 $ V(-x) = V(x) $,那么整个哈密顿量在空间反演下不变:$ \hat{P} \hat{H} \hat{P}^{-1} = \hat{H} $,或者说 $ [\hat{H}, \hat{P}] = 0 $(哈密顿量与宇称算符对易)。
		\item 如果 $ V(x) $ 不是偶函数,这个对易关系一般不成立。
	\end{itemize}
\end{itemize}

\textbf{对易 $ [\hat{H}, \hat{P}] = 0 $ 的深刻物理意义}:
\begin{enumerate}
	\item \textbf{宇称守恒}:如果一个系统初始处于一个具有确定宇称的态,那么随着时间演化,它将始终保持相同的宇称。
	\item \textbf{能量本征态可以同时是宇称本征态}:因为 $ \hat{H} $ 和 $ \hat{P} $ 可以对易,它们存在一组\textbf{共同的本征函数}。这意味着我们可以找到这样一些态,它们\textbf{同时是能量本征态(定态)和宇称本征态}。
\end{enumerate}
\subsection{结论与推论}
因此,势能是偶函数 $ V(x)=V(-x) $ 直接导致了:
\begin{itemize}
	\item \textbf{宇称是守恒量}:体系具有空间反演对称性。
	\item \textbf{能量本征函数 $ \psi_n(x) $ 具有确定的奇偶性}:每一个束缚态能量本征函数,必定要么是偶函数(偶宇称),要么是奇函数(奇宇称)。这是求解定态薛定谔方程时一个极其强大的性质。
\end{itemize}

\subsection{一个经典例子:一维谐振子}
谐振子势 $ V(x) = \frac{1}{2} m\omega^2 x^2 $ 显然是偶函数。
其能量本征态为厄米多项式与高斯函数的乘积:
\[
\psi_n(x) = H_n(\alpha x) e^{-\alpha^2 x^2/2}
\]
\begin{itemize}
	\item 当量子数 $ n $ 为\textbf{偶数}时,$ \psi_n(x) $ 是偶函数(偶宇称)。
	\item 当量子数 $ n $ 为\textbf{奇数}时,$ \psi_n(x) $ 是奇函数(奇宇称)。
\end{itemize}
这完美地印证了上述理论。

\subsection{宇称在选择定则中的重要性}
量子跃迁的概率(如吸收、发射光子)由跃迁矩阵元决定。对于电偶极跃迁,矩阵元为:
\[
\langle f | \hat{\mathbf{d}} | i \rangle = \int \psi_f^* (\mathbf{r}) \, e\mathbf{r} \, \psi_i(\mathbf{r}) \, d^3\mathbf{r}
\]
其中 $\hat{\mathbf{d}} = e\mathbf{r}$ 是电偶极矩算符。注意:$\mathbf{r}$ 是奇宇称算符(因为空间反演下 $\mathbf{r} \rightarrow -\mathbf{r}$)。

根据宇称分析:
\begin{itemize}
	\item 如果初态 $|i\rangle$ 和末态 $|f\rangle$ 具有相同的宇称(同奇或同偶),那么被积函数 $\psi_f^* (\mathbf{r}) \, \mathbf{r} \, \psi_i(\mathbf{r})$ 的整体宇称为:$\text{末态宇称} \times \text{奇宇称} \times \text{初态宇称}$。由于 $\mathbf{r}$ 是奇函数,当两态宇称相同时,被积函数为奇函数(奇×偶×偶=奇,或偶×奇×奇=奇)。在全空间积分时,奇函数的积分为零。
	\item 如果初态和末态宇称相反,则被积函数为偶函数,积分可能不为零。
\end{itemize}

因此,电偶极跃迁的\textbf{选择定则}为:跃迁前后体系的宇称必须改变,即 $\Delta P = -1$(从偶宇称到奇宇称,或反之)。

这个选择定则在原子物理、分子光谱和凝聚态物理中至关重要:
\begin{enumerate}
	\item 解释光谱线的出现与缺失:例如,在氢原子中,从$1s$(偶宇称)到$2s$(偶宇称)的跃迁是禁戒的(电偶极近似下),而从$1s$到$2p$(奇宇称)是允许的。
	\item 判断跃迁类型:电偶极跃迁(E1)要求宇称改变;而磁偶极跃迁(M1)和电四极跃迁(E2)等高级跃迁具有不同的宇称选择定则,可用于鉴别光谱的来源。
	\item 指导能级布局和激光物理:在设计激光介质时,需要利用允许跃迁来产生受激辐射,而宇称选择定则帮助确定哪些能级之间容易实现粒子数反转。
\end{enumerate}

\subsection{为什么这很重要?}
\begin{enumerate}
	\item \textbf{简化计算}:在求解薛定谔方程时,我们可以先根据对称性将解预设为奇函数或偶函数,从而只需在 $ x \geq 0 $ 的区域求解并满足边界条件,这大大降低了计算复杂度。
	\item \textbf{选择定则}:在跃迁过程中,宇称守恒会给出严格的跃迁选择定则(例如,电偶极跃迁要求初末态宇称相反),这对理解原子光谱等现象至关重要。
	\item \textbf{物理理解}:它体现了对称性如何深刻地决定和简化了物理系统的结构。
\end{enumerate}

\subsubsection*{总结}
\textbf{逻辑链}:
势能是偶函数 $ V(x)=V(-x) $ $\Rightarrow$ 体系具有\textbf{空间反演对称性} $\Rightarrow$ 哈密顿量与宇称算符对易 $ [\hat{H}, \hat{P}]=0 $ $\Rightarrow$ \textbf{宇称守恒},且\textbf{能量本征态具有确定的奇偶性}。

宇称守恒不仅决定了定态波函数的对称性,还通过选择定则支配了量子跃迁的概率,这是理解原子分子光谱、量子光学和许多凝聚态现象的基础。
\newpage

\section{简并性}
Darling,Let's back to the example, the  matrix in the basis $\{\ket{ \uparrow \uparrow }, \ket{\uparrow \downarrow}, \ket{\downarrow \uparrow}, \ket{\downarrow \downarrow}\}$:
\begin{equation}
	{S_z} = \hbar \left[ {\begin{array}{*{20}{c}}
			1&0&0&0\\
			0&0&0&0\\
			0&0&0&0\\
			0&0&0&{ - 1}
	\end{array}} \right]
\end{equation}
我们知道:"对于每一个厄米算符$\Omega$,(至少)存在一个由其正交本征矢组成的基。它在这个本征基下是对角的,并且它的本征值就作为它的对角元。"如果存在简并(一个本征值对应多个本征矢),这种情况就会出现. \par 对于${S_z}$就有这种情况,为了一般性,我们考虑算符$\Omega$,我们让$\left| {{\omega }} \right\rangle $作为本征矢,考虑\newline
$\omega = {\omega _1} = {\omega _2}$,那么存在
\begin{equation}
	\begin{array}{l}
		\Omega \left| {{\omega _1}} \right\rangle  = \omega \left| {{\omega _1}} \right\rangle \\
		\Omega \left| {{\omega _2}} \right\rangle  = \omega \left| {{\omega _2}} \right\rangle 
	\end{array}
\end{equation}
由此导出,对于任意的$\alpha ,\beta $,有
\begin{equation}
	\Omega \left( {\alpha \left| {{\omega _1}} \right\rangle  + \beta \left| {{\omega _2}} \right\rangle } \right) 
	= \omega \left( {\alpha \left| {{\omega _1}} \right\rangle  + \beta \left| {{\omega _2}} \right\rangle } \right)
\end{equation}
由于矢量$\left| {{\omega _1}} \right\rangle $,$\left| {{\omega _2}} \right\rangle $是正交的,我们发现存在由$\left| {{\omega _1}} \right\rangle $和$\left| {{\omega _2}} \right\rangle $张成的二维子空间,其中的元素都是本征值为$\omega $的$\Omega $的本征矢。我们称这样的空间是本征值为$\omega $的$\Omega $本征空间。\par
我们还可以由方程(1)发现,我们有无穷多对正交的
\begin{equation}
	\left| {\omega '} \right\rangle  = \alpha \left| {{\omega _1}} \right\rangle  + \beta \left| {{\omega _2}} \right\rangle
\end{equation}
作为 $\Omega $ 的本征矢\par
\textcolor{purple}{这是非常非常非常有意义的!因为你无法通过 $\omega $ 分辨本征矢}

\newpage

	\section{Complete Set of Commuting Observables}
	CSCO定义:In quantum mechanics, a complete set of commuting observables (CSCO) is a set of commuting operators whose common eigenvectors can be used as a basis to express any quantum state. 
	
\newpage
	\section{测量}
\subsection{离散谱}
假设 $ A $ 是对应某一动力学变量的厄米算符。我们认为,若对 $ A $ 的测量得到结果 $ a $, 则测量行为会使波函数坍缩到一个新的状态,在该状态下对 $ A $ 的测量必然得到结果 $ a $. 那么,什么样的波函数 $ \psi $ 能使得对 $ A $ 的测量必然得到确定结果 $ a $ 呢?将 $ \psi $ 表示为 $ A $ 的本征态的线性组合,可得
\begin{equation}
	\psi = \sum_{i} c_{i} \psi_{i} \label{eq:m1}
\end{equation}
其中 $ \psi_{i} $ 是 $ A $ 对应于本征值 $ a_{i} $ 的本征态。若对 $ A $ 的测量必然得到结果 $ a $, 则有
\begin{equation}
	\langle A \rangle = a,
\end{equation}
且
\begin{equation}
	\sigma_A^2 = \langle A^2 \rangle - \langle A \rangle^2 = 0. 
\end{equation}
易证
\begin{align}
	\langle A \rangle &= \sum_{i} |c_{i}|^{2} a_{i}, \\
	\langle A^{2} \rangle &= \sum_{i} |c_{i}|^{2} a_{i}^{2}.
\end{align}
因此
\begin{equation}
	\sum_{i} a_{i}^{2} |c_{i}|^{2} - \left( \sum_{i} a_{i} |c_{i}|^{2} \right)^{2} = 0. 
\end{equation}
此外,归一化条件要求
\begin{equation}
	\sum_{i} |c_{i}|^{2} = 1. \label{eq:m2}
\end{equation}

例如,假设仅有两个本征态。此时上述两式可简化为 $ |c_{1}|^{2} = x $,  $ |c_{2}|^{2} = 1 - x $ (其中 $ 0 \leq x \leq 1 $ ),且
\begin{equation}
	(a_{1} - a_{2})^{2} x (1 - x) = 0.
\end{equation}
该方程的唯一解为 $ x = 0 $ 和 $ x = 1 $.  

这一结果可轻易推广到本征态数量多于两个的情形。由此可知,与 $ A $ 的确定值相关的状态,是其中一个 $ |c_{i}|^{2} $ 为 1、其余均为 0 的状态。换言之,仅当状态为 $ A $ 的本征态时,对 $ A $ 的测量才能得到确定值。这立即得出结论:对 $ A $ 测量的结果必定是 $ A $ 的某个本征值。进一步,若将一般波函数按 $ A $ 的本征态展开 (如式(\ref{eq:m1})),则测量 $ A $ 得到本征值 $ a_{i} $ 的概率恰好为 $ |c_{i}|^{2} $  ($ c_{i} $ 为展开式中第 $ i $ 个本征态前的系数)。由式(\ref{eq:m2})可知,这些概率满足归一化条件:即对 $ A $ 的测量得到任意可能结果的总概率为1。最后,若对 $ A $ 的测量得到本征值 $ a_{i} $, 则测量完成后系统会立即处于对应于 $ a_{i} $ 的本征态。

考虑由两个厄米算符 $ A $ 和 $ B $ 表示的两个物理动力学变量。在何种情况下能对这两个变量进行精确的同时测量呢?对 $ A $ 和 $ B $ 测量的可能结果分别是 $ A $ 和 $ B $ 的本征值。因此,要对 $ A $ 和 $ B $ 进行精确的同时测量,必须存在同时为 $ A $ 和 $ B $ 本征态的状态。事实上,若要在所有情况下都能同时测量 $ A $ 和 $ B $, 则 $ A $ 的所有本征态也必须是 $ B $ 的本征态(反之亦然),这样所有与 $ A $ 的唯一值相关的状态也与 $ B $ 的唯一值相关(反之亦然)。

若 $ A $ 和 $ B $ 不对易(即 $ AB \neq BA $ ),则无法对它们进行同时测量。这表明,同时测量的条件是 $ A $ 和 $ B $ 对易。假设 $ A $ 和 $ B $ 对易,且 $ \psi_{i} $ 和 $ a_{i} $ 分别为 $ A $ 的归一化本征态和本征值,则有
\begin{equation}
	(AB - BA) \psi_i = (AB - B a_i) \psi_i = (A - a_i) B \psi_i = 0,
\end{equation}
或
\begin{equation}
	A (B \psi_i) = a_i (B \psi_i). 
\end{equation}
因此, $ B \psi_{i} $ 是 $ A $ 对应于本征值 $ a_{i} $ 的本征态(不一定归一化)。换言之, $ B \psi_{i} \propto \psi_{i} $, 即
\begin{equation}
	B \psi_i = b_i \psi_i
\end{equation}
其中 $ b_{i} $ 为比例常数。由此可知, $ \psi_{i} $ 是 $ B $ 的本征态,因而也是 $ A $ 和 $ B $ 的共同本征态。我们得出结论:若 $ A $ 和 $ B $ 对易,则它们存在共同本征态,从而可被精确同时测量。

\subsection{连续谱}
此前我们讨论的是具有分立本征值和平方可积本征态的算符。但部分算符——最典型的是位置算符 $ x $ 和动量算符 $ p $ ——\textbf{具有连续取值范围的本征值,且其本征态非平方可积}(事实上,这两个性质是相伴出现的)。因此,我们接下来研究位置算符和动量算符的本征态与本征值。

设 $ \psi_x(x, x') $ 为位置算符 $ x $ 对应于本征值 $ x' $ 的本征态,则对所有 $ x $ 满足
\begin{equation}
	x \psi_x(x, x') = x' \psi_x(x, x')
\end{equation}
考虑狄拉克δ函数 $ \delta(x - x') $, 可写出
\begin{equation}
	x \delta(x - x') = x' \delta(x - x'),
\end{equation}
这是因为 $ \delta(x - x') $ 仅在 $ x = x' $ 的无穷小邻域内非零。显然, $ \psi_{x}(x, x') $ 与 $ \delta(x - x') $ 成正比则
\begin{equation}
	\psi_x(x, x') = \delta(x - x').
\end{equation}
易证
\begin{equation}
	\int_{-\infty}^{\infty} \delta(x - x') \delta(x - x'') \, dx = \delta(x' - x''). 
\end{equation}
因此, $ \psi_{x}(x, x') $ 满足狄拉克正交归一条件
\begin{equation}
	\int_{-\infty}^{\infty} \psi_x^*(x, x') \psi_x(x, x'') \, dx = \delta(x' - x''). \label{eq:m3}
\end{equation}
该条件与平方可积本征态满足的正交归一条件类似。根据定义, $ \delta(x - x') $ 满足
\begin{equation}
	\int_{-\infty}^{\infty} f(x) \delta(x - x') \, dx = f(x')
\end{equation}
其中 $ f(x) $ 为任意函数。因此,可将一般波函数 $ \psi(x) $ 展开为
\begin{equation}
	\psi(x) = \int_{-\infty}^{\infty} c(x') \psi_{x}(x, x') \, dx'
\end{equation}
其中 $ c(x') = \psi(x') $, 或写作
\begin{equation}
	c(x') = \int_{-\infty}^{\infty} \psi_x^*(x, x') \psi(x) \, dx. 
\end{equation}
也就是说,可将一般波函数 $ \psi(x) $ 表示为位置算符本征态 $ \psi_{x}(x, x') $ 的线性组合。测量位置 $ x $ 得到值 $ x' $ 的概率密度为 $ |c(x')|^{2} $, 这与标准结果 $ |\psi(x')|^{2} $ 一致。此外,若 $ \psi(x) $ 满足归一化条件,则这些概率也满足归一化:即
\begin{equation}
	\int_{-\infty}^{\infty} |c(x')|^2 \, dx' = \int_{-\infty}^{\infty} |\psi(x')|^2 \, dx' = 1. 
\end{equation}
最后,若对 $ x $ 的测量得到值 $ x' $, 则测量后系统立即处于对应的位置本征态 $ \psi_{x}(x, x') $, 即波函数坍缩为“尖峰函数”: $ \delta(x - x') $.  

动量算符 $ p \equiv -i \hbar \partial / \partial x $ 对应于本征值 $ p' $ 的本征态满足
\begin{equation}
	-i \hbar \frac{\partial \psi_p(x, p')}{\partial x} = p' \psi_p(x, p').
\end{equation}
显然,该方程的解为
\begin{equation}
	\psi_p(x, p') \propto e^{+i p' x / \hbar}. \label{eq:m4}
\end{equation}
我们要求 $ \psi_p(x, p') $ 满足与式(\ref{eq:m3})类似的正交归一条件:
\begin{equation}
	\int_{-\infty}^{\infty} \psi_p^*(x, p') \psi_p(x, p'') \, dx = \delta(p' - p'').
\end{equation}
根据狄拉克 \(\delta\) 函数的积分表示
\begin{equation}
	\delta (p - {p_0}) = \frac{1}{{2\pi \hbar }}\int_{ - \infty }^\infty  {{{\rm{e}}^{ + i\left( {p - {p_0}} \right)x/\hbar }}} dx
\end{equation}
可得式(\ref{eq:m4})中的比例常数应为 $ (2 \pi \hbar)^{-1/2} $, 即
\begin{equation}
	\displaystyle \psi_p(x, p') = \frac{1}{(2 \pi \hbar)^{1/2}}e^{i p' x / \hbar}. 
\end{equation}
进一步,可将一般波函数 $ \psi(x) $ 展开为
\begin{equation}
	\psi(x) = \int_{-\infty}^{\infty} c(p') \psi_p(x, p') \, dp'
\end{equation}
其中 $ c(p') = \phi(p') $, 或写作
\begin{equation}
	c(p') = \int_{-\infty}^{\infty} \psi_p^*(x, p') \psi(x) \, dx. 
\end{equation}
也就是说,可将一般波函数 $ \psi(x) $ 表示为动量算符本征态 $ \psi_{p}(x, p') $ 的线性组合。测量动量 $ p $ 得到结果 $ p' $ 的概率密度为 $ |c(p')|^{2} $, 这与标准结果 $ |\phi(p')|^{2} $ 一致。若 $ \psi(x) $ 满足归一化条件,则这些概率也满足归一化:
\begin{equation}
	\int_{-\infty}^{\infty} |c(p')|^2 \, dp' = \int_{-\infty}^{\infty} |\phi(p')|^2 \, dp' = \int_{-\infty}^{\infty} |\psi(x')|^2 \, dx' = 1. 
\end{equation}
最后,若对 $ p $ 的测量得到值 $ p' $, 则测量后系统立即处于对应的动量本征态 $ \psi_{p}(x, p') $.  

\newpage

\section{测量的不确定性}
\subsection{从实验及对称性的视角出发}
在这里,我希望用一个直观的例子而不是公式来展示这一影响深远的“测不准”关系,我们需要从一个典型的实验即Stern-Gerlach experiment开始\par
当我们选择${\hat S_z}$的本征态作为基矢时,我们记为:
\[
\left|  \uparrow  \right\rangle  = \left( {\begin{array}{*{20}{c}}
		1\\
		0
\end{array}} \right);\quad \left|  \downarrow  \right\rangle  = \left( {\begin{array}{*{20}{c}}
		0\\
		1
\end{array}} \right)
\]
根据实验事实我们可以得到三个关系式:(也就是大名鼎鼎的泡利矩阵关系)
首先,实验表明银原子束通过非均匀磁场后分裂成两束,证明存在内禀角动量(自旋),且其在$z$方向的投影只能取两个离散值:
\[
\hat{S}_z \ket{\uparrow} = +\frac{\hbar}{2} \ket{\uparrow}, \quad \hat{S}_z \ket{\downarrow} = -\frac{\hbar}{2} \ket{\downarrow}
\]
于是我们轻而易举就得到了
\[
\hat{S}_z = \frac{\hbar}{2} \left( \ket{\uparrow}\bra{\uparrow} - \ket{\downarrow}\bra{\downarrow} \right)
\]
矩阵形式为
\[
\hat{S}_z = \frac{\hbar}{2} \begin{pmatrix} 1 & 0 \\ 0 & -1 \end{pmatrix}
\]
同时我们根据基本的对称性理论(也就是角动量理论),我们可以得到角动量对易关系
\[
[\hat{S}_i, \hat{S}_j] = i\hbar \epsilon_{ijk} \hat{S}_k
\]
定义升降算符:
\begin{equation*}
	\hat{S}_+ = \hat{S}_x + i\hat{S}_y, \quad \hat{S}_- = \hat{S}_x - i\hat{S}_y 
\end{equation*}
根据角动量基本代数关系得到:
\[\hat{S}_+ \ket{\downarrow} = \hbar \ket{\uparrow}, \quad \hat{S}_- \ket{\uparrow} = \hbar \ket{\downarrow}\]
反解得:
\[
\hat{S}_x = \frac{1}{2}(\hat{S}_+ + \hat{S}_-), \quad \hat{S}_y = \frac{1}{2i}(\hat{S}_+ - \hat{S}_-)
\]
现在,我们将升降算符在${{\hat S}_z}$的基底下表示出来,根据其作用效应:
\begin{itemize}
\item $\hat {S}_ + $:$\hat {S}_ + $把$\ket{\downarrow}$变为$\ket{\uparrow}$,所以它的矩阵形式是
	\[
	\hat{S}_+ = \hbar \begin{pmatrix} 0 & 0 \\ 1 & 0 \end{pmatrix}
	\]
外积形式为 $\hat{S}_+ =\hbar \ket{\uparrow}\bra{\downarrow} $	

\item $\hat {S}_ - $:同理,$\hat {S}_ - $把$\ket{\uparrow}$变成$\ket{\downarrow}$,矩阵形式为
	\[
	\hat{S}_- = \hbar \begin{pmatrix} 0 & 1 \\ 0 & 0 \end{pmatrix}
	\]
外积形式为 $\hat{S}_- =\hbar \ket{\downarrow}\bra{\uparrow}$
\end{itemize}
\par
引入泡利矩阵:
\[
\sigma_x = \begin{pmatrix} 0 & 1 \\ 1 & 0 \end{pmatrix}, \quad
\sigma_y = \begin{pmatrix} 0 & -i \\ i & 0 \end{pmatrix}, \quad
\sigma_z = \begin{pmatrix} 1 & 0 \\ 0 & -1 \end{pmatrix}
\]\par

\textbf{一个补充:}$\sigma_{x}$ 是自旋在 $x$ 方向的算符,它不与 $\sigma_{z}$ 对易(即 $[\sigma_{x},\sigma_{z}]\neq 0$)。因此,当 $\sigma_{x}$ 作用在 $\sigma_{z}$ 的本征态上时,不会得到相同的本征态,而是将态翻转到另一个本征态。这反映了自旋在不同方向上的测量是不兼容的。

自旋算符的最终表达式为:
\[
\boxed{\hat{S}_x = \frac{\hbar}{2} \sigma_x, \quad
	\hat{S}_y = \frac{\hbar}{2} \sigma_y, \quad
	\hat{S}_z = \frac{\hbar}{2} \sigma_z}
\]

外积形式的等价表达式:
\[
\hat{S}_x = \frac{\hbar}{2} \left( \ket{\uparrow}\bra{\downarrow} + \ket{\downarrow}\bra{\uparrow} \right), \quad
\hat{S}_y = \frac{\hbar}{2i} \left( \ket{\uparrow}\bra{\downarrow} - \ket{\downarrow}\bra{\uparrow} \right),\quad
\hat{S}_z = \frac{\hbar}{2} \left( \ket{\uparrow}\bra{\uparrow} - \ket{\downarrow}\bra{\downarrow} \right)
\]

在标准基下,泡利算符的矩阵表示为:
\[
\sigma_{x} = \begin{pmatrix}
	0 & 1 \\
	1 & 0
\end{pmatrix}, \quad
\ket{\uparrow} = \begin{pmatrix}
	1 \\
	0
\end{pmatrix}, \quad
\ket{\downarrow}= \begin{pmatrix}
	0 \\
	1
\end{pmatrix}
\]

计算可得:

\[
\sigma_{x}\ket{\uparrow} = \begin{pmatrix}
	0 & 1 \\
	1 & 0
\end{pmatrix} \begin{pmatrix}
	1 \\
	0
\end{pmatrix} = \begin{pmatrix}
	0 \\
	1
\end{pmatrix} = \ket{\downarrow}
\]

\[
\sigma_{x}\ket{\downarrow} = \begin{pmatrix}
	0 & 1 \\
	1 & 0
\end{pmatrix} \begin{pmatrix}
	0 \\
	1
\end{pmatrix} = \begin{pmatrix}
	1 \\
	0
\end{pmatrix} = \ket{\uparrow}
\]
$\sigma_{x}$ 作用在 $z$ 方向本征态上的效果是使自旋翻转。在布洛赫球上体现为绕 X 轴​旋转 180°, 右手拇指指向+X,四指弯曲方向为旋转正方向\par
\[
\sigma_{y}\ket{\uparrow} = \begin{pmatrix}0 & -i\\ i & 0\end{pmatrix}\begin{pmatrix}1\\0\end{pmatrix}=\begin{pmatrix}0\\ i\end{pmatrix}=i\ket{\downarrow}
\]
\[
\sigma_{y}\ket{\downarrow} = \begin{pmatrix}0 & -i\\ i & 0\end{pmatrix}\begin{pmatrix}0\\1\end{pmatrix}=\begin{pmatrix}-i\\0\end{pmatrix}=-i\ket{\uparrow}
\]
\(\sigma_y\) 操作在翻转自旋方向的同时,附加了特定的相位\(\pm i\)。在布洛赫球上,这对应于绕 Y 轴旋转180°

\[
\sigma_{z}\ket{\uparrow} = \begin{pmatrix}1 & 0\\0 & -1\end{pmatrix}\begin{pmatrix}1\\0\end{pmatrix}=\begin{pmatrix}1\\0\end{pmatrix}=\ket{\uparrow}
\]
\[
\sigma_{z}\ket{\downarrow} = \begin{pmatrix}1 & 0\\0 & -1\end{pmatrix}\begin{pmatrix}0\\1\end{pmatrix}=\begin{pmatrix}0\\-1\end{pmatrix}=-\ket{\downarrow}
\]
\(\sigma_z\) 操作不改变自旋方向(即不自旋翻转),但会改变自旋向下态的相对相位。在布洛赫球上,这对应于绕 Z 轴旋转180°

\[
\boxed{\text{Raising and Lowering Operators } (\sigma_+, \sigma_-)}
\]

These are non-Hermitian combinations of \(\sigma_x \) and \(\sigma_y\) that specifically raise or lower the atomic state.

\(\sigma_+ = (\sigma_x + i\sigma_y)/2 = \ket{e}\bra{g} \) (lowers atom from \(\ket{e}\) to
\(\ket{g}\))

\(\sigma_- = (\sigma_x - i\sigma_y)/2 = \ket{g}\bra{e} \) (lowers atom from \(\ket{g}\) to
\(\ket{e}\))

让我们考虑一个任意的 state $\ket{\alpha} $,并计算其分量的期望值。在 \(z\)-基 中计算最容易,因为我们已经知道各算符的表示了。现在我们将$\ket{\alpha} $表示为$\ket{\uparrow},\ket{\downarrow}$的线性组合:
\begin{equation}
	|\alpha\rangle=\alpha_{\uparrow}|\uparrow\rangle+\alpha_{\downarrow}|\downarrow\rangle,\quad\langle\alpha|=\left\langle\uparrow\left|\alpha_{\uparrow}^*+\langle\downarrow\right|\alpha_{\downarrow}^*\right.,
\end{equation}
并将这些表达式代入可观测量 $S_{z}$的期待值公式,我们得到
\begin{equation}
	\begin{aligned}
		\langle S_z \rangle &= \left( \alpha_{\uparrow}^* \langle \uparrow \rvert + \alpha_{\downarrow}^* \langle \downarrow \rvert \right) \hat{S}_z \left( \alpha_{\uparrow} \lvert \uparrow \rangle + \alpha_{\downarrow} \lvert \downarrow \rangle \right) \\
		&= \lvert \alpha_{\uparrow} \rvert^2 \langle \uparrow \rvert \hat{S}_z \lvert \uparrow \rangle
		+ \lvert \alpha_{\downarrow} \rvert^2 \langle \downarrow \rvert \hat{S}_z \lvert \downarrow \rangle
		+ \alpha_{\uparrow} \alpha_{\downarrow}^* \langle \downarrow \rvert \hat{S}_z \lvert \uparrow \rangle
		+ \alpha_{\downarrow} \alpha_{\uparrow}^* \langle \uparrow \rvert \hat{S}_z \lvert \downarrow \rangle.\\
		&=\frac{\hbar}{2}\left(\alpha_{\uparrow}\alpha_{\uparrow}^*-\alpha_{\downarrow}\alpha_{\downarrow}^*\right).
	\end{aligned}
\end{equation}
完全类似的计算给出:
\begin{equation}
	\begin{aligned}
		\left\langle S_x\right\rangle&=\alpha_{\uparrow}\alpha_{\uparrow}^*\left\langle\uparrow\left|\hat{S}_x\right|\uparrow\right\rangle+\alpha_{\downarrow}\alpha_{\downarrow}^*\left\langle\downarrow\left|\hat{S}_x\right|\downarrow\right\rangle+\alpha_{\uparrow}\alpha_{\downarrow}^*\left\langle\downarrow\left|\hat{S}_x\right|\uparrow\right\rangle+\alpha_{\downarrow}\alpha_{\uparrow}^*\left\langle\uparrow\left|\hat{S}_x\right|\downarrow\right\rangle\\
		&=\frac{\hbar}{2}\left(\alpha_{\uparrow}\alpha_{\downarrow}^*+\alpha_{\downarrow}\alpha_{\uparrow}^*\right)
	\end{aligned}
\end{equation}
\begin{equation}
	\begin{aligned}
		\left\langle S_y\right\rangle&=\alpha_{\uparrow}\alpha_{\uparrow}^*\left\langle\uparrow\left|\hat{S}_y\right|\uparrow\right\rangle+\alpha_{\downarrow}\alpha_{\downarrow}^*\left\langle\downarrow\left|\hat{S}_y\right|\downarrow\right\rangle+\alpha_{\uparrow}\alpha_{\downarrow}^*\left\langle\downarrow\left|\hat{S}_y\right|\uparrow\right\rangle+\alpha_{\downarrow}\alpha_{\uparrow}^*\left\langle\uparrow\left|\hat{S}_y\right|\downarrow\right\rangle\\
		&=i\frac{\hbar}{2}\left(\alpha_{\uparrow}\alpha_{\downarrow}^*-\alpha_{\downarrow}\alpha_{\uparrow}^*\right)
	\end{aligned}
\end{equation}

\par
其实到这里你应该很惊讶了,明明是 \(z\) 方向的叠加态,我们居然可能在 $x,y$ 方向测量到数值(所谓涨落),我们先考虑一个经典的图像::一个纯的自旋向上态 $\ket{\uparrow}$, 在 $x$ 和 $y$ 方向的平均值必然为零。这其实很好证明:因为在这个态中,$\alpha_{\uparrow}=1$ 且 $\alpha_{\downarrow}=0$,于是非常轻易地就得到了
\begin{equation}
	\left\langle S_z\right\rangle=\frac{\hbar}{2},\quad\left\langle S_x\right\rangle=\left\langle S_y\right\rangle=0.
\end{equation}
为了考虑量子情形,现在我们来做一些确定度计算,根据数学关系:
\begin{equation}
{\left( {\delta X} \right)^2} = \left\langle {{X^2}} \right\rangle  - {\left\langle X \right\rangle ^2}
\end{equation}
同时,每个自旋分量平方的算符等于 $(\hbar/ 2)^{2}\hat{I}$ ($i.e.~\hat S_i^2 = (\hbar/ 2)^{2}\hat I$), 于是得到:
\begin{equation}
	\begin{gathered}
		\left(\delta S_z\right)^2=\left\langle S_z^2\right\rangle-\left\langle S_z\right\rangle^2=\left\langle\uparrow\left|\hat{S}_z^2\right|\uparrow\right\rangle-\left(\frac{\hbar}{2}\right)^2=\left(\frac{\hbar}{2}\right)^2\langle\uparrow|\hat{I}|\uparrow\rangle-\left(\frac{\hbar}{2}\right)^2=0,\\
		\left(\delta S_x\right)^2=\left\langle S_x^2\right\rangle-\left\langle S_x\right\rangle^2=\left\langle\uparrow\left|\hat{S}_x^2\right|\uparrow\right\rangle-0=\left(\frac{\hbar}{2}\right)^2\langle\uparrow|\hat{I}|\uparrow\rangle=\left(\frac{\hbar}{2}\right)^2,\\
		\left(\delta S_y\right)^2=\left\langle S_y^2\right\rangle-\left\langle S_y\right\rangle^2=\left\langle\uparrow\left|\hat{S}_y^2\right|\uparrow\right\rangle-0=\left(\frac{\hbar}{2}\right)^2\langle\uparrow|\hat{I}|\uparrow\rangle=\left(\frac{\hbar}{2}\right)^2.
	\end{gathered}
\end{equation}
\par
现在我们感受到了大名鼎鼎的 Uncertainty Principle,而且我们还感受到了量子涨落,我想说“微风吹起了她的面纱,我着迷了”

\begin{center}
	{\large \textcolor{purple}{那一刻的美,请我们都要记住,现在让我们用严谨的数学来勾勒那一眼吧}}
\end{center}

\subsection{ Derivation of the Uncertainty Relations}

Let $\Omega$ and $\Lambda$ be two Hermitian operators, with a commutator
\begin{equation}
	[\Omega, \Lambda] = i\Gamma
\end{equation}

You may readily verify that $\Gamma$ is also Hermitian. Let us start with the uncertainty product in a normalized state $|\psi\rangle$:
\begin{equation}
	(\Delta\Omega)^2(\Delta\Lambda)^2 = \langle\psi|(\Omega - \langle\Omega\rangle)^2|\psi\rangle\langle\psi|(\Lambda - \langle\Lambda\rangle)^2|\psi\rangle
\end{equation}
where $\langle\Omega\rangle = \langle\psi|\Omega|\psi\rangle$ and $\langle\Lambda\rangle = \langle\psi|\Lambda|\psi\rangle$. Let us next define the pair
\begin{equation}
	\begin{aligned}
	\hat{\Omega} &= \Omega - \langle\Omega\rangle \\
	\hat{\Lambda} &= \Lambda - \langle\Lambda\rangle
	\end{aligned}
\end{equation}
which has the same commutator as $\Omega$ and $\Lambda$ (verify this). In terms of $\hat{\Omega}$ and $\hat{\Lambda}$
\begin{equation}
	\begin{aligned}
		(\Delta\Omega)^2(\Delta\Lambda)^2 &= \langle\psi|\hat{\Omega}^2|\psi\rangle\langle\psi|\hat{\Lambda}^2|\psi\rangle \\
		&= \langle\hat{\Omega}\psi|\hat{\Omega}\psi\rangle\langle\hat{\Lambda}\psi|\hat{\Lambda}\psi\rangle
	\end{aligned}
	\label{Eq.9.2.4}
\end{equation}
since
\[
\hat{\Omega}^2 = \hat{\Omega}\hat{\Omega} = \hat{\Omega}^{\dagger}\hat{\Omega}
\]
and
\begin{equation}
	\hat{\Lambda}^2 = \hat{\Lambda}^{\dagger}\hat{\Lambda}
\end{equation}

If we apply the Schwartz inequality
\begin{equation}
	|V_1|^2|V_2|^2 \geq \left|\langle V_1|V_2\rangle\right|^2
\end{equation}
(where the equality sign holds only if $|V_1\rangle = c|V_2\rangle$, where $c$ is a constant) to the states $|\hat{\Omega}\psi\rangle$ and $|\hat{\Lambda}\psi\rangle$, we get from equation (\ref{Eq.9.2.4}),	
\begin{equation}
	(\Delta\Omega)^2(\Delta\Lambda)^2 \geq |\langle\hat{\Omega}\psi|\hat{\Lambda}\psi\rangle|^2
\end{equation}
Let us now use the fact that
\begin{equation}
	\langle\hat{\Omega}\psi|\hat{\Lambda}\psi\rangle = \langle\psi|\hat{\Omega}^{\dagger}\hat{\Lambda}|\psi\rangle = \langle\psi|\hat{\Omega}\hat{\Lambda}|\psi\rangle
\end{equation}
to rewrite the above inequality as
\begin{equation}
	(\Delta\Omega)^2(\Delta\Lambda)^2 \geq |\langle\psi|\hat{\Omega}\hat{\Lambda}|\psi\rangle|^2 \label{Eq.9.2.9}
\end{equation}

Now, we know that the commutator has to enter the picture somewhere. This we arrange through the following identity:
\begin{align}
	\hat{\Omega}\hat{\Lambda} &= \frac{\hat{\Omega}\hat{\Lambda} + \hat{\Lambda}\hat{\Omega}}{2} + \frac{\hat{\Omega}\hat{\Lambda} - \hat{\Lambda}\hat{\Omega}}{2} \nonumber \\
	&= \frac{1}{2}[\hat{\Omega},\hat{\Lambda}]_{+} + \frac{1}{2}[\hat{\Omega},\hat{\Lambda}]\label{Eq.9.2.10}
\end{align}
where $[\hat{\Omega},\hat{\Lambda}]_{+}$ is called the anticommutator. Feeding equation (\ref{Eq.9.2.10}) into the inequality (\ref{Eq.9.2.9}), we get
\begin{equation}
	(\Delta\Omega)^{2}(\Delta\Lambda)^{2} \geq \left|\langle\psi|\frac{1}{2}[\hat{\Omega},\hat{\Lambda}]_{+} + \frac{1}{2}[\hat{\Omega},\hat{\Lambda}]|\psi\rangle\right|^{2}
\end{equation}

We next use the fact that
\begin{enumerate}
	\item since $[\hat{\Omega},\hat{\Lambda}] = i\Gamma$, where $\Gamma$ is Hermitian, the expectation value of the commutator is pure imaginary;
	\item since $[\hat{\Omega},\hat{\Lambda}]_{+}$ is Hermitian, the expectation value of the anticommutator is real.
\end{enumerate}
Recalling that $|a+ib|^{2} = a^{2} + b^{2}$, we get
\begin{align}
	(\Delta\Omega)^{2}(\Delta\Lambda)^{2} &\geq \frac{1}{4}\left|\langle\psi|[\hat{\Omega},\hat{\Lambda}]_{+}|\psi\rangle + i\langle\psi|\Gamma|\psi\rangle\right|^{2} \nonumber \\
	&\geq \frac{1}{4}\langle\psi|[\hat{\Omega},\hat{\Lambda}]_{+}|\psi\rangle^{2} + \frac{1}{4}\langle\psi|\Gamma|\psi\rangle^{2}
\end{align}

This is the general uncertainty relation between any two Hermitian operators and is evidently state dependent. Consider now canonically conjugate operators, for which $\Gamma = \hbar$. In this case
\begin{equation}
	(\Delta\Omega)^{2}(\Delta\Lambda)^{2} \geq \frac{1}{4}\langle\psi|[\hat{\Omega},\hat{\Lambda}]_{+}|\psi\rangle^{2} + \frac{\hbar^{2}}{4}
\end{equation}

Since the first term is positive definite, we may assert that for any $|\psi\rangle$
\[
(\Delta\Omega)^{2}(\Delta\Lambda)^{2} \geq \hbar^{2}/4
\]
or
\begin{equation}
	\Delta\Omega \cdot \Delta\Lambda \geq \hbar/2
\end{equation}
which is the celebrated uncertainty relation. Let us note that the above inequality becomes an equality only if
\par
	(1) $\hat{\Omega}|\psi\rangle = c\hat{\Lambda}|\psi\rangle$ 
\newline
and
\par
	(2) $\langle\psi|[\hat{\Omega},\hat{\Lambda}]_{+}|\psi\rangle = 0$

	\textbf{Generalized uncertainty principle:}
	\[\boxed{
	{\left( {\Delta A\Delta B} \right)^2} \ge {\left( {\frac{1}{{2i}}\left\langle {\left[ {\hat A,\hat B} \right]} \right\rangle } \right)^2}
	}\]
\subsection{顺序测量}
	\begin{problembox}{顺序测量}
	可观测量 A 用算符 $\hat{A}$ 表示,它的两个归一化本征态是 $\psi_{1}$ 和 $\psi_{2}$,本征值分别为 $a_{1}$ 和 $a_{2}$。算符 $\hat{B}$ 表示可观测量 B,它的两个归一化本征态是 $\phi_{1}$ 和 $\phi_{2}$,本征值分别为 $b_{1}$ 和 $b_{2}$。两组本征态之间关系为
	\[
	\psi_{1} = (3\phi_{1} + 4\phi_{2})/5, \quad \psi_{2} = (4\phi_{1} - 3\phi_{2})/5.
	\]
		
	\begin{enumerate}
		\item 测量可观测量 A,所得结果为 $a_{1}$。那么在测量之后(瞬时)体系处在什么态?
		\item 如果再测量 B,可能的结果是什么?它们出现的几率是多少?
		\item 在恰好测出 B 之后,再次测量 A。那么结果为 $a_{1}$ 的几率是多少?(注意如果已经告诉你测量 B 的结果,答案会完全不同。)
	\end{enumerate}
	\end{problembox}

	
	\subsection*{解答}
	\begin{enumerate}
		\item 当对体系测量 $\hat{A}$ 得到 $a_{1}$ 时,体系的波函数会坍缩为 $\hat{A}$ 本征值为 $a_{1}$ 的本征态 $\psi_{1}$,所以在测量之后(瞬时)体系在 $\psi_{1}$ 态。
		
		\item 由 $\psi_{1} = (3\phi_{1} + 4\phi_{2})/5$ 是 $\hat{B}$ 的本征态 $\phi_{1}$ 和 $\phi_{2}$ 的线性叠加,当对 $\psi_{1}$ 态测量 $\hat{B}$ 时,可能得到 $b_{1}$ 或者 $b_{2}$,得到 $b_{1}$ 的几率为 $9/25$,得到 $b_{2}$ 的几率为 $16/25$。
		
		\item 如果在测量 $\hat{B}$ 时得到的结果是 $b_{1}$,则波函数坍缩到 $\phi_{1}$ 态(几率为 $9/25$),由
		\[
		\psi_{1} = (3\phi_{1} + 4\phi_{2})/5, \quad \psi_{2} = (4\phi_{1} - 3\phi_{2})/5,
		\]
		可以解出
		\[
		\phi_{1} = (3\psi_{1} + 4\psi_{2})/5,
		\]
		所以再测量 $\hat{A}$ 时,得到 $a_{1}$ 的几率为 $9/25$。
		
		同理,如果在测量 $\hat{B}$ 时得到的是 $b_{2}$,则波函数坍缩到 $\phi_{2}$ 态(几率为 $16/25$)
		\[
		\phi_{2} = (4\psi_{1} - 3\psi_{2})/5,
		\]
		所以再测量 $\hat{A}$ 时得到 $a_{1}$ 的几率为 $16/25$。
		
		所以测量 $\hat{B}$,再测量 $\hat{A}$ 得到 $a_{1}$ 的几率为
		\[
		P = \frac{9}{25} \cdot \frac{9}{25} + \frac{16}{25} \cdot \frac{16}{25} = \frac{337}{625} = 0.5392.
		\]
	\end{enumerate}
\newpage

\section{Multi-Particle Systems}
In this chapter, we shall extend the single particle, one-dimensional formulation of non-relativistic quantum mechanics, introduced in the previous chapters, in order to investigate one-dimensional systems containing multiple particles.
\subsection{Fundamental Concepts of Multi-Particle Systems}

We have already seen that the instantaneous state of a system consisting of a single non-relativistic particle, whose position coordinate is $x$, is fully specified by a complex wavefunction $\psi(x, t)$. This wavefunction is interpreted as follows. The probability of finding the particle between $x$ and $x+dx$ at time $t$ is given by $|\psi(x, t)|^{2} dx$. This interpretation only makes sense if the wavefunction is normalized such that
\begin{equation}
	\int_{-\infty}^{\infty}|\psi(x, t)|^2 dx = 1
\end{equation}
at all times. The physical significance of this normalization requirement is that the probability of the particle being found anywhere on the $x$-axis must always be unity (which corresponds to certainty).

Consider a system containing $N$ non-relativistic particles, labeled $i=1, N$, moving in one dimension. Let $x_{i}$ and $m_{i}$ be the position coordinate and mass, respectively, of the $i$-th particle. By analogy with the single-particle case, the instantaneous state of a multi-particle system is specified by a complex wavefunction $\psi(x_{1}, x_{2},\ldots, x_{N}, t)$. The probability of finding the first particle between $x_{1}$ and $x_{1}+dx_{1}$, the second particle between $x_{2}$ and $x_{2}+dx_{2}$, et cetera, at time $t$ is given by $|\psi(x_{1}, x_{2},\ldots, x_{N}, t)|^{2} dx_{1} dx_{2}\ldots dx_{N}$. It follows that the wavefunction must satisfy the normalization condition
\begin{equation}
	\int |\psi(x_{1}, x_{2},\ldots, x_{N}, t)|^{2} dx_{1} dx_{2}\ldots dx_{N} = 1
	\label{eq:5.1.2}
\end{equation}
at all times, where the integration is taken over all $x_{1}x_{2}\ldots x_{N}$ space.

In a single-particle system, position is represented by the algebraic operator $x$, whereas momentum is represented by the differential operator $-i\hbar\partial/\partial x$. By analogy, in a multi-particle system, the position of the $i$-th particle is represented by the algebraic operator $x_{i}$, whereas the corresponding momentum is represented by the differential operator
\begin{equation}
	p_{i} = -i\hbar\frac{\partial}{\partial x_{i}}.
	\label{eq:5.1.3}
\end{equation}

Because the $x_{i}$ are independent variables (i.e., $\partial x_{i}/\partial x_{j} = \delta_{ij}$), we conclude that the various position and momentum operators satisfy the following commutation relations:
\begin{align}
	[x_i, x_j] &= 0, \label{eq:5.1.4} \\
	[p_i, p_j] &= 0, \label{eq:5.1.5} \\
	[x_i, p_j] &= i\hbar \delta_{ij}. \label{eq:5.1.6}
\end{align}

Now, we know that two dynamical variables can only be (exactly) measured simultaneously if the operators that represent them in quantum mechanics commute with one another. Thus, it is clear, from the previous commutation relations, that the only restriction on measurement in a one-dimensional multi-particle system is that it is impossible to simultaneously measure the position and momentum of the same particle. Note, in particular, that a knowledge of the position or momentum of a given particle does not in any way preclude a similar knowledge for a different particle. The commutation relations (\ref{eq:5.1.4})-(\ref{eq:5.1.6}) illustrate an important point in quantum mechanics: namely, that operators corresponding to different degrees of freedom of a dynamical system tend to commute with one another. In this case, the different degrees of freedom correspond to the different motions of the various particles making up the system.

Finally, if $H(x_{1}, x_{2},\ldots, x_{N}, t)$ is the Hamiltonian of the system then the multi-particle wavefunction $\psi(x_{1}, x_{2},\ldots, x_{N}, t)$ satisfies the usual time-dependent Schrödinger equation
\begin{equation}
	i\hbar\frac{\partial\psi}{\partial t} = H\psi.
	\label{eq:5.1.7}
\end{equation}
Likewise, a multi-particle state of definite energy $E$ (i.e., an eigenstate of the Hamiltonian with eigenvalue $E$) is written
\begin{equation}
	\psi(x_{1}, x_{2},\ldots, x_{N}, t) = \psi_{E}(x_{1}, x_{2},\ldots, x_{N}) e^{-i E t/\hbar},
	\label{eq:5.1.8}
\end{equation}
where the stationary wavefunction $\psi_{E}$ satisfies the time-independent Schrödinger equation
\begin{equation}
	H\psi_E = E\psi_E.
	\label{eq:5.1.9}
\end{equation}
Here, $H$ is assumed not to be an explicit function of $t$.

\subsection{Non-interacting Particles}
\label{sec:5.2}

In general, we expect the Hamiltonian of a multi-particle system to take the form
\begin{equation}
	H(x_{1}, x_{2},\ldots, x_{N}, t) = \sum_{i=1}^{N}\frac{p_{i}^{2}}{2 m_{i}} + V(x_{1}, x_{2},\ldots, x_{N}, t).
	\label{eq:5.2.1}
\end{equation}
Here, the first term on the right-hand side represents the total kinetic energy of the system, whereas the potential $V$ specifies the nature of the interaction between the various particles making up the system, as well as the interaction of the particles with any external forces.

Suppose that the particles do not interact with one another. This implies that each particle moves in a common potential: that is,
\begin{equation}
	V(x_{1}, x_{2},\ldots, x_{N}, t) = \sum_{i=1}^{N} V(x_{i}, t).
	\label{eq:5.2.2}
\end{equation}
Hence, we can write
\begin{equation}
	H(x_{1}, x_{2},\ldots, x_{N}, t) = \sum_{i=1}^{N} H_{i}(x_{i}, t)
	\label{eq:5.2.3}
\end{equation}
where
\begin{equation}
	H_i = \frac{p_i^2}{2 m_i} + V(x_i, t).
	\label{eq:5.2.4}
\end{equation}
In other words, for the case of non-interacting particles, the multi-particle Hamiltonian of the system can be written as the sum of $N$ independent single-particle Hamiltonians. Here, $H_{i}$ represents the energy of the $i$-th particle, and is completely unaffected by the energies of the other particles. Furthermore, given that the various particles that make up the system are non-interacting, we expect their instantaneous positions to be completely uncorrelated with one another. This immediately implies that the multi-particle wavefunction $\psi(x_{1}, x_{2},\ldots x_{N}, t)$ can be written as the product of $N$ independent single-particle wavefunctions: that is,
\begin{equation}
	\psi(x_{1}, x_{2},\ldots, x_{N}, t) = \psi_{1}(x_{1}, t)\psi_{2}(x_{2}, t)\ldots\psi_{N}(x_{N}, t).
	\label{eq:5.2.5}
\end{equation}

If the single-particle wavefunctions are normalized such that
\begin{equation}
	\int_{-\infty}^{\infty}|\psi_{i}(x_{i}, t)|^{2} dx_{i} = 1
	\label{eq:5.2.6}
\end{equation}
then the normalization constraint (\ref{eq:5.1.2}) for the multi-particle wavefunction is automatically satisfied. Equation (\ref{eq:5.2.5}) illustrates an important point in quantum mechanics: namely, that we can generally write the total wavefunction of a many degree of freedom system as a product of different wavefunctions corresponding to each degree of freedom.

According to Equations (\ref{eq:5.2.3}) and (\ref{eq:5.2.5}), the time-dependent Schrödinger equation (\ref{eq:5.1.7}) for a system of $N$ non-interacting particles factorizes into $N$ independent equations of the form
\begin{equation}
	i\hbar\frac{\partial\psi_{i}}{\partial t} = H_{i}\psi_{i}.
	\label{eq:5.2.7}
\end{equation}
Assuming that $V(x, t) \equiv V(x)$, the time-independent Schrödinger equation (\ref{eq:5.1.9}) also factorizes to give
\begin{equation}
	H_{i}\psi_{E_{i}} = E_{i}\psi_{E_{i}}
	\label{eq:5.2.8}
\end{equation}
where $\psi_{i}(x_{i}, t) = \psi_{E_{i}}(x_{i})\exp(-i E_{i} t/\hbar)$, and $E_{i}$ is the energy of the $i$-th particle. Hence, a multi-particle state of definite energy $E$ has a wavefunction of the form
\begin{equation}
	\psi(x_{1}, x_{2},\ldots, x_{n}, t) = \psi_{E}(x_{1}, x_{2},\ldots, x_{N}) e^{-i E t/\hbar},
	\label{eq:5.2.9}
\end{equation}
where
\begin{equation}
	\psi_{E}(x_{1},x_{2},...,x_{N}) = \psi_{E_{1}}(x_{1})\psi_{E_{2}}(x_{2})\ldots\psi_{E_{N}}(x_{N}),
	\label{eq:5.2.10}
\end{equation}
and
\begin{equation}
	E = \sum_{i=1}^{N} E_{i}.
	\label{eq:5.2.11}
\end{equation}
Clearly, for the case of non-interacting particles, the energy of the whole system is simply the sum of the energies of the component particles.


\subsection{Two-Particle Systems}
\label{sec:5.3}

Consider a system consisting of two particles, mass $m_{1}$ and $m_{2}$, interacting via a potential $V(x_{1}-x_{2})$ that only depends on the relative positions of the particles. According to Equations (\ref{eq:5.2.1}) and (\ref{eq:5.2.2}), the Hamiltonian of the system is written
\begin{equation}
	H(x_{1}, x_{2}) = -\frac{\hbar^{2}}{2 m_{1}}\frac{\partial^{2}}{\partial x_{1}^{2}} - \frac{\hbar^{2}}{2 m_{2}}\frac{\partial^{2}}{\partial x_{2}^{2}} + V(x_{1}-x_{2}).
	\label{eq:5.3.1}
\end{equation}
Let
\begin{equation}
	x' = x_{1} - x_{2}
	\label{eq:5.3.2}
\end{equation}
be the particles' relative position coordinate, and
\begin{equation}
	X = \frac{m_1 x_1 + m_2 x_2}{m_1 + m_2}
	\label{eq:5.3.3}
\end{equation}
the coordinate of the center of mass. It is easily demonstrated that
\begin{align}
	\frac{\partial}{\partial x_1} &= \frac{m_1}{m_1+m_2}\frac{\partial}{\partial X} + \frac{\partial}{\partial x'}, \label{eq:5.3.4a} \\
	\frac{\partial}{\partial x_2} &= \frac{m_2}{m_1+m_2}\frac{\partial}{\partial X} - \frac{\partial}{\partial x'}. \label{eq:5.3.4b}
\end{align}
Hence, when expressed in terms of the new variables, $x'$ and $X$, the Hamiltonian becomes
\begin{equation}
	H(x', X) = -\frac{\hbar^{2}}{2 M}\frac{\partial^{2}}{\partial X^{2}} - \frac{\hbar^{2}}{2\mu}\frac{\partial^{2}}{\partial x^{\prime 2}} + V(x'),
	\label{eq:5.3.5}
\end{equation}
where
\begin{equation}
	M = m_{1} + m_{2}
	\label{eq:5.3.6}
\end{equation}
is the total mass of the system, and
\begin{equation}
	\mu = \frac{m_{1} m_{2}}{m_{1} + m_{2}}
	\label{eq:5.3.7}
\end{equation}
the so-called reduced mass. Note that the total momentum of the system can be written
\begin{equation}
	P = -i\hbar\left(\frac{\partial}{\partial x_1} + \frac{\partial}{\partial x_2}\right) = -i\hbar\frac{\partial}{\partial X}.
	\label{eq:5.3.8}
\end{equation}

The fact that the Hamiltonian (\ref{eq:5.3.5}) is separable when expressed in terms of the new coordinates [i.e., $H(x', X) = H_{x'}(x') + H_{X}(X)$] suggests, that the wavefunction can be factorized: that is,
\begin{equation}
	\psi(x_1, x_2, t) = \psi_{x'}(x', t)\psi_X(X, t).
	\label{eq:5.3.9}
\end{equation}
Hence, the time-dependent Schrödinger equation (\ref{eq:5.1.7}) also factorizes to give
\begin{equation}
	i\hbar\frac{\partial\psi_{x'}}{\partial t} = -\frac{\hbar^{2}}{2\mu}\frac{\partial^{2}\psi_{x'}}{\partial x^{\prime 2}} + V(x')\psi_{x'},
	\label{eq:5.3.10}
\end{equation}
and
\begin{equation}
	i\hbar\frac{\partial\psi_{X}}{\partial t} = -\frac{\hbar^{2}}{2 M}\frac{\partial^{2}\psi_{X}}{\partial X^{2}}.
	\label{eq:5.3.11}
\end{equation}
The previous equation can be solved to give
\begin{equation}
	\psi_{X}(X, t) = \psi_{0} e^{i(P' X/\hbar - E' t/\hbar)},
	\label{eq:5.3.12}
\end{equation}
where $\psi_{0}$, $P'$, and $E' = P^{\prime 2}/ 2 M$ are constants. It is clear, from Equations (\ref{eq:5.3.8}), (\ref{eq:5.3.11}), and (\ref{eq:5.3.12}), that the total momentum of the system takes the constant value $P'$. In other words, momentum is conserved.

Suppose that we work in the centre of mass frame of the system, which is characterized by $P' = 0$. It follows that $\psi_{X} = \psi_{0}$. In this case, we can write the wavefunction of the system in the form $\psi(x_{1}, x_{2}, t) = \psi_{x'}(x', t)\psi_{0} \equiv \psi(x_{1}-x_{2}, t)$, where
\begin{equation}
	i\hbar\frac{\partial\psi}{\partial t} = -\frac{\hbar^{2}}{2\mu}\frac{\partial^{2}\psi}{\partial x^{2}} + V(x)\psi.
	\label{eq:5.3.13}
\end{equation}
In other words, in the center of mass frame, two particles of mass $m_{1}$ and $m_{2}$, moving in the potential $V(x_{1}-x_{2})$, are equivalent to a single particle of mass $\mu$, moving in the potential $V(x)$, where $x = x_{1}-x_{2}$. This is a familiar result from classical dynamics.

\subsection{Identical Particles}
\label{sec:5.4}

Consider a system consisting of two identical particles of mass $m$. As before, the instantaneous state of the system is specified by the complex wavefunction $\psi(x_{1}, x_{2}, t)$. This wavefunction tells us is that the probability of finding the first particle between $x_{1}$ and $x_{1}+dx_{1}$, and the second between $x_{2}$ and $x_{2}+dx_{2}$, at time $t$ is $|\psi(x_{1}, x_{2}, t)|^{2} dx_{1} dx_{2}$. However, because the particles are identical, this must be the same as the probability of finding the first particle between $x_{2}$ and $x_{2}+dx_{2}$, and the second between $x_{1}$ and $x_{1}+dx_{1}$, at time $t$ (because, in both cases, the result of the measurement is exactly the same). Hence, we conclude that
\begin{equation}
	|\psi(x_1, x_2, t)|^2 = |\psi(x_2, x_1, t)|^2,
	\label{eq:5.4.1}
\end{equation}
or
\begin{equation}
	\psi(x_1, x_2, t) = e^{i\varphi}\psi(x_2, x_1, t),
	\label{eq:5.4.2}
\end{equation}
where $\varphi$ is a real constant. However, if we swap the labels on particles 1 and 2 (which are, after all, arbitrary for identical particles), and repeat the argument, we also conclude that
\begin{equation}
	\psi(x_2, x_1, t) = e^{i\varphi}\psi(x_1, x_2, t).
	\label{eq:5.4.3}
\end{equation}
Hence,
\begin{equation}
	e^{2i\varphi} = 1.
	\label{eq:5.4.4}
\end{equation}
The only solutions to the previous equation are $\varphi = 0$ and $\varphi = \pi$. Thus, we infer that, for a system consisting of two identical particles, the wavefunction must be either symmetric or anti-symmetric under interchange of particle labels. That is, either
\begin{equation}
	\psi(x_2, x_1, t) = \psi(x_1, x_2, t),
	\label{eq:5.4.5}
\end{equation}
or
\begin{equation}
	\psi(x_2, x_1, t) = -\psi(x_1, x_2, t).
	\label{eq:5.4.6}
\end{equation}
The previous argument can easily be extended to systems containing more than two identical particles.

It turns out that the question of whether the wavefunction of a system containing many identical particles is symmetric or anti-symmetric under interchange of the labels on any two particles is determined by the nature of the particles themselves. Particles with wavefunctions that are symmetric under label interchange are said to obey Bose-Einstein statistics, and are called bosons. For instance, photons are bosons. Particles with wavefunctions that are anti-symmetric under label interchange are said to obey Fermi-Dirac statistics, and are called fermions. For instance, electrons, protons, and neutrons are fermions.

Consider a system containing two identical and non-interacting bosons. Let $\psi(x, E)$ be a properly normalized, single-particle, stationary wavefunction corresponding to a state of definite energy $E$. The stationary wavefunction of the whole system is written
\begin{equation}
	\psi_{E \text{ boson}}(x_{1}, x_{2}) = \frac{1}{\sqrt{2}}[\psi(x_{1}, E_{a})\psi(x_{2}, E_{b}) + \psi(x_{2}, E_{a})\psi(x_{1}, E_{b})],
	\label{eq:5.4.7}
\end{equation}
when the energies of the two particles are $E_{a}$ and $E_{b}$. This expression automatically satisfies the symmetry requirement on the wavefunction. Incidentally, because the particles are identical, we cannot be sure which particle has energy $E_{a}$, and which has energy $E_{b}$ - only that one particle has energy $E_{a}$, and the other $E_{b}$.

For a system consisting of two identical and non-interacting fermions, the stationary wavefunction of the whole system takes the form
\begin{equation}
	\psi_{E \text{ fermion}}(x_{1}, x_{2}) = \frac{1}{\sqrt{2}}[\psi(x_{1}, E_{a})\psi(x_{2}, E_{b}) - \psi(x_{2}, E_{a})\psi(x_{1}, E_{b})],
	\label{eq:5.4.8}
\end{equation}
Again, this expression automatically satisfies the symmetry requirement on the wavefunction. Note that if $E_{a} = E_{b}$ then the total wavefunction becomes zero everywhere. Now, in quantum mechanics, a null wavefunction corresponds to the absence of a state. We thus conclude that it is impossible for the two fermions in our system to occupy the same single-particle stationary state.

Finally, if the two particles are somehow distinguishable then the stationary wavefunction of the system is simply
\begin{equation}
	\psi_{E \text{ dist}}(x_{1}, x_{2}) = \psi(x_{1}, E_{a})\psi(x_{2}, E_{b}).
	\label{eq:5.4.9}
\end{equation}

Let us evaluate the variance of the distance, $x_{1}-x_{2}$, between the two particles, using the previous three wavefunctions. It is easily demonstrated that if the particles are distinguishable then
\begin{equation}
	\langle (x_1 - x_2)^2 \rangle_{\text{dist}} = \langle x^2 \rangle_a + \langle x^2 \rangle_b - 2\langle x \rangle_a \langle x \rangle_b,
	\label{eq:5.4.10}
\end{equation}
where
\begin{equation}
	\langle x^{n} \rangle_{a, b} = \int_{-\infty}^{\infty} \psi^{*}(x, E_{a, b}) x^{n} \psi(x, E_{a, b}) dx.
	\label{eq:5.4.11}
\end{equation}
For the case of two identical bosons, we find
\begin{equation}
	\langle (x_{1} - x_{2})^{2} \rangle_{\text{boson}} = \langle (x_{1} - x_{2})^{2} \rangle_{\text{dist}} - 2|\langle x \rangle_{ab}|^{2},
	\label{eq:5.4.12}
\end{equation}
where
\begin{equation}
	\langle x \rangle_{ab} = \int_{-\infty}^{\infty} \psi^{*}(x, E_{a}) x \psi(x, E_{b}) dx.
	\label{eq:5.4.13}
\end{equation}
Here, we have assumed that $E_{a} \neq E_{b}$, so that
\begin{equation}
	\int_{-\infty}^{\infty} \psi^{*}(x, E_{a}) \psi(x, E_{b}) dx = 0.
	\label{eq:5.4.14}
\end{equation}
Finally, for the case of two identical fermions, we obtain
\begin{equation}
	\langle (x_1 - x_2)^2 \rangle_{\text{fermion}} = \langle (x_1 - x_2)^2 \rangle_{\text{dist}} + 2|\langle x \rangle_{ab}|^2.
	\label{eq:5.4.15}
\end{equation}
Equation (\ref{eq:5.4.12}) indicates that the symmetry requirement on the total wavefunction of two identical bosons causes the particles to be, on average, closer together than two similar distinguishable particles. Conversely, Equation (\ref{eq:5.4.15}) indicates that the symmetry requirement on the total wavefunction of two identical fermions causes the particles to be, on average, further apart than two similar distinguishable particles. However, the strength of this effect depends on square of the magnitude of $\langle x \rangle_{ab}$, which measures the overlap between the wavefunctions $\psi(x, E_{a})$ and $\psi(x, E_{b})$. It is evident, then, that if these two wavefunctions do not overlap to any great extent then identical bosons or fermions will act very much like distinguishable particles.

For a system containing $N$ identical and non-interacting fermions, the anti-symmetric stationary wavefunction of the system is written
\begin{equation}
	\psi_{E}(x_{1}, x_{2},\ldots x_{N}) = \frac{1}{\sqrt{N!}}
	\begin{vmatrix}
		\psi(x_{1}, E_{1}) & \psi(x_{2}, E_{1}) & \ldots & \psi(x_{N}, E_{1}) \\
		\psi(x_{1}, E_{2}) & \psi(x_{2}, E_{2}) & \ldots & \psi(x_{N}, E_{2}) \\
		\vdots & \vdots & \vdots & \vdots \\
		\psi(x_{1}, E_{N}) & \psi(x_{2}, E_{N}) & \ldots & \psi(x_{N}, E_{N})
	\end{vmatrix}.
	\label{eq:5.4.16}
\end{equation}
This expression is known as the Slater determinant, and automatically satisfies the symmetry requirements on the wavefunction. Here, the energies of the particles are $E_{1}, E_{2},\ldots, E_{N}$. Note, again, that if any two particles in the system have the same energy (i.e., if $E_{i} = E_{j}$ for some $i \neq j$) then the total wavefunction is null. We conclude that it is impossible for any two identical fermions in a multi-particle system to occupy the same single-particle stationary state. This important result is known as the Pauli exclusion principle.

\newpage
	\section{一维问题}
	\begin{center}
	\textit{	\fbox{
			\textbf{一维情况下,}不存在简并束缚态 
	}}
	\end{center}
	\begin{proof*}
	设 $\psi_{1},\psi_{2}$ 是定态薛定谔方程具有同一个能量 $E$ 的不同的解, 即
	\[
	-\frac{\hbar^{2}}{2 m} \frac{\mathrm{d}^{2} \psi_{1}}{\mathrm{~d} x^{2}}+V \psi_{1}=E \psi_{1},
	\]
	\[
	-\frac{\hbar^{2}}{2 m} \frac{\mathrm{d}^{2} \psi_{2}}{\mathrm{~d} x^{2}}+V \psi_{2}=E \psi_{2}.
	\]
	
	第一个式子乘以 $\psi_{2}$, 第二个式子乘以 $\psi_{1}$, 两式相减, 得
	\[
	\psi_{2}\frac{\mathrm{d}^{2}\psi_{1}}{\mathrm{~d} x^{2}}-\psi_{1}\frac{\mathrm{d}^{2}\psi_{2}}{\mathrm{~d} x^{2}}=0,
	\]
	
	进一步得
	\[
	\frac{\mathrm{d}}{\mathrm{d} x}\left(\psi_{2}\frac{\mathrm{d}\psi_{1}}{\mathrm{~d} x}-\psi_{1}\frac{\mathrm{d}\psi_{2}}{\mathrm{~d} x}\right)=0.
	\]
	
	这表明
	\[
	\psi_{2}\frac{\mathrm{d}\psi_{1}}{\mathrm{~d} x}-\psi_{1}\frac{\mathrm{d}\psi_{2}}{\mathrm{~d} x}=K(\text{常数}).
	\]
	
	考虑到 $x\to\infty$ 时 $\psi_{1}\to 0,\psi_{2}\to 0$ (束缚态归一化的要求), 所以 $K=0$. 因此
	\[
	\frac{1}{\psi_{1}}\frac{\mathrm{d}\psi_{1}}{\mathrm{~d} x}=\frac{1}{\psi_{2}}\frac{\mathrm{d}\psi_{2}}{\mathrm{~d} x},
	\]
	
	得
	\[
	\ln \psi_{1}=\ln \psi_{2}+ \text{常数},\quad \text{即}\quad \psi_{1}= \text{常数} \times \psi_{2}.
	\]
	
	即这两个解仅相差一个常数, 归一化后, 它们仅相差一个相因子, 代表同一个物理态.
	\end{proof*}
		\begin{center}
			\fbox{
				\begin{minipage}{0.9\linewidth}  % 添加minipage确保正确换行
					\begin{itemize}[leftmargin=*, nosep]
						\item \textbf{波函数本身连续:}在空间任意位置(包括势能不连续点,除非势能为无穷大),波函数 $\psi(x)$ 必须是连续的。
						
						\item \textbf{波函数一阶导数连续:}在势能有限的不连续点处,波函数的一阶导数 $\dfrac{d\psi}{dx}$ 必须连续。
						
						\textit{\textbf{例外:}}若势能有无穷大跳跃(如无限深势阱边界),则一阶导数可以不连续。
					\end{itemize}
				\end{minipage}
			}
		\end{center}
	\subsection{对一个粒子的量子描述;波包}
	\subsection*{1. 自由粒子}
	从自由粒子的哈密顿量出发
	\[\hat H = \frac{{{{\hat p}^2}}}{{2m}}\]
	
	我们可以得知
	\[\left[ {H,p} \right] = 0\]
	
	由\textbf{Ehrenfest theorem}, 我们可以得到力学量的演化
	\[\left\langle p \right\rangle=\text{常数} \]
	
	如果你看到这很高兴,我们来求解坐标表象下的本征态\\
	当 $ V(\boldsymbol{r},t)=0 $ 时,薛定谔方程变为
	\begin{equation}
		i\hbar\frac{\partial}{\partial t}\psi(\boldsymbol{r}, t)=-\frac{\hbar^{2}}{2 m}\nabla ^2 \psi(\boldsymbol{r}, t) \quad \label{(C-1)}
	\end{equation}
	显然,这个微分方程具有下列形式的解:
	\begin{equation}
		\psi(\boldsymbol{r}, t)=A \mathrm{e}^{\mathrm{i}(\boldsymbol{k}\cdot\boldsymbol{r}-\omega t)} \quad \label{(C-2)}
	\end{equation}
	(式中 $A$ 为常数),其中 $\boldsymbol{k}$ 与 $\omega$ 之间必须有下列关系:
	\begin{equation}
		\omega=\frac{\hbar\boldsymbol{k}^{2}}{2 m} \quad \label{(C-3)}
	\end{equation}
	请注意,引用德布罗意关系式 $p = \hbar k,E = \hbar \omega $ ,便可以从条件 (\ref{(C-3)}) 得到一个自由粒子的能量 $E$ 和动量 $\boldsymbol{p}$ 的关系式
	\begin{equation}
		E=\frac{\boldsymbol{p}^{2}}{2 m} \quad \label{(C-4)}
	\end{equation}
	这是经典力学中一个熟知的关系式。到后面我们再来讨论 (\ref{(C-2)}) 式所表示的态的物理意义;不过在这里我们已经看到,由于
	\begin{equation}
		|\psi(\boldsymbol{r}, t)|^{2}=|A|^{2} \quad \label{(C-5)}
	\end{equation}
	所以一个这种类型的平面波代表这样一个粒子,它在空间各点出现的概率都一样。
	\paragraph*{一点讨论:}
	(\ref{(C-2)}) 式这种类型的平面波,它的模在空间处处为常数 (见 (\ref{(C-5)}) 式),这种函数并不是平方可积的;严格说来,它不能表示粒子的物理状态(同样,在光学中,一个单色平面波在物理上是不能实现的)。反之,平面波的叠加,却完全是平方可积的。
	
	叠加原理告诉我们,适合 (\ref{(C-3)}) 式的各平面波的一切线性组合,也是方程 (\ref{(C-1)}) 的解。这样的叠加可以写作
	\begin{equation}
		\psi(\boldsymbol{r}, t)=\frac{1}{(2\pi)^{3 /2}}\int g(\boldsymbol{k})\mathrm{e}^{\mathrm{i}[\boldsymbol{k}\cdot\boldsymbol{r}-\omega(k) t]}\mathrm{d}^{3} k \quad \label{(C-6)}
	\end{equation}
	(按定义,$\mathrm{d}^{3}k$ 表示 $\boldsymbol{k}$ 空间的体积元 $\mathrm{d}k_x \mathrm{d}k_y \mathrm{d}k_z$);$g(\boldsymbol{k})$ 可以是复函数,但必须是充分正规的,以保证可以在积分号下求它的微商。此外,我们可以证明,方程 (\ref{(C-1)}) 的一切平方可积的解都可以写成 (\ref{(C-6)}) 式的形式。
	
	形如 (\ref{(C-6)}) 式的波函数,即平面波的叠加,叫做一个三维“波包”。为简单起见,我们研究一维波包的情况。平行于 $Ox$ 轴传播的诸平面波的叠加便是一维波包,因而它的波函数只依赖于 $x$ 和 $t$,即
	\begin{equation}
		\psi (x, t)=\frac{1}{\sqrt{2 \pi}} \int_{-\infty}^{+\infty} g(k) \mathrm{e}^{\mathrm{i}[k x-\omega(k) t]} \mathrm{d} k \quad \label{(C-7)}
	\end{equation}
	我们讨论波包在指定时刻的形状。若将这个时刻选作时间的起点,则波函数应为:
	\begin{equation}
		\psi (x, 0)=\frac{1}{\sqrt{2 \pi}} \int g(k) \mathrm{e}^{\mathrm{i} k x} \mathrm{~d} k \quad \label{(C-8)}
	\end{equation}
	我们看到,$g(k)$ 其实就是 $\psi (x, 0)$ 的傅里叶变换,即
	\begin{equation}
		g(k)=\frac{1}{\sqrt{2\pi}}\int\psi(x, 0)e^{-ikx}dx \quad \label{(C-9)}
	\end{equation}
	因而,公式 (\ref{(C-8)}) 的适用范围并不限于自由粒子。就是说,不论存在什么样的势,都可以将 $\psi (x, 0)$ 写成这种形式。
	
	
	
	\subsection*{2. 波包在指定时刻的形状}
	式 (\ref{(C-8)}) 中的 $\psi (x, 0)$ 对 $x$ 的依赖关系决定着波包的形状。
	
	先考虑一个很简单的特例,以便着手定性地研究 $\psi (x, 0)$ 的行为。假设 $\psi (x, 0)$ 不是像公式 (\ref{(C-8)}) 中那样的无穷多个平面波 $\mathrm{e}^{\mathrm{i}kx}$ 的叠加,而仅仅是三个平面波之和;这些平面波的波矢为 $k_{0}$, $k_{0}-\frac{\Delta k}{2}$, $k_{0}+\frac{\Delta k}{2}$,它们的振幅分别正比于 1, $\frac{1}{2}$ 和 $\frac{1}{2}$。于是我们有:
	\begin{equation}
		\begin{aligned}
			\psi(x) &= \frac{g(k_{0})}{\sqrt{2\pi}}\left[e^{i k_{0}x}+\frac{1}{2}e^{i(k_{0}-\frac{\Delta k}{2})x}+\frac{1}{2}e^{i(k_{0}+\frac{\Delta k}{2})x}\right] \\
			&= \frac{g(k_{0})}{\sqrt{2\pi}}e^{ik_{0}x}\left[1+\cos\left(\frac{\Delta k}{2}x\right)\right] \quad \label{(C-10)}
		\end{aligned}
	\end{equation}
	容易看出,在 $x=0$ 处 $|\psi(x)|$ 有极大值。造成这个结果的原因是下述事实:当 $x$ 取这个值的时候,三个波是同相位的,因而它们的干涉是相长的,如图 1-4 所示。在 $x$ 逐渐偏离 0 值以后,三个波的相位便互有差异,于是 $|\psi(x)|$ 便减小了。当 $e^{ik_{0}x}$ 和 $e^{i(k_{0}\mp\frac{\Delta k}{2})x}$ 之间的相位差等于 $\pm\pi$ 时,它们的干涉便是完全相消的;当 $x=\pm\frac{\Delta x}{2}$ 时,$\psi(x)$ 等于零,$\Delta x$ 由
	\begin{equation}
		\Delta x\cdot\Delta k=4\pi \quad \label{(C-11)}
	\end{equation}
	给出。此式表明,函数 $|g(k)|$ 的宽度 $\Delta k$ 越小,函数 $|\psi(x)|$ 的宽度 $\Delta x$($|\psi(x)|$ 的两个零点间的距离)就越大。
	
	\subsection*{附注:}
	公式 (\ref{(C-10)}) 表明,$|\psi(x)|$ 对于 $x$ 具有周期性,因而具有一系列极大和极小,其原因在于,$\psi(x)$ 是有限多个(这里是三个)波的叠加;若是无限多个波的连续叠加(像在公式 (\ref{(C-8)}) 中那样),便不会出现这样的现象,而 $|\psi(x,0)|$ 只会有一个极大值。

	
	\subsection{Particle in Box}
	
	势能函数定义:
	\[
	V(x) = 
	\begin{cases} 
		0, & 0 \leq x \leq a \\
		\infty, & \text{otherwise}
	\end{cases}
	\]
	
	在区域 $x \in (0, a)$ 内,$V(x) = 0$,定态薛定谔方程为:
	\[
	\frac{p^2}{2m} \psi(x) = E \psi(x)
	\]
	在位置表象中,动量算符 $p = -i\hbar \frac{d}{dx}$,因此 $p^2/(2m) = -\frac{\hbar^2}{2m} \frac{d^2}{dx^2}$,方程化为:
	\[
	-\frac{\hbar^2}{2m} \frac{d^2 \psi(x)}{dx^2} = E \psi(x)
	\]
	
	整理得:
	\[
	\frac{d^2 \psi(x)}{dx^2} + \frac{2mE}{\hbar^2} \psi(x) = 0
	\]
	
\begin{center}
	\textit{	\fbox{
			\textbf{注意:} 束缚态问题的每一个归一化解,$E$ 必须大于 $V(x)$ 的最小值。
	}}
\end{center}
\begin{proofbox}
	束缚态波函数满足归一化条件, 考虑哈密顿量在态 $\psi$ 中的期望值:
	\[
	\langle \hat{H} \rangle = \int_{-\infty}^{+\infty} \psi^*(x) \, \hat{H} \, \psi(x) \, dx
	= \int_{-\infty}^{+\infty} \left( \frac{\hbar^2}{2m} \left| \frac{d\psi}{dx} \right|^2 + V(x) |\psi(x)|^2 \right) dx.
	\]
	由于动能项 $\dfrac{\hbar^2}{2m} \left| \dfrac{d\psi}{dx} \right|^2$ 处处非负,因此
	\[
	\langle \hat{H} \rangle \;\geq\; \int_{-\infty}^{+\infty} V(x) |\psi(x)|^2 \, dx
	\;\geq\; V_{\text{min}} \int_{-\infty}^{+\infty} |\psi(x)|^2 \, dx
	= V_{\text{min}} .
	\]
	对于归一化的能量本征态,$\langle \hat{H} \rangle = E$,故有
	\[
	E \;\geq\; V_{\text{min}} .
	\]
	
	若等号成立 $E = V_{\text{min}}$,则上述不等式链成为等式,这要求动能项为零,即
	\[
	\left| \frac{d\psi}{dx} \right|^2 \equiv 0 \quad \text{(几乎处处成立)},
	\]
	因此 $\psi(x)$ 为常数函数。但常数波函数在无穷区间上不可归一化,这与束缚态的归一化条件矛盾。因此等号不可能达到,必须有
	\[
	E \;>\; V_{\text{min}} .
	\]
\end{proofbox}	
	由于 $E > V_{\min} = 0$,令 $\displaystyle k = \frac{\sqrt{2mE}}{\hbar}>0$,则方程简化为:
	\[
	\frac{d^2 \psi(x)}{dx^2} + k^2 \psi(x) = 0
	\]
	该方程的通解为:
	\[
	\psi(x) = A e^{ikx} + B e^{-ikx}
	\]
	其中 $A$ 和 $B$ 为复常数。
	
	应用边界条件:在 $x=0$ 和 $x=a$ 处,波函数为零(因为势阱外 $V=\infty$)。
	\begin{align*}
		x=0: &\quad \psi(0) = A + B = 0 \\
		x=a: &\quad \psi(a) = A e^{ika} + B e^{-ika} = 0
	\end{align*}
	通过行列式为零:
	\[
	\left| \begin{array}{cc}
		1 & 1 \\
		e^{ika} & e^{-ika}
	\end{array} \right| = 0 \quad \Rightarrow \quad e^{-ika} - e^{ika} = 0
	\]
	这等价于 $e^{2ika} = 1$,即:
	\[
	2ka = 2n\pi \quad (n = 1, 2, 3, \dots)
	\]
	解得:
	\[
	k = \frac{n\pi}{a} = \frac{\sqrt{2mE}}{\hbar}
	\]
	代入 $k$ 得能量本征值:
	\[
	E_n = \frac{\hbar^2 k^2}{2m} = \frac{\hbar^2 \pi^2 n^2}{2m a^2}
	\]
	利用边界条件 $B = -A$,波函数为:
	\[
	\psi(x) = A (e^{ikx} - e^{-ikx}) = 2iA \sin(kx)
	\]
	令 $C = 2iA$,则 $\psi(x) = C \sin(kx)$,其中 $k = n\pi/a$。
	
	通过归一化确定常数 $C$:
	\[
	\int_0^a |\psi(x)|^2 \,dx = 1
	\]
	代入 $\psi(x) = C \sin(kx)$:
	\[
	\int_0^a |C|^2 \sin^2(kx) \,dx = |C|^2 \int_0^a \sin^2\left( \frac{n\pi x}{a} \right) dx = 1
	\]
	\[
	|C|^2 \cdot \frac{a}{2} = 1 \quad \Rightarrow \quad |C| = \sqrt{\frac{2}{a}}
	\]
	则归一化波函数为:
	\[
	\psi_n(x) = \sqrt{\frac{2}{a}} \sin\left( \frac{n\pi x}{a} \right)
	\]
	\paragraph*{一点讨论}
	根据不确定性原理,当位置受到 $\left|x\right|\leqslant L/2$ 的限制时,不允许粒子具有一个确定的零动量。这反过来又导致能量的下限,这一点我们推导如下。我们从
	\[
	H=\frac{P^{2}}{2m}
	\]
	所以
	\begin{equation}
			\left\langle H\right\rangle=\frac{\left\langle P^{2}\right\rangle}{2m}\label{5.2.19}
	\end{equation}
	现在,在任一束缚态下 $\left\langle P\right\rangle=0$,原因如下。因为束缚态是一个定态,$\left\langle P\right\rangle$ 与时间无关。如果这个 $\left\langle P\right\rangle\neq0$,粒子一定(在平均意义上)向右或向左漂移,最终逃逸到无穷远处,在束缚态下是不可能发生的。
	因此,我们可以把 (\ref{5.2.19}) 式改写为
	\[
	\left\langle H\right\rangle=\frac{\left\langle(P-\left\langle P\right\rangle)^{2}\right\rangle}{2m}=\frac{\left(\Delta P\right)^{2}}{2m}
	\]
	如果我们现在利用不确定度关系
	\[
	\Delta P\cdot\Delta X\geqslant\hbar/2
	\]
	我们发现
	\[
	\left\langle H\right\rangle\geqslant\frac{\hbar^{2}}{8m(\Delta X)^{2}}
	\]
	
	\noindent
	由于变量 $x$ 受到 $-L/2 \leq x \leq L/2$ 的限制,所以其标准偏差 $\Delta X$ 不可能超过 $L/2$。因此,
	\[
	\langle H \rangle \geq \frac{\hbar^{2}}{2mL^{2}}
	\]
	在能量本征态上,$\langle H \rangle = E$,所以
	\[
	E \geq \frac{\hbar^{2}}{2mL^{2}}
	\]
	我们需要强调无限深势肼是束缚态:体现为波函数在无穷远处趋向于零
	\begin{center}
		\textit{	\fbox{
				\textbf{束缚态:} 粒子被限制在空间有限区域内运动,不能逃离到无穷远
		}}
	\end{center}
	
	与束缚态相对的是散射态。散射态粒子的能量足以让它运动到无穷远处,其能量谱是连续的,波函数在无穷远处也不为零。数学上散射态波函数本身是不可归一化的!物理上,通过波包叠加(代表真实粒子)或数学上通过盒子归一化技巧来处理其归一化问题,从而计算有物理意义的可观测量(如散射截面)
	
	\textbf{\textit{Condition:}}
	\begin{equation*}
		\begin{cases}
			E < V(-\infty) \text{ and } V(+\infty) \Rightarrow \text{ bound state,} \\
			E > V(-\infty) \text{ or } V(+\infty) \Rightarrow \text{ scattering state.}
		\end{cases}	
	\end{equation*}
	
	% 中间说明文字
	In real life most \textbf{potentials go to zero at infinity}, in which case the criterion simplifies even further:
	
	% 公式 (2.113)
	\begin{equation*}
		\begin{cases}
			E < 0 \Rightarrow \text{ bound state,} \\
			E > 0 \Rightarrow \text{ scattering state.}
		\end{cases}
	\end{equation*}
	
	
		
		\subsection{Delta势}
		首先,考虑束缚态
		
		势能定义为:
		\[
		V(x) = \begin{cases}
			\delta(x), & x = 0 \\
			0, & x \neq 0
		\end{cases}
		\]
		
		引入参数 $\alpha$(或 $\alpha_1$)使得 $V(x) = \alpha \delta(x)$。
		
		考虑定态薛定谔方程:
		\[
		-\frac{\hbar^2}{2m} \frac{\partial^2 \psi}{\partial x^2} + V(x) \psi(x) = E \psi(x)
		\]
		
		当 $x < 0$ 时,$V(x) = 0$,方程简化为:
		\[
		-\frac{\hbar^2}{2m} \frac{\partial^2 \psi}{\partial x^2} = E \psi(x)
		\]
		
		令 $k^2 = -\dfrac{2mE}{\hbar^2}$(注意 $E < 0$),则通解为:
		\[
		\psi(x) = A_1 e^{kx} + A_2 e^{-kx}
		\]
		
		由于 $k > 0$,当 $x \to -\infty$ 时要求波函数有限,故 $A_2 = 0$,即:
		\[
		\psi(x) = A_1 e^{kx}, \quad x < 0
		\]
		
		同理,当 $x > 0$ 时,解为:
		\[
		\psi(x) = B_1 e^{-kx}, \quad x > 0
		\]
		
		在 $x = 0$ 处波函数连续:$\psi(0^-) = \psi(0^+)$,可得 $A_1 = B_1 = C$,因此:
		\[
		\psi(x) = 
		\begin{cases}
			C e^{kx}, & x < 0 \\
			C e^{-kx}, & x > 0
		\end{cases}
		\]
		
		接下来考虑 $x = 0$ 处的导数跃变。对薛定谔方程在 $(-\varepsilon, \varepsilon)$ 积分并取极限 $\varepsilon \to 0$:
		\[
		\lim_{\varepsilon \to 0} \int_{-\varepsilon}^{+\varepsilon} \left[ -\frac{\hbar^2}{2m} \frac{\partial^2 \psi}{\partial x^2} + \alpha \delta(x) \psi(x) \right] dx = \lim_{\varepsilon \to 0} \int_{-\varepsilon}^{+\varepsilon} E \psi(x) dx
		\]
		
		由于 $E \psi(x)$ 有限,右侧积分为零。左侧第一项给出:
		\[
		\lim_{\varepsilon \to 0} \int_{-\varepsilon}^{+\varepsilon} \frac{\partial^2 \psi}{\partial x^2} dx = \lim_{\varepsilon \to 0} \left[ \psi'(\varepsilon) - \psi'(-\varepsilon) \right] = \psi'(0^+) - \psi'(0^-)
		\]
		
		利用波函数形式,$\psi'(0^+) = -kC$,$\psi'(0^-) = kC$,故该积分为 $-2kC$。左侧第二项为:
		\[
		\lim_{\varepsilon \to 0} \int_{-\varepsilon}^{+\varepsilon} \alpha \delta(x) \psi(x) dx = \alpha \psi(0) = \alpha C
		\]
		
		代入方程整理得:
		\[
		-\frac{\hbar^2}{2m} (-2kC) + \alpha C = 0 \quad \Rightarrow \quad k =- \frac{m\alpha}{\hbar^2}
		\]
		
		由于 $k>0$, 要求 $\alpha<0$, 结合 $k^2 = -\dfrac{2mE}{\hbar^2}$, 解得能量:
		\[
		E = -\frac{m\alpha^2}{2\hbar^2}
		\]
		
		最后进行归一化:
		\[
		\int_{-\infty}^{\infty} |\psi(x)|^2 dx = \int_{-\infty}^{0} C^2 e^{2kx} dx + \int_{0}^{\infty} C^2 e^{-2kx} dx = \frac{C^2}{k} = 1  \Rightarrow C = \sqrt k 
		\]
		
		故归一化波函数为:
		\[\psi \left( x \right) = \sqrt k {e^{-k\left| x \right|}}\]
		
		\subsection{散射态($E>0$)的推导}
		
		\subsubsection{波函数形式}
		
		对于 \(E>0\),令 \(k = \sqrt{\dfrac{2mE}{\hbar^2}} > 0\).
		
		考虑从左侧入射的平面波:
		
		\begin{itemize}
			\item 当 \(x<0\) 时,波函数包含入射波和反射波:
			\[
			\psi(x) = A e^{ikx} + B e^{-ikx}
			\]
			其中 \(A\) 是入射振幅,\(B\) 是反射振幅.
			
			\item 当 \(x>0\) 时,波函数只有透射波:
			\[
			\psi(x) = C e^{ikx}
			\]
			其中 \(C\) 是透射振幅.
		\end{itemize}
		
		\subsubsection{边界条件}
		
		在 \(x=0\) 处:
		
		1. 波函数连续:
		\begin{equation}
		A + B = C \label{1}
		\end{equation}
		
		2. 波函数导数跃变(由 \(\delta\) 势导致):
		\[
		\psi'(0^+) - \psi'(0^-) = \frac{2m\alpha}{\hbar^2} \psi(0)
		\]
		计算导数:
		\[
		\psi'(0^-) = ikA - ikB, \quad \psi'(0^+) = ikC
		\]
		代入得:
		\[
		ikC - ik(A - B) = \frac{2m\alpha}{\hbar^2} C
		\]
		整理得:
		\begin{equation}
			ik(C - A + B) = \frac{2m\alpha}{\hbar^2} \label{2}
		\end{equation}
		
		\subsubsection{反射系数与透射系数}
		
		将 (\ref{1}) 式代入 (\ref{2}) 式:
		\[
		ik[(A+B) - A + B] = \frac{2m\alpha}{\hbar^2} (A+B)
		\]
		\[
		2ikB = \frac{2m\alpha}{\hbar^2} (A+B)
		\]
		令 \(\beta = \dfrac{m\alpha}{\hbar^2 k}\),则上式化为:
		\[
		B = \beta (A+B)
		\]
		解得:
		\[
		B = \frac{\beta}{1-\beta} A, \quad C = A+B = \frac{1}{1-\beta} A
		\]
		
		反射系数 \(R\) 和透射系数 \(T\) 分别为:
		\[
		R = \left|\frac{B}{A}\right|^2 = \frac{ \beta ^2}{ 1-\beta ^2}, \quad T = \left| \frac{C}{A} \right|^2 = \frac{1}{ 1-\beta ^2}
		\]
		
		其中:
		\[
		\beta = \frac{m\alpha}{\hbar^2 k} = \frac{\alpha}{\hbar} \sqrt{\frac{m}{2E}}
		\]
		
		可以验证,\(R + T = 1\),满足概率守恒.
		
		\subsubsection{束缚态和散射态情况总结}
		
		\paragraph*{吸引势(\(\alpha<0\))}
		
		\begin{itemize}
			\item 束缚态:存在一个束缚态,能量为 \(E = -\dfrac{m\alpha^2}{2\hbar^2}\).
			\item 散射态:当 \(E>0\) 时,粒子可能被部分反射或透射.
		\end{itemize}
		
		\paragraph*{排斥势(\(\alpha>0\))}
		
		\begin{itemize}
			\item 束缚态:不存在束缚态.
			\item 散射态:当 \(E>0\) 时,粒子被部分反射或透射,反射系数和透射系数由上述公式给出.
		\end{itemize}
		
		\paragraph*{特殊情形} 
		\begin{itemize}
			\item 当 \(E \to \infty\) 时,\(T \to 1\),\(R \to 0\)(高能粒子几乎完全透射).
			\item 当 \(E \to 0^+\) 时,\(T \to 0\),\(R \to 1\)(低能粒子几乎完全反射).
		\end{itemize}
		
			
			\subsubsection{有限深方势阱}
			
			势能函数:
			\[
			V(x) = 
			\begin{cases}
				-V_0, & -a \leq x \leq a \\
				0, & |x| > a
			\end{cases}
			\]
			
			定态薛定谔方程:
			\[
			\left[ -\frac{\hbar^2}{2m} \frac{d^2}{dx^2} + V(x) \right] \psi(x) = E \psi(x)
			\]
			
			由于考虑束缚态,于是有$E$ 必须大于 $V(x)$ 的最小值,同时 $E<V(\infty), E<0$
			\subsubsection*{区域 $|x| > a$}
			此时 $V(x) = 0$,方程为:
			\[
			-\frac{\hbar^2}{2m} \frac{d^2 \psi(x)}{dx^2} = E \psi(x)
			\]
			\(E<0\),令 $k_1^2 = \dfrac{-2mE}{\hbar^2} > 0$,则方程解为:
			\[
			\psi(x) = A_1 e^{k_1 x},\quad x < -a; \qquad \psi(x) = B_1 e^{-k_1 x},\quad x > a.
			\]
			
			\subsubsection*{区域 $x \in [-a, a]$}
			此时 $V(x) = -V_0$,方程为:
			\[
			-\frac{\hbar^2}{2m} \frac{d^2 \psi(x)}{dx^2} = (E + V_0) \psi(x)
			\]
			令 $k_2^2 = \dfrac{2m(E + V_0)}{\hbar^2}$(束缚态下 $E + V_0 > 0$,故 $k_2^2 > 0$),方程解为:
			\[
			\psi(x) = C_1 e^{i k_2 x} + C_2 e^{-i k_2 x},\quad -a \leq x \leq a.
			\]
			
			综上,波函数表达式为:
			\[
			\psi(x) = 
			\begin{cases}
				A_1 e^{k_1 x}, & x < -a \\
				C_1 e^{i k_2 x} + C_2 e^{-i k_2 x}, & -a \leq x \leq a \\
				B_1 e^{-k_1 x}, & x > a
			\end{cases}
			\]
			
			\subsubsection*{根据边界条件}
			波函数 $\psi(x)$ 连续,$\psi'(x)$ 连续。在 $x = \pm a$ 处有:
			\[
			\begin{cases}
				A_1 e^{-k_1 a} = C_1 e^{-i k_2 a} + C_2 e^{i k_2 a}, & \psi(-a) \\
				B_1 e^{-k_1 a} = C_1 e^{i k_2 a} + C_2 e^{-i k_2 a}, & \psi(a) \\
				k_1 A_1 e^{-k_1 a} = i k_2 C_1 e^{-i k_2 a} - i k_2 C_2 e^{i k_2 a}, & \psi'(-a) \\
				-k_1 B_1 e^{-k_1 a} = i k_2 C_1 e^{i k_2 a} - i k_2 C_2 e^{-i k_2 a}, & \psi'(a)
			\end{cases}
			\]
			\begin{center}
				\textit{	\fbox{
						\textbf{定理:} 若 $V(x)$ 是偶函数, 此时宇称守恒, $\psi(x)$ 总可以取偶函数或奇函数。
				}}
			\end{center}
			
			
			\subsubsection*{(1) 偶函数(偶宇称)情形}
			$A_1 = B_1,\; C_1 = C_2$,代入边界条件得:
			\[
			\begin{cases}
				A_1 e^{-k_1 a} = 2 C_1 \cos(k_2 a) \\
				k_1 A_1 e^{-k_1 a} = 2 k_2 C_1 \sin(k_2 a)
			\end{cases}
			\]
			两式相除得:
			\[
			k_1 = k_2 \tan(k_2 a).
			\]
			
			\subsubsection*{(2) 奇函数(奇宇称)情形}
			$A_1 = -B_1,\; C_1 = -C_2$,代入边界条件得:
			\[
			\begin{cases}
				A_1 e^{-k_1 a} = -2i C_1 \sin(k_2 a) \\
				k_1 A_1 e^{-k_1 a} = 2i k_2 C_1 \cos(k_2 a)
			\end{cases}
			\]
			两式相除得:
			\[
			k_1 = -k_2 \cot(k_2 a).
			\]
		
		引入无量纲参数:
		\[
		\xi = k_2 a, \quad \eta = k_1 a, \quad R^2 = \xi^2 + \eta^2 = \frac{2m V_0 a^2}{\hbar^2}
		\]
		其中 $R$ 是衡量势阱"强度"的参数。
		
		本征值方程改写为:
		\begin{align*}
			\text{偶宇称:} & \quad \eta = \xi \tan \xi \\
			\text{奇宇称:} & \quad \eta = -\xi \cot \xi
		\end{align*}
		约束条件为 $\xi^2 + \eta^2 = R^2$。
		
		束缚态数目由参数 $R$ 决定
		\begin{itemize}
			\item 当 $0 < R \leq \pi/2$ 时,只有 1 个束缚态(偶宇称)
			\item 当 $\pi/2 < R \leq \pi$ 时,有 2 个束缚态(1 偶宇称,1 奇宇称)
			\item 当 $\pi < R \leq 3\pi/2$ 时,有 3 个束缚态(2 偶宇称,1 奇宇称)
			\item 一般地,当 $\displaystyle (N-1)\frac{\pi}{2} < R \leq N\frac{\pi}{2}$ 时,有 $N$ 个束缚态
		\end{itemize}
		
		\subsection*{偶宇称波函数的归一化}
		偶宇称波函数形式为:
		\[
		\psi_e(x) = 
		\begin{cases}
			A e^{k_1 x}, & x < -a \\
			C \cos(k_2 x), & -a \le x \le a \\
			A e^{-k_1 x}, & x > a
		\end{cases}
		\]
		由连续性条件得 $A = C \cos(k_2 a) e^{k_1 a}$。归一化常数 $C$ 由下式确定:
		\[
		C = \frac{1}{\sqrt{a + \frac{1}{k_1}}}
		\]
		
		\subsection*{奇宇称波函数的归一化}
		奇宇称波函数形式为:
		\[
		\psi_o(x) = 
		\begin{cases}
			A e^{k_1 x}, & x < -a \\
			C \sin(k_2 x), & -a \le x \le a \\
			-A e^{-k_1 x}, & x > a
		\end{cases}
		\]
		由连续性条件得 $A = -C \sin(k_2 a) e^{k_1 a}$。归一化常数 $C$ 由下式确定:
		\[
		C = \frac{1}{\sqrt{a - \frac{\sin(2k_2 a)}{2k_2}}}
		\]
		
		\subsection*{讨论与结论}
		
		\subsubsection*{能级特性}
		\begin{itemize}
			\item 束缚态能级 $E_n$ 满足 $-V_0 < E_n < 0$,且能级总数有限
			\item 最低能级为偶宇称态,随后宇称奇偶交替出现
			\item 能级间距随量子数增加而减小
		\end{itemize}
		
		\subsubsection*{势阱参数的影响}
		参数 $R = a\sqrt{2mV_0}/\hbar$ 对体系性质有重要影响:
		\begin{itemize}
			\item $R$ 越大,束缚态越多,能级间隔越小
			\item 当 $R$ 很小(浅而窄的势阱)时,可能只存在一个束缚态,甚至没有束缚态
			\item 实际物理体系中,$R$ 的大小决定了量子效应的显著程度
		\end{itemize}

\newpage
\section{谐振子}
\subsection{谐振子的解}
\begin{equation}
	V = \frac{1}{2} m \omega^{2} x^{2}, \quad H = \frac{p^{2}}{2m} + \frac{1}{2} m \omega^{2} x^{2}
\end{equation}

定义湮灭算符 $a$ , 产生算符 $a^\dag$ 
\begin{equation}
	a = \frac{1}{\sqrt{2 m \hbar \omega}} (m \omega x + i p), \quad a^\dag = \frac{1}{\sqrt{2 m \hbar \omega}} (m \omega x - i p)
\end{equation}

于是有
\begin{align}
	a^\dag a &= \frac{1}{2 m \hbar \omega} \left(m^{2} \omega x^{2} - i m \omega p x + i m \omega x p + p^{2}\right) \notag \\
	&= \frac{1}{2 m \hbar \omega} \left(m^{2} \omega x^{2} + i m \omega[x, p] + p^{2}\right).
\end{align}

由于
\begin{align}
	[x, p] f &= -i \hbar x \frac{\partial f}{\partial x} + i \hbar \frac{\partial}{\partial x}(x f) \notag \\
	&= -i \hbar x \frac{\partial f}{\partial x} + i \hbar x \frac{\partial f}{\partial x} + i \hbar f = i \hbar f,
\end{align}

于是得到
\begin{equation}
	[x, p] = i \hbar.
\end{equation}

可知
\begin{equation}
	a^\dag a = \frac{1}{2 m \hbar \omega} \left(m^{2} \omega x^{2} - m \omega \hbar + p^{2}\right).
\end{equation}

进而有
\begin{equation}
	\hbar \omega \left(a^\dag a + \frac{1}{2}\right) = \frac{p^{2}}{2m} + \frac{1}{2} m \omega^{2} x^{2} = H.
\end{equation}

于是得到 $H = \hbar \omega (a^\dag a + \frac{1}{2})$,令 $a^\dag a = N$,这也被称为粒子数算符,则
\begin{equation}
	\boxed{H = \hbar \omega \left(N + \frac{1}{2}\right)}.
\end{equation}

可以看出 $N$ 为厄米算符,且与哈密顿量$H$对易。根据本征方程
\begin{equation}
	N\left| n \right\rangle = n\left| n \right\rangle.
\end{equation}

本征值满足
\begin{equation}
	n = \langle n | a^\dag a | n \rangle = \langle a n | a n \rangle \geqslant 0.
\end{equation}

考虑
\begin{align}
	a a^\dag &= \frac{1}{2 m \hbar \omega} \left(m^{2} \omega x^{2} - i m \omega[x, p] + p^{2}\right), \\
	a^\dag a &= \frac{1}{2 m \hbar \omega} \left(m^{2} \omega x^{2} + i m \omega[x, p] + p^{2}\right).
\end{align}

有对易关系
\begin{equation}
	[a, a^\dag] = 1, \quad [N, a] = -a, \quad [N, a^\dag] = a^\dag.
\end{equation}

接下来我们求解本征值 $n$ 与本征态矢 $| n \rangle$(不存在简并的前提下)。

根据对易关系 $[N, a] = -a, \ [N, a^\dag] = a^\dag$,有
\begin{align*}
	N a | n \rangle &= (a N - a) | n \rangle \\
	&= a(n - 1) | n \rangle.
\end{align*}

即 $a\left| n \right\rangle$ 的本征值为 $n-1$,故
\begin{equation}
	a | n \rangle = C_{n} | n-1 \rangle.
\end{equation}

根据
\begin{equation}
	\langle n | a^\dag a | n \rangle = | C_{n} |^{2} = n,
\end{equation}

得到 $C_{n} = \sqrt{n}$,即
\begin{equation}
	a | n \rangle = \sqrt{n} | n-1 \rangle.
\end{equation}

以及
\begin{align*}
	N a^\dag | n \rangle &= (a^\dag N + a^\dag) | n \rangle \\
	&= (n+1) a^\dag | n \rangle.
\end{align*}

故
\begin{equation}
	a^\dag | n \rangle = C_{n+1} | n+1 \rangle.
\end{equation}

而 $a a^\dag = N + 1$,因此
\begin{equation}
	\langle n | a a^\dag | n \rangle = | C_{n+1} |^{2} = n+1.
\end{equation}

于是有 $C_{n+1} = \sqrt{n+1}$,即
\begin{equation}
	a^\dag | n \rangle = \sqrt{n+1} | n+1 \rangle.
\end{equation}

综合有
\[
\boxed{
	\begin{aligned}
		a| n \rangle &= \sqrt{n}| n-1 \rangle, \\
		a^\dagger| n \rangle &= \sqrt{n+1}| n+1 \rangle.
	\end{aligned}
}
\]

以及一个非常重要的性质:
\[
	\boxed{
		\begin{array}{c}
			\text{If } \psi_{n} \text{ is an eigenfunction of } N \text{ with eigenvalue } n,\\[6pt]
			\text{then } a^\dag \psi_{n} \text{ also is an eigenfunction of } N \text{ with eigenvalue } (n + 1),\\[6pt]
			\text{then } a\psi_{n} \text{ also is an eigenfunction of } N \text{ with eigenvalue } (n-1).
		\end{array}
	}
\]

\begin{itemize}
	\item 由 $a|\psi_{n}\rangle = \sqrt{n}\,|n-1\rangle$, 且$n \ge 0 $, 若 $n$ 为非整数(如 $n=0.5$),则 $a|0.5\rangle = \sqrt{0.5}\,|-0.5\rangle$,但 $|-0.5\rangle$ 无物理意义(因为它的本征值是``-0.5",我们已经证明了 $N$ 所有的本征值不能为负),对于大于1的非整数你可以利用 $a$ 推导到小于1的部分,然后证伪
	\item 从 $a | n \rangle = \sqrt{n} | n-1 \rangle, a|0\rangle = 0$ 可知 $n=0$ 是最小本征值,且 $a^{\dagger}$ 提升量子数,故 $n=0,1,2,\dots$。
\end{itemize}

从 $a | 0 \rangle = 0$ 出发,根据 $\displaystyle a = \frac{1}{\sqrt{2m \hbar \omega}} (m \omega x + i p)$,得到
\begin{equation}
	\frac{m \omega x}{\hbar} \psi_{0} + \frac{d \psi_{0}}{dx} = 0
\end{equation}
\begin{equation}
	\int {\frac{{d{\psi _0}}}{{{\psi _0}}}}  =  - \int d x\frac{{m\omega x}}{\hbar } \Rightarrow {\psi _0} = \lambda \exp \left( { - \frac{{m\omega {x^2}}}{{2\hbar }}} \right)
\end{equation}


利用高斯积分公式
\begin{equation}
	\int_{-\infty}^{\infty} e^{-\alpha x^{2}} dx = \sqrt{\frac{\pi}{\alpha}}.
\end{equation}

由归一化可得
\begin{equation}
	\lambda = \left( \frac{m \omega}{\pi \hbar} \right)^{1/4}.
\end{equation}

从而有
\begin{equation}
	\psi_{0} = \left( \frac{m \omega}{\pi \hbar} \right)^{1/4} \exp \left( -\frac{m \omega x^{2}}{2 \hbar} \right).
\end{equation}

通过 $\displaystyle a^\dag| n \rangle = \sqrt{n+1}\,| n+1 \rangle$ 我们可以得到所有的本征态:
\begin{align}
	\psi_{n}(x) &= \frac{1}{\sqrt{n !}}\left(a^{\dagger}\right)^{n}\psi_{0}(x) \notag \\
	&= \frac{1}{\sqrt{n !}} \cdot \left(\frac{m \omega}{\pi \hbar}\right)^{\frac{1}{4}} \left(a^{\dagger}\right)^{n} \exp \left(-\frac{m \omega}{2 \hbar} x^{2}\right) \notag \\
	&= \left( \frac{m\omega}{\pi\hbar} \right)^{1/4} \frac{1}{\sqrt{2^{n} n!}} H_{n}(\xi) e^{-\xi^{2}/2}, \quad \xi={\sqrt {\frac{{m\omega }}{\hbar }} x}.
\end{align}

注意到只需要做替换:$\displaystyle (a^\dag)^n \to (\frac{1}{{\sqrt 2 }})^n{H_n}\left( \xi  \right) $

\subsection{Matrix Representation of the Harmonic Oscillator}

Starting from the eigenfunctions of the harmonic oscillator ${\psi_{n}(x)}$, which form a complete orthonormal system of the corresponding Hilbert space, we change our notation and label the states only by their index number $n$, getting the following vectors:
\begin{equation}
	\psi_{0} \to\begin{pmatrix}1 \\ 0 \\ 0 \\ \vdots\end{pmatrix}=\ket{0}, \quad \psi_{1} \to\begin{pmatrix}0 \\ 1 \\ 0 \\ \vdots \end{pmatrix}=\ket{1}, \quad \psi_{n} \underset{n\text{-th row}}{\to}\begin{pmatrix}\vdots \\ 0 \\ 1 \\ 0 \\ \vdots \end{pmatrix}=\ket{n} .
\end{equation}

The vectors $\ket{n}$ now also form a complete orthonormal system, but one of the so called Fock space or occupation number space. We will not give a mathematical definition of this space but will just note here, that we can apply our rules and calculations as before on the Hilbert space. The ladder operators however have a special role in this space as they allow us to construct any vector from the ground state.

Let us state the most important relations here:
\begin{equation}
	\braket{n}{m} = \delta_{n,m} \label{5.57}
\end{equation}

\begin{equation}
	\ket{n}=\frac{1}{\sqrt{n !}}\left(a^{\dagger}\right)^{n}\ket{0} \label{5.58}
\end{equation}

\begin{equation}
	a^{\dagger}\ket{n}=\sqrt{n+1}\ket{n+1}.
\end{equation}

\begin{equation}
	a\ket{n}=\sqrt{n}\ket{n-1}.
\end{equation}

\begin{equation}
	N\ket{n}=a^{\dagger}a\ket{n}=n\ket{n}.
\end{equation}

\begin{equation}
	H\ket{n}=\hbar\omega\left(N+\frac{1}{2}\right)\ket{n}=\hbar\omega\left(n+\frac{1}{2}\right)\ket{n}.
\end{equation}

With this knowledge we can write down matrix elements as transition amplitudes. We could, for example, consider Eq. (\ref{5.57}) as the matrix element from the $n$-th row and the $m$-th column:
\begin{equation}
	\braket{n}{m} = \delta_{n,m} \leftrightarrow\begin{pmatrix} 1 & 0 & \cdots \\ 0 & 1 & \\ \vdots & & \ddots \end{pmatrix} .
\end{equation}

In total analogy we can get the matrix representation of the creation operator:
\begin{equation}
	\braket{k|a^\dagger|n}=\sqrt{n+1} \underbrace{\braket{k}{n+1}}_{\delta_{k,n+1}} \leftrightarrow a^{\dagger}=\begin{pmatrix} 0 & 0 & 0 & \cdots \\ \sqrt{1} & 0 & 0 & \\ 0 & \sqrt{2} & 0 & \\ \vdots & & \sqrt{3} & \\ & & & \ddots \end{pmatrix}.
\end{equation}

As well as of the annihilation operator, occupation number operator and Hamiltonian:
\begin{equation}
	\braket{k|a|n}=\sqrt{n} \underbrace{\braket{k}{n-1}}_{\delta_{k,n-1}} \leftrightarrow a=\begin{pmatrix}0 & \sqrt{1} & 0 & \cdots & \\ 0 & 0 & \sqrt{2} & & \\ 0 & 0 & 0 & \sqrt{3} & \\ \vdots & & & & \ddots \end{pmatrix}.
\end{equation}

\begin{equation}
	\braket{k|N|n}=n \underbrace{\braket{k}{n}}_{\delta_{k,n}} \leftrightarrow N=\begin{pmatrix}0 & 0 & 0 & \cdots & \\ 0 & 1 & 0 & & \\ 0 & 0 & 2 & & \\ \vdots & & & 3 & \\ & & & & \ddots\end{pmatrix}.
\end{equation}

\begin{equation}
	\braket{k|H|n}=\hbar\omega\left(n+\frac{1}{2}\right) \underbrace{\braket{k}{n}}_{\delta_{k,n}} \leftrightarrow H=\hbar\omega\begin{pmatrix} \frac{1}{2} & 0 & 0 & \cdots & \\ 0 & \frac{3}{2} & 0 & & \\ 0 & 0 & \frac{5}{2} & & \\ \vdots & & & \frac{7}{2} & \\ & & & & \ddots \end{pmatrix} .
\end{equation}
\subsection{三维谐振子}
	\begin{problembox}{三维谐振子}
		考虑三维谐振子,其势函数为
		\[
		V(r) = \frac{1}{2} m \omega^{2} r^{2}.
		\]
		
		\begin{enumerate}
			\item[(a)] 证明:通过在笛卡儿坐标系中分离变量可以得到三个一维谐振子,并利用所学知识给出能量允许值。
			
			\item[(b)] 确定 \(E_{n}\) 的简并度 \(d(n)\).
		\end{enumerate}
	\end{problembox}
	
	
	\subsection*{解答}
	
	\subsection*{(a) 变量分离与能量本征值}
	
	三维谐振子的定态薛定谔方程为:
	\[
	-\frac{\hbar^{2}}{2m} \nabla^{2}\psi + \frac{1}{2}m\omega^{2}r^{2}\psi = E\psi,
	\]
	其中在笛卡尔坐标系下,拉普拉斯算符表示为:
	\[
	\nabla^{2} = \frac{\partial^{2}}{\partial x^{2}} + \frac{\partial^{2}}{\partial y^{2}} + \frac{\partial^{2}}{\partial z^{2}}.
	\]
	
	采用分离变量法,设波函数为:
	\[
	\psi(x,y,z) = X(x)Y(y)Z(z).
	\]
	代入薛定谔方程,两边同时除以\(\psi = XYZ\),得到:
	\begin{align*}
		&-\frac{\hbar^{2}}{2m} \left( \frac{1}{X}\frac{d^{2}X}{dx^{2}} + \frac{1}{Y}\frac{d^{2}Y}{dy^{2}} + \frac{1}{Z}\frac{d^{2}Z}{dz^{2}} \right) + \frac{1}{2}m\omega^{2}(x^{2} + y^{2} + z^{2}) = E.
	\end{align*}
	
	重组项后可得:
	\begin{align}
		\left[ -\frac{\hbar^{2}}{2m} \frac{1}{X} \frac{d^{2}X}{dx^{2}} + \frac{1}{2}m\omega^{2}x^{2} \right] 
		+ \left[ -\frac{\hbar^{2}}{2m} \frac{1}{Y} \frac{d^{2}Y}{dy^{2}} + \frac{1}{2}m\omega^{2}y^{2} \right] \nonumber \\
		+ \left[ -\frac{\hbar^{2}}{2m} \frac{1}{Z} \frac{d^{2}Z}{dz^{2}} + \frac{1}{2}m\omega^{2}z^{2} \right] = E. \label{eq:separated}
	\end{align}
	
	由于方程左边三项分别只依赖于单一变量\(x\)、\(y\)、\(z\),而右边为常数,因此每一项必须等于常数:
	\begin{align*}
		E_x + E_y + E_z = E.
	\end{align*}
	
	由此得到三个独立的一维谐振子方程:
	\begin{align}
		-\frac{\hbar^{2}}{2m} \frac{d^{2}X}{dx^{2}} + \frac{1}{2}m\omega^{2}x^{2}X &= E_x X, \label{eq:x_eq} \\
		-\frac{\hbar^{2}}{2m} \frac{d^{2}Y}{dy^{2}} + \frac{1}{2}m\omega^{2}y^{2}Y &= E_y Y, \label{eq:y_eq} \\
		-\frac{\hbar^{2}}{2m} \frac{d^{2}Z}{dz^{2}} + \frac{1}{2}m\omega^{2}z^{2}Z &= E_z Z. \label{eq:z_eq}
	\end{align}
	
	每个一维谐振子的能量本征值为:
	\[
	E_x = \left(n_x + \frac{1}{2}\right)\hbar\omega, \quad 
	E_y = \left(n_y + \frac{1}{2}\right)\hbar\omega, \quad 
	E_z = \left(n_z + \frac{1}{2}\right)\hbar\omega,
	\]
	其中 \(n_x, n_y, n_z = 0, 1, 2, \dots\)
	
	因此,三维谐振子的总能量为:
	\[
	E = E_x + E_y + E_z = \left(n_x + n_y + n_z + \frac{3}{2}\right)\hbar\omega = \left(n + \frac{3}{2}\right)\hbar\omega,
	\]
	其中 \(n = n_x + n_y + n_z\) 为非负整数。
	
	\subsection*{(b) 能级简并度的确定}
	
	对于给定的总量子数 \(n\),简并度 \(d(n)\) 等于方程
	\[
	n_x + n_y + n_z = n
	\]
	的非负整数解的个数。
	
	固定 \(n_x = k\)(其中 \(k = 0, 1, \dots, n\)),则 \(n_y + n_z = n - k\)。此时 \(n_y\) 可以取 \(0, 1, \dots, n-k\),共 \((n - k + 1)\) 个值,而 \(n_z\) 随之确定。
	
	因此,总简并度为:
	\[
	d(n) = \sum_{k=0}^{n} (n - k + 1) = \sum_{j=1}^{n+1} j = \frac{(n+1)(n+2)}{2}.
	\]
	
	方程 \(n_x + n_y + n_z = n\)(其中 \(n\) 为固定非负整数)的非负整数解 \((n_x, n_y, n_z)\) 的个数,等价于从 \(n\) 个不可区分的物品和 \(2\) 个隔板中排列所有物品和隔板的组合数,其计算公式如下:
	
	\[
	\binom{n+2}{2} = \frac{(n+2)(n+1)}{2}
	\]
	
	因此,解的总数为 \(\boxed{\dfrac{(n+2)(n+1)}{2}}\)
	\subsection{相干态}
	
	\begin{problembox}{谐振子的相干态}
		在谐振子定态中\(\displaystyle {\psi _n} = \frac{1}{{\sqrt {n!} }}{\left( {{{\hat a}_ + }} \right)^n}{\psi _0}\),仅当 $n=0$ 时的态符合不确定性原理的极限 $(\sigma_{x}\sigma_{p}=\hbar/2)$;一般情况下有 $\sigma_{x}\sigma_{p}=(2n+1)\hbar/2$。但是,某些线性叠加(所谓的相干态)也会减小不确定度的乘积。它们是湮灭算符的本征函数:
		\[
		\hat{a}\ket{\alpha}=\alpha\ket{\alpha},
		\]
		(其中本征值 $\alpha$ 可以是任何复数。)
		\begin{enumerate}
			\item[(a)] 对态 $\ket{\alpha}$ 计算 $\expval{x}$, $\expval{x^{2}}$, $\expval{p}$, $\expval{p^{2}}$. 提示:注意 $\hat{a}^\dagger$ 是 $\hat{a}$ 的厄米共轭算符. 不要假定 $\alpha$ 是实数。
			\item[(b)] 求 $\sigma_{x}$ 和 $\sigma_{p}$;证明:$\sigma_{x}\sigma_{p}=\hbar/2$。
			\item[(c)] 像任何其他的波函数一样,相干态可以利用能量本征态展开:
			\[
			\ket{\alpha}=\sum_{n=0}^{\infty}c_{n}\ket{n}.
			\]
			证明:展开系数是
			\[
			c_{n}=\frac{\alpha^{n}}{\sqrt{n!}}c_{0}.
			\]
			\item[(d)] 通过归一化 $\ket{\alpha}$ 确定 $c_{0}$.
			\item[(e)] 引入时间因子
			\[
			\ket{n}\rightarrow e^{-iE_{n}t/\hbar}\ket{n},
			\]
			证明:$\ket{\alpha(t)}$ 仍然是 $\hat{a}$ 的本征态,但本征值是随时间演化的,即
			\[
			\alpha(t)=e^{-i\omega t}\alpha.
			\]
			因此一个相干态将维持相干,并继续减小不确定积。
			\item[(f)] 基于(a),(b)和(e)中得到的结果,求出作为时间函数的 $\expval{x}$ 和 $\sigma_{x}$。
			
			如果把复数 $\alpha$ 写成如下形式将会大有帮助:
			\[
			\alpha=C\sqrt{\frac{m\omega}{2\hbar}}e^{i\phi},
			\]
			其中 $C$ 和 $\phi$ 是实数。注:在某种意义上,相干态的行为是准经典的。
			\item[(g)] 基态 $(\ket{n=0})$ 本身是相干态吗?如果是,它的本征值是什么?
		\end{enumerate}
		(注:升阶算符没有可归一化的本征函数。)
	\end{problembox}
	
	\subsection*{解答}
	\begin{enumerate}
		\item[(a)] 因为 $\hat{a}^\dagger$ 是 $\hat{a}$ 的厄米共轭算符,所以
		\begin{align*}
			\expval{x} &= \mel{\alpha}{x}{\alpha} = \mel{\alpha}{\sqrt{\frac{\hbar}{2m\omega}}(\hat{a}^\dagger + \hat{a})}{\alpha} \\
			&= \sqrt{\frac{\hbar}{2m\omega}} \left( \mel{\alpha}{\hat{a}^\dagger}{\alpha} + \mel{\alpha}{\hat{a}}{\alpha} \right) \\
			&= \sqrt{\frac{\hbar}{2m\omega}} \left( \bra{\alpha}\hat{a}^\dagger\ket{\alpha} + \bra{\alpha}\hat{a}\ket{\alpha} \right) \\
			&= \sqrt{\frac{\hbar}{2m\omega}} \left( \bra{\hat{a}\alpha}\ket{\alpha} + \alpha \right) \quad \text{(因为 } \hat{a}\ket{\alpha}=\alpha\ket{\alpha} \text{)} \\
			&= \sqrt{\frac{\hbar}{2m\omega}} \left( \alpha^* + \alpha \right) \\
			&= \sqrt{\frac{2\hbar}{m\omega}} \operatorname{Re}(\alpha), \\
			\expval{x^{2}} &= \mel{\alpha}{x^2}{\alpha} \\
			&= \frac{\hbar}{2m\omega} \mel{\alpha}{(\hat{a}^\dagger + \hat{a})^2}{\alpha} \\
			&= \frac{\hbar}{2m\omega} \mel{\alpha}{\hat{a}^\dagger\hat{a}^\dagger + \hat{a}^\dagger\hat{a} + \hat{a}\hat{a}^\dagger + \hat{a}\hat{a}}{\alpha} \\
			&= \frac{\hbar}{2m\omega} \left( \mel{\alpha}{\hat{a}^\dagger\hat{a}^\dagger}{\alpha} + \mel{\alpha}{\hat{a}^\dagger\hat{a}}{\alpha} + \mel{\alpha}{\hat{a}\hat{a}^\dagger}{\alpha} + \mel{\alpha}{\hat{a}\hat{a}}{\alpha} \right) \\
			&= \frac{\hbar}{2m\omega} \left( \alpha^{*2} + \alpha^{*}\alpha + \mel{\alpha}{\hat{a}\hat{a}^\dagger + 1}{\alpha} + \alpha\alpha \right) \quad ([\hat{a},\hat{a}^\dagger]=1)\\
			&= \frac{\hbar}{2m\omega} \left( \alpha^{*}\alpha^{*} + \alpha\alpha + 2\alpha^{*}\alpha + 1 \right) \\
			&= \frac{\hbar}{2m\omega} \left[ (\alpha^{*} + \alpha)^2 + 1 \right] \\
			&= \frac{\hbar}{2m\omega} \left[ 4\operatorname{Re}^2(\alpha) + 1 \right].
		\end{align*}
		类似地,
		\begin{align*}
			\expval{p} &= \mel{\alpha}{p}{\alpha} = \mel{\alpha}{ i\sqrt{\frac{\hbar m\omega}{2}}(\hat{a}^\dagger - \hat{a}) }{\alpha} \\
			&= i\sqrt{\frac{\hbar m\omega}{2}} \left( \mel{\alpha}{\hat{a}^\dagger}{\alpha} - \mel{\alpha}{\hat{a}}{\alpha} \right) \\
			&= i\sqrt{\frac{\hbar m\omega}{2}} \left( \alpha^* - \alpha \right) \\
			&= \sqrt{2\hbar m\omega} \operatorname{Im}(\alpha), \\
			\expval{p^{2}} &= \mel{\alpha}{p^2}{\alpha} \\
			&= -\frac{\hbar m\omega}{2} \mel{\alpha}{(\hat{a}^\dagger - \hat{a})^2}{\alpha} \\
			&= -\frac{\hbar m\omega}{2} \mel{\alpha}{\hat{a}^\dagger\hat{a}^\dagger - \hat{a}^\dagger\hat{a} - \hat{a}\hat{a}^\dagger + \hat{a}\hat{a}}{\alpha} \\
			&= -\frac{\hbar m\omega}{2} \left( \alpha^{*}\alpha^{*} - \alpha^{*}\alpha - \mel{\alpha}{\hat{a}^\dagger\hat{a} + 1}{\alpha} + \alpha\alpha \right) \\
			&=- \frac{\hbar m\omega}{2} \left( \alpha^{*}\alpha^{*} + \alpha\alpha - 2\alpha^{*}\alpha - 1 \right) \\
			&= -\frac{\hbar m\omega}{2} \left[ (\alpha^{*} - \alpha)^2 - 1 \right] \\
			&= \frac{\hbar m\omega}{2} \left[ 4\operatorname{Im}^2(\alpha) + 1 \right].
		\end{align*}
		
		\item[(b)]
		\begin{align*}
			\sigma_{x} &= \sqrt{\expval{x^{2}} - \expval{x}^{2}} \\
			&= \sqrt{\frac{\hbar}{2m\omega}\left[4\operatorname{Re}^{2}(\alpha) + 1\right] - \frac{2\hbar}{m\omega}\operatorname{Re}^{2}(\alpha)} \\
			&= \sqrt{\frac{\hbar}{2m\omega}}, \\
			\sigma_{p} &= \sqrt{\expval{p^{2}} - \expval{p}^{2}} \\
			&= \sqrt{\frac{\hbar m\omega}{2}\left[4\operatorname{Im}^{2}(\alpha) + 1\right] - 2\hbar m\omega \operatorname{Im}^{2}(\alpha)} \\
			&= \sqrt{\frac{\hbar m\omega}{2}},
		\end{align*}
		所以,
		\[
		\sigma_{x} \sigma_{p} = \sqrt{\frac{\hbar}{2m\omega}} \sqrt{\frac{\hbar m\omega}{2}} = \frac{\hbar}{2}.
		\]
		
		\item[(c)] 由
		\[
		\ket{\alpha} = \sum_{n=0}^{\infty} c_{n}\ket{n}, \quad c_{n} = \bra{n}\ket{\alpha}, \quad \ket{n} = \frac{1}{\sqrt{n!}}\left(\hat{a}^\dagger\right)^{n}\ket{0},
		\]
		所以,
		\[
		c_{n} = \bra{n}\ket{\alpha} = \frac{1}{\sqrt{n!}} \bra{0} \hat{a}^{n} \ket{\alpha} = \frac{\alpha^{n}}{\sqrt{n!}} \bra{0}\ket{\alpha} = \frac{\alpha^{n}}{\sqrt{n!}} c_{0}.
		\]
		
		\item[(d)] 归一化:
		\begin{align*}
			1 &= \bra{\alpha}\ket{\alpha} = \sum_{n=0}^{\infty} |c_{n}|^{2} = \sum_{n=0}^{\infty} \frac{1}{n!} \alpha^{*n} c_{0}^{*} \alpha^{n} c_{0} = |c_{0}|^{2} \sum_{n=0}^{\infty} \frac{\left(|\alpha|^{2}\right)^{n}}{n!} = |c_{0}|^{2} e^{|\alpha|^{2}},
		\end{align*}
		所以
		\[
		c_{0} = e^{-|\alpha|^{2}/2} \quad\text{不考虑任意相因子 $e^{i\delta}$}
		\]
		
		
		\item[(e)]
		\begin{align*}
			\hat{a}\ket{\alpha(t)} &= \sum_{n=0}^{\infty} \frac{\alpha^{n}}{\sqrt{n!}} c_{0} e^{-iE_{n}t/\hbar} \hat{a}\ket{n} \\
			&= \sum_{n=1}^{\infty} \frac{\alpha^{n}}{\sqrt{(n-1)!}} c_{0} e^{-iE_{n}t/\hbar} \ket{n-1} \quad (\hat a\ket{n}=\sqrt{n}\ket{n-1})\\
			&= \sum_{n=0}^{\infty} \frac{\alpha^{n+1}}{\sqrt{n!}} c_{0} e^{-iE_{n+1}t/\hbar} \ket{n} \\
			&= \alpha e^{-i\omega t} \sum_{n=0}^{\infty} \frac{\alpha^{n}}{\sqrt{n!}} c_{0} e^{-iE_{n}t/\hbar} \ket{n} \\
			&= \alpha e^{-i\omega t} \ket{\alpha(t)}.
		\end{align*}
		其中用到了 $E_{n+1} = E_{n} + \hbar\omega$。所以,$\ket{\alpha(t)}$ 仍然是 $\hat{a}$ 的本征态,其本征值为 $\alpha e^{-i\omega t}$。
		
		\item[(f)] 由(a)知
		\[
		\expval{x} = \sqrt{\frac{\hbar}{2m\omega}} \left[ \alpha^{*}(t) + \alpha(t) \right],
		\]
		由(e)知
		\[
		\alpha(t) = e^{-i\omega t} \alpha,
		\]
		因此,
		\begin{align*}
			\expval{x} &= \sqrt{\frac{\hbar}{2m\omega}} \left[ \alpha^{*} e^{i\omega t} + \alpha e^{-i\omega t} \right] \\
			&= \sqrt{\frac{\hbar}{2m\omega}} \left( C\sqrt{\frac{m\omega}{2\hbar}} e^{i\phi} e^{-i\omega t} + C\sqrt{\frac{m\omega}{2\hbar}} e^{-i\phi} e^{i\omega t} \right) \\
			&= \frac{C}{2} \left[ e^{-i(\omega t - \phi)} + e^{i(\omega t - \phi)} \right] \\
			&= C \cos(\omega t - \phi).
		\end{align*}
		由(b)知,$\sigma_{x} = \sqrt{\frac{\hbar}{2m\omega}}$,与时间无关。坐标 $\expval{x}$ 以经典频率振动。
		
		\item[(g)] 因为 $\hat{a}\ket{0} = 0\ket{0}$,故基态 $\ket{0}$ 是 $\hat{a}$ 的本征值为 $0$ 的本征态,所以是相干态。
	\end{enumerate}
\newpage
\section{对称性理论}
对称性原理指出:当我们以特定方式改变视角时,自然规律保持不变。例如,移动或旋转实验室不应改变实验室中观测到的自然规律。这种改变视角的特定方式称为对称变换。该定义并不意味着对称变换不改变物理态,而是指变换后的新态与原态满足相同的自然规律。\par
\textbf{特别地,对称变换必须不改变跃迁概率。}若系统处于归一化希尔伯特空间向量$\ket{\Psi}$描述的态,且测量(例如测量一组对易厄米算符表示的可观测量)将系统置于标准正交态组$\ket{\Phi_i}$中的某一态,则系统处于$\ket{\Phi_i}$的概率由 The coefficient of $\ket{\Psi}$ 决定
\begin{equation}
	{P_{\left| \Psi  \right\rangle  \to \left| {{\Phi _i}} \right\rangle }} = |\left\langle {{{\Phi _i}}}
	\mathrel{\left | {\vphantom {{{\Phi _i}} \Psi }}
		\right. \kern-\nulldelimiterspace}
	{\Psi } \right\rangle {|^2}.
\end{equation}
请注意,这是任何形式的“对称变换”都需要满足的基本条件,或称“第一性原理”\par
\subsection{对称变换}
对称变换必须保持所有$|\left\langle {{{\Phi _i}}}
\mathrel{\left | {\vphantom {{{\Phi _i}} \Psi }}
	\right. \kern-\nulldelimiterspace}
{\Psi } \right\rangle {|^2}$不变。满足该条件的一种方式是:对称变换将任意态向量$\Psi$映射为$U\Psi$,其中$U$是满足幺正性的线性算符——对任意两个态向量$\Phi$和$\Psi$,
\begin{equation}
	\left\langle {{U{\Phi _i}}}
	\mathrel{\left | {\vphantom {{U{\Phi _i}} {U\Psi }}}
		\right. \kern-\nulldelimiterspace}
	{{U\Psi }} \right\rangle  = \left\langle {{{\Phi _i}}}
	\mathrel{\left | {\vphantom {{{\Phi _i}} \Psi }}
		\right. \kern-\nulldelimiterspace}
	{\Psi } \right\rangle \label{eq:3.4.2}
\end{equation}
\par
Now, we have a way in which symmetry transformations can act on physical states.But is this the only way that symmetry transformations can act on physical states?
\par
答案当然是否定的,我们需要接受一个这样的事实:一个量子系统所有可能量子态的集合构成希尔伯特空间。在物理上,我们常常默认量子态已经归一化了,但即使对于已经归一化的量子态 $|\psi\rangle$,它其实也并不唯一确定,因为根据态叠加原理,$e^{i\theta}|\psi\rangle$ 和 $|\psi\rangle$ 描述的是同一个量子态!所以,严格来说,一个量子态其实是希尔伯特空间归一化矢量的一个等价类,满足 $e^{i\theta}|\psi\rangle \sim |\psi\rangle$,其中 "$\sim$" 表示等价关系。\par
量子力学中的物理态不由希尔伯特空间中的特定归一化向量表示,but by a "Ray", the whole class of normalized state vectors that differ from one another only by phase factors, numerical factors with modulus unity。我们无权假设对称变换必须将希尔伯特空间中的任意向量映射为另一确定向量,仅能要求对称变换将"Ray"映射为"Ray":即作用于表示某物理态的“相差相位因子的归一化向量组”,得到表示另一物理态的“相差相位因子的归一化向量组”。\par
根据量子力学的玻恩规则,对于处在 $|\psi\rangle$ 态上的系统,我们在 $|\phi\rangle$ 态上测到它的概率为 $|\langle\phi|\psi\rangle|^{2}$,其中两个量子态的内积 $\langle\phi|\psi\rangle$ 也称之为概率幅,因为它的模方给出了概率。

在量子力学中,概率(而不是量子态)才是我们在物理上真正测量的东西。

Eugene Wigner的基本定理指出,对称变换方式仅有两种:一种是选择适当相位,使得幺正变换 $U$ 对任意态向量 $\Psi$ 的作用为$\Psi\mapsto U\Psi$($U$是满足(\ref{eq:3.4.2})式的线性幺正算符);另一种是$\Theta$为\textbf{反幺正算符}(antiunitary operator),即
\begin{align}
	\Theta(\alpha\Psi + \alpha'\Psi') &= \alpha^* \Theta\Psi + \alpha'^* \Theta\Psi' \\
	\langle \Theta\Phi | \Theta\Psi \rangle &= \langle \Phi | \Psi \rangle^*\label{eq:3.4.10}
\end{align}

(注意反幺正算符不可能是线性的——否则$\alpha \Theta\Psi=(\Theta\alpha\Psi)=\Theta(\alpha\Psi)=\alpha^*\Theta\Psi$,这对复数$\alpha$不成立。)反幺正算符的伴随算符定义为
\begin{equation}
\left\langle {{{\Theta^\dag }\Phi }}
\mathrel{\left | {\vphantom {{{U^\dag }\Phi } \Psi }}
	\right. \kern-\nulldelimiterspace}
{\Psi } \right\rangle  = {\left\langle {\Phi }
	\mathrel{\left | {\vphantom {\Phi  {\Theta\Psi }}}
		\right. \kern-\nulldelimiterspace}
	{{\Theta\Psi }} \right\rangle ^*} \label{eq:3.4.10a}
\end{equation}
根据 (\ref{eq:3.4.10})和(\ref{eq:3.4.10a})我们可以得到:
$\langle \Theta\Phi |\Theta\Psi \rangle  = {\langle \Phi |\Psi \rangle ^*}
\Leftrightarrow {\langle \Phi |{\Theta^\dag }\Theta\Psi \rangle ^*} = {\left\langle {\Phi }
	\mathrel{\left | {\vphantom {\Phi  \Psi }}
		\right. \kern-\nulldelimiterspace}
	{\Psi } \right\rangle ^*}$,
即 ${\Theta^\dag }\Theta = \hat I$,所以所谓的反幺正算符其实就是满足和幺正算符类似的 
$\Theta^{\dagger} = \Theta^{-1}$ 的反线性算符。\par
很显然,幺正变换乘以幺正变换,结果必定还是幺正变换,而反幺正变换乘以反幺正变换结果也是幺正变换,只有幺正变换乘以反幺正变换结果才可能是反幺正变换。正因为如此,任何反幺正变换总是能写成某个幺正变换和另一个反幺正变换的乘积。
\par
由反线性反幺正算符表示的对称性均涉及时间流向的反转。\textbf{我们先主要讨论由线性幺正算符表示的对称性。}\par
单位算符$\hat I$表示平凡对称性(不改变态向量),显然是线性幺正算符。若$U_1$和$U_2$均表示对称变换,则$U_1U_2$也表示对称变换。该性质(结合逆变换的存在性和平凡变换$\hat I$)意味着:所有表示对称变换的算符构成一个群。\par
现在考虑无穷小变换——$U$可任意接近$\hat I$。这类对称算符可方便地表示为
\begin{equation}
U_\epsilon=\hat I+i\epsilon T+O(\epsilon^2), \label{eq:3.4.11} 
\end{equation}
其中$\epsilon$是任意实无穷小量,$T$是与$\epsilon$无关的算符。幺正性条件要求
\begin{equation}
(\hat I-i\epsilon T^\dagger+O(\epsilon^2))(\hat I+i\epsilon T+O(\epsilon^2))=\hat I,
\end{equation}
对$\epsilon$取一阶近似,可得
\begin{equation}
T=T^\dagger.
\end{equation}
因此,无穷小对称性自然导致\textbf{厄米算符}的出现。令$\epsilon=\theta/N$($\theta$是与$N$无关的有限参数),将对称变换重复$N$次并令$N\to\infty$,得到变换算符
\begin{equation}
U(\theta)=\lim_{N\to\infty}\left(\hat I+i\frac{\theta }{N}T\right)^N=\exp(i\theta T).\label{eq:3.4.13} 
\end{equation}
式中, $T$称为对称操作的生成元。如我们将看到的,量子力学中表示可观测量的算符大多(若非全部)是对称性的生成元:例如,总动量是空间坐标平移的生成元;哈密顿量是时间平移的生成元;总角动量是空间旋转的生成元。 \par
故事还没有结束,还有另外精彩的一面。在对称变换$\Psi\mapsto U\Psi$下,任意可观测量$A$的期望值变换为
\begin{equation}
\langle\Psi|A|\Psi\rangle=\langle A\rangle_{\Psi}\mapsto\langle U\Psi|A| U\Psi\rangle=\left\langle\Psi\left|U^{\dagger} AU\right|\Psi\right\rangle=\left\langle U^{\dagger} AU\right\rangle_{\Psi}
\end{equation}
因此,可通过将可观测量变换为$A\mapsto U^\dagger A U$,得到期望值的变换性质。这类变换称为\textbf{相似变换}。注意相似变换保持代数关系:
\begin{equation}
U^\dagger AU \cdot U^\dagger BU=U^\dagger(AB)U, \quad U^\dagger AU+U^\dagger BU=U^\dagger(A+B)U.
\end{equation}
此外,相似变换不改变算符的本征值——若$\Psi$是$A$的本征向量(本征值$a$),则$U^\dagger\Psi$是$U^\dagger AU$的本征向量(本征值$a$)。当$U$取式(\ref{eq:3.4.11})的无穷小形式时,任意算符$A$的变换为
\begin{equation}
A\mapsto A-i\epsilon[T, A]+O(\epsilon^2)
\end{equation}
因此,无穷小对称变换对任意算符的作用,由对称性生成元与该算符的对易子描述。当算符$A$本身是对称性生成元时,这一点尤为重要——此时对易子反映了对称群的性质。
\subsection{Unitary Transform}

A \textbf{unitary transform} is a linear transformation represented by a \textbf{unitary matrix} $\mathbf{U}$ that \textbf{preserves the inner product structure} of a vector space.

\subsubsection*{Mathematical Definition}

A matrix $\mathbf{U} \in \mathbb{C}^{n \times n}$ is unitary if it satisfies:

\[
\mathbf{U}^* \mathbf{U} = \mathbf{U} \mathbf{U}^* = \mathbf{I}
\]

where:
\begin{itemize}
	\item $\mathbf{U}^*$ is the conjugate transpose (adjoint) of $\mathbf{U}$
	\item $\mathbf{I}$ is the identity matrix
\end{itemize}

Equivalently, the columns (and rows) of $\mathbf{U}$ form an orthonormal basis:

\[
\langle \mathbf{u}_i, \mathbf{u}_j \rangle = \mathbf{u}_i^* \mathbf{u}_j = \delta_{ij}
\]

where $\delta_{ij}$ is the Kronecker delta.

\subsubsection*{Key Properties}

\begin{enumerate}
	\item \textbf{Norm Preservation:} $\|\mathbf{Ux}\| = \|\mathbf{x}\|$ for all $\mathbf{x}$
	\item \textbf{Inner Product Preservation:} $\langle \mathbf{Ux}, \mathbf{Uy} \rangle = \langle \mathbf{x}, \mathbf{y} \rangle$
	\item \textbf{Reversibility:} $\mathbf{U}^{-1} = \mathbf{U}^*$ exists
	\item \textbf{Eigenvalue Property:} All eigenvalues satisfy $|\lambda| = 1$
\end{enumerate}

\subsubsection*{Important Examples}

\begin{itemize}
	\item \textbf{Fourier Transform:} $F(k) = \frac{1}{\sqrt{N}} \sum_{n=0}^{N-1} f(n) e^{-2\pi i kn/N}$
	\item \textbf{Quantum Gates:} Pauli matrices, Hadamard gate $H = \frac{1}{\sqrt{2}} \begin{bmatrix} 1 & 1 \\ 1 & -1 \end{bmatrix}$
	\item \textbf{Rotation Matrices:} $R(\theta) = \begin{bmatrix} \cos\theta & -\sin\theta \\ \sin\theta & \cos\theta \end{bmatrix}$
\end{itemize}

\subsubsection*{Geometric Interpretation}

Unitary transformations preserve:
\begin{itemize}
	\item Lengths of vectors
	\item Angles between vectors
	\item Orthogonality relationships
\end{itemize}

In quantum mechanics, unitary transforms describe reversible evolution that preserves probability amplitudes:

\[
|\psi'\rangle = \mathbf{U}|\psi\rangle \quad \text{with} \quad \langle\psi'|\psi'\rangle = \langle\psi|\psi\rangle
\]
\subsection{空间平移对称性}
作为具有重要物理意义的对称变换示例,我们考虑空间平移对称性:自然规律不应随空间坐标系原点的移动而改变,即任意粒子坐标$X_n$($n$标记单个粒子)变换为$X_n+a$($a$是任意三维向量)。因此,必存在幺正算符 $T(\varepsilon )$使得:
\begin{equation}
	T(\varepsilon)|x\rangle = |x + \varepsilon\rangle \label{11.2.6}
\end{equation}

换句话说,如果粒子最初在\(x\)处,它最终必须在\(x + \varepsilon\)处。注意\(T(\varepsilon)\)是幺正的:它作用于正交归一基\(|x\rangle\)(\(-\infty \leq x \leq \infty\)),得到另一个正交归一基\(|x + \varepsilon\rangle\)(\(-\infty \leq x + \varepsilon \leq \infty\))。一旦知道了\(T(\varepsilon)\)在一个完全基上的作用,它在任意右矢\(|\psi\rangle\)上的作用就可以推导出来:
\begin{equation}
	\begin{aligned}
		|\psi_\varepsilon\rangle &= T(\varepsilon)|\psi\rangle = T(\varepsilon) \int_{-\infty}^{\infty} |x\rangle\langle x|\psi\rangle dx = \int_{-\infty}^{\infty} |x + \varepsilon\rangle\langle x|\psi\rangle dx \\
		&=\int_{ - \infty }^\infty  | x'\rangle \langle x' - \varepsilon |\psi \rangle dx'\;\;\;{\kern 1pt} (x' = x + \varepsilon ){\rm{ }}
	\end{aligned}
\end{equation}

换句话说,如果
\begin{equation}
	\langle x|\psi\rangle = \psi(x)
\end{equation}

那么
\begin{equation}
	\langle x|T(\varepsilon)|\psi\rangle = \psi(x - \varepsilon)
\end{equation}

这里的物理意义是非常明显的,平移后的波函数$T(\varepsilon )\psi \left( x \right)$在$x$处的值等于未平移波函数$\psi \left( x \right)$在$x-\varepsilon $处的值\par
从该结论出发,利用一个优美的数学等式:\par
We can express \(\hat{T}(\varepsilon)\) in terms of the momentum operator, to which it is intimately related. To that end, we replace \(\psi(x - \varepsilon)\) by its Taylor series
\begin{equation}
	\begin{aligned}
		\hat{T}(\varepsilon) \psi(x) &= \psi(x-\varepsilon) = \sum_{n=0}^{\infty} \frac{1}{n!} (-\varepsilon)^n \frac{d^n}{dx^n} \psi(x) \\
		&= \sum_{n=0}^{\infty} \frac{1}{n!} \left( \frac{-i \varepsilon}{\hbar} \hat{p} \right)^n \psi(x).
	\end{aligned}
\end{equation}
The right-hand side of this equation is the exponential function, so
\begin{equation}
	\hat{T}(\varepsilon) = \exp\left[ -\frac{i \varepsilon}{\hbar} \hat{p} \right].
\end{equation}
We say that momentum is the \textbf{``generator'' of translations}.
\subsection{时间平移对称性}
自然的基本对称性之一是时间平移不变性——自然规律不应依赖于时钟的校准方式。因此,无论态矢量$\Psi(t)$的时间依赖性如何,时间平移$\tau$(任意量)后的结果$\Psi(t+\tau)$在物理上等价,必存在线性幺正算符$\mathscr{U}(\tau)$使得系统在时刻$t$的态变换为:
\begin{equation}
\mathscr{U}(\tau)\Psi(t)=\Psi(t+\tau). \label{eq:3.6.1}
\end{equation}
由于$\tau$是连续变量,$\mathscr{U}(\tau)$可表示为(\ref{eq:3.4.13})式的形式。对于时间平移,用厄米算符$-H/\hbar$替换\newline(\ref{eq:3.4.13})式中的一般厄米算符$T$,因此:
\begin{equation}
\mathscr{U}(\tau)=\exp(-iH\tau/\hbar).
\end{equation}
这可作为哈密顿量$H$的定义。

令(\ref{eq:3.6.1})式中$t=0$,再将$\tau$替换为$t$,可得任意物理态向量的时间依赖性:
\begin{equation}
\Psi(t)=\exp(-iHt/\hbar)\Psi(0). \label{(3.6.3)}
\end{equation}
与所有由线性幺正算符表示的对称变换一样,该变换保持内积不变:
\begin{equation}
\left\langle {{\Phi (t)}}
\mathrel{\left | {\vphantom {{\Phi (t)} {\Psi (t)}}}
	\right. \kern-\nulldelimiterspace}
{{\Psi (t)}} \right\rangle  = \left\langle {{\Phi (0)}}
\mathrel{\left | {\vphantom {{\Phi (0)} {\Psi (0)}}}
	\right. \kern-\nulldelimiterspace}
{{\Psi (0)}} \right\rangle 
\end{equation}

于是得到含时薛定谔方程
\begin{equation}
i\hbar\dot{\Psi}(t)=H\Psi(t). 
\end{equation}

哈密顿量决定了大多数物理量的时间依赖性。任意与哈密顿量对易且不明显依赖于时间的算符$A$是守恒的,意味着该可观测量的期望值与时间无关。

对称性原理为物理理论中守恒量的存在提供了自然解释。若观察者看到态$\Psi(t)$按\newline
(\ref{(3.6.3)})式演化,则另一认为自然规律相同的观察者,必看到态$\mathscr{U}\Psi(t)$按同一方程演化:
\begin{equation}
U\Psi(t)=\exp(-iHt/\hbar)\mathscr{U}\Psi(0). \label{(3.6.8)}
\end{equation}
为使该式对所有态与 (\ref{(3.6.3)})式一致,需满足:
\begin{equation}
\mathscr{U}\exp(-iHt/\hbar)=\exp(-iHt/\hbar)\mathscr{U},\label{(3.6.9)}
\end{equation}
因此,若$\mathscr{U}$是线性算符,则
\begin{equation}
\mathscr{U}^{-1}H\mathscr{U}=H. \label{(3.6.10)}
\end{equation}
即哈密顿量在对称变换下不变。对于$\mathscr{U}$取(\ref{eq:3.4.11})式的无穷小对称变换,这意味着:
\begin{equation}
[H, T]=0,
\end{equation}
因此,哈密顿量对称性生成元表示的可观测量与哈密顿量对易。空间平移和时间平移不变性分别导致动量守恒和能量守恒。

注意,若$\mathscr{U}$是反线性算符,上述结论不成立。此时,由于(\ref{(3.6.9)})式指数中的$i$,(\ref{(3.6.10)})式替换为$\mathscr{U}^{-1}H\mathscr{U}=-H$。这意味着:哈密顿量的每个本征值$E$对应的本征态$\Phi$,存在另一本征态$U\Phi$对应本征值$-E$——这与观测结果和物质稳定性明显矛盾。反线性算符表示的对称性要避免这一结论,必须假设这类对称性反转时间流向,而非满足(\ref{(3.6.8)})式:
\begin{equation}
\mathscr{U}\Psi(t)=\exp(+iHt/\hbar)\mathscr{U}\Psi(0). \label{(3.6.12)}
\end{equation}
此时,与(\ref{(3.6.3)})式一致的条件替换为(\ref{(3.6.9)}))式:
\begin{equation}
\exp(+iHt/\hbar)\mathscr{U}=\mathscr{U}\exp(-iHt/\hbar). \label{(3.6.13)}
\end{equation}
对于反线性$\mathscr{U}$,这再次导致$\mathscr{U}$与$H$对易,避免了负能态的困境。因此,反线性算符表示的对称性是可能的,但必然涉及时间流向的反转。

\newpage

\section{角动量理论}
\subsection{Fundamental Commutation Relations of Angular Momentum}
首先我们需要强调的就是,一个物理操作比如绕z轴旋转角度$\theta $,我们记作$R\left( \theta  \right)$,我们是不可以直接说这个操作作用到我们的量子态上面的,而是由$R\left( \theta  \right)$诱导出的对称变换$U\left( {R\left( \theta  \right)} \right)$,这才是我们可以作用于量子态上的算符\par
我们记旋转操作$R$所诱导的对称变换为$U(R)$。首先考察绕$z$轴的\textbf{有限大$\theta$角}旋转$R_z(\theta)$。假设先绕$z$轴旋转$\theta_1$,接着再绕$z$轴旋转$\theta_2$,显然总的效果相当于绕$z$轴旋转$\theta_1+\theta_2$,因此我们有$R_z(\theta_2)R_z(\theta_1)=R_z(\theta_1+\theta_2)$。类似的,对于相应的希尔伯特空间幺正量子变换而言,必有
\begin{equation}
	U(R_z(\theta_2))U(R_z(\theta_1))=U(R_z(\theta_1+\theta_2)). 
\end{equation}

这种可加性意味着,$U(R_z(\theta))$可以写成
\begin{equation}
	U(R_z(\theta))=\exp\left(-i\theta J_z/\hbar\right). \label{(6.33)}
\end{equation}
式中厄米算符 $J_z$ 为绕$z$轴旋转的生成元,我们称之为 $z-axis$ 角动量\par
现在,我们对式(\ref{(6.33)})进行泰勒展开,并取无穷小$\theta$, 可以得到\textbf{无穷小$\varepsilon$角}旋转诱导的变换为:
\begin{equation}
	U({R_z}(\varepsilon )) = 1 - i\varepsilon {J_z}/\hbar  \label{(6.34)}
\end{equation}
More generally,we have
\begin{equation}
	U({\bf{\hat n}},d\phi ) = 1 - i\left( {\frac{{{\bf{J}} \cdot {\bf{\hat n}}}}{\hbar }} \right)d\phi  \label{(6.35)}
\end{equation}
for a rotation about the ${{\bf{\hat n}}}$ direction by an infinitesimal angle $d\phi $,and we define the notation “${\bf{J}}$” as angular-momentum operator.\par
由旋转操作,我们可以得到一个有趣的定理:
	\begin{equation}
	R_{x}(\varepsilon) R_{y}(\varepsilon)-R_{y}(\varepsilon) R_{x}(\varepsilon) =R_{z}(\varepsilon^{2})-1,
	\end{equation}
由该定理,我们可以很快地得到
\begin{equation}
U\left( {{R_x}(\varepsilon )} \right)U\left( {{R_y}(\varepsilon )} \right) - U\left( {{R_y}(\varepsilon )} \right)U\left( {{R_x}(\varepsilon )} \right) = U\left( {{R_z}({\varepsilon ^2})} \right) - I
\end{equation}
展开得到
\begin{equation}
	\begin{aligned}
		&\left(1-\frac{iJ_{x}\varepsilon}{\hbar}-\frac{J_{x}^{2}\varepsilon^{2}}{2\hbar^{2}}\right)
		\left(1-\frac{iJ_{y}\varepsilon}{\hbar}-\frac{J_{y}^{2}\varepsilon^{2}}{2\hbar^{2}}\right) \\
		&\quad -\left(1-\frac{iJ_{y}\varepsilon}{\hbar}-\frac{J_{y}^{2}\varepsilon^{2}}{2\hbar^{2}}\right)
		\left(1-\frac{iJ_{x}\varepsilon}{\hbar}-\frac{J_{x}^{2}\varepsilon^{2}}{2\hbar^{2}}\right)
		= 1-\frac{iJ_{z}\varepsilon^{2}}{\hbar}-1. \label{3.18}
	\end{aligned}
\end{equation}
Terms of order $\varepsilon$ automatically drop out. Equating terms of order $\varepsilon^{2}$ on both sides of (\ref{3.18}), we obtain
\begin{equation}
	[J_{x},J_{y}]=i\hbar J_{z}.
	\label{eq:3.19}
\end{equation}
Repeating this kind of argument with rotations about other axes, we obtain
\begin{equation}
\boxed{	[J_{i},J_{j}]=i\hbar\varepsilon_{ijk}J_{k}}
	\label{eq:3.20}
\end{equation}
known as the \textbf{fundamental commutation relations of angular momentum} ( The Lie algebra of the roattion group SO(3) )

\subsection{角动量:本征态与本征值}

我们继续研究
\begin{equation}
	[J_{i},J_{j}]=i\hbar\varepsilon_{ijk}J_{k},
	\label{6.17}
\end{equation}

根据
\[
\Delta A\Delta B \ge \frac{1}{2}\left| {\left\langle {\left[ {A,B} \right]} \right\rangle } \right|
\]

我们得到
\[\Delta {L_x}\Delta {L_y} \ge \frac{\hbar }{2}\left| {\left\langle {{L_z}} \right\rangle } \right|\]

这也就意味着角动量的可观测量的不同分量不能“同时”测量,这体现在它们没有共同的本征态,现在我们需要找到一组算符,它们有相同的本征态也就是,它们之间彼此对易

定义角动量矢量的平方 ${{\vec J}^2} = J_x^2 + J_y^2 + J_z^2$ , 然后得到
\begin{equation}\label{6.18}
	\left[ {{J^2},{J_i}} \right] = 0
\end{equation}
\noindent\rule{\textwidth}{0.4pt}
\begin{proof*}
	我们通过角动量算符的对易关系和对易子的性质来推导。
	
	\textbf{步骤 1:回忆角动量算符的定义与对易关系}
	
	角动量算符的平方为 \( \vec{J}^{\,2} = J_x^2 + J_y^2 + J_z^2 = J_i J_i \)(爱因斯坦求和约定,$i$ 从 1 到 3 求和)。
	
	角动量分量间的对易关系为:
	\[
	[J_i, J_j] = i\hbar \epsilon_{ijk} J_k
	\]
	其中 \( \epsilon_{ijk} \) 是莱维-奇维塔符号。
	
	\textbf{步骤 2:利用对易子的双线性性质展开}
	
	对易子满足 \( [AB, C] = A[B, C] + [A, C]B \)。对于 \( \vec{J}^{\,2} = J_j J_j \),我们有:
	\[
	[\vec{J}^{\,2}, J_i] = [J_j J_j, J_i] = J_j [J_j, J_i] + [J_j, J_i] J_j
	\]
	
	\textbf{步骤 3:代入角动量分量的对易关系}
	
	将 \( [J_j, J_i] = i\hbar \epsilon_{j i k} J_k \) 代入上式:
	\[
	[J_j J_j, J_i] = J_j \cdot (i\hbar \epsilon_{j i k} J_k) + (i\hbar \epsilon_{j i k} J_k) \cdot J_j = i\hbar \epsilon_{j i k} (J_j J_k + J_k J_j)
	\]
	
	\textbf{步骤 4:分析反称张量与对称表达式的缩并}
	
	注意到:
	\begin{itemize}
		\item \( \epsilon_{j i k} \) 是关于 \(j, k\) 的反称张量(交换 \(j\) 和 \(k\) 时,\( \epsilon_{k i j} = -\epsilon_{j i k} \))。
		\item \( J_j J_k + J_k J_j \) 是关于 \(j, k\) 的对称表达式(交换 \(j\) 和 \(k\) 时,表达式不变)。
	\end{itemize}
	
	反称张量与对称张量的缩并为 0(因为反称张量要求交换指标变号,对称张量要求交换指标不变,缩并后正负项抵消)。
	
	此外,当 \(j = k\) 时,\( \epsilon_{j i j} = 0 \)(莱维-奇维塔符号要求下标无重复,否则为 0)。因此,整个表达式在求和后为 0:
	\[
	[\vec{J}^{\,2}, J_i] = i\hbar \epsilon_{j i k} (J_j J_k + J_k J_j) = 0
	\]
\end{proof*}
\noindent\rule{\textwidth}{0.4pt}

我们找到了一组算符,它们对易,有相同的本征态,我们用两个量子数去表示这个本征态,记作 $\ket{\lambda ,\mu }$
即
\begin{equation}
	\begin{aligned}
		{J^2}\left| {\lambda ,\mu } \right\rangle  &= \lambda \left| {\lambda ,\mu } \right\rangle  \,, \\
		{J_z}\left| {\lambda ,\mu } \right\rangle  &= \mu \left| {\lambda ,\mu } \right\rangle  \,.
	\end{aligned} 
\end{equation}
接下来我们计算本征值

我们将使用升降算符,其定义如下:

角动量\textbf{升降算符} \(J_\pm\) 定义为:
\begin{equation}
	J_\pm := J_x \pm i J_y
\end{equation}
其中 \(J_+\) 称为升算符,\(J_-\) 称为降算符。

利用式 (\ref{6.17} - \ref{6.18}),可轻松重写角动量算符的李代数:
\begin{align}
	[\vec{J}^{\,2}, J_\pm] &= 0 \label{6.22} \\
	[J_z, J_\pm] &= \pm \hbar J_\pm \label{6.21} \\
	[J_+, J_-] &= 2\hbar J_z \label{6.23}
\end{align}
一个非常重要的性质:
\[
\boxed{
	\begin{gathered}
		\text{若 } f \text{ 是 } \vec{J}^{\,2} \text{ 和 } J_z \text{ 的本征函数} \\[6pt]
		\text{则 } J_{\pm} f \text{ 也同样是它们的本征函数}
	\end{gathered}
}
\]
\begin{proof*}
 我们分两步证明:先证明 \(J_\pm f\) 是 \(\vec{J}^{\,2}\) 的本征函数,再证明它是 \(J_z\) 的本征函数。

首先证明对 \(\vec{J}^{\,2}\) 的本征性:
\begin{equation}
	\vec{J}^{\,2} J_\pm f \stackrel{\text{式 (\ref{6.22})}}{=} J_\pm \vec{J}^{\,2} f = \lambda J_\pm f \label{6.24}
\end{equation}

再证明对 \(J_z\) 的本征性:
\begin{equation}
	\begin{aligned}
		J_z J_\pm f &= J_z J_\pm f - J_\pm J_z f + J_\pm J_z f \\
		&= [J_z, J_\pm] f + J_\pm J_z f \\
		&\stackrel{\text{式 (\ref{6.21})}}{=} (\mu \pm \hbar) J_\pm f \,.
	\end{aligned} \label{6.25}
\end{equation}

可以看到,\(J_\pm f\) 是 \(\vec{J}^{\,2}\) 的本征函数,但是不会改变其本征值; \(J_\pm f\) 是 \(J_z\) 的本征函数,本征值为 \((\mu \pm \hbar)\)。

因此,从某个本征值 \(\mu\) 出发,升降算符可在 \(J_z\) 的所有可能本征值之间“切换”:\(J_+\) 对应“上升”本征值,\(J_-\) 对应“下降”本征值。
\end{proof*}
\noindent\rule{\textwidth}{0.3pt}

由于 \(\vec{J}^{\,2}\) 的本征值 \(\lambda\) 对升降算符作用后得到的所有本征函数都相同,因此对给定的 \(\vec{J}^{\,2}\) 本征值 \(\lambda\),\(J_z\) 的取值必被 \(\sqrt{\lambda}\) 约束。\textbf{故存在 \(J_z\) 最大可能值的函数} \(f_{\text{top}}\), 满足:
\begin{equation}
	J_+ f_{\text{top}} = 0 \label{6.26}
\end{equation}

进一步假设 \(f_{\text{top}}\) 对应的 \(J_z\) 本征值为 \(\hbar j\),则有:
\begin{equation}
	\begin{aligned}
		J_z f_{\text{top}} &= \hbar j f_{\text{top}} \,, \\
		\vec{J}^{\,2} f_{\text{top}} &= \lambda f_{\text{top}} \,.
	\end{aligned} \label{6.27}
\end{equation}

在继续之前,我们先推导一个后续计算中会用到的等式:
\[
\begin{aligned}
	J_\pm J_\mp &= (J_x \pm i J_y)(J_x \mp i J_y) \\
	&= J_x^2 + J_y^2 \mp i [J_x, J_y] \\
	&= \underbrace{J_x^2 + J_y^2 + J_z^2 }_{\vec{J}^{\,2}}- J_z^2 \pm \hbar J_z \\
\end{aligned} 
\]
\begin{equation}
	\boxed{\vec{J}^{\,2} = J_\pm J_\mp + J_z^2 \mp \hbar J_z \label{6.28}} 
\end{equation}


我们可利用式 (\ref{6.28}),通过 \(J_z\) 的本征值 \(\hbar j\) 计算 \(\vec{J}^{\,2}\) 的本征值:
\begin{equation}
	\vec{J}^{\,2} f_{\text{top}} = (J_- J_+ + J_z^2 + \hbar J_z) f_{\text{top}} = \hbar^2 j(j + 1) f_{\text{top}} \label{6.29}
\end{equation}
其中用到了式 (\ref{6.26}) 和式 (\ref{6.27})。由此可得角动量平方的本征值 \(\lambda\) 为:
\begin{equation}
	\lambda = \hbar^2 j(j + 1) \label{6.30}
\end{equation}
其中 \(\hbar j\) 是 \(J_z\) 的最大可能取值。

对应 \(J_z\) 最小可能本征值(固定 \(\lambda\))的本征函数 \(f_{\text{bottom}}\) 做类似计算,假设其本征值为 \(\hbar \bar{j}\),则本征方程为:
\begin{equation}
	\begin{aligned}
		J_z f_{\text{bottom}} &= \hbar \bar{j} f_{\text{bottom}} \,, \\
		\vec{J}^{\,2} f_{\text{bottom}} &= \lambda f_{\text{bottom}} \,.
	\end{aligned} \label{6.31}
\end{equation}

再次利用式 (\ref{6.28}),将 \(\vec{J}^{\,2}\) 作用于 \(f_{\text{bottom}}\):
\begin{equation}
	\vec{J}^{\,2} f_{\text{bottom}} = (J_+ J_- + J_z^2 - \hbar J_z) f_{\text{bottom}} = \hbar^2 \bar{j}(\bar{j} - 1) f_{\text{bottom}} \label{6.32}
\end{equation}

由于 \(\lambda = \hbar^2 \bar{j}(\bar{j} - 1)\),将其与式 (\ref{6.30}) 的结果联立,可得:
\begin{equation}
	\bar{j} = -j \label{6.33}
\end{equation}

\noindent\textbf{结果:} \(J_z\) 的本征值取值范围为 \(-\hbar j\) 到 \(+\hbar j\),相邻本征值的间隔为 \(\hbar\) 的整数倍。因此存在整数 \(N\) 表示从 \(-j\) 到 \(j\) 的阶梯数,满足:
\begin{equation}
	j = -j + N \quad \Leftrightarrow \quad j = \frac{N}{2} \label{6.34}
\end{equation}
这意味着 \(j\) 可以是整数(\(0,1,2,\cdots\)),也可以是半整数(\(\frac{1}{2},\frac{3}{2},\frac{5}{2},\cdots\))。

\(j\) 称为\textbf{角量子数}。

我们将 \(J_z\) 的本征值记为 \(\mu = \hbar m\),其中:
\begin{equation}
	m = -j,-j+1,\cdots,0,\cdots,j-1,+j \label{6.35}
\end{equation}

$m$ 称为\textbf{磁量子数}。

对给定的 \(j\),\(m\) 共有 \(2j+1\) 个取值,且满足 \(|m| \leq j\)。

本征函数 \(f\) 实际上是\textbf{球谐函数} \(Y_{j,m}\),其下标对标记了角动量可观测量对应的本征值。

直接给出球谐函数的表达式:
\begin{equation}
	{Y_{jm}}(\theta ,\phi ) = {\left[ {\frac{{2j + 1}}{{4\pi }}\frac{{(j - m)!}}{{(j + m)!}}} \right]^{\frac{1}{2}}}{e^{im\phi }}P_j^m(\cos \theta )
\end{equation}

用球谐函数重写本征方程如下:
\begin{align}
	\vec{J}^{\,2} Y_{j,m} &= \hbar^2 j(j + 1) Y_{j,m} \label{6.36} \\
	J_z Y_{j,m} &= \hbar m Y_{j,m} \label{6.37} \\
	J_\pm Y_{j,m} &= \hbar \sqrt{j(j + 1) - m(m \pm 1)} Y_{j,m\pm1} \label{6.38}
\end{align}

其中,式(\ref{6.38})的证明如下:
\begin{proof*}
令\(J_{\pm} Y_{j,m}=C_{\pm}(j,m \pm 1)Y_{j,m}\), 由厄米性, $J_+^\dagger = J_-$,故
\[
\left\langle {{{J_ \pm }{Y_{j,m}}}}
\mathrel{\left | {\vphantom {{{J_ \pm }{Y_{j,m}}} {{J_ \pm }{Y_{j,m}}}}}
	\right. \kern-\nulldelimiterspace}
{{{J_ \pm }{Y_{j,m}}}} \right\rangle  = \left\langle {{Y_{j,m}}} \right.\left| {{J_ \mp }{J_ \pm }\left| {{Y_{j,m}}} \right\rangle } \right.
\]
根据\(\vec{J}^{\,2} = J_\pm J_\mp + J_z^2 \mp \hbar J_z\),有
\begin{align*}
	J_{\mp} J_{\pm} Y_{j,m} &= (J^2 - J_z^2 \mp \hbar J_z) Y_{j,m} \\
	&= \bigl[ \hbar^2 j(j+1) - \hbar^2 m^2 \mp \hbar^2 m \bigr] Y_{j,m} \\
	&= \hbar^2 \bigl[ j(j+1) - m(m\pm 1) \bigr] Y_{j,m}.
\end{align*}
于是得到
\[\left\langle {{{J_ \pm }{Y_{j,m}}}}
\mathrel{\left | {\vphantom {{{J_ \pm }{Y_{j,m}}} {{J_ \pm }{Y_{j,m}}}}}
	\right. \kern-\nulldelimiterspace}
{{{J_ \pm }{Y_{j,m}}}} \right\rangle  = \left\langle {{Y_{j,m}}} \right.\left| {{J_ \mp }{J_ \pm }\left| {{Y_{j,m}}} \right\rangle } \right. = {\hbar ^2}[j(j + 1) - m(m \pm 1)]\langle {Y_{j,m}}|{Y_{j,m}}\rangle .\]
从而得到
\[
|C_{\pm}(j,m \pm 1)|^2 = \hbar^2 \bigl[ j(j+1) - m(m\pm 1) \bigr].
\]

取通常的 Condon-Shortley 相位约定,使 $C_{\pm}(j,m \pm 1)$ 为正实数,则
\[
C_{\pm}(j,m \pm 1) = \hbar \sqrt{ j(j+1) - m(m\pm 1) }.
\]

因此,
\[
\boxed{ J_{\pm} Y_{j,m} = \hbar \sqrt{ j(j+1) - m(m\pm 1) } \; Y_{j,m\pm 1} }.
\]
\end{proof*}
\noindent\rule{\textwidth}{0.4pt}  % 宽度为文本宽度,厚度0.4pt

\textbf{One More thing}
\[
J_\pm := J_x \pm i J_y
\]

我们可以写出
\[
\boxed{
	\begin{aligned}
		J_x &= \frac{1}{2} \bigl( J_{+} + J_{-} \bigr) \\
		J_y &= \frac{1}{2i} \bigl( J_{+} - J_{-} \bigr)
	\end{aligned}
}
\]

如果我们考虑自旋 $s=1/2$, 本征态为 \(\ket{\uparrow},\ket{\downarrow}\), 我们可以轻而易举地写出
\[
\begin{aligned}
	S_x |\! \uparrow \rangle &= \frac{1}{2} S_{-} |\! \uparrow \rangle = \frac{\hbar}{2} |\! \downarrow \rangle, \\[4pt]
	S_x |\! \downarrow \rangle &= \frac{1}{2} S_{+} |\! \downarrow \rangle = \frac{\hbar}{2} |\! \uparrow \rangle, \\[6pt]
	S_y |\! \uparrow \rangle &= -\frac{1}{2i} S_{-} |\! \uparrow \rangle = -i\frac{\hbar}{2} |\! \downarrow \rangle, \\[4pt]
	S_y |\! \downarrow \rangle &= \frac{1}{2i} S_{+} |\! \downarrow \rangle = i\frac{\hbar}{2} |\! \uparrow \rangle .
\end{aligned}
\]

以及
\[{S_z}\left|  \uparrow  \right\rangle  = \frac{\hbar }{2}\left|  \uparrow  \right\rangle ;\quad {S_z}\left|  \downarrow  \right\rangle = -\frac{\hbar }{2}\left|  \downarrow  \right\rangle\]

同时我们易知
\[\begin{gathered}
	{S_ + }\left|  \downarrow  \right\rangle  = \hbar \left|  \uparrow  \right\rangle \,\, \Rightarrow \hbar \left|  \uparrow  \right\rangle \left\langle  \downarrow  \right|\\
	{S_ - }\left|  \uparrow  \right\rangle  = \hbar \left|  \downarrow  \right\rangle \,\, \Rightarrow \hbar \left|  \downarrow  \right\rangle \left\langle  \uparrow  \right|
\end{gathered}\]

根据\({S_x},{S_y}\)的表达式:我们可以得到各方向角动量算符的外积形式:
\[
\boxed{
	\begin{aligned}
		S_x &= \frac{\hbar}{2} \bigl( \ket{\uparrow}\bra{\downarrow} + \ket{\downarrow}\bra{\uparrow} \bigr) \\[6pt]
		S_y &= -i\frac{\hbar}{2} \bigl( \ket{\uparrow}\bra{\downarrow} - \ket{\downarrow}\bra{\uparrow} \bigr) \\[6pt]
		S_z &= \frac{\hbar}{2} \bigl( \ket{\uparrow}\bra{\uparrow} - \ket{\downarrow}\bra{\downarrow} \bigr)
	\end{aligned}
}
\]

我们已经引入了泡利矩阵
\[
{S_i} = \frac{\hbar }{2}{\sigma _i}
\]

在此给出泡利矩阵的重要的运算性质
\[\boxed{
\begin{array}{c}
	\sigma _i^2 = 1\\
	{\sigma _i}{\sigma _j} + {\sigma _j}{\sigma _i} = 0,\;\;{\mkern 1mu} {\kern 1pt} {\rm{for }}\,\, i \ne j
\end{array}
}\]

or equivalently, anticommutation relation
\[
\boxed{
\{ \sigma_i, \sigma_j \} = 2\delta_{ij}
}
\]

Also, we have:
\[
\boxed{
[ \sigma_i, \sigma_j ] = 2i \epsilon_{ijk} \sigma_k
}
\]

\subsubsection{Angular Momentum Operator in Three-Dimensional}
我们直接从角动量的经典定义出发(实际上应该从对称性出发)。首先,我们需要将$L_x$ $L_y$和 $L_z$ 改写为球坐标形式。现在,角动量$\mathbf{L} = -i\hbar (\mathbf{r} \times \nabla)$,而球坐标下的梯度为:

\begin{equation}\label{eq:4.123}
	\nabla = \hat{r}\frac{\partial}{\partial r} + \hat{\theta}\frac{1}{r}\frac{\partial}{\partial \theta} + \hat{\phi}\frac{1}{r\sin\theta}\frac{\partial}{\partial \phi};
\end{equation}


同时,$\mathbf{r} = r\hat{r}$,因此
\[
\mathbf{L} = -i\hbar \left[ r\left(\hat{r} \times \hat{r}\right)\frac{\partial}{\partial r} + \left(\hat{r} \times \hat{\theta}\right)\frac{\partial}{\partial \theta} + \left(\hat{r} \times \hat{\phi}\right)\frac{1}{\sin\theta}\frac{\partial}{\partial \phi} \right].
\]

而$\left(\hat{r} \times \hat{r}\right) = 0$,$\left(\hat{r} \times \hat{\theta}\right) = \hat{\phi}$,且$\left(\hat{r} \times \hat{\phi}\right) = -\hat{\theta}$,因此

\begin{equation}\label{eq:4.124}
	\mathbf{L} = -i\hbar \left( \hat{\phi}\frac{\partial}{\partial \theta} - \hat{\theta}\frac{1}{\sin\theta}\frac{\partial}{\partial \phi} \right).
\end{equation}


单位矢量$\hat{\theta}$和$\hat{\phi}$可分解为直角坐标分量:

\begin{equation}\label{eq:4.125}
	\hat{\theta} = (\cos\theta \cos\phi)\hat{i} + (\cos\theta \sin\phi)\hat{j} - (\sin\theta)\hat{k};
\end{equation}

\begin{equation}\label{eq:4.126}
	\hat{\phi} = -(\sin\phi)\hat{i} + (\cos\phi)\hat{j}.
\end{equation}


因此
\[
\mathbf{L} = -i\hbar \left[ \left(-\sin\phi \hat{i} + \cos\phi \hat{j}\right)\frac{\partial}{\partial \theta} - \left( \cos\theta \cos\phi \hat{i} + \cos\theta \sin\phi \hat{j} - \sin\theta \hat{k} \right)\frac{1}{\sin\theta}\frac{\partial}{\partial \phi} \right].
\]

由此可得

\begin{equation}\label{eq:4.127}
	L_x = -i\hbar \left( -\sin\phi \frac{\partial}{\partial \theta} - \cos\phi \cot\theta \frac{\partial}{\partial \phi} \right).
\end{equation}

\begin{equation}\label{eq:4.128}
	L_y = -i\hbar \left( +\cos\phi \frac{\partial}{\partial \theta} - \sin\phi \cot\theta \frac{\partial}{\partial \phi} \right).
\end{equation}

以及

\begin{equation}\label{eq:4.129}
	L_z = -i\hbar \frac{\partial}{\partial \phi}.
\end{equation}


我们还需要升降算符:
\[
L_\pm = L_x \pm iL_y = -i\hbar \left[ (-\sin\phi \pm i\cos\phi)\frac{\partial}{\partial \theta} - (\cos\phi \pm i\sin\phi)\cot\theta \frac{\partial}{\partial \phi} \right].
\]
而$\cos\phi \pm i\sin\phi = e^{\pm i\phi}$,因此

\begin{equation}\label{eq:4.130}
	L_\pm = \pm\hbar e^{\pm i\phi} \left( \frac{\partial}{\partial \theta} \pm i\cot\theta \frac{\partial}{\partial \phi} \right).
\end{equation}


特别地:

\begin{equation}\label{eq:4.131}
	L_+L_- = -\hbar^2 \left( \frac{\partial^2}{\partial \theta^2} + \cot\theta \frac{\partial}{\partial \theta} + \cot^2\theta \frac{\partial^2}{\partial \phi^2} + i\frac{\partial}{\partial \phi} \right).
\end{equation}

进而由$\vec{J}^{\,2} = J_\pm J_\mp + J_z^2 \mp \hbar J_z $:

\begin{equation}\label{eq:4.132}
	L^2 = -\hbar^2 \left[ \frac{1}{\sin\theta}\frac{\partial}{\partial \theta}\left( \sin\theta \frac{\partial}{\partial \theta} \right) + \frac{1}{\sin^2\theta}\frac{\partial^2}{\partial \phi^2} \right].
\end{equation}


我们可以确定球谐函数$Y_\ell^m(\theta, \phi)$是$L^2$的本征函数,本征值为$\hbar^2 \ell(\ell + 1)$:
\[
L^2 Y_\ell^m(\theta, \phi) = -\hbar^2 \left[ \frac{1}{\sin\theta}\frac{\partial}{\partial \theta}\left( \sin\theta \frac{\partial}{\partial \theta} \right) + \frac{1}{\sin^2\theta}\frac{\partial^2}{\partial \phi^2} \right] Y_\ell^m(\theta, \phi) = \hbar^2 \ell(\ell + 1) Y_\ell^m(\theta, \phi).
\]

同时它也是$L_z$的本征函数,本征值为$m\hbar$:
\[
L_z Y_\ell^m(\theta, \phi) = -i\hbar \frac{\partial}{\partial \phi} Y_\ell^m(\theta, \phi) = \hbar m Y_\ell^m(\theta, \phi),
\]

结论:球谐函数是$L^2$和$L_z$的共同本征函数。在中心势场中,我们用分离变量法解薛定谔方程时,其实是在构造$H, L^2$和$L_z$这三个对易算符的共同本征函数:
\begin{equation}\label{eq:4.133}
	H\psi = E\psi,\quad L^2\psi = \hbar^2 \ell(\ell + 1)\psi,\quad L_z\psi = \hbar m \psi.
\end{equation}


顺便一提,我们可以利用式(\ref{eq:4.132})将薛定谔方程写得更简洁:
\[
\frac{1}{2mr^2} \left[ -\hbar^2 \frac{\partial}{\partial r}\left( r^2 \frac{\partial}{\partial r} \right) + L^2 \right] \psi + V\psi = E\psi.
\]

\textbf{Central Potentials}
\[
H = \frac{\vb{p}^2}{2m} + V(r)
\]

It is easy to show commutation relations:
\[
[\vb{L}, \vb{p}^2] = [\vb{L}, \vb{x}^2] = 0
\]

and therefore:
\[
[\vb{L}, H] = [L^2, H] = 0
\]

So $H$, $L^2$ and $L_z$ can be chosen as the operators to label eigenstates of $H$.

Let's denote the eigenstates as $\ket{nlm}$:
\begin{align}
	H\ket{nlm} &= E_n \ket{nlm} \\
	L^2\ket{nlm} &= l(l+1)\hbar^2 \ket{nlm} \\
	L_z\ket{nlm} &= m\hbar \ket{nlm}
\end{align}

In position representation, we write:
\[
\braket{r,\theta,\phi}{nlm} = R_{nl}(r) Y_l^m(\theta, \phi)
\]
where $Y_l^m(\theta, \phi)$ are the spherical harmonics function, and $R_{nl}(r)$ needs to determine the potential energy to solve
\vspace{0.5cm}  % 段落间距

\noindent \textbf{备注 I:} 在 $1/r$ 势中存在关于角动量的简并性是一个有趣的性质。这暗示了除了旋转对称性之外还存在额外的对称性。这种对称性在经典理论中也有对应,即在这种势中,Laplace-Runge-Lenz Vector是一个运动常量。

\newpage
\section{Spin and Spin–Addition}
\subsection{Two Particles with Spin 1/2}
两个自旋的耦合与一般角动量的耦合遵循相同的数学规则。为描述耦合后的自旋态,我们引入张量积运算“\(\otimes\)”,它将分属不同(希尔伯特)空间(例如\(\mathcal{H}^A\)和\(\mathcal{H}^B\))的态矢量组合成新的态。所有这样的组合(及其线性叠加)构成了一个更大的复合系统希尔伯特空间\(\mathcal{H}^{AB}\):
\begin{equation}
	\mathcal{H}^{AB} = \mathcal{H}^A \otimes \mathcal{H}^B \label{7.42}
\end{equation}

为了更具体地说明,我们考虑两个自旋为 \(\displaystyle \frac{1}{2}\) 的粒子,每个粒子均可处于态 \(\ket{\uparrow}\) 或 \(\ket{\downarrow}\). 由此可得到这两个态的4种不同组合方式:\(\ket{\uparrow\uparrow}\)、\(\ket{\uparrow\downarrow}\)、\(\ket{\downarrow\uparrow}\) 和 \(\ket{\downarrow\downarrow}\),这些组合通过张量积运算构造(为简洁起见,记号中常省略张量积符号\(\otimes\)):
\begin{equation}
	|\uparrow\rangle \otimes |\uparrow\rangle = |\uparrow\rangle|\uparrow\rangle = |\uparrow\uparrow\rangle \label{7.43}
\end{equation}
\begin{equation}
	|\uparrow\rangle \otimes |\downarrow\rangle = |\uparrow\rangle|\downarrow\rangle = |\uparrow\downarrow\rangle \label{7.44}
\end{equation}
\begin{equation}
	|\downarrow\rangle \otimes |\uparrow\rangle = |\downarrow\rangle|\uparrow\rangle = |\downarrow\uparrow\rangle \label{7.45}
\end{equation}
\begin{equation}
	|\downarrow\rangle \otimes |\downarrow\rangle = |\downarrow\rangle|\downarrow\rangle = |\downarrow\downarrow\rangle \label{7.46}
\end{equation}

现在我们根据希尔伯特空间的张量积结构(见式(\ref{7.42}))定义复合系统的\textbf{自旋算符}。已知单个自旋算符 \(\vec{S}^{(A)}\) 和 \(\vec{S}^{(B)}\) 分别作用于 \(\mathcal{H}^A\) 和 \(\mathcal{H}^B\), 我们构造复合自旋算符,使得单个算符作用于各自的子空间,即:
\begin{equation}
	\vec{S}^{(AB)} = \vec{S}^{(A)} \otimes \mathbb{I}^{(B)} + \mathbb{I}^{(A)} \otimes \vec{S}^{(B)} = \vec{S}^{(A)} + \vec{S}^{(B)} \label{7.47}
\end{equation}

同理,我们构造 $z$ 方向的自旋分量算符:
\begin{equation}
	S_z^{(AB)} = S_z^{(A)} \otimes \mathbb{I}^{(B)} + \mathbb{I}^{(A)} \otimes S_z^{(B)} = S_z^{(A)} + S_z^{(B)} \label{7.48}
\end{equation}

接下来,利用上述关系计算态\(|\uparrow\uparrow\rangle\)的 $z$ 方向自旋分量:
\begin{equation}
	\begin{aligned}
		S_z^{(AB)} \ket{\uparrow\uparrow} &= \left(S_z^{(A)} + S_z^{(B)}\right) \ket{\uparrow} \otimes \ket{\uparrow} \\
		&= \left(S_z^{(A)} \ket{\uparrow}\right) \otimes \ket{\uparrow} + \ket{\uparrow} \otimes \left(S_z^{(B)} \ket{\uparrow}\right) \\
		&= \left(\frac{\hbar}{2} \ket{\uparrow}\right) \otimes \ket{\uparrow} + \ket{\uparrow} \otimes \left(\frac{\hbar}{2} \ket{\uparrow}\right) \\
		&= \left(\frac{\hbar}{2} + \frac{\hbar}{2}\right) \ket{\uparrow} \otimes \ket{\uparrow} = \hbar \ket{\uparrow\uparrow}
	\end{aligned}
\end{equation}

采用相同方法,对其他组合(式(\ref{7.44})-(\ref{7.46}))计算可得:
\begin{equation}
	S_z \ket{\downarrow\downarrow} = -\hbar \ket{\downarrow\downarrow} \label{7.50}
\end{equation}
\begin{equation}
	S_z \ket{\uparrow\downarrow} = S_z \ket{\downarrow\uparrow} = 0 \label{7.51} 
\end{equation}

为简化记号,我们省略了下标\(AB\)。除非另有说明,否则所有算符均作用于全空间,与式(\ref{7.47})和(\ref{7.48})的定义一致。\par
\textbf{遗憾的是,这些简单的自旋组合虽然是 \(S_z\) 算符的本征态,但并非都是自旋平方算符 \(\vec{S}^2\) 的共同本征态}。我们将\({\vec S^2} = {\left( {{{\vec S}_1} + {{\vec S}_2}} \right)^2}\)用张量积表示为:

\begin{equation}\label{7.52}
	\begin{aligned}
		\vec{S}^2 &= \left(\vec{S}^{(AB)}\right)^2 = \left(\vec{S}^{(A)} \otimes \mathbb{I}^{(B)} + \mathbb{I}^{(A)} \otimes \vec{S}^{(B)}\right)^2 \\
		&= \left(\vec{S}^{(A)}\right)^2 \otimes \mathbb{I}^{(B)} + 2\vec{S}^{(A)} \otimes \vec{S}^{(B)} + \mathbb{I}^{(A)} \otimes \left(\vec{S}^{(B)}\right)^2
	\end{aligned}
\end{equation}

利用泡利矩阵的性质 \(\sigma_i^2 = \mathbb{I}\) ,计算\(\left(\vec{S}\right)^2\):

\begin{equation}
	\left(\vec{S}\right)^2 = \frac{\hbar^2}{4}\left(\sigma_x^2 + \sigma_y^2 + \sigma_z^2\right) = \frac{\hbar^2}{4} \cdot 3\mathbb{I} = \frac{3}{4}\hbar^2 \label{7.53}
\end{equation}

将此结果代入式(\ref{7.52}),改写可得:

\begin{equation}
	\begin{aligned}
		\vec{S}^2 &= \frac{3}{4}\hbar^2 \mathbb{I}^{(A)} \otimes \mathbb{I}^{(B)} + 2\vec{S}^{(A)} \otimes \vec{S}^{(B)} + \frac{3}{4}\hbar^2 \mathbb{I}^{(A)} \otimes \mathbb{I}^{(B)} \\
		&= \frac{\hbar^2}{4}\left[6\mathbb{I} \otimes \mathbb{I} + 2\left(\sigma_x \otimes \sigma_x + \sigma_y \otimes \sigma_y + \sigma_z \otimes \sigma_z\right)\right]
	\end{aligned}
\end{equation}

计算\(\vec{S}^2\)对自旋态(式(\ref{7.43}-\ref{7.46}))的作用,发现\(|\uparrow\uparrow\rangle\)和\(|\downarrow\downarrow\rangle\)确实是\(\vec{S}^2\)的本征态,对应量子数\(s=1\):

\begin{equation}
	\begin{aligned}
		\vec{S}^2|\uparrow\uparrow\rangle &= \frac{\hbar^2}{4}\left[6\mathbb{I} \otimes \mathbb{I} + 2\left(\sigma_x \otimes \sigma_x + \sigma_y \otimes \sigma_y + \sigma_z \otimes \sigma_z\right)\right]|\uparrow\rangle \otimes |\uparrow\rangle \\
		&= \frac{\hbar^2}{4}\left[6|\uparrow\uparrow\rangle + 2|\downarrow\downarrow\rangle + 2(i)^2|\downarrow\downarrow\rangle + 2|\uparrow\uparrow\rangle\right] \\
		&= \frac{\hbar^2}{4} \cdot 8|\uparrow\uparrow\rangle = 2\hbar^2|\uparrow\uparrow\rangle = \hbar^2 \cdot 1(1+1)|\uparrow\uparrow\rangle
	\end{aligned}
\end{equation}

\begin{equation}
	\vec{S}^2|\downarrow\downarrow\rangle = 2\hbar^2|\downarrow\downarrow\rangle \label{7.56}
\end{equation}

而态\(|\uparrow\downarrow\rangle\)和\(|\downarrow\uparrow\rangle\)并非自旋平方算符的本征态:
\begin{equation}
	\vec{S}^2|\uparrow\downarrow\rangle = \vec{S}^2|\downarrow\uparrow\rangle = \hbar^2\left(|\uparrow\downarrow\rangle + |\downarrow\uparrow\rangle\right)
\end{equation}

不过,我们可以对 \(|\uparrow\downarrow\rangle\) 和 \(|\downarrow\uparrow\rangle\) 进行线性组合,并选择合适的归一化权重 \(\displaystyle \frac{1}{\sqrt{2}}\)(注:该权重为两个自旋 \(\frac{1}{2}\) 耦合的克莱布施-戈登系数),得到量子数\(s=1\)和\(s=0\)对应的本征态:
\begin{align}
	\vec{S}^{\,2} \, \frac{1}{\sqrt{2}} \bigl( \ket{\uparrow\downarrow} + \ket{\downarrow\uparrow} \bigr) &= 2\hbar^{2} \, \frac{1}{\sqrt{2}} \bigl( \ket{\uparrow\downarrow} + \ket{\downarrow\uparrow} \bigr) \\
	\vec{S}^{\,2} \, \frac{1}{\sqrt{2}} \bigl( \ket{\uparrow\downarrow} - \ket{\downarrow\uparrow} \bigr) &= 0
\end{align}


由此得到自旋总量子数\(s=1\)、磁自旋量子数\(m_s=-1,0,1\)的三重态\(|s,m_s\rangle\):
\begin{equation}
	|1,-1\rangle = |\downarrow\downarrow\rangle
\end{equation}
\begin{equation}
	|1,0\rangle = \frac{1}{\sqrt{2}}\left(|\uparrow\downarrow\rangle + |\downarrow\uparrow\rangle\right) \label{7.61}
\end{equation}
\begin{equation}
	|1,+1\rangle = |\uparrow\uparrow\rangle \label{7.62}
\end{equation}

以及自旋总量子数\(s=0\)、磁自旋量子数\(m_s=0\)的单态:

\begin{equation}
	|0,0\rangle = \frac{1}{\sqrt{2}}\left(|\uparrow\downarrow\rangle - |\downarrow\uparrow\rangle\right) \label{7.63}
\end{equation}

这说明了
\[
\mathbb{V}_{s_1 = \frac{1}{2}} \otimes \mathbb{V}_{s_2 = \frac{1}{2}} = \mathbb{V}_{s = 1} \oplus \mathbb{V}_{s = 0}
\]

\textbf{\(s_1 = \frac{1}{2}\) 与 \(s_2 = \frac{1}{2}\)的张量积空间,实际上是两个子空间的直和},在以 ``\(\left|  \uparrow  \right\rangle  \otimes \left|  \downarrow  \right\rangle \)'' 为基的\(\vec{S}^{(AB)}\)的矩阵表达式中体现为块对角化

\textbf{注1}:显然,我们也可以通过将降算符\(S_-\)作用于\(|\uparrow\uparrow\rangle\)态来得到式(\ref{7.61})的态:
\begin{equation}
	S_-|\uparrow\uparrow\rangle = \left(S_-^A + S_-^B\right)|\uparrow\uparrow\rangle = \hbar\left(|\uparrow\downarrow\rangle + |\downarrow\uparrow\rangle\right) \label{7.64}
\end{equation}

\textbf{注2}:统计性质。我们将粒子分为两类:自旋为整数的粒子称为玻色子,自旋为半整数的粒子称为费米子。由此可得出结论:描述玻色子的波函数必须是偶对称函数,而描述费米子的波函数必须是奇反对称函数。这导致两类粒子遵循不同的统计规律:费米子遵循费米-狄拉克统计,玻色子遵循玻色-爱因斯坦统计。

自旋-统计关系的一个重要推论是泡利不相容原理:

\textbf{泡利不相容原理: 任何两个或多个费米子不能同时占据同一个量子态。 }

因此,若两个(或多个)电子处于相同的自旋态(对称自旋态),则它们不能位于同一位置——因为位置态也会是对称态,进而导致总波函数为对称态,这与费米子的波函数要求矛盾。

\subsection{Combining the Angular Momenta of Two States}
一般情况,我们以旋转群 \(SO(3)\) 为例,假设\textbf{两个子系统各自处于总角动量确定的状态}——从数学角度来说,这意味着它们按照 \(SO(3)\) 的不可约表示变换——对应的总角动量量子数分别为 \(j_1\) 和 \(j_2\). 我们希望将子系统的张量积表示为若干希尔伯特空间的直和形式:
\begin{equation}
	\mathcal{H}_{j_1} \otimes \mathcal{H}_{j_2} = \bigoplus_j \mathcal{H}_j
\end{equation}

每个 \(\mathcal{H}_j\) 描述的是整个系统总角动量量子数为 \(j\) 的状态。从物理意义上看,这种分解等价于将两个子系统的角动量相加,从而得到整个系统总角动量的可能取值。

例如,在氢原子中,质子和电子都因自旋而具有 \(\hbar/2\) 的角动量,此外电子绕质子的轨道运动还可能带来额外的角动量。若我们希望将原子视为一个整体,关注的重点就会是整个系统的总角动量,而非电子和质子各自的角动量。
\subsection{ 两个态的角动量组合}
考虑一个封闭系统内包含两个子系统,第一个系统的总角动量量子数为 \(j_1\) ,第二个系统的总角动量量子数为 \(j_2\). 考虑:基态表示、组合系统态空间、算符及整个系统的态构造。


\subsubsection{ 子系统的基态表示}
第一个系统的态基为:
\begin{equation}
	\left\{ \left| j_1, m_1 \right> \right\}, \quad \text{其中 } m_1 \in \left\{ -j_1, -j_1+1, \dots, j_1 \right\} 
\end{equation}
式中 \(m_1\) 为第一个系统角动量的  \(z\)  分量量子数,取值共 \(2j_1+1\) 个(对应角动量在 \(z\) 轴的不同投影)。

同理,第二个系统的态基为:
\begin{equation}
	\left\{ \left| j_2, m_2 \right> \right\}, \quad \text{其中 } m_2 \in \left\{ -j_2, -j_2+1, \dots, j_2 \right\} 
\end{equation}
其 \(z\) 分量量子数 \(m_2\) 的取值共 \(2j_2+1\) 个。


\subsubsection{ 组合系统的态空间}
由恒等变换,组合系统的任意态可表示为子系统基态的张量积线性组合:
\begin{equation}\label{6.4}
	\ket{j_1 j_2; j m} = \sum_{m_1, m_2} \bra{j_1 m_1; j_2 m_2}\ket{j_1 j_2; j m} \ket{j_1 m_1}\ket{	j_2 m_2}
\end{equation}

由于 \(m_1\) 和 \(m_2\) 可独立取值,组合系统的总态数为 \((2j_1+1)(2j_2+1)\) ,这也正是张量积空间 \(\mathcal{H}_{j_1} \otimes \mathcal{H}_{j_2}\) 的维度。

我们的核心目标是:理解式(\ref{6.4})中的哪些线性组合对应“系统整体具有确定角动量”的态,即总角动量算符的本征态。

\subsubsection{总角动量算符的表示}
在量子力学中,组合系统的总角动量算符定义为两个子系统角动量算符的“张量积和”:
\begin{equation}
	\hat{J} = \left( \hat{J}_1 \otimes \mathbb{I}_{\mathcal{H}_{j_2}} \right) + \left( \mathbb{I}_{\mathcal{H}_{j_1}} \otimes \hat{J}_2 \right) 
\end{equation}
为简化记号,\textbf{通常省略张量积和单位算符},直接写为:
\begin{equation}
	\hat{J} = \hat{J}_1 + \hat{J}_2 
\end{equation}
对总角动量算符取平方(需注意张量积的分配律):
\begin{equation}
	\hat{J}^2 = \hat{J}_1^2 + \hat{J}_2^2 + 2\hat{J}_1 \cdot \hat{J}_2 \label{6.7b}
\end{equation}
式中 \(\hat{J}_1 \cdot \hat{J}_2 = \hat{J}_{1x}\hat{J}_{2x} + \hat{J}_{1y}\hat{J}_{2y} + \hat{J}_{1z}\hat{J}_{2z}\) (角动量分量的标量积)。
得到:
\[\boxed{
{\hat J_1} \cdot {\hat J_2} = \frac{1}{2}\left( {{{\hat J}^2} - \hat J_1^2 - \hat J_2^2} \right)
}\]
比如轨道-自旋耦合  \(\displaystyle \hat{S} \cdot \hat{L} = \frac{1}{2} (\hat{J}^2 - \hat{L}^2 - \hat{S}^2)\) 
\subsubsection{总角动量算符的展开(用升降算符)}
为明确  \(\hat{J}^2\)  对基态 \(\left| j_1, m_1 \right>\left| j_2, m_2 \right>\) 的作用,需用 \(\textbf{升降算符}\) ( \(\hat{J}_+ = \hat{J}_x + i\hat{J}_y\) , \(\hat{J}_- = \hat{J}_x - i\hat{J}_y\) )重写 \(\hat{J}_1 \cdot \hat{J}_2\) 。

首先,由升降算符的定义可得角动量直角分量:
\begin{equation}
	\hat{J}_x = \frac{\hat{J}_+ + \hat{J}_-}{2}, \quad \hat{J}_y = \frac{\hat{J}_+ - \hat{J}_-}{2i}
\end{equation}
将其代入标量积:
\begin{align}
	2\hat{J}_1 \cdot \hat{J}_2 &= 2\left( \hat{J}_{1x}\hat{J}_{2x} + \hat{J}_{1y}\hat{J}_{2y} + \hat{J}_{1z}\hat{J}_{2z} \right) \\
	&= 2\left( \frac{\hat{J}_{1+}+\hat{J}_{1-}}{2} \cdot \frac{\hat{J}_{2+}+\hat{J}_{2-}}{2} + \frac{\hat{J}_{1+}-\hat{J}_{1-}}{2i} \cdot \frac{\hat{J}_{2+}-\hat{J}_{2-}}{2i} + \hat{J}_{1z}\hat{J}_{2z} \right) \\
	&= \hat{J}_{1+}\hat{J}_{2-} + \hat{J}_{1-}\hat{J}_{2+} + 2\hat{J}_{1z}\hat{J}_{2z}
\end{align}

将此结果代入式(\ref{6.7b}),总角动量平方算符最终展开为:
\begin{equation}
	\boxed{\hat{J}^2 = \hat{J}_1^2 + \hat{J}_2^2 + \hat{J}_{1+}\hat{J}_{2-} + \hat{J}_{1-}\hat{J}_{2+} + 2\hat{J}_{1z}\hat{J}_{2z}}
	\label{6.9}
\end{equation}
式中各项均能直接作用于基态 \(\left| j_1, m_1 \right>\left| j_2, m_2 \right>\) (利用升降算符对基态的作用规则:)
\begin{align}
	\hat{J}_+ \left| j, m \right> = \hbar\sqrt{j(j+1)-m(m+1)} \left| j, m+1 \right>\\
	\hat{J}_- \left| j, m \right> = \hbar\sqrt{j(j+1)-m(m-1)} \left| j, m-1 \right>
\end{align}


\subsubsection{总角动量本征态的构造}
我们从“最高权重态”(总角动量  \(z\)  分量最大的态)开始构造总角动量本征态。

\paragraph{1. 最高权重态( \(j = j_1 + j_2\) )}

考虑两个子系统均沿 \(z\) 轴最大投影的态: \(\left| j_1, j_1 \right>\left| j_2, j_2 \right>\) 。

首先验证其为 \(\hat{J}_z\) 的本征态:由 \(\hat{J}_z = \hat{J}_{1z} + \hat{J}_{2z}\) ,代入 \(\hat{J}_{z} \left| j, m \right> = \hbar m \left| j, m \right>\) 得:
\begin{equation}
	\hat{J}_z \left| j_1, j_1 \right>\left| j_2, j_2 \right> = (j_1 + j_2)\hbar \left| j_1, j_1 \right>\left| j_2, j_2 \right> \tag{6.10}
\end{equation}
可见其为 \(\hat{J}_z\) 的本征态,本征值为 \((j_1+j_2)\hbar\) 。

再验证其为 \(\hat{J}^2\) 的本征态:将式(\ref{6.9})代入,注意到 \(\hat{J}_{1+} \left| j_1, j_1 \right> = 0\) (最高投影态被上升算符湮灭)、 \(\hat{J}_{2+} \left| j_2, j_2 \right> = 0\) ,且 \(\hat{J}_1^2 \left| j_1, m_1 \right> = j_1(j_1+1)\hbar^2 \left| j_1, m_1 \right>\) ,可得:
\begin{align}
	\hat{J}^2 \left| j_1, j_1 \right>\left| j_2, j_2 \right> &= \left( \hat{J}_1^2 + \hat{J}_2^2 + 2\hat{J}_{1z}\hat{J}_{2z} \right) \left| j_1, j_1 \right>\left| j_2, j_2 \right> \\
	&= \left[ j_1(j_1+1) + j_2(j_2+1) + 2j_1j_2 \right] \hbar^2 \left| j_1, j_1 \right>\left| j_2, j_2 \right> \\
	&= (j_1+j_2)(j_1+j_2+1)\hbar^2 \left| j_1, j_1 \right>\left| j_2, j_2 \right>
\end{align}
因此,该态是 \(\hat{J}^2\) 的本征态,对应总角动量量子数 \(j = j_1 + j_2\) ,记为:
\begin{equation}
	\left| j, j \right> = \left| j_1, j_1 \right>\left| j_2, j_2 \right> \label{6.12}
\end{equation}
此态为 \(j = j_1 + j_2\) 对应的“最高权重态”( \(m = j\) )。


\paragraph{2. 用降算符构造其他态( \(m = j-1, j-2, \dots\) )}

利用降算符 \(\hat{J}_- = \hat{J}_{1-} + \hat{J}_{2-}\) , (此时已经固定  \(j\)  )可从最高权重态生成\textbf{同一  \(j\)  }下\textbf{不同  \(m\) } 的态。

首先,降算符对最高权重态 (式(\ref{6.12})的左侧) 的作用:
\begin{equation}
	\hat{J}_- \left| j, j \right> = \hbar\sqrt{j(j+1) - j(j-1)} \left| j, j-1 \right> = \hbar\sqrt{2j} \left| j, j-1 \right> \label{6.13}
\end{equation}

另一方面,将 \(\hat{J}_- = \hat{J}_{1-} + \hat{J}_{2-}\) 作用于式(\ref{6.12})的右侧:
\begin{align}
	(\hat{J}_{1-} + \hat{J}_{2-}) \ket{j_1, j_1} \ket{j_2, j_2} 
	&= \hbar\sqrt{j_1(j_1+1) - j_1(j_1-1)} \ket{j_1, j_1-1} \ket{j_2, j_2} \nonumber \\
	&\quad + \hbar\sqrt{j_2(j_2+1) - j_2(j_2-1)} \ket{j_1, j_1} \ket{j_2, j_2-1}  \nonumber\\
	&= \hbar\sqrt{2j_1} \ket{j_1, j_1-1} \ket{j_2, j_2} + \hbar\sqrt{2j_2} \ket{j_1, j_1} \ket{j_2, j_2-1}\label{6.14}
\end{align}

联立式(\ref{6.13})与式(\ref{6.14}),并代入 \(j = j_1 + j_2\) ,可得:
\begin{equation}
	\left| j, j-1 \right> = \sqrt{\frac{j_1}{j}} \left| j_1, j_1-1 \right>\left| j_2, j_2 \right> + \sqrt{\frac{j_2}{j}} \left| j_1, j_1 \right>\left| j_2, j_2-1 \right> \label{6.15}
\end{equation}
该态是 \(\hat{J}_z\) 的本征态 (本征值 \((j-1)\hbar\) ), 且因 \([\hat{J}^2, \hat{J}_-] = 0\) ,其总角动量量子数仍为 \(j = j_1 + j_2\) 。重复应用 \(\hat{J}_-\) ,可生成 \textbf{该 (固定的)  \(j\) } 下所有 \(m\) ( \(m = j, j-1, \dots, -j\) )的态。

\paragraph{3. 较低总角动量的态( \(j' = j_1 + j_2 - 1, \dots, |j_1 - j_2|\) )}
粗略地说,最高的总角动量发生在各单角动量平行排列时候,最低地角动量发生在它们反平行排列时,于是可知
\[
\left\lfloor {{j_1} - {j_2}} \right\rfloor  \le j \le {j_1} + {j_2}
\]
当两个子系统不共线时,系统总角动量量子数小于 \(j_1 + j_2\) , 大于  \(|j_1 - j_2|\) 

以 \(j'= j - 1\) 为例,其最高权重态( \(m = j'\) )记为 \(\left| j-1, j-1 \right>\) ,需满足两个条件:

1.  \(\hat{J}_z\) 守恒: \(m_1 + m_2 = j-1 = j_1 + j_2 - 1\) ,因此 \((m_1, m_2)\) 在满足  \(m_1 + m_2 = j-1 = j_1 + j_2 - 1;{m_1} \le {j_1}{\rm{; }}{m_2} \le {j_2}\)  的前提下所有组合都可以组合为最高权重态

2. 正交性条件:与 \(\left| j, j-1 \right>\) (式(\ref{6.15}))正交(因二者是厄米算符 \(\hat{J}^2\) 的不同本征值对应的本征态)。

由此可构造:
\begin{equation}
	\left| j-1, j-1 \right> = \sqrt{\frac{j_2}{j}} \left| j_1, j_1-1 \right>\left| j_2, j_2 \right> - \sqrt{\frac{j_1}{j}} \left| j_1, j_1 \right>\left| j_2, j_2-1 \right>
\end{equation}
式中负号确保正交性。重复应用 \(\hat{J}_-\) ,可生成 \(j = j_1 + j_2 - 1\) 下所有 \(m\) 的态。

\noindent \textbf{备注 I:}对于每个总角动量  \(j_n\)  有简并度  \(2j_n+1\) 

\subsubsection{总态数验证}
由经典类比可猜想:系统总角动量量子数的取值范围为 \(j = |j_1 - j_2|, |j_1 - j_2| + 1, \dots, j_1 + j_2\) 。我们通过态数验证该猜想:

不妨设 \(j_1 \geq j_2\) ,则总态数为各 \(j\) 对应的态数(每个 \(j\) 对应 \(2j + 1\) 个 \(m\) )之和:
\begin{align}
	\sum_{j = j_1 - j_2}^{j_1 + j_2} (2j + 1) &= 2\sum_{j = j_1 - j_2}^{j_1 + j_2} j + (2j_2 + 1) \\
	&= 2 \cdot \frac{(j_1 + j_2) + (j_1 - j_2)}{2} \cdot (2j_2 + 1) + (2j_2 + 1) \\
	&= 2j_1(2j_2 + 1) + (2j_2 + 1) \\
	&= (2j_1 + 1)(2j_2 + 1)
\end{align}
该结果与张量积空间的维度(式(\ref{6.4})的总态数)完全一致,证明总角动量量子数的取值范围猜想正确。


\subsubsection{克莱布施-戈登系数}
在角动量组合中,“耦合基”(总角动量本征态 \(\left| j, m \right>\) )与“非耦合基”(子系统基态张量积 \(\left| j_1, m_1 \right> \otimes \left| j_2, m_2 \right>\) )之间的变换系数称为 \(\textbf{克莱布施-戈登系数}\) ,定义为:
\begin{equation}
	C_{j,m}^{{j_1},{m_1};{j_2},{m_2}} = {\rm{ }}\left\langle {{{j_1},{m_1};{j_2},{m_2}}}
	\mathrel{\left | {\vphantom {{{j_1},{m_1};{j_2},{m_2}} {{j_1},{j_2};j,m}}}
		\right. \kern-\nulldelimiterspace}
	{{{j_1},{j_2};j,m}} \right\rangle {\rm{  }}
\end{equation}
\(\ket{j_1, m_1; j_2, m_2}\)为非耦合基(子系统基的张量积),\(\ket{j_1,j_2;j, m}\)为耦合基(总角动量本征态)

\paragraph{物理意义}:克莱布施-戈登系数表示“当系统处于总角动量态\(\left| j, m \right>\)时,测量子系统发现其处于\(\left| j_1, m_1 \right>\)和\(\left| j_2, m_2 \right>\)的概率幅”,测量概率为系数的平方(\(|C_{j,m}^{j_1,m_1;j_2,m_2}|^2\))。
\begin{problembox}{Calculation of Degeneracy and Energy Levels}
	Consider a particle in a central potential with orbital angular momentum $\hat{\mathbf{L}}$ such that $L^2 = 6\,\hbar^2$ (implying $l=2$ since $L^2 = \hbar^2 l(l+1)$) and spin angular momentum $\hat{\mathbf{S}}$ such that $S^2 = 3/4\,\hbar$ (implying $s=1/2$ since $S^2 = \hbar^2 s(s+1)$). The spin-orbit interaction is given by:
	\[
	H_{\text{int}} = A \hat{\mathbf{S}} \cdot \hat{\mathbf{L}}
	\]
	where $A$ is a constant. Determine the energy levels and their degeneracies associated with this interaction.
\end{problembox}
\textbf{Solution:}

The energy levels are found by evaluating the expectation value of $H_{\text{int}}$. Using the identity:
\[
\hat{\mathbf{S}} \cdot \hat{\mathbf{L}} = \frac{1}{2} (\hat{\mathbf{J}}^2 - \hat{\mathbf{L}}^2 - \hat{\mathbf{S}}^2)
\]
where $\hat{\mathbf{J}} = \hat{\mathbf{L}} + \hat{\mathbf{S}}$ is the total angular momentum operator, with eigenvalues $\hat{\mathbf{J}}^2 = \hbar^2 j(j+1)$.

For $l=2$ and $s=1/2$, the possible total angular momentum quantum numbers $j$ are:
\[
j = |l-s|, \ldots, l+s \quad \Rightarrow \quad j_1 = \frac{3}{2}, \quad j_2 = \frac{5}{2}.
\]

Compute the eigenvalue of $\hat{\mathbf{S}} \cdot \hat{\mathbf{L}}$ for each $j$:

\begin{enumerate}[label=(\arabic*)]
	\item For $j = \dfrac{3}{2}$:
	\begin{align*}
		\langle \hat{\mathbf{S}} \cdot \hat{\mathbf{L}} \rangle_{j=3/2} 
		&= \frac{\hbar^2}{2} \left[ j(j+1) - l(l+1) - s(s+1) \right] \\
		&= \frac{\hbar^2}{2} \left[ \frac{3}{2}\left(\frac{5}{2}\right) - 2(3) - \frac{1}{2}\left(\frac{3}{2}\right) \right] \\
		&= -\frac{3}{2}\hbar^2.
	\end{align*}
	Thus, the energy level is:
	\[
	E_{3/2} = A \langle \hat{\mathbf{S}} \cdot \hat{\mathbf{L}} \rangle_{j=3/2} = -\frac{3}{2} A \hbar^2.
	\]
	
	\item For $j = \dfrac{5}{2}$:
	\begin{align*}
		\langle \hat{\mathbf{S}} \cdot \hat{\mathbf{L}} \rangle_{j=5/2} 
		&= \frac{\hbar^2}{2} \left[ j(j+1) - l(l+1) - s(s+1) \right] \\
		&= \frac{\hbar^2}{2} \left[ \frac{5}{2}\left(\frac{7}{2}\right) - 2(3) - \frac{1}{2}\left(\frac{3}{2}\right) \right] \\
		&= \hbar^2.
	\end{align*}
	Thus, the energy level is:
	\[
	E_{5/2} = A \langle \hat{\mathbf{S}} \cdot \hat{\mathbf{L}} \rangle_{j=5/2} = A \hbar^2.
	\]
\end{enumerate}

The degeneracy of each level is given by $ 2j_n + 1$:
\begin{itemize}
	\item For $\displaystyle j = 3/2, E_{3/2} = -\frac{3}{2} A \hbar^2$, degeneracy $g_{3/2} = 2 \times \frac{3}{2} + 1 = 4$.
	\item For $j = 5/2, E_{5/2} = A \hbar^2$, degeneracy $g_{5/2} = 2 \times \frac{5}{2} + 1 = 6$.
\end{itemize}
	\begin{problembox}{CG系数的计算}
		考虑两个态: \(j_1 = 1\), \(j_2 = 1/2\), 求解耦合的态的表达
	\end{problembox}
	\textbf{Solution:}
	
	可能的 \(j\) 值:\(3/2\) 和 \(1/2\)。
	
	\noindent\textbf{1. 计算 \(j = 3/2\) 的CG系数} 可知\(m=3/2,1/2,-1/2,-3/2\)
	\begin{itemize}
		\item \textbf{最高权态} \(m = 3/2\):
		\[
		|3/2, 3/2\rangle = |1, 1\rangle |1/2, 1/2\rangle, \quad \langle 1,1; 1/2,1/2 | 3/2,3/2\rangle = 1.
		\]
		
		\item \textbf{用 \(J_-\) 求 \(m = 1/2\)}:
		\[
		J_- |3/2,3/2\rangle = \hbar\sqrt{3} |3/2,1/2\rangle,
		\]
		又
		\[
		J_- |1,1\rangle |1/2,1/2\rangle = \hbar\left( \sqrt{2} |1,0\rangle |1/2,1/2\rangle + |1,1\rangle |1/2,-1/2\rangle \right).
		\]
		比较得
		\[
		|3/2,1/2\rangle = \sqrt{\frac{2}{3}} |1,0\rangle |1/2,1/2\rangle + \sqrt{\frac{1}{3}} |1,1\rangle |1/2,-1/2\rangle.
		\]
		\item \textbf{用 \(J_-\) 求 \(m = -1/2\)}:
		\[
		J_- |3/2,1/2\rangle = 2\hbar |3/2,-1/2\rangle,
		\]
		计算左边得
		\[
		|3/2,-1/2\rangle = \sqrt{\frac{1}{3}} |1,-1\rangle |1/2,1/2\rangle + \sqrt{\frac{2}{3}} |1,0\rangle |1/2,-1/2\rangle.
		\]
		\item \textbf{用 \(J_-\) 求 \(m = -3/2\)}:
		\[
		J_- |3/2,-1/2\rangle = \hbar\sqrt{3} |3/2,-3/2\rangle,
		\]
		计算左边得
		\[
		|3/2,-3/2\rangle = |1,-1\rangle |1/2,-1/2\rangle.
		\]
	\end{itemize}
	
	\vspace{0.5em}
	\noindent\textbf{2. 计算 \(j = 1/2\) 的CG系数}
	\begin{itemize}
		\item \textbf{最高权态} \(m = 1/2\):设
		\[
		|1/2,1/2\rangle = a |1,1\rangle |1/2,-1/2\rangle + b |1,0\rangle |1/2,1/2\rangle.
		\]
		\textbf{与 \(|3/2,1/2\rangle\) 正交}:
		\[
		\sqrt{\frac{2}{3}} b + \sqrt{\frac{1}{3}} a = 0 \quad \Rightarrow \quad a = -\sqrt{2} b.
		\]
		\textbf{归一化}:\(|a|^2 + |b|^2 = 1 \Rightarrow 3|b|^2 = 1 \Rightarrow b = 1/\sqrt{3}\)(取正),\(a = -\sqrt{2/3}\)。所以
		\[
		|1/2,1/2\rangle = -\sqrt{\frac{2}{3}} |1,1\rangle |1/2,-1/2\rangle + \sqrt{\frac{1}{3}} |1,0\rangle |1/2,1/2\rangle.
		\]
		\item \textbf{用 \(J_-\) 求 \(m = -1/2\)}:
		\[
		J_- |1/2,1/2\rangle = \hbar |1/2,-1/2\rangle,
		\]
		计算左边得
		\[
		|1/2,-1/2\rangle = \sqrt{\frac{2}{3}} |1,-1\rangle |1/2,1/2\rangle - \frac{1}{\sqrt{3}} |1,0\rangle |1/2,-1/2\rangle.
		\]
	\end{itemize}

\newpage
\section{氢原子}
\subsection{氢原子的解}
从中心势\(V(r)\)
\[i\hbar\frac{\partial H}{\partial t}=H\ket{\psi } ,\hat{H}=\frac{\hat{p}^{2}}{2m}+V, \hat{p}=-i\hbar\nabla\]

转换到球坐标系

\[
\nabla^{2}=\frac{1}{r^{2}}\frac{\partial}{\partial r}\left(r^{2}\frac{\partial}{\partial r}\right)+\frac{1}{r^{2}\sin\theta}\frac{\partial}{\partial\theta}\left(\sin\theta\frac{\partial}{\partial\theta}\right)+\frac{1}{r^{2}\sin^{2}\theta}\frac{\partial^{2}}{\partial\phi^{2}}
\]

分离变量\[\psi(r,\theta,\phi)=R(r)Y(\theta,\phi)\]

以及归一化条件:
\[\int \psi  (r,\theta ,\phi ){r^2}\sin \theta {\mkern 1mu} dr{\mkern 1mu} d\theta {\mkern 1mu} d\phi  = \int R (r){r^2}{\mkern 1mu} dr\int Y (\theta ,\phi )\sin \theta d\theta {\mkern 1mu} d\phi =1\]
\[
\begin{array}{l}
	\displaystyle \int R(r) r^{2} \, dr = 1 \\
	\displaystyle \int {Y_l^m\left( {\theta ,\phi } \right)} \sin \theta {\mkern 1mu} d\theta d\phi  = 1
\end{array}
\]

得到

\begin{equation}\label{central_equation}
	\frac{1}{R}\frac{d}{{dr}}\left( {{r^2}\frac{{dR}}{{dr}}} \right) - \frac{{2m{r^2}}}{{{\hbar ^2}}}[V(r) - E] = l( l + 1)
\end{equation}

\begin{equation}\label{H_V}
	\frac{1}{Y}\left\{ {\frac{1}{{\sin \theta }}\frac{\partial }{{\partial \theta }}\left( {\sin \theta \frac{{\partial Y}}{{\partial \theta }}} \right) + \frac{1}{{{{\sin }^2}\theta }}\frac{{{\partial ^2}Y}}{{\partial {\phi ^2}}}} \right\} =  - l(l + 1)		
\end{equation}

由
\[ L_z = - i\hbar \frac{\partial }{{\partial \phi }}\]
\[L_z^2 =  - {\hbar ^2}\left[ {\frac{1}{{\sin \theta }}\frac{\partial }{{\partial \theta }}\left( {\sin \theta \frac{\partial }{{\partial \theta }}} \right) + \frac{1}{{{{\sin }^2}\theta }}\frac{{{\partial ^2}}}{{\partial {\phi ^2}}}} \right]\]

求解(\ref{H_V})式 (省略了通过分离变量法引入磁量子数 \(m\)的过程) 得
\[Y_l^m(\theta ,\phi ) = \sqrt {\frac{{(2l + 1)}}{{4\pi }} - \frac{{(l - m)!}}{{(l + m)!}}} \;\;{\mkern 1mu} {\kern 1pt} {e^{im\phi }}P_l^m(\cos \theta )\]

其中 $\displaystyle Y_0^0 = \frac{1}{2}\sqrt {\frac{1}{\pi }} $, 以及
\[{L^2}Y_l^m(\theta ,\phi ) = l(l + 1){\hbar ^2} \quad {L_z}Y_l^m(\theta ,\phi ) = m\hbar \]

现在我们考虑径向方程
\[ V(r) = -\frac{e^{2}}{4\pi\varepsilon_{0}} \frac{1}{r} \] 

令 $\displaystyle u = rR(r)$, 由(\ref{central_equation})可得
\[
-\frac{\hbar^{2}}{2m_e} \frac{d^{2}u}{dr^{2}} + \left[ -\frac{e^{2}}{4\pi\varepsilon_{0}} \frac{1}{r} + \frac{\hbar^{2}}{2m_e} \frac{l(l+1)}{r^{2}} \right] u = Eu
\]

通过渐进行为、级数展开等办法可以得到径向解 $\displaystyle {R_{nl}}\left( r \right)$, 归一化后得到氢原子的解
\[\boxed{
\psi_{n\ell m}(r,\theta,\phi) = 
\sqrt{ \left( \frac{2}{n a} \right)^3 \frac{(n-\ell-1)!}{2n[(n+\ell)!]} } 
e^{-r/(n a)} 
\left( \frac{2r}{n a} \right)^{\ell}
L_{n-\ell-1}^{2\ell+1}\!\left( \frac{2r}{n a} \right)
Y_{\ell}^{m}(\theta, \phi)
}
\]

值得注意的是,氢原子的基态 (\(a\) 为玻尔半径)
\[
\scalebox{1.1}{
	$\displaystyle
	\boxed{\psi_{100}(r, \theta, \phi) = \frac{1}{\sqrt{\pi a^3}} e^{-r / a}}$
}
\]
对于中心势场问题,利用 Virial theorem 可以很方便地得到能量
\begin{equation*}\boxed{
		2\langle T \rangle = \langle \mathbf{r} \cdot \nabla V \rangle
	}
\end{equation*}
\paragraph{齐次势能特例}
若 $V(k \mathbf{r}) = k^n V(\mathbf{r})$($n$ 次齐次函数),则欧拉定理给出:
\[
\mathbf{r} \cdot \nabla V = n V
\]
代入位力定理:
\begin{equation*}
	\boxed{2\langle T \rangle = n \langle V \rangle}
\end{equation*}

\begin{itemize}
	\item \textbf{库仑势} ($V \propto -1/r$, $n = -1$):
	在氢原子中存在\[\left\langle T \right\rangle  + \left\langle V \right\rangle  = {E_n}\]
	\[
	2\langle T \rangle = -\langle V \rangle \implies \langle T \rangle = -E_n, \quad  E_n= \frac{1}{2}\langle V \rangle
	\]
\end{itemize}

于是得到基态能量
\begin{align}
E_1&=\frac{1}{2}\langle V \rangle = \frac{1}{2}\int \psi_{100}^* \, V(r) \, \psi_{100} \, dV \nonumber\\
&= \frac{1}{2}\int_0^\infty \int_0^\pi \int_0^{2\pi} 
\left[ \frac{1}{\sqrt{\pi a^3}} e^{-r/a} \right] 
\left( -\frac{e^2}{4\pi\epsilon_0 \, r} \right) 
\left[ \frac{1}{\sqrt{\pi a^3}} e^{-r/a} \right] 
\, r^2 \sin\theta \, dr \, d\theta \, d\phi \\
&= \frac{1}{2} (-\frac{e^2}{4\pi\epsilon_0}) \frac{1}{a} =-13.6 eV\nonumber
\end{align}

其它能级的能量满足:
\[{E_n} = \frac{{{E_n}}}{{{n^2}}}\]

我们可以用五个量子数来描述电子的态,分别是\(n;l,m_l;s,m_s\)
\[
|n;l,m_l;s,m_s\rangle \longrightarrow n\,^{2S+1}L_j
\]

其中:
\begin{itemize}
	\item $n$: 主量子数
	\item $L$: 轨道角动量量子数对应的字母符号(S, P, D, F, \ldots)
	\item $j$: 总角动量量子数
	\item $2S+1$: 自旋多重度
\end{itemize}
\subsection{氢原子能级简并度的计算}

在非相对论量子力学中,氢原子的能级仅由主量子数 \( n \) 决定。其简并度可通过以下步骤计算。

\subsubsection*{1. 量子数回顾}
氢原子束缚态由以下量子数描述:
\begin{itemize}
	\item 主量子数:\( n = 1, 2, 3, \dots \)
	\item 角量子数:\( l = 0, 1, 2, \dots, n-1 \)
	\item 磁量子数:\( m = -l, -l+1, \dots, l \)
\end{itemize}
电子自旋为 \( s = \frac{1}{2} \),自旋投影 \( m_s = \pm \frac{1}{2} \)。

\subsubsection*{2. 空间部分的简并度(不考虑自旋)}
对给定 \( n \),空间简并度为
\[
g_{\text{space}}(n) = \sum_{l=0}^{n-1} (2l + 1).
\]
该和为前 \( n \) 个奇数之和,结果为
\[
g_{\text{space}}(n) = n^2.
\]

\subsubsection*{3. 包含自旋的总简并度}
由于每个空间态可与两个自旋态组合,总简并度为
\[
g_{\text{total}}(n) = 2 \times n^2.
\]

\subsubsection*{4. 示例}
\begin{itemize}
	\item \( n = 1 \):空间简并度 \( = 1 \),总简并度 \( = 2 \)。
	\item \( n = 2 \):空间简并度 \( = 4 \),总简并度 \( = 8 \)。
\end{itemize}

\noindent
\textbf{注意}:当考虑精细结构(如自旋-轨道耦合)或外加磁场时,简并会被部分或完全解除。

\noindent \textbf{备注 I:} 本节中对氢原子的处理完全是非相对论的。当我们使用正确的相对论描述,例如相对论性能量-动量关系和自旋时,我们会发现原本简并的能级会出现额外的分裂。这些效应包括每组量子数 $n$、$l$ 和 $m$ 对应的两个可能的自旋能级,以及由电子运动引起的磁矩相互作用(称为自旋-轨道相互作用),这导致了氢原子的精细结构。当我们进一步考虑电子与质子自旋的相互作用时,我们得到了所谓的超精细结构。所有这些效应都可以通过狄拉克方程(薛定谔方程的相对论对应物)更好地理解,尽管它们也可以通过微扰方法处理以获得令人满意的结果。利用这种微扰方法,我们还可以处理外场的影响,其中磁场导致塞曼效应,电场导致斯塔克效应。

\newpage
\section{不含时微扰理论}
\subsection{微扰理论的出发点}
微扰的出发点:
\begin{equation}
	\begin{array}{l}
		H=H_{0}+\lambda V \\
		E=E_{0}+\lambda E_{1}+\lambda^{2} E_{2}+\cdots \\
		\psi=\psi_{0}+\lambda \psi_{1}+\lambda^{2} \psi_{2}+\cdots
	\end{array}
\end{equation}
\subsection{非简并微扰}
全部的哈密顿量:考虑了微扰 $V$

\[H=H^0+\lambda V\]
我们的出发点是 $H^0$ 及 $V$ 都是已知的, $H^0$ 的选择是任意的,唯一的要求是 $\phi_n^0$ 是 $H^0$ 的本征态,你选择不同的 $H_0$, 那么对应的 $\phi_n^0$ 及 $E_n^0$ 也不同
\begin{tcolorbox}[
	colback=white, 
	colframe=black, 
	sharp corners, 
	boxrule=1pt, 
	width=0.6\textwidth, 
	center,
	valign=center,  % 垂直对齐为居中
	halign=center,   % 水平对齐为居中
	height=3.8cm,    % 设置固定高度,使垂直居中效果更明显
	before upper={\centering},  % 内容前使用居中命令
	after upper={\vfill}  % 内容后添加垂直填充
	]
	\begin{gather*}
		\text{已知} \quad H^0,\quad V,\quad \left\{ \phi_{n}^{(0)} \right\}, \quad E_{n}^{(0)} \\
		\downarrow  \, \text{{\footnotesize 一阶修正}} \\
		\text{已知} \quad \left\{ \phi_{n}^{(0)} \right\}, \quad E_{n}^{(0)}, \quad E_{n}^{(1)}, \quad \left\{ \phi_{n}^{(1)} \right\}  \\
		\downarrow  \, \text{{\footnotesize 二阶修正}} \\
		\text{得到} \quad E_{n}^{(2)}
	\end{gather*}
\end{tcolorbox}
\subsubsection{一阶微扰理论推导}

零阶近似:
\[
H^0 \phi_n^0 = E_n^0 \phi_n^0
\]

一阶近似:
\[
H^0 \phi_n^1 + V \phi_n^0 = E_n^0 \phi_n^1 + E_n^1 \phi_n^0
\]

二阶近似:
\[
H^0 \phi_n^2 + V \phi_n^1 = E_n^0 \phi_n^2 + E_n^1 \phi_n^1 + E_n^2 \phi_n^0
\]

\textbf{能量一阶修正:}
\[
\boxed{E_n^1 = \langle \phi_n^0 | V | \phi_n^0 \rangle}
\]

求一阶波函数修正 \(\phi_n^1\):
\[
(H^0 - E_n^0) \phi_n^1 = (E_n^1 - V) \phi_n^0
\]

利用 \(\left\{ \phi_{n}^{(0)} \right\}\) 展开 \(\phi_n^1\):
\[\phi _n^1 = \sum\limits_{m \ne n} {\bra{\phi _m^0}\ket{\phi _n^1}\phi _m^0} \]
\begin{problembox}{\(m \ne n\) 满足波函数的归一化约定}
	在微扰理论中,我们通常采用中间归一化条件:
	\[
	\braket{\phi_n^0}{\phi_n} = 1
	\]
	其中总波函数 $\ket{\phi_n} = \ket{\phi_n^0} + \lambda \ket{\phi_n^1} + \cdots$。将此展开代入:
	\[
	\bra{\phi_n^0} \bigl( \ket{\phi_n^0} + \lambda \ket{\phi_n^1} + \cdots \bigr) = \braket{\phi_n^0}{\phi_n^0} + \lambda \braket{\phi_n^0}{\phi_n^1} + \cdots = 1
	\]
	由于 $\braket{\phi_n^0}{\phi_n^0} = 1$,要满足上式,必须要求:
	\[
	\braket{\phi_n^0}{\phi_n^1} = 0
	\]
	这意味着,一阶修正波函数 $\ket{\phi_n^1}$ 中不能包含与零级波函数 $\ket{\phi_n^0}$ 平行的分量。在展开式 $\ket{\phi_n^1} = \sum_m c_m^{(1)} \ket{\phi_m^0}$ 中,这个分量正是 $m = n$ 的项 $c_n^{(1)} \ket{\phi_n^0}$。因此,物理上的归一化要求直接导致了 $c_n^{(1)} = 0$,从而在求和中无需包含 $m = n$ 的项。
	
	
\end{problembox}



代入方程:
\[
\sum_{m \neq n} (H^0 - E_n^0) \bra{\phi _m^0}\ket{\phi _n^1} \phi_m^0 = (E_n^1 - V) \phi_n^0
\]

利用正交性简化:
\[
(E_m^0 - E_n^0) \bra{\phi _m^0}\ket{\phi _n^1} = -\langle \phi_m^0 | V | \phi_n^0 \rangle
\]

因此系数为:
\[
\bra{\phi _m^0}\ket{\phi _n^1} = \frac{\langle \phi_m^0 | V | \phi_n^0 \rangle}{E_n^0 - E_m^0}
\]

\textbf{一阶修正波函数:}
\[
\boxed{\phi_n^1 = \sum_{m \neq n} \bra{\phi _m^0}\ket{\phi _n^1} \phi_m^0 = \sum_{m \neq n} \frac{\langle \phi_m^0 | V | \phi_n^0 \rangle}{E_n^0 - E_m^0} \phi_m^0}
\]

\subsubsection{二阶能量修正推导}

二阶近似方程:
\[
H^0 \phi_n^2 + V \phi_n^1 = E_n^0 \phi_n^2 + E_n^1 \phi_n^1 + E_n^2 \phi_n^0
\]

取内积 \(\langle \phi_n^0 | \cdot \)
\[
\langle \phi_n^0 | V | \phi_n^1 \rangle = E_n^1\langle \phi_n^0 | \phi_n^1 \rangle + E_n^2
\]

整理得:
\[
E_n^2 = \langle \phi_n^0 | V | \phi_n^1 \rangle - E_n^1 \langle \phi_n^0 | \phi_n^1 \rangle
\]

由于 \(\boxed{\langle \phi_n^0 | \phi_n^1 \rangle = 0}\)\,, 简化:
\[
E_n^2 = \langle \phi_n^0 | V | \phi_n^1 \rangle
\]

代入一阶波函数展开:
\[
\phi_n^1 = \sum_{m \neq n} \frac{\langle \phi_m^0 | V | \phi_n^0 \rangle}{E_n^0 - E_m^0} \phi_m^0
\]

得到\textbf{二阶能量修正}:
\[
\boxed{E_n^2 = \sum_{m \neq n} \frac{|\langle \phi_m^0 | V | \phi_n^0 \rangle|^2}{E_n^0 - E_m^0}}
\]
\subsection{简并微扰}
从微扰的出发点:
\begin{equation*}
	\begin{array}{l}
		H=H_{0}+\lambda V \\
		E=E_{0}+\lambda E_{1}+\lambda^{2} E_{2}+\cdots \\
		\psi=\psi_{0}+\lambda \psi_{1}+\lambda^{2} \psi_{2}+\cdots
	\end{array}
\end{equation*}

我们想要 $E$ 和 $\psi$ 关于 $\lambda$ 有很好的性质或者说是解析的,那么一个很重要的特点就是当 $\lambda\to 0$ 时,$E\to E_{0}$ 而 $\psi\to \psi_{0}$。下面我们就来看看微扰会产生什么影响。

首先我们来看微扰会造成什么结果。当然如果是非简并的情况,微扰只是在原来的各个能级(本征值)上进行修正,这没有问题。但是如果原来能级是简并的,加上微扰如果破坏了对称性,这个简并能级就会分裂成多个能级了。

在我们讨论非简并微扰的时候,直接取了 $\psi_{m}$,对应的微扰也仅仅对能量 $E_{m}$ 进行修正。当 $\lambda\to 0$ 时,修正后的能量 $E$ 退化回 $E_{m}$,由于非简并,$\psi$ 只能退化到确定的 $\psi_{m}$。因此非简并微扰中直接选定 $\psi_{m}$ 的做法满足我们对 $\lambda$ 解析的期望,没有啥问题。

\subsubsection{简并微扰理论}
首先这时候就有很多本征矢都同属于 $E_{l}$ 的本征值,选哪一个作为 $\phi_{0}$ 才合适呢?

其次假如说加入微扰过后原来的能级 $E_{l}$ 解除了简并,得到了两个非简并的能级,它们对应的本征矢分别是:
\begin{equation}
	\psi^{(1)}=\psi_{0}^{(1)}+\lambda\psi_{1}^{(1)}+\cdots, \quad \psi^{(2)}=\psi_{0}^{(2)}+\lambda\psi_{1}^{(2)}+\cdots.
\end{equation}
当 $\lambda\to 0$ 时,分裂的两个能级必然要退化到 $E_{l}$,同时本征矢 $\psi^{(1)}$、$\psi^{(2)}$ 也必须退化到 $E_{l}$ 子空间中的两个特别的矢量上。也就是说为了保证 $\lambda\to 0$ 的连贯性,$\psi_{0}^{(1)}$ 和 $\psi_{0}^{(2)}$ 不是任意的!

主要的问题是,如果你忽视简并,那么你将会从一个态变成多个态,这是不合理的。 $E_{l}$ 简并时如果要做微扰,不能直接把 $\psi_{0}$ 取作 $\phi_{l}$ 或 $\phi_{m}$,而应设 $\psi_{0}=a_{l}\phi_{l}+a_{m}\phi_{m}$,然后通过某些手段确定 $a_{l},a_{m}$ 的值,才能继续做微扰。

\subsubsection{一级简并微扰}
如果能量 $E_l$ 是简并的,在不影响叙述问题的前提下,我们不妨设 $E_l$ 的子空间中有两个独立的本征矢 $|l\rangle$, $|m\rangle$。设 $\boxed{|\psi_0\rangle = a_l |l\rangle + a_m |m\rangle}$,利用一级关系式
\begin{equation}
	(H_0 - E_l)|\psi_1\rangle = (E_1 - V)|\psi_0\rangle,
\end{equation}
代入 $|\psi_0\rangle$ 得到:
\begin{equation}
	(H_0 - E_l)|\psi_1\rangle = a_l (E_1 - V)|l\rangle + a_m (E_1 - V)|m\rangle.
\end{equation}
两边同时左乘 $\langle l|$ 或 $\langle m|$ 得到:
\begin{align}
	0 &= a_l\langle l|(E_1 - V)|l\rangle + a_m\langle l|(E_1 - V)|m\rangle, \\
	0 &= a_l\langle m|(E_1 - V)|l\rangle + a_m\langle m|(E_1 - V)|m\rangle.
\end{align}
写成矩阵形式如下:
\begin{equation}
	\begin{bmatrix}
		\langle l|V|l\rangle - E_1 & \langle l|V|m\rangle \\
		\langle m|V|l\rangle & \langle m|V|m\rangle - E_1
	\end{bmatrix}
	\begin{bmatrix}
		a_l \\ a_m
	\end{bmatrix} = 0.
\end{equation}
要确定系数 $a_l$, $a_m$,就要先保证
\begin{equation}
	\left|\begin{array}{cc}
		\langle l|V|l\rangle - E_1 & \langle l|V|m\rangle \\
		\langle m|V|l\rangle & \langle m|V|m\rangle - E_1
	\end{array}\right| = 0,
\end{equation}
也就是研究 $V$ 的矩阵
\begin{equation}
	\begin{bmatrix}
		\langle l|V|l\rangle & \langle l|V|m\rangle \\
		\langle m|V|l\rangle & \langle m|V|m\rangle
	\end{bmatrix},
\end{equation}
它的本征值为 $E_1$。\textbf{事实上,微扰矩阵的本征值给出了能量一阶修正},对应有本征矢量 $\begin{bmatrix} a_l \\ a_m \end{bmatrix}$。如果有两个不一样的本征值,那么简并就解除了;如果两个本征值相同,则简并还是没有解除,我们需要在二级微扰中再尝试解除简并。如果两个本征值不同,那么我们就可以确定两组 $a_l$, $a_m$,分别从它们出发得到两个 $|\psi_0\rangle$,然后进一步得到两个微扰修正波函数 $|\psi_1\rangle$。

和非简并的步骤类似,我们得到了 $E_1$ 并且确定了 $a_l$, $a_m$,也就是确定了 $|\psi_0\rangle$ 下面我们要进一步求 $|\psi_1\rangle$。

通过\textbf{展开一阶修正波函数} $\displaystyle \ket{\psi _1}  = \sum\limits_{k \ne l,m} {a_k^{(1)}} \ket{k} $,我们让所有的修正波函数 $|\psi_s\rangle$(其中 $s>0$)都不含 $|\psi_0\rangle = a_l |l\rangle + a_m |m\rangle$ 的成分(必须剔除简并子空间, 否则分母会爆炸), 因此我们将维持类似于非简并情况下的良好性质 $\langle\psi_0|\psi_s^1\rangle=0$。

考虑方程 
\[(H_0 - E_l)|\psi_1\rangle = a_l (E_1 - V)|l\rangle + a_m (E_1 - V)|m\rangle\]
 
两边同时左乘 $\langle k|$($k \neq l,m$)后得到:
\begin{equation}
	a_k^{(1)}(E_k - E_l) = -a_l\langle k|V|l\rangle - a_m\langle k|V|m\rangle,
\end{equation}

即
\begin{equation}
	a_k^{(1)} = \frac{a_l\langle k|V|l\rangle + a_m\langle k|V|m\rangle}{E_l - E_k}.
\end{equation}

\subsection{二级简并微扰}
如果在一级处理中得到的 $E_1$ 的两个本征值不同,那么各个 $|\psi_0\rangle$ 都在一级微扰中妥善处理了,二级微扰就是在 $|\psi_0\rangle$ 和 $|\psi_1\rangle$ 的基础上继续进行非简并处理就可以了。

但是如果一级处理中 $E_1$ 的两个本征值相同,那么还需要在二级微扰中进一步解除简并。这种情况下我们已知 $|\psi_0\rangle = a_l|l\rangle + a_m|m\rangle$,以及 $|\psi_1\rangle = \sum_{k \neq l,m} a_k^{(1)}|k\rangle$,还有 $E_1$,但 $a_l$, $a_m$ 还未确定。

因此我们要用到二级近似的关系式:
\begin{equation}
	(H_0 - E_l)|\psi_2\rangle = (E_1 - V)|\psi_1\rangle + E_2|\psi_0\rangle.
\end{equation}
注意到 $\langle\psi_0|\psi_s\rangle=0 \,(s>0)$, 两边同时左乘 $\langle l|$ 或 $\langle m|$ 得到:
\begin{align}
	0 &= -\sum_{k \neq l,m} a_k^{(1)}\langle l|V|k\rangle + a_l E_2, \\
	0 &= -\sum_{k \neq l,m} a_k^{(1)}\langle m|V|k\rangle + a_m E_2.
\end{align}
代入条件 $\displaystyle a_k^{(1)} = \frac{a_l\langle k|V|l\rangle + a_m\langle k|V|m\rangle}{E_l - E_k}$ 并化简得:
\begin{align}
	&\left(\sum_{k \neq l,m} \frac{\langle l|V|k\rangle\langle k|V|l\rangle}{E_l - E_k} - E_2\right) a_l + \sum_{k \neq l,m} \frac{\langle l|V|k\rangle\langle k|V|m\rangle}{E_l - E_k} a_m = 0, \\
	&\sum_{k \neq l,m} \frac{\langle m|V|k\rangle\langle k|V|l\rangle}{E_l - E_k} a_l + \left(\sum_{k \neq l,m} \frac{\langle m|V|k\rangle\langle k|V|m\rangle}{E_l - E_k} - E_2\right) a_m = 0.
\end{align}
所以要想解除简并,得关键求解上面的方程组,其中二阶能量修正 $E_2$ 就是如下矩阵的特征值:
\begin{equation}
	\begin{bmatrix}
		\displaystyle\sum_{k \neq l,m} \frac{\langle l|V|k\rangle\langle k|V|l\rangle}{E_l - E_k} & \displaystyle\sum_{k \neq l,m} \frac{\langle l|V|k\rangle\langle k|V|m\rangle}{E_l - E_k} \\[12pt]
		\displaystyle\sum_{k \neq l,m} \frac{\langle m|V|k\rangle\langle k|V|l\rangle}{E_l - E_k} & \displaystyle\sum_{k \neq l,m} \frac{\langle m|V|k\rangle\langle k|V|m\rangle}{E_l - E_k}
	\end{bmatrix},
\end{equation}
 $\begin{bmatrix} a_l \\ a_m \end{bmatrix}$ 就是上述矩阵的特征向量。

如果 $V$ 是厄米的,那么矩阵可以写成更加简练的形式:
\begin{equation}
	\begin{bmatrix}
		\displaystyle\sum_{k \neq l,m} \frac{|\langle l|V|k\rangle|^2}{E_l - E_k} & \displaystyle\sum_{k \neq l,m} \frac{\langle l|V|k\rangle\langle k|V|m\rangle}{E_l - E_k} \\[12pt]
		\displaystyle\sum_{k \neq l,m} \frac{\langle m|V|k\rangle\langle k|V|l\rangle}{E_l - E_k} & \displaystyle\sum_{k \neq l,m} \frac{|\langle m|V|k\rangle|^2}{E_l - E_k}
	\end{bmatrix}.
\end{equation}

从上面的推导可以看出,上面的矩阵方法可以直接推广到一个能级对应更多个未扰本征态的情况,只需要简单地把矩阵的规模扩大就可以了。



\subsection{应用:氢原子的一级斯塔克效应}
由均匀的外电场引起的原子能级的变化称为斯塔克效应。这里还是用氢原子举例来说明如何用微扰来解决问题。

实际上容易看出来在电场 $\boldsymbol{E}$ 的作用下,如果电场方向沿 $z$ 轴正方向,则电子有附加能量
\begin{equation}
	H' = eEz = eEr\cos\theta.
\end{equation}
氢原子的波函数我们用 $|nlm\rangle$ 来表示,则我们的任务是计算 \[\langle nlm|H'|n'l'm'\rangle = \langle nlm|eEr\cos\theta|n'l'm'\rangle\]

我们需要借助对称性来排除一些项的计算。
即跃迁选择定律(Selection Rules):
\begin{align}
	\Delta l &= l_f - l_i = \pm 1 \quad & \text{(角动量变化)} \\
	\Delta m &= m_f - m_i = 0, \pm 1 \quad & \text{(角动量投影变化)}
\end{align}
其中 $l$ 为轨道角动量量子数,$m$ 为 $z$ 分量

显然非简并态,例如基态氢原子,是不可能有一级斯塔克效应的,因为这个态下波函数只有一个,则 $\langle nlm|H'|n'l'm'\rangle \equiv 0$。

下面回到氢原子。当 $n=1$ 时,只能有 $l=0,m=0$,$H'$ 的矩阵元为 0.

当氢原子 $n=2$ 时,$(l,m)$ 的可能取值有 4 个,即:$(0,0),(1,0),(1,1),(1,-1)$,简并度为 4。

$\Delta l = {l_f} - {l_i} =  \pm 1$ 是必须要要遵守的,而\(\Delta m = m_f - m_i = 0, \pm 1\),我们需要进行一点的判断:因为微扰 \(H' = eEz = eEr\cos\theta\) 与波函数 $\left| {nlm} \right\rangle $ 的 $\phi $ 部分 ${e^{im\phi }}$ 无关,因此要求 \(\Delta m = m_f - m_i = 0\)

另一种解释:考虑到 \(\left[L_{i}, x_{j}\right] = i\hbar\epsilon_{i j k} x_{k}\), 可知 \(z\) 与 \(L_z\) 对易,即 \(H'\) 与\(z\) 对易,也就是说如果 \(m \ne m'\), 那么 $\langle nlm|H'|n'l'm'\rangle $ 就等于 \(0 \)

我们发现第一激发态中只有 $\langle 200 | H' | 210 \rangle$ 以及 $\langle 210 | H' | 200 \rangle$ 不为 0。

因此 $H'$ 给出矩阵:
\begin{equation}
	\begin{bmatrix}
		0 & \langle 200 | H' | 210 \rangle & 0 & 0 \\
		\langle 210 | H' | 200 \rangle & 0 & 0 & 0 \\
		0 & 0 & 0 & 0 \\
		0 & 0 & 0 & 0
	\end{bmatrix}.
\end{equation}
我们需要计算这两个矩阵元,但由于 $H'$ 本身是厄米的,因此有 $\langle 210 | H' | 200 \rangle = \langle 200 | H' | 210 \rangle^{*}$,我们只需要算一次就可以了。
\paragraph{补充:}多电子原子与总角动量
在 $LS$ 耦合下,电偶极跃迁规则为:
\begin{align*}
	\Delta L &= 0, \pm 1 \quad (L=0 \to L'=0 \text{ 禁止}) \\
	\Delta S &= 0 \quad \text{(电偶极算符不作用于自旋)} \\
	\Delta J &= 0, \pm 1 \quad (J=0 \to J'=0 \text{ 禁止})
\end{align*}




\newpage
\section{氢原子的精细结构}
对无外场的氢原子,我们至少需要考虑三个哈密顿量,分别是 Bohr, Relativity, Orbit-Spin Coupling 即:
\[\hat H = \hat H_{Bohr} + \hat H_{Rel} + \hat H_{S-O}\]

氢原子玻尔哈密顿量:
\begin{equation*}
	\begin{aligned}
		H_{\text{Bohr}} &= T + V \\
		&= \frac{{{p^2}}}{{2m}} - \frac{e^{2}}{4\pi\varepsilon_{0}}\frac{1}{r}
	\end{aligned}
\end{equation*}

我们用力学量组\(\{ H,{L^2},{L_z},S^2,S_z\} \)确定的本征态,取为$ \left| {n,l,s,m_l,m_s} \right\rangle $

我们需要用到几个有用的等式:
\[\boxed{
	\begin{aligned}
		\left\langle \frac{1}{r} \right\rangle &= \frac{1}{n^{2}a} \\
		\left\langle \frac{1}{r^{2}} \right\rangle 
		&= \frac{1}{\left(l+\frac{1}{2}\right) n^{3} a^{2}} \\
		\left\langle \frac{1}{r^{3}} \right\rangle 
		&= \frac{1}{l\left(l+\frac{1}{2}\right)(l+1) n^{3} a^{3}}
	\end{aligned}
}
\]
其中,\(a\) 是玻尔半径
\subsection{相对论修正}
考虑相对论动能,并近似展开:
\begin{equation*}
	\begin{aligned}
		T &= \sqrt{p^2c^2 + m^2c^4} - mc^2 \\
		&= \frac{p^2}{2m} - \frac{p^4}{8m^3c^2} + \cdots
	\end{aligned}
\end{equation*}

修正量作为微扰:
\[
\text{修正量 } H_{Rel} = -\frac{p^{4}}{8m^{3}c^{2}} \text{ 看作微扰变量}
\]
幸运的是,$H_{Rel}$ 与 ${L^2},{L_z}$ 对易,我们可以继续使用原来的态矢
由此${E_n}$的一阶能量修正为:
\[
\langle n | H_{Rel} | n \rangle = -\frac{1}{8m^{3}c^{2}} \langle n | p^{4} | n \rangle
\]

由玻尔哈密顿量的本征方程:
\[
H_{\text{Bohr}} | n \rangle = E_{n} | n \rangle \Rightarrow \left( \frac{p^{2}}{2m} + V \right) | n \rangle = E_{n} | n \rangle
\]

可得:
\[
p^{2} | n \rangle = 2m(E_{n} - V) | n \rangle
\]

于是:
\[
\langle n | H_{Rel} | n \rangle  =  - \frac{1}{{8{m^3}{c^2}}}\left\langle {{{p^2}n}}
\mathrel{\left | {\vphantom {{{p^2}n} {{p^2}n}}}
	\right. \kern-\nulldelimiterspace}
{{{p^2}n}} \right\rangle = -\frac{(2m)^{2}}{8m^{3}c^{2}} \langle n | (E_{n} - V)^{2} | n \rangle
\]

简化得:
\[
E_{r}^{1} = -\frac{1}{2mc^{2}} \langle (E_{n} - V)^{2} \rangle
\]

考虑氢原子库仑势:
\[
V(r) = -\frac{1}{4\pi\varepsilon_{0}}\frac{e^{2}}{r}
\]

经过繁复的计算:
\[
E_{r}^{1} = -\frac{(E_{n})^{2}}{2mc^{2}} \left[ \frac{4n}{l + 1/2} - 3 \right]
\]


\subsection{自旋-轨道耦合}
质子绕着电子旋转产生一个磁场$\vec B$,会引入额外的哈密顿量
\[{H_{S - O}} =  - \vec \mu  \cdot \vec B\]
其中$\displaystyle B = \frac{{{\mu _0}I}}{{2r}}$,考虑到
\[I = \frac{e}{T},L = \frac{{2\pi m{r^2}}}{T} = rmv\]
得到
\[\vec B = \frac{1}{{4\pi {\varepsilon _0}}}\frac{e}{{m{c^2}{r^3}}}\vec L\]
同时有
\[\vec \mu  = g\gamma \vec S = 2\left( { - \frac{e}{{2m}}} \right)\vec S =  - \frac{e}{m}\vec S\]
于是得到
\[{H_{S - O}} = \frac{1}{{4\pi {\varepsilon _0}}}\frac{{{e^2}}}{{{m^2}{c^2}{r^3}}}\vec S \cdot \vec L\]
由于托马斯进动,需要在上式再乘一个$\displaystyle \frac{1}{2}$,于是有
\[{H_{S - O}} = \frac{1}{{8\pi {\varepsilon _0}}}\frac{{{e^2}}}{{{m^2}{c^2}{r^3}}}\vec S \cdot \vec L\]

不幸的是,由于存在自旋-轨道耦合作用,自旋量子数和轨道角量子数都不再守恒,考虑 \(H_{S-O}\) 修正后,哈密顿量不再与 ${\vec L, \vec S}$ 对易,幸运的是 ${H_{S - O}}$与${L^2},{S^2}$ 及总角动量平方\({J^2}\), 以及总角动量$\vec J=\vec L+\vec S$ 对易,因此这些量仍然是守恒量
\begin{center}
	\framebox{
			\centering
			这提醒我们需要用新的本征态 \(\displaystyle \left| n,l,s,j,m_j \right\rangle\)
	}
\end{center}
\begin{proofbox}
	\[\hat H = \hat H_{Bohr} + \hat H_{Rel} + \hat H_{S-O}\]
	在 $\hat H =\hat H_{Bohr} + \hat H_{Rel}$ 下,$[H, \mathbf{L}] = 0$ 和 $[H, \mathbf{S}] = 0$,因此 $\mathbf{L}^2$, $L_z$, $\mathbf{S}^2$, $S_z$ 均为守恒量,对应量子数 $l$, $m_l$, $s$, $m_s$。
	\begin{itemize}
	\item $H_{\text{rel}}$ 与 $\mathbf{L}$ 对易,但与 $\mathbf{S}$ 无关,故 $[H_{\text{rel}}, \mathbf{L}] = 0$,$[H_{\text{rel}}, \mathbf{S}] = 0$
	\end{itemize}
	
	但加入 $H_{\text{fs}}$ 后:
	\begin{itemize}
		
		\item $H_{\text{so}}$ 包含 $\mathbf{L} \cdot \mathbf{S}$,导致 $\mathbf{L}$ 和 $\mathbf{S}$ 不单独守恒,因为 $[H_{\text{so}}, L_z] \neq 0$,$[H_{\text{so}}, S_z] \neq 0$。
	\end{itemize}
	
	然而,总角动量 $\mathbf{J} = \mathbf{L} + \mathbf{S}$ 与 $H_{\text{so}}$ 对易:
	\[
	[H_{\text{so}}, \mathbf{J}] = [H_{\text{so}}, \mathbf{L}] + [H_{\text{so}}, \mathbf{S}] = 0,
	\]
	因为 $\mathbf{L} \cdot \mathbf{S}$ 与 $\mathbf{J}^2$ 对易。同时,$H_{\text{rel}}$ 也与 $\mathbf{J}$ 对易。因此,总哈密顿量 $H$ 满足:
	\[
	[H, \mathbf{J}^2] = 0, \quad [H, J_z] = 0,
	\]
	且 $[H, \mathbf{L}^2] = 0$(因为 $H$ 仍为球对称),但 $[H, L_z] \neq 0$,$[H, S_z] \neq 0$。故好量子数变为 $n$, $l$, $j$, $m_j$,其中 $j$ 是总角动量量子数,$m_j$ 是其投影
\end{proofbox}
and we conclude that
 \[E_{\mathrm{S-O}}^{1}=\left\langle H_{\mathrm{S-O}}\right\rangle=\frac{e^{2}}{8 \pi \epsilon_{0}} \frac{1}{m^{2} c^{2}} \frac{\left(\hbar^{2} / 2\right)\left[j(j+1)-\ell(\ell+1)-s(s+1)\right]}{\ell(\ell+1 / 2)(\ell+1) n^{3} a^{3}},\]
or, expressing it all in terms of $E_{n}$:
\[ E_{\mathrm{S-O}}^{1}=\frac{\left(E_{n}\right)^{2}}{m c^{2}}\left\{\frac{n\left[j(j+1)-\ell(\ell+1)-3 / 4\right]}{\ell(\ell+1 / 2)(\ell+1)}\right\} \]
\subsection{Fine Structure}
考虑相对论效应核自旋-轨道耦合后,那么哈密顿修正项为
\[{H_{fs}} = {H_{rel}} + {H_{S - O}}\]

此时,氢原子的电子的哈密顿量为
\[H = {H_{Bohr}} + {H_{rel}} + {H_{S - O}}\]

值得注意的是,尽管涉及的物理机制完全不同,相对论修正与自旋-轨道耦合的量级是相同的(\(E_n^2/mc^2\))。将二者相加,我们得到完整的精细结构公式

\begin{equation}
	E_{\text{fs}}^1 = \frac{(E_n)^2}{2mc^2} \left( 3 - \frac{4n}{j + 1/2} \right). 
\end{equation}

将此式与玻尔公式结合,我们得到包含精细结构的氢原子能级最终结果:

\begin{equation}
		{E_{nj}} =  - \frac{{13.6eV}}{{{n^2}}}\left[ {1 + \frac{{{\alpha ^2}}}{{{n^2}}}\left( {\frac{n}{{j + 1/2}} - \frac{3}{4}} \right)} \right]
\end{equation}

精细结构打破了\(\ell\)的简并性(即对于给定的\(n\),不同的允许\(\ell\)值不再具有相同的能量),但它仍保留了\(j\)的简并性。轨道角动量与自旋角动量的 \(z\) 分量本征值(\(m_l\) 和 \(m_s\))不再是“好的”量子数——定态是这些量不同取值对应状态的线性组合;“好”量子数是$n,\ell ,s,j,{m_j}$
\begin{table}[htbp]
	\centering
	\caption{总结:精细结构下的守恒量}
	\begin{tabular}{l c l}
		\toprule
		物理量 & 是否守恒 & 说明 \\
		\midrule
		$L$ & $\times$ & 自旋-轨道耦合破坏其守恒 \\
		$S$ & $\times$ & 同上 \\
		$\mathbf{J}=\mathbf{L}+\mathbf{S}$ & $\checkmark$ & 关键守恒量 \\
		$\mathbf{J}^{2}$, $J_{z}$ & $\checkmark$ & 好量子数 \\
		能量 $H_{\text{fs}}$ & $\checkmark$ & 哈密顿量本身守恒 \\
		$L^{2}$, $S^{2}$ & $\checkmark$ & (数值守恒) \\
		\bottomrule
	\end{tabular}
	
	\vspace{0.5em}
	\footnotesize
	注:$L^{2}$和$S^{2}$实际上与$H_{\text{so}}\propto\mathbf{L}\cdot\mathbf{S}$对易,所以它们的本征值$l(l+1)\hbar^{2}$、$s(s+1)\hbar^{2}$仍然是好量子数。但$\mathbf{L}$和$\mathbf{S}$的方向不固定。
\end{table}
\subsection{塞曼效应}

当原子置于均匀外磁场 \( B_{\text{ext}} \) 中时,其能级会发生移动,这一现象被称为塞曼效应。对于单个电子,微扰哈密顿量为
\begin{equation}
	H_Z' = -(\vec{\mu}_l + \vec{\mu}_s) \cdot \vec{B}_{\text{ext}}
\end{equation}

其中,
\begin{equation}
	\vec{\mu}_s = -\frac{e}{m} \vec{S}
\end{equation}

是与电子自旋相关的磁偶极矩
\begin{equation}
	\vec{\mu}_l = -\frac{e}{2m} \vec{L}
\end{equation}

是与轨道运动相关的磁偶极矩。因此:
\begin{equation}
	H_Z' = \frac{e}{2m} (\vec{L} + 2\vec{S}) \cdot \vec{B}_{\text{ext}}
\end{equation}

若 \( B_{\text{ext}} \ll B_{\text{int}} \),则精细结构起主导作用,可将 \( H_Z' \) 视为小微扰;

若 \( B_{\text{ext}} \gg B_{\text{int}} \),则塞曼效应起主导作用,精细结构 \( H_{Rel} + H_{S-O} \) 成为微扰;
\begin{table}[htbp]
	\centering
	\caption{总结:外磁场下的守恒量}
	\label{tab:conservation-external-B}
	\begin{tabular}{p{4cm} p{4cm} p{3.2cm} }
		\toprule
		\textbf{情况} & \textbf{守恒量} & \textbf{好量子数}  \\
		\midrule
		无磁场(仅精细结构) & 
		$H$, $J^{2}$, $J_{z}$, $L^{2}$, $S^{2}$ & 
		$n, l, j, m_j$   \\
		\midrule
		弱磁场(反常塞曼) & 
		$H$, $J^{2}$, $J_{z}$, $L^{2}$, $S^{2}$ & 
		$n, l, j, m_j$  \\
		\midrule
		强磁场 & 
		$H$, $L_{z}$, $S_{z}$, $L^{2}$, $S^{2}$ & 
		$n, l, m_l, m_s$  \\
		\bottomrule
	\end{tabular}
\end{table}
\subsubsection{未考虑电子自旋的塞曼效应}
未考虑精细结构的氢原子在磁场强度为 $H$ 的均匀磁场内的情况。

对于带负电 $-e$ 的电子,带电磁作用的薛定谔方程的哈密顿量
\begin{equation}
	H = \frac{1}{2 \mu} \left( \boldsymbol{p} + \frac{e}{c} \boldsymbol{A} \right)^{2} - e \phi,
\end{equation}
其中没有 $\boldsymbol{H}$,但我们知道 $\boldsymbol{H} = \nabla \times \boldsymbol{A}$,写成积分就是
\begin{equation}
	\oint_{\partial S} \boldsymbol{A} \cdot d\boldsymbol{r} = \int_{S} \boldsymbol{H} \cdot d\boldsymbol{S} = \oint_{\partial S} \boldsymbol{H} \cdot \frac{1}{2}\boldsymbol{r} \times d\boldsymbol{r} = \oint_{\partial S} \left( \frac{1}{2} \boldsymbol{H} \times \boldsymbol{r} \right) \cdot d\boldsymbol{r}.
\end{equation}
我们看看能不能有 $\boldsymbol{A} = \frac{1}{2} \boldsymbol{H} \times \boldsymbol{r}$。由于 $\boldsymbol{H}$ 是常量,可以验证确实能满足 $\boldsymbol{H} = \nabla \times \boldsymbol{A}$,于是我们把这个关系式带到哈密顿量中得到:
\begin{align}
	H &= \frac{1}{2 \mu} \left( \boldsymbol{p} + \frac{e}{2c} \boldsymbol{H} \times \boldsymbol{r} \right)^2 + e \phi \nonumber \\
	&= \frac{\boldsymbol{p}^{2}}{2 \mu} + \frac{e}{2 \mu c} \boldsymbol{H} \times \boldsymbol{r} \cdot \boldsymbol{p} + \frac{e^{2}}{8 \mu c^{2}}(\boldsymbol{H} \times \boldsymbol{r}) \cdot (\boldsymbol{H} \times \boldsymbol{r}) - e \phi \nonumber \\
	&= \left( \frac{\boldsymbol{p}^{2}}{2 \mu} - e \phi \right) + \frac{e}{2 \mu c} \boldsymbol{H} \cdot \boldsymbol{L} + \frac{e^{2}}{8 \mu c^{2}}(\boldsymbol{H} \times \boldsymbol{r}) \cdot (\boldsymbol{H} \times \boldsymbol{r}).
\end{align}
注意到 $\boldsymbol{H}$ 是常量,$\boldsymbol{H} \times \boldsymbol{r}$ 和 $\boldsymbol{p}$ 是对易的。

可以看到等号右边第一项就是氢原子本身的哈密顿量,也就是未扰动的哈密顿量 $H_{0}$,而第三项显然远远小于第二项,因此从近似的角度来看取 $H'=\frac{e}{2\mu c}\boldsymbol{H}\cdot \boldsymbol{L}$ 就可以了。

由于氢原子的能级存在简并,因此我们不能用非简并的微扰方法。我们先看看能不能通过一级微扰来解除简并。为了方便,我们取 $\boldsymbol{H}$ 的方向为 $z$ 轴,因此 $H'=\frac{e}{2\mu c}H L_{z}$。然后为了尝试通过微扰解除简并,我们需要计算:
\begin{equation}
	\langle k|H'|m\rangle = \langle k|\frac{e}{2\mu c}H L_z|m\rangle = \frac{eH}{2\mu c} m\hbar \delta_{mk}.
\end{equation}
我们发现矩阵是一个对角阵,写成矩阵就是
\begin{equation}
	\frac{eH\hbar}{2\mu c}
	\begin{bmatrix}
		-l & 0 & 0 & \cdots & 0 \\
		0 & -(l-1) & 0 & \cdots & 0 \\
		\vdots & \vdots & \vdots & \ddots & \vdots \\
		0 & 0 & 0 & \cdots & l
	\end{bmatrix},
\end{equation}
$l$ 是轨道角动量量子数。所以该矩阵的本征值(也就是能量的一级修正值 $E_{1}$)就是:
\begin{equation}
	E_{1} = \frac{eH}{2\mu c} m\hbar, \qquad m=0,\pm 1,\cdots,\pm l.
\end{equation}
它们互不相同,我们发现这个问题中直接通过一级微扰就解除了简并。

\subsubsection{弱场塞曼效应}

当 \( B_{\text{ext}} \ll B_{\text{int}} \) 时,精细结构起主导作用。我们将 \( H_{\text{Bohr}} + H_{\text{fs}}' \) 作为"未微扰"哈密顿量,\( H_Z' \) 作为微扰。此时,"未微扰"的本征态为适用于精细结构的 \( |n\ell j m_j\rangle \),"未微扰"能量为 \( E_{nj} \)

在一阶微扰理论中,塞曼效应的能量修正为:
\begin{equation}
	E_Z^1 = \langle n\ell j m_j | H_Z' | n\ell j m_j \rangle = \frac{e}{2m} B_{\text{ext}} \hat{k} \cdot \langle \vec{L} + 2\vec{S} \rangle
\end{equation}

注意到 \( \vec{L} + 2\vec{S} = \vec{J} + \vec{S} \),但我们无法直接得到 \( \vec{S} \) 的期望值,可通过以下方法推导:总角动量 \( \vec{J} = \vec{L} + \vec{S} \) 是守恒量,\( \vec{L} \) 和 \( \vec{S} \) 绕该固定矢量快速进动。特别地,\( \vec{S} \) 的(时间)平均值即为其在 \( \vec{J} \) 方向上的投影:
\begin{equation}\boxed{
	\vec{S}_{\text{ave}} = \frac{(\vec{S} \cdot \vec{J})}{J^2} \vec{J}
}
\end{equation}

由 \( \vec{L} = \vec{J} - \vec{S} \) 可得 
\[L^2 = J^2 + S^2 - 2\vec{J} \cdot \vec{S} \] 

因此:
\begin{equation}
	\vec{S} \cdot \vec{J} = \frac{1}{2} \left( J^2 + S^2 - L^2 \right) = \frac{\hbar^2}{2} \left[ j(j+1) + s(s+1) - \ell(\ell+1) \right]
\end{equation}

进而可推得
\begin{equation}
	\langle \vec{L} + 2\vec{S} \rangle = \left\langle \left( 1 + \frac{\vec{S} \cdot \vec{J}}{J^2} \right) \vec{J} \right\rangle = \left[ 1 + \frac{j(j+1) - \ell(\ell+1) + s(s+1)}{2j(j+1)} \right] \langle \vec{J} \rangle
\end{equation}

方括号中的项称为朗德 \( g \) 因子\( g_J \), 定义玻尔磁子:
\begin{equation}
	\mu_B \equiv \frac{e\hbar}{2m} = 5.788 \times 10^{-5} \, \text{eV/T}
\end{equation}

最终,塞曼效应的一阶能量修正为:
\begin{equation}
	E_Z^1 = \mu_B g_J B_{\text{ext}} m_j
\end{equation}

总能量为精细结构能量与塞曼修正之和。

\subsubsection{强场塞曼效应}

当 \( B_{\text{ext}} \gg B_{\text{int}} \) 时,塞曼效应起主导作用,我们将 \( H_{\text{Bohr}} + H_Z' \) 作为"未微扰"哈密顿量,\( H_{\text{fs}}' \) 作为微扰。此时选取本征态 $\left| {n,l,{m_l},{m_s}} \right\rangle $,塞曼哈密顿量为:
\begin{equation}
	H_Z' = \frac{e}{2m} B_{\text{ext}} (L_z + 2S_z)
\end{equation}

"未微扰"能量的计算较为直接:
\begin{equation}
	E_{n m_l m_s} = -\frac{13.6 \, \text{eV}}{n^2} + \mu_B B_{\text{ext}} (m_l + 2m_s)
\end{equation}

在一阶微扰理论中,这些能级的精细结构修正为:
\begin{equation}
	E_{\text{fs}}^1 = \langle n\ell m_l m_s | (H_r' + H_{\text{so}}') | n\ell m_l m_s \rangle
\end{equation}

相对论修正与之前相同;对于自旋-轨道项,需计算:
\begin{equation}
	\langle \vec{S} \cdot \vec{L} \rangle = \langle S_x \rangle \langle L_x \rangle + \langle S_y \rangle \langle L_y \rangle + \langle S_z \rangle \langle L_z \rangle \equiv \hbar^2 m_l m_s
\end{equation}

(注意:对于 \( S_z \) 和 \( L_z \) 的本征态,\( \langle S_x \rangle = \langle S_y \rangle = \langle L_x \rangle = \langle L_y \rangle = 0 \))。综合以上所有结果,可得:
\begin{equation}
	E_{\text{fs}}^1 = \frac{13.6 \, \text{eV}}{n^3} \alpha^2 \left\{ \frac{3}{4n} - \left[ \frac{\ell(\ell+1) - m_l m_s}{\ell(\ell+\frac{1}{2})(\ell+1)} \right] \right\}
\end{equation}

(当 \( \ell=0 \) 时,方括号中的项无定义;此时其正确值为1)总能量为塞曼能量与精细结构修正之和。
	 \subsection{Hyperfine Structure}
	\subsubsection{Physical Definition}
	The hyperfine structure results from the interaction between the total electronic angular momentum $\mathbf{J}$ and the nuclear spin angular momentum $\mathbf{I}$. This interaction is primarily driven by:
	\begin{itemize}
		\item \textbf{Magnetic Dipole Interaction:} The nuclear magnetic moment interacting with the magnetic field of the electrons.
		\item \textbf{Electric Quadrupole Interaction:} The interaction between the nuclear electric quadrupole moment and the electric field gradient.
	\end{itemize}
	
	\subsubsection{Quantum Numbers and Coupling}
	The total atomic angular momentum $\mathbf{F}$ is defined by the vector sum:
	\begin{equation}
		\mathbf{F} = \mathbf{I} + \mathbf{J}
	\end{equation}
	
	The associated quantum number $F$ takes values in the range:
	\begin{equation}
		F = |J - I|, |J - I| + 1, \dots, J + I
	\end{equation}
	
	For a given $F$, the magnetic projection quantum number $m_F$ is:
	\begin{equation}
		m_F \in \{-F, -F+1, \dots, F-1, F\}
	\end{equation}
	
	\subsubsection{Energy Eigenvalues}
	The Hamiltonian for the magnetic dipole interaction is:
	\begin{equation}
		\hat{H}_{hfs} = A \cdot (\mathbf{I} \cdot \mathbf{J})
	\end{equation}
	
	Using the identity $\mathbf{F}^2 = \mathbf{I}^2 + \mathbf{J}^2 + 2\mathbf{I} \cdot \mathbf{J}$, the energy shift $\Delta E_{hfs}$ is derived as:
	\begin{equation}
		\Delta E_{hfs} = \frac{A}{2} [F(F+1) - I(I+1) - J(J+1)]
	\end{equation}
	where $A$ is the hyperfine structure constant.
	
	\textbf{This means that after considering the nuclear spin effect, for a ground state with total angular momentum J, it can be divided into two or more different "ground states" of different F}
	\subsubsection{Lande Interval Rule}
	The separation between two adjacent hyperfine levels $F$ and $F-1$ is proportional to $F$:
	\begin{equation}
		\Delta E_{F} - \Delta E_{F-1} = AF
	\end{equation}
	
	\subsubsection{The Hyperfine Quantum Number Set}
	
	In the presence of hyperfine coupling, the individual projections $m_I$ and $m_J$ are no longer "good" quantum numbers. We shift to the coupled basis, defined by the following set:
	
	\begin{itemize}
		\item \textbf{$I$ (Nuclear Spin):} Fixed for a given nucleus. 
		\item \textbf{$J$ (Electronic Angular Momentum):} The total electronic momentum from fine structure.
		\item \textbf{$F$ (Total Atomic Angular Momentum):} The principal quantum number for hyperfine structure. 
		\item \textbf{$m_F$ (Magnetic Quantum Number):} The projection of $F$ along the quantization axis.
	\end{itemize}
	
	\subsubsection{Degeneracy and Dimensionality}
	For a given $J$ and $I$, the total number of hyperfine states must equal the product of the individual multiplicities:
	\begin{equation}
		\sum_{F=|J-I|}^{J+I} (2F + 1) = (2J + 1)(2I + 1)
	\end{equation}
	
	\subsubsection{The Zeeman Effect in Hyperfine Structure}
	When an external magnetic field $B$ is applied, the $m_F$ levels shift according to:
	\begin{equation}
		\Delta E_{mag} = g_F \mu_B m_F B
	\end{equation}
	where $g_F$ is the Landé g-factor for the hyperfine level:
	\begin{equation}
		g_F \approx g_J \frac{F(F+1) + J(J+1) - I(I+1)}{2F(F+1)}
	\end{equation}
\newpage
\section{斯塔克效应}
\begin{problembox}{斯塔克效应}
	原子置于一个恒定外电场 $E_{\mathrm{ext}}$ 中,其电子能级将发生分裂---该现象被称为斯塔克效应(它在电学上和塞曼效应相对应)。本题研究氢原子 $n=1$ 和 $n=2$ 态能级的斯塔克效应。令电场沿 $z$ 轴方向,因此电子的势能为
	\[
	H'_S = e E_{\mathrm{ext}} z = e E_{\mathrm{ext}} r \cos\theta.
	\]
	将其看成加在玻尔哈密顿量上的微扰。(自旋和这个问题无关,所以我们将其忽略,忽略精细结构的影响。)
	
	\begin{enumerate}
		\item 证明:在一阶修正下,基态能量不受微扰的影响。
		
		\item 第一激发态是四重简并的:$\psi_{200}, \psi_{211}, \psi_{210}, \psi_{21-1}$。利用简并微扰理论确定能量的一阶修正。$E_2$ 将分裂为几条能级?
		
		\item 问题 (2) 的"好"波函数是什么?求出在这些"好"态中电偶极矩 ($\mathbf{p}_e = -e\mathbf{r}$) 的期望值。注意结果将与施加场没有关系---显然,处在第一激发态的氢原子可以具有恒定的电偶极矩。
		
		\textbf{提示:} 本题中涉及很多积分需要计算,但几乎所有的积分都为零。所以在你计算每一个积分前,要仔细分析:如果 $\phi$ 积分为零,那么无需计算 $r$ 和 $\theta$ 的积分。
	\end{enumerate}
\end{problembox}

\subsection*{解答}

(a) 氢原子基态为 $|1\,0\,0\rangle = \dfrac{1}{\sqrt{\pi a^3}} e^{-r/a}$,非简并,能量一阶修正为
\begin{align*}
	E_0^1 &= \langle 1\,0\,0 | H'_S | 1\,0\,0 \rangle \\
	&= \frac{e E_{\mathrm{ext}}}{\pi a^3} \int e^{-2r/a} r \cos\theta \, r^2 \sin\theta \, dr\, d\theta\, d\phi \\
	&= \frac{e E_{\mathrm{ext}}}{\pi a^3} \int_0^\infty e^{-2r/a} r^3 \, dr \underbrace{\int_0^\pi \cos\theta \sin\theta \, d\theta}_{0} \int_0^{2\pi} d\phi = 0.
\end{align*}

(b) 第一激发态 \(n=2\) 四重简并:
\begin{align*}
	|1\rangle &= \psi_{200} = \frac{1}{\sqrt{8\pi a^3}} \left(1 - \frac{r}{2a}\right) e^{-r/(2a)}, \\
	|2\rangle &= \psi_{211} = -\frac{1}{\sqrt{64\pi a^5}} r e^{-r/(2a)} \sin\theta \, e^{i\phi}, \\
	|3\rangle &= \psi_{210} = \frac{1}{\sqrt{32\pi a^5}} r e^{-r/(2a)} \cos\theta, \\
	|4\rangle &= \psi_{21-1} = \frac{1}{\sqrt{64\pi a^5}} r e^{-r/(2a)} \sin\theta \, e^{-i\phi}.
\end{align*}
\textbf{简并态需要重新选择本征态}

分析微扰算符 \(H'_S\) 的角动量特征:将 \(H'_S\) 写成 \(eE_{\text{ext}} r \cos \theta = eE_{\text{ext}} r \cdot Y_{10}(\theta, \phi)\),其中 \(Y_{10}\) 是 \(l=1, m=0\) 的球谐函数

 这意味着,从角动量算符 \(L_z = -i\hbar \frac{\partial}{\partial \phi}\) 来看,\(Y_{10}\) 与 \(\phi\) 无关,是 \(L_z\) 的本征值为 0 的本征态。可以说,算符 \(H'_S\) 带有角动量投影 \(m=0\), 沿 z 方向的电场不改变系统绕 z 轴的旋转对称性。  
\[
\Delta l = \pm 1, \quad \Delta m = 0
\] 
在简并子空间计算 $H'_S$ 矩阵元,发现仅有 $\langle 1 | H' | 3 \rangle$, $\langle 3 | H' | 1 \rangle$ 不为零,其余为零。
\begin{align*}
	\langle 1 | H'_S | 3 \rangle &= \frac{1}{\sqrt{8\pi a^3}} \frac{1}{\sqrt{32\pi a^5}} e E_{\mathrm{ext}} \underbrace{\int_0^\infty \left(1 - \frac{r}{2a}\right) e^{-r/a} r^4 \, dr}_{4!a^5 - 5!a^6/(2a)} \underbrace{\int_0^\pi \cos^2\theta \sin\theta \, d\theta}_{2/3} \underbrace{\int_0^{2\pi} d\phi}_{2\pi} \\
	&= -3ae E_{\mathrm{ext}}.
\end{align*}

$\mathcal{H}'_S$ 矩阵为
\[
\mathcal{H}'_S = -3ae E_{\mathrm{ext}} 
\begin{pmatrix}
	0 & 0 & 1 & 0 \\
	0 & 0 & 0 & 0 \\
	1 & 0 & 0 & 0 \\
	0 & 0 & 0 & 0
\end{pmatrix}.
\]

设能量本征值为 $-3ae E_{\mathrm{ext}} \lambda$。解久期方程:
\[
\begin{vmatrix}
	-\lambda & 0 & 1 & 0 \\
	0 & -\lambda & 0 & 0 \\
	1 & 0 & -\lambda & 0 \\
	0 & 0 & 0 & -\lambda
\end{vmatrix}
= -\lambda \begin{vmatrix}
	-\lambda & 0 & 0 \\
	0 & -\lambda & 0 \\
	0 & 0 & -\lambda
\end{vmatrix}
+ \begin{vmatrix}
	0 & 1 & 0 \\
	-\lambda & 0 & 0 \\
	0 & 0 & -\lambda
\end{vmatrix}
= 0,
\]
得
\[
\lambda^4 - \lambda^2 = 0 \Rightarrow \lambda^2(\lambda^2 - 1) = 0 \Rightarrow \lambda = 0, 0, 1, -1.
\]

对应能量的一阶修正为
\[
E = E_2, E_2, E_2 - 3ae E_{\mathrm{ext}}, E_2 + 3ae E_{\mathrm{ext}}.
\]

(c) 将本征值 $\lambda = 0, 0, 1, -1$ 代入本征方程,求本征函数:
\[
\begin{pmatrix}
	0 & 0 & 1 & 0 \\
	0 & 0 & 0 & 0 \\
	1 & 0 & 0 & 0 \\
	0 & 0 & 0 & 0
\end{pmatrix}
\begin{pmatrix}
	c_1 \\ c_2 \\ c_3 \\ c_4
\end{pmatrix}
= \lambda
\begin{pmatrix}
	c_1 \\ c_2 \\ c_3 \\ c_4
\end{pmatrix}.
\]

当 $\lambda = 0$ 时,$c_1 = 0, c_3 = 0$,考虑归一化,两个态可选为 $(c_2 = 1, c_4 = 0)$ 或 $(c_2 = 0, c_4 = 1)$:
\[
|\psi_{\lambda=0_1}\rangle = |2\rangle = \begin{pmatrix} 0 \\ 1 \\ 0 \\ 0 \end{pmatrix}, \quad
|\psi_{\lambda=0_2}\rangle = |4\rangle = \begin{pmatrix} 0 \\ 0 \\ 0 \\ 1 \end{pmatrix}.
\]

当 $\lambda = \pm 1$ 时,$c_2 = 0, c_4 = 0, c_1 = \pm c_3$,归一化后:
\[
|\psi_{\lambda=1}\rangle = \frac{1}{\sqrt{2}} \begin{pmatrix} 1 \\ 0 \\ 1 \\ 0 \end{pmatrix}, \quad
|\psi_{\lambda=-1}\rangle = \frac{1}{\sqrt{2}} \begin{pmatrix} 1 \\ 0 \\ -1 \\ 0 \end{pmatrix}.
\]

零阶近似波函数("好"的波函数)为
\[
|\psi\rangle = \sum_{n=1}^4 c_n |\psi_{n}\rangle.
\]

对 $\lambda = 0$:
\[
|\psi_{\lambda=0_1}\rangle = (0\ 1\ 0\ 0) \begin{pmatrix} \psi_{200} \\ \psi_{211} \\ \psi_{210} \\ \psi_{21-1} \end{pmatrix} = \psi_{211},
\]
\[
|\psi_{\lambda=0_2}\rangle = (0\ 0\ 0\ 1) \begin{pmatrix} \psi_{200} \\ \psi_{211} \\ \psi_{210} \\ \psi_{21-1} \end{pmatrix} = \psi_{21-1}.
\]

对 $\lambda = 1$:
\[
|\psi_{\lambda=1}\rangle = \frac{1}{\sqrt{2}} (1\ 0\ 1\ 0) \begin{pmatrix} \psi_{200} \\ \psi_{211} \\ \psi_{210} \\ \psi_{21-1} \end{pmatrix} = \frac{1}{\sqrt{2}} (\psi_{200} + \psi_{210}).
\]

对 $\lambda = -1$:
\[
|\psi_{\lambda=-1}\rangle = \frac{1}{\sqrt{2}} (1\ 0\ -1\ 0) \begin{pmatrix} \psi_{200} \\ \psi_{211} \\ \psi_{210} \\ \psi_{21-1} \end{pmatrix} = \frac{1}{\sqrt{2}} (\psi_{200} - \psi_{210}).
\]

在这四个零阶近似波函数中求电偶极矩 ($\mathbf{p}_e = -e\mathbf{r}$) 的期望值:

对 $\lambda = 0$ 的两个态:
\begin{align*}
	\langle \psi_{\lambda=0_1} | \mathbf{p}_e | \psi_{\lambda=0_1} \rangle 
	&= \langle 2 | \mathbf{p}_e | 2 \rangle \\
	&= \int \psi_{211}^*(-er)\psi_{211} \, d^3r \\
	&= -\frac{e}{64\pi a^5} \int_0^\infty \int_0^\pi \int_0^{2\pi} 
	r^2 e^{-r/a} \sin^2\theta \times \\
	&\quad \times \bigl( r\sin\theta\cos\phi\,\hat{\imath} 
	+ r\sin\theta\sin\phi\,\hat{\jmath} 
	+ r\cos\theta\,\hat{k} \bigr) \, r^2 \sin\theta 
	\, dr\,d\theta\,d\phi \\
	&= -\frac{e}{64\pi a^5} \biggl[ 
	\underbrace{\int_0^\infty r^5 e^{-r/a} dr 
		\int_0^\pi \sin^4\theta\,d\theta 
		\int_0^{2\pi} \cos\phi\,d\phi}_{0} \,\hat{\imath} \\
	&\quad + \underbrace{\int_0^\infty r^5 e^{-r/a} dr 
		\int_0^\pi \sin^4\theta\,d\theta 
		\int_0^{2\pi} \sin\phi\,d\phi}_{0} \,\hat{\jmath} \\
	&\quad + \underbrace{\int_0^\infty r^5 e^{-r/a} dr 
		\int_0^\pi \sin^3\theta\cos\theta\,d\theta 
		\int_0^{2\pi} d\phi}_{0} \,\hat{k} \biggr] \\
	&= 0.
\end{align*}

同样有
\[
\langle \psi_{\lambda=0_2} | \mathbf{p}_e | \psi_{\lambda=0_2} \rangle = \langle 4 | \mathbf{p}_e | 4 \rangle = 0.
\]

对 $\lambda = \pm 1$ 的两个态:
\begin{align*}
	\langle \psi_{\lambda=1} | \mathbf{p}_e | \psi_{\lambda=1} \rangle &= \frac{1}{2} (\langle 1| + \langle 3|) \mathbf{p}_e (|1\rangle + |3\rangle) \\
	&= \frac{1}{2} \int (\psi_{200} + \psi_{210})^*(-er)(\psi_{200} + \psi_{210}) d^3r \\
	&= \int \psi_{200} \psi_{210} (-er) d^3r \\
	&= -e \frac{1}{8\pi a^3} \int_0^\infty \int_0^\pi \int_0^{2\pi} \left(1 - \frac{r}{2a}\right) \left(\frac{r}{2a} \cos\theta\right) e^{-r/a} \\
	&\quad \times (r\sin\theta\cos\phi\,\hat{i} + r\sin\theta\sin\phi\,\hat{j} + r\cos\theta\,\hat{k}) r^2 \sin\theta\,dr\,d\theta\,d\phi \\
	&= -e \frac{1}{8\pi a^3} \frac{1}{2a} \int_0^\infty \left(1 - \frac{r}{2a}\right) r^4 e^{-r/a} dr \int_0^\pi \sin^2\theta \cos\theta\,d\theta \underbrace{\int_0^{2\pi} \cos\phi\,d\phi}_{0} \hat{i} + \cdots \\
	&= -e \frac{1}{12a^4} \int_0^\infty \left(1 - \frac{r}{2a}\right) r^4 e^{-r/a} dr \cdot \frac{2}{3} \hat{k} \\
	&= 3ea \hat{k}.
\end{align*}

同理可得
\[
\langle \psi_{\lambda=-1} | \mathbf{p}_e | \psi_{\lambda=-1} \rangle = -3ea \hat{k}.
\]
\newpage
\begin{problembox}{斯塔克效应}
	考虑 $n=3$ 时氢原子态的斯塔克效应。开始时有九个简并态 $\psi_{3\ell m}$(和以前一样,忽略自旋),然后在沿 $z$ 轴方向上加一个电场。
	
	\begin{enumerate}
		\item 构造 $9\times9$ 的矩阵表示微扰哈密顿量。部分答案:
		\[
		\langle 300 | z | 310 \rangle = -3\sqrt{6}a, \quad
		\langle 310 | z | 320 \rangle = -3\sqrt{3}a, \quad
		\langle 31\pm1 | z | 32\pm1 \rangle = -\frac{9}{2}a.
		\]
		
		\item 确定其本征值和简并度。
	\end{enumerate}
\end{problembox}


\subsection*{解答}

(a) 简并子空间的 9 个态是:
\begin{align*}
	|1\rangle &= |3\,0\,0\rangle = R_{30}Y_0^0, \quad
	|2\rangle = |3\,1\,0\rangle = R_{31}Y_1^0, \\
	|3\rangle &= |3\,2\,0\rangle = R_{32}Y_2^0, \quad
	|4\rangle = |3\,1\,1\rangle = R_{31}Y_1^1, \\
	|5\rangle &= |3\,2\,1\rangle = R_{32}Y_2^1, \quad
	|6\rangle = |3\,1\,-1\rangle = R_{31}Y_1^{-1}, \\
	|7\rangle &= |3\,2\,-1\rangle = R_{32}Y_2^{-1}, \quad
	|8\rangle = |3\,2\,2\rangle = R_{32}Y_2^2, \\
	|9\rangle &= |3\,2\,-2\rangle = R_{32}Y_2^{-2}.
\end{align*}

由于 $H'_S = e E_{\mathrm{ext}} z = e E_{\mathrm{ext}} r \cos\theta$ 不依赖 $\phi$,所以
\[
\langle n\,\ell'\,m' | H'_S | n\,\ell\,m \rangle = \{\cdots\} \int_0^{2\pi} e^{-im'\phi} e^{im\phi} d\phi.
\]

当 $m \ne m'$ 时,矩阵元为零。对于对角元:
\[
\langle n\,\ell\,m | H'_S | n\,\ell\,m \rangle = \{\cdots\} \int_0^\pi [P_\ell^m(\cos\theta)]^2 \cos\theta \sin\theta\,d\theta,
\]
由于 $[P_\ell^m(\cos\theta)]^2$ 是 $\cos\theta$ 偶次幂的多项式,而每一项的积分为零,故所有对角元为零。

当 $m = m'$ 且 $\Delta l =  \pm 1$ 时,$P_\ell^m(\cos\theta) P_{\ell'}^m(\cos\theta)$ 也是 $\cos\theta$ 偶次幂的多项式,积分为零。只需计算以下三个非零矩阵元:
\[
\langle 3\,0\,0 | H'_S | 3\,1\,0 \rangle , \quad
\langle 3\,1\,0 | H'_S | 3\,2\,0 \rangle, \quad
 \langle 3\,0\,\pm1 | H'_S | 3\,2\,\pm1 \rangle
\]

计算得:
\[\begin{array}{l}
	\langle 3\,0\,0 | H'_S | 3\,1\,0 \rangle = -3\sqrt{6}a e E_{\mathrm{ext}}\\
	\langle 3\,1\,0 | H'_S | 3\,2\,0 \rangle = -3\sqrt{3}a e E_{\mathrm{ext}},\\
	\displaystyle \langle 3\,1\,\pm1 | H'_S | 3\,2\,\pm1 \rangle = -\frac{9}{2}a e E_{\mathrm{ext}}.
\end{array}\]

所以微扰矩阵为
\[
-a e E_{\mathrm{ext}}
\begin{pmatrix}
	0 & 3\sqrt{6} & 0 & 0 & 0 & 0 & 0 & 0 & 0 \\
	3\sqrt{6} & 0 & 3\sqrt{3} & 0 & 0 & 0 & 0 & 0 & 0 \\
	0 & 3\sqrt{3} & 0 & 0 & 0 & 0 & 0 & 0 & 0 \\
	0 & 0 & 0 & 0 & 9/2 & 0 & 0 & 0 & 0 \\
	0 & 0 & 0 & 9/2 & 0 & 0 & 0 & 0 & 0 \\
	0 & 0 & 0 & 0 & 0 & 9/2 & 0 & 0 & 0 \\
	0 & 0 & 0 & 0 & 0 & 0 & 9/2 & 0 & 0 \\
	0 & 0 & 0 & 0 & 0 & 0 & 0 & 0 & 0 \\
	0 & 0 & 0 & 0 & 0 & 0 & 0 & 0 & 0
\end{pmatrix}.
\]

(b) 该矩阵可约化为一个 $3\times3$、两个 $2\times2$ 和两个 $1\times1$ 子矩阵。设能量本征值为 $-a e E_{\mathrm{ext}} \lambda$。

对 $3\times3$ 矩阵,久期方程为:
\[
\begin{vmatrix}
	-\lambda & 3\sqrt{6} & 0 \\
	3\sqrt{6} & -\lambda & 3\sqrt{3} \\
	0 & 3\sqrt{3} & -\lambda
\end{vmatrix}
= -\lambda^3 + 81\lambda = 0 \Rightarrow \lambda = 0, \pm 9.
\]

所以能量一阶修正为:
\[
E_1^1 = 0, \quad E_2^1 = 9ae E_{\mathrm{ext}}, \quad E_3^1 = -9ae E_{\mathrm{ext}}.
\]

对 $2\times2$ 矩阵:
\[
\begin{vmatrix}
	-\lambda & 9/2 \\
	9/2 & -\lambda
\end{vmatrix}
= \lambda^2 - \left(\frac{9}{2}\right)^2 = 0 \Rightarrow \lambda = \pm \frac{9}{2}.
\]

所以能量一阶修正为:
\[
E_4^1 = \frac{9}{2}ae E_{\mathrm{ext}}, \quad E_5^1 = -\frac{9}{2}ae E_{\mathrm{ext}}, \quad E_6^1 = \frac{9}{2}ae E_{\mathrm{ext}}, \quad E_7^1 = -\frac{9}{2}ae E_{\mathrm{ext}}.
\]

对两个 $1\times1$ 矩阵,$\lambda = 0$,所以 $E_8^1 = E_9^1 = 0$。

原来 9 重简并能级分裂为 5 个能级:
\begin{itemize}
	\item $E_1^1 = E_8^1 = E_9^1 = 0$(3 重简并);
	\item $E_4^1 = E_6^1 = \dfrac{9}{2}ae E_{\mathrm{ext}}$(2 重简并);
	\item $E_5^1 = E_7^1 = -\dfrac{9}{2}ae E_{\mathrm{ext}}$(2 重简并);
	\item $E_2^1 = 9ae E_{\mathrm{ext}},\ E_3^1 = -9ae E_{\mathrm{ext}}$(非简并)。
\end{itemize}
\newpage

	\section{the WKB Approximation}
\subsection{Introduction and Core Idea}
The Wentzel-Kramers-Brillouin (WKB) method is a \textbf{semiclassical approximation technique} for solving the one-dimensional, time-independent Schr\"odinger equation:
\[
-\frac{\hbar^2}{2m} \psi''(x) + V(x)\psi(x) = E\psi(x).
\]
Its central idea is to treat $\hbar$ as a formal small parameter and expand the wavefunction and relevant quantities as asymptotic series in $\hbar$. This approach is particularly powerful for obtaining estimates of bound state energies when the quantum number is large (so that the classical action is large compared to $\hbar$) and for understanding tunneling phenomena.

\subsection{The Basic WKB Formalism}
\subsubsection{The WKB Ansatz and Expansion}
The starting point is to write the wavefunction in the form:
\[
\psi(x) = \exp\left[\frac{i}{\hbar} \int^x Y(x') dx'\right].
\]
Substituting this into the Schr\"odinger equation transforms it into a Riccati equation for $Y(x)$:
\[
Y^2(x) - i\hbar \frac{dY(x)}{dx} = p^2(x), \quad \text{where } p(x) = \sqrt{2m(E - V(x))}.
\]
One then solves for $Y(x)$ as a formal power series in $\hbar$:
\[
Y(x) = \sum_{k=0}^{\infty} Y_k(x) \hbar^k.
\]
The leading term is simply the classical momentum: $Y_0(x) = p(x)$.

\subsubsection{The WKB Wavefunctions}
To all orders in $\hbar$, a general solution can be written as a linear combination of two functions:
\[
\psi(x) \sim \frac{A_+}{\sqrt{S'(x, \hbar)}} \exp\left(+\frac{i}{\hbar} S(x, \hbar)\right) + \frac{A_-}{\sqrt{S'(x, \hbar)}} \exp\left(-\frac{i}{\hbar} S(x, \hbar)\right),
\]
where $S(x, \hbar) = \sum_{n\ge 0} S_n(x) \hbar^{2n}$ and $S_0(x) = \int^x p(x') dx'$.

In the \textbf{traditional WKB approximation}, one truncates this series at leading order, neglecting the Schwarzian derivative terms. This yields the familiar \textbf{basic WKB solutions}:
\begin{itemize}
	\item In the \textbf{classically allowed region} ($E > V(x)$, $p(x)$ real):
	\[
	\psi_{\pm}(x) \approx \frac{1}{\sqrt{p(x)}} \exp\left[ \pm \frac{i}{\hbar} \int^x p(x') dx' \right].
	\]
	\item In the \textbf{classically forbidden region} ($E < V(x)$, $p(x)$ imaginary, define $p_1(x) = \sqrt{2m(V(x)-E)} > 0$):
	\[
	\psi(x) \approx \frac{A}{\sqrt{p_1(x)}} \exp\left[ +\frac{1}{\hbar} \int^x p_1(x') dx' \right] + \frac{B}{\sqrt{p_1(x)}} \exp\left[ -\frac{1}{\hbar} \int^x p_1(x') dx' \right].
	\]
\end{itemize}

\subsection{Limitations and the Uniform WKB Method}
The basic WKB solutions diverge at \textbf{turning points} where $p(x)=0$ ($E = V(x)$). To connect solutions across turning points, a more refined treatment is needed.

The \textbf{uniform WKB method} employs a more general ansatz:
\[
\psi(x) = \frac{1}{\sqrt{\phi'(x)}} f(\phi(x)),
\]
where the auxiliary function $f(\phi)$ is chosen to match the local behavior near a turning point. For a standard linear turning point (where $V(x)-E \approx k(x-x_0)$ near $x_0$), the natural choice is $f(\phi) = a\operatorname{Ai}(\hbar^{-2/3}\phi) + b\operatorname{Bi}(\hbar^{-2/3}\phi)$, where $\operatorname{Ai}$ and $\operatorname{Bi}$ are Airy functions. The function $\phi(x)$ is then determined by solving a nonlinear equation order-by-order in $\hbar^2$.

This method yields a wavefunction $\psi(x)$ that remains finite and accurate even at the turning point, seamlessly interpolating between the oscillatory and exponential regions.

\subsection{Connection Formulae and Quantization Conditions}
\subsubsection{The Voros-Silverstone Connection Formula}
Using the uniform WKB method and Borel resummation techniques, one can derive \textbf{exact connection formulae} that relate the wavefunction on either side of a turning point. For a wavefunction approaching a turning point from the classically allowed side with the form:
\[
\psi_{\text{I}}(x) \sim \left(S'(x,\hbar)\right)^{-1/2} \left[ (b-ia)e^{+iS/\hbar + i\pi/4} + (b+ia)e^{-iS/\hbar - i\pi/4} \right],
\]
the connection to the classically forbidden side ($x > x_0$) is given by:
\[
\psi_{\text{II}}(x) \sim \left(-Q'(x,\hbar)\right)^{-1/2} \left\{ 2b e^{+Q/\hbar} + (a \pm i b) e^{-Q/\hbar} \right\},
\]
where $Q(x,\hbar) = iS(e^{i\pi}(x_0-x)+x_0, \hbar)$ is a real function in the forbidden region. The $\pm$ sign corresponds to the choice of lateral Borel resummation.

\subsubsection{Bohr-Sommerfeld Quantization Condition}
For a bound state in a potential well with two turning points $x_-$ and $x_+$, imposing decay in the outer forbidden regions and using the connection formulae leads to a condition on the action integral. In its leading-order form, this recovers the famous \textbf{Bohr-Sommerfeld quantization condition}:
\[
\frac{1}{\hbar} \oint p(x) \, dx = \frac{1}{\hbar} \int_{x_-}^{x_+} p(x) \, dx = \pi \left(n + \frac{1}{2}\right), \quad n = 0, 1, 2, \dots
\]
This condition provides an asymptotic formula for the energy eigenvalues $E_n$ in the limit of large quantum number $n$. The all-orders WKB expansion yields systematic higher-order corrections in $\hbar$ to this condition.

\subsection{Non-Perturbative Effects and the Complex WKB Method}
The WKB expansion in $\hbar$ is typically a divergent asymptotic series. Crucially, the exact quantization condition also contains \textbf{exponentially small, non-perturbative terms} that are not captured by the power series in $\hbar$. These terms are essential for describing phenomena like tunneling splittings in symmetric double-well potentials.

To capture these effects systematically, one must extend the analysis to the \textbf{complex plane}. This leads to the \textbf{exact or complex WKB method}. Here, one considers period integrals (action integrals) for complex turning points. The exact quantization condition often takes the form:
\[
\frac{1}{2\cos\left(\frac{1}{2\hbar}\oint_{\gamma} p\,dz\right)} + \text{(non-perturbative terms)} = 0,
\]
where $\gamma$ is an appropriate cycle on the complex Riemann surface defined by $p(z)=\sqrt{2m(E-V(z))}$. The non-perturbative terms are typically of the form $\exp(-S_{\text{inst}}/\hbar)$, where $S_{\text{inst}}$ is the instanton action related to tunneling.

\subsection{Generalization: The EBK Quantization for Integrable Systems}
For higher-dimensional, completely integrable systems, the WKB method generalizes to the \textbf{Einstein-Brillouin-Keller (EBK)} quantization condition. For a system with $n$ degrees of freedom and action variables $I_j = \frac{1}{2\pi}\oint_{\gamma_j} \mathbf{p} \cdot d\mathbf{q}$ defined over independent cycles $\gamma_j$ on the invariant torus, the quantization condition is:
\[
I_j(\mathbf{f}) = \hbar \left( n_j + \frac{\mu_j}{4} \right), \quad n_j \in \mathbb{Z}_{\ge 0},
\]
where $\mu_j$ are Maslov indices. This reduces to the one-dimensional Bohr-Sommerfeld condition for $n=1$.

\subsection{Summary of Key Points}
\begin{enumerate}
	\item \textbf{Core Purpose}: The WKB method is a systematic semiclassical ($\hbar \to 0$) expansion for solving the Schr\"odinger equation, providing asymptotic series for wavefunctions and energy levels.
	\item \textbf{Basic Solution}: In regions where $V(x)$ varies slowly, the wavefunction resembles plane waves (allowed region) or exponentials (forbidden region), with amplitude proportional to $1/\sqrt{p(x)}$.
	\item \textbf{Turning Point Problem}: The basic approximation fails at turning points. The \textbf{uniform WKB method} (using Airy functions) provides a globally valid solution.
	\item \textbf{Quantization}: Matching conditions across turning points lead to the \textbf{Bohr-Sommerfeld rule} at leading order, with higher-order corrections available from the all-orders expansion.
	\item \textbf{Exact WKB}: Considering the full asymptotic series and its Borel resummation, along with contributions from complex trajectories, leads to \textbf{exact quantization conditions} that include non-perturbative tunneling effects.
	\item \textbf{Higher Dimensions}: For integrable systems, the method generalizes to the \textbf{EBK quantization} condition.
	\item \textbf{Validity}: The approximation requires the local de Broglie wavelength $\lambda(x) = 2\pi\hbar/|p(x)|$ to vary slowly: $\left| \frac{d\lambda}{dx} \right| \ll 1$. It becomes exact in the formal limit $\hbar \to 0$ or for large quantum numbers.
\end{enumerate}

In essence, the WKB approximation is a powerful bridge between classical and quantum mechanics, providing a framework to systematically compute quantum corrections to classical results and to understand deep concepts like tunneling and non-perturbative effects.
\newpage



\section{含时微扰理论}
在含时微扰理论框架下,量子系统从初态 $|\psi_i\rangle$(能量 $E_i$)跃迁到末态 $|\psi_f\rangle$(能量 $E_f$)的概率幅正比于跃迁矩阵元:
\begin{equation}
	V_{fi} = \langle \psi_f | \hat{V} | \psi_i \rangle
\end{equation}
其中 $\hat{V}$ 为微扰哈密顿量(如电磁场相互作用的电偶极项 $-\mathbf{e} \cdot \mathbf{r}$)。%选择定律的本质是判断 $\mathcal{M}_{fi}$ 是否为零:
\begin{itemize}
	\item \textbf{允许跃迁}:$V_{fi} \neq 0$(概率非零)
	\item \textbf{禁止跃迁}:$V_{fi} = 0$(概率为零或极小)
\end{itemize}
%选择定律由初态与末态的对称性(角动量、宇称等)决定,源于微扰算符 $\hat{V}$ 与量子态的对称性匹配要求。
\subsection{时间演化算符}
对于哈密顿量与时间无关的系统的量子力学。在这种情况下,态矢的时间演化可以通过时间演化算子来描述
\begin{equation*}
	\hat{U}=e^{-i\hat{H}t/\hbar}
\end{equation*}
当用哈密顿量的本征态$\hat{H}\ket{n}=E_n\ket{n}$表示时,态矢可写为:
\begin{equation}\label{12.1}
	\ket{\psi(t)}=e^{-i\hat{H}t/\hbar}\ket{\psi(0)}=\sum_n e^{-iE_n t/\hbar}c_n(0)\ket{n}
\end{equation}
尽管该框架适用于任何封闭量子力学系统,但它无法描述系统与外部环境的相互作用(例如外部电磁场施加的作用)。在这类情况下,更方便的做法是通过含时相互作用$V(t)$来描述小孤立系统$\hat{H}_0$的诱导相互作用。实例包括:描述量子力学自旋与外部含时磁场相互作用的磁共振问题,以及原子对外部电磁场的响应问题。下文将构建处理含时微扰的理论体系。

\subsection{含时势:一般理论框架}
考虑哈密顿量$\hat{H}=\hat{H}_0+V(t)$,其中所有时间依赖性均来自势场$V(t)$。在薛定谔表象中,系统的动力学由含时波函数$\ket{\psi(t)}_S$通过薛定谔方程
\begin{equation}
	i\hbar\partial_t\ket{\psi(t)}_S=\hat{H}\ket{\psi(t)}_S
\end{equation}
确定。然而,在许多情况下,采用\textbf{相互作用表象}会更为便捷,其定义为:
\begin{equation}
	\ket{\psi(t)}_I=e^{i\hat{H}_0 t/\hbar}\ket{\psi(t)}_S
\end{equation}
其中$\ket{\psi(0)}_I=\ket{\psi(0)}_S$,根据这一定义,可以证明波函数满足以下运动方程:
\begin{equation}
	i\hbar\partial_t\ket{\psi(t)}_I=V_I(t)\ket{\psi(t)}_I
\end{equation}
式中
\begin{equation}
	V_I(t)=e^{i\hat{H}_0 t/\hbar}Ve^{-i\hat{H}_0 t/\hbar}
\end{equation}
若将波函数按本征函数展开为$\ket{\psi(t)}_I=\sum_n c_n(t)\ket{n}$,并将运动方程与一般态$\bra{n}$内积,可得:
\begin{equation}
	i\hbar\dot{c}_m(t)=\sum_n V_{mn}(t)e^{i\omega_{mn}t}c_n(t)
\end{equation}
其中矩阵元$V_{mn}(t)=\matrixel{m}{V(t)}{n}$,且$\omega_{mn}=(E_m-E_n)/\hbar=-\omega_{nm}$。

\subsection{详细计算}
现在我们考虑一般的含时哈密顿量——遗憾的是,这类哈密顿量通常无法获得解析解!此时必须采用微扰分析,将基矢系数$c_n(t)$按相互作用的幂次展开:
\begin{equation}
	c_n(t)=c_n^{(0)}+c_n^{(1)}(t)+c_n^{(2)}(t)+\cdots
\end{equation}
其中$c_n^{(m)}\sim O(V^m)$,$c_n^{(0)}$为与时间无关的初始态。完成该级数展开的过程直接但涉及较多技巧。

▷说明:\textbf{在相互作用表象中},态$\ket{\psi(t)}_I$可通过时间演化算子$U_I(t,t_0)$与初始态$\ket{\psi(t_0)}_I$关联,即$\ket{\psi(t)}_I=U_I(t,t_0)\ket{\psi(t_0)}_I$。由于该式对任意初始态$\ket{\psi(t_0)}_I$均成立,结合式(\ref{12.1})可得:
\begin{equation}
	i\hbar\partial_t U_I(t,t_0)=V_I(t)U_I(t,t_0)
\end{equation}
边界条件为$U_I(t_0,t_0)=\mathbb{I}$(单位算子)。将该方程从$t_0$积分至$t$,形式上可得:
\begin{equation}
	U_I(t,t_0)=\mathbb{I}-\frac{i}{\hbar}\int_{t_0}^t dt' V_I(t')U_I(t',t_0)
\end{equation}
该结果为$U_I(t,t_0)$提供了自洽方程:将表达式代入积分项中的$U_I(t',t_0)$,可得:
\begin{equation}
	U_I(t,t_0)=\mathbb{I}-\frac{i}{\hbar}\int_{t_0}^t dt' V_I(t')+\left(-\frac{i}{\hbar}\right)^2\int_{t_0}^t dt' V_I(t')\int_{t_0}^{t'} dt'' V_I(t'')U_I(t'',t_0)
\end{equation}
重复该迭代过程,最终得到:
\begin{equation}\label{12.3}
	U_I(t,t_0)=\sum_{n=0}^\infty \left(-\frac{i}{\hbar}\right)^n \int_{t_0}^t dt_1\cdots\int_{t_0}^{t_{n-1}} dt_n V_I(t_1)V_I(t_2)\cdots V_I(t_n)
\end{equation}
式中$n=0$项对应单位算子$\mathbb{I}$。注意算子$V_I(t)$按时间有序排列,满足$t_0\leq t_n\leq t_{n-1}\leq\cdots\leq t_1\leq t$。基于这一理解,可将表达式更简洁地写为:
\begin{equation}
	U_I(t,t_0)=T\left[e^{-\frac{i}{\hbar}\int_{t_0}^t dt' V_I(t')}\right]
\end{equation}
其中"$T$"表示时间排序算子,其作用由式(\ref{12.3})定义。

若系统在时刻$t=t_0$制备于初始态$\ket{i}$,则在后续时刻$t$,系统将处于末态:
\begin{equation}
	\ket{i,t_0,t}=U_I(t,t_0)\ket{i}=\sum_n \ket{n} \overbrace{\matrixel{n}{U_I(t,t_0)}{i}}^{c_n(t)}
\end{equation}
利用式(\ref{12.3})和单位分解$\sum_m \ket{m}\bra{m}=\mathbb{I}$,可得:
\begin{multline}
	c_n(t) = \overbrace{\delta_{ni}}^{c_n^{(0)}} 
	\overbrace{-\frac{i}{\hbar}\int_{t_0}^t dt'\matrixel{n}{V_I(t')}{i}}^{c_n^{(1)}} \\
	\overbrace{-\frac{1}{\hbar^2}\int_{t_0}^t dt'\int_{t_0}^{t'} dt''\sum_m \matrixel{n}{V_I(t')}{m}\matrixel{m}{V_I(t'')}{i}}^{c_n^{(2)}}
	+\cdots
\end{multline}
回顾$V_I=e^{i\hat{H}_0 t/\hbar}Ve^{-i\hat{H}_0 t/\hbar}$,因此:
\begin{equation}\label{12.4}
	 \boxed
	{
	\begin{aligned}
		c_n^{(1)}(t)&=-\frac{i}{\hbar}\int_{t_0}^t dt' e^{i\omega_{ni}t'} V_{ni}(t')\\
		c_n^{(2)}(t)&=-\frac{1}{\hbar^2}\sum_m \int_{t_0}^t dt'\int_{t_0}^{t'} dt'' e^{i\omega_{nm}t'+i\omega_{mi}t''} V_{nm}(t')V_{mi}(t'')
	\end{aligned}
	}
\end{equation}
式中$V_{nm}(t)=\matrixel{n}{V(t)}{m}$,$\omega_{nm}=(E_n-E_m)/\hbar$等。

特别地,从态$\ket{i}$到态$\ket{n}$($n\neq i$)的跃迁概率为:
\begin{equation}
	P_{i\to n}=|c_n(t)|^2=|c_n^{(1)}(t)+c_n^{(2)}(t)+\cdots|^2
\end{equation}

示例:\textit{\textbf{The kicked oscillator}}

考虑一个简谐振子在时刻$t=-\infty$制备于基态$\ket{0}$。若其受到微扰含时势
\[V(t)=-eE x e^{-t^2/\tau^2}\]
求在时刻$t=+\infty$时系统处于第一激发态$\ket{1}$的概率?

在一阶微扰近似下,跃迁概率为$P_{0\to1}\simeq|c_1^{(1)}|^2$,其中:
\begin{equation}
	c_1^{(1)}(t)=-\frac{i}{\hbar}\int_{t_0}^t dt' e^{i\omega_{10}t'} V_{10}(t')
\end{equation}
$\displaystyle V_{10}(t')=-eE\matrixel{1}{x}{0}e^{-t'^2/\tau^2}$, 令 $\omega_{10}=\omega$。

利用升降算符理论,$\ket{1}=a^\dagger\ket{0}$且$x=\sqrt{\frac{\hbar}{2m\omega}}(a+a^\dagger)$,可得$\matrixel{1}{x}{0}=\sqrt{\frac{\hbar}{2m\omega}}$。利用积分恒等式
\[\int_{-\infty}^\infty dt' \exp\left[i\omega t'-t'^2/\tau^2\right]=\sqrt{\pi}\tau\exp\left[-\omega^2\tau^2/4\right]\]
可得跃迁振幅:
\begin{equation}
	c_1^{(1)}(t\to\infty)=ieE\tau\sqrt{\frac{\pi}{2m\hbar\omega}}e^{-\omega^2\tau^2/4}
\end{equation}
因此,跃迁概率为:
\begin{equation}
	P_{0\to1}\simeq(eE\tau)^2\left(\frac{\pi}{2m\hbar\omega}\right)e^{-\omega^2\tau^2/2}
\end{equation}
注意当$\tau\sim1/\omega$时,概率达到最大值。

\section{“Sudden” perturbation}
为进一步探讨含时微扰理论,我们现在考虑快速(或"突然")微扰的作用。此处定义 “Sudden” perturbation 为:

哈密顿量从一个与时间无关的$\hat{H}_0$切换至另一个$\hat{H}_0'$的时间远短于系统的任何自然周期。

在这种情况下,微扰理论不再适用:若系统初始处于$\hat{H}_0$的本征态$\ket{n}$,则切换后的时间演化将遵循$\hat{H}_0'$的规律,\textbf{需将初始态按}$\hat{H}_0'$\textbf{的本征态展开}:$\ket{n}=\sum_{n'}\ket{n'}\braket{n'}{n}$。该问题的核心在于判断哈密顿量的变化是否足够"突然",这需要估算哈密顿量变化的实际时间,以及与态$\ket{n}$相关的运动周期及其向邻近态的跃迁周期。

\subsection{简谐微扰:费米黄金规则}
现在仅考虑\textbf{离散谱},系统初始处于态$\ket{i}$,并受到周期性简谐势$V(t)=Ve^{-i\omega t}$的微扰(该势在时刻$t=0$突然施加)。这一模型可描述原子受到外部振荡电场(如入射光波)的微扰。求在后续 $t$ 时刻系统处于态$\ket{f}$的概率?

根据式(\ref{12.4}),在一阶微扰近似下:
\begin{equation}
	c_f^{(1)}(t)=-\frac{i}{\hbar}\int_0^t dt'\matrixel{f}{V}{i}e^{i(\omega_{fi}-\omega)t'}=-\frac{i}{\hbar}\matrixel{f}{V}{i}\frac{e^{i(\omega_{fi}-\omega)t}-1}{i(\omega_{fi}-\omega)}
\end{equation}
因此,时刻$t$的跃迁概率为:
\begin{equation}
	P_{i\to f}(t)\simeq|c_f^{(1)}(t)|^2=\frac{1}{\hbar^2}|\matrixel{f}{V}{i}|^2\left(\frac{\sin\left((\omega_{fi}-\omega)t/2\right)}{(\omega_{fi}-\omega)/2}\right)^2
\end{equation}
令$\alpha=(\omega_{fi}-\omega)/2$,则概率 $P \propto {\sin ^2}(\alpha t)/{\alpha ^2}$,其在$\alpha=0$处出现峰值,最大值为$t^2$,宽度约为$1/t$,总权重约为$t$。该函数在$\alpha t=(n+1/2)\pi$($n$为整数)处存在更多峰值,峰值大小受分母$1/\alpha^2$限制。当$t\to\infty$时,该函数趋近于函数 $``\delta \left( x \right)t"$ ,即跃迁概率与经历的时间成反比。因此,需通过除以时间$t$得到跃迁速率。

利用积分\[\int_{-\infty}^\infty d\alpha\left(\frac{\sin(\alpha t)}{\alpha}\right)^2=\pi t\]

可进行替换:
\begin{equation*}
	\lim_{t\to\infty}\frac{1}{t}\left(\frac{\sin(\alpha t)}{\alpha}\right)^2=\pi\delta(\alpha)=2\pi\delta(2\alpha)
\end{equation*}
最终得到跃迁速率的表达式:
\begin{equation}\label{12.5}
	\boxed{
	R_{i\to f}=\lim_{t\to\infty}\frac{P_{i\to f}(t)}{t}=\frac{2\pi}{\hbar^2}|\matrixel{f}{V}{i}|^2\delta(\omega_{fi}-\omega)
}
\end{equation}
该表达式称为\textbf{费米黄金法则}。

现在假设末态构成\textbf{连续谱},即存在态密度 $\rho(E_f)$,使得能量在 $[E_f, E_f + dE_f]$ 内的态数为 $\rho(E_f)\, dE_f$。

跃迁到所有末态的总概率为
\begin{equation}
P_{\text{tot}}(t) = \int P_{i \to f}(t)\, \rho(E_f)\, dE_f.
\end{equation}

由于 $\alpha=(\omega_{fi}-\omega)/2$,  $E_f = E_i + 2\hbar \alpha$,故 $dE_f = 2\hbar\, d\alpha$,于是
\begin{equation*}
P_{\text{tot}}(t) = \int_{-\infty}^{\infty} \frac{1}{\hbar^2}|\langle f| V |i\rangle|^2 \left( \frac{\sin(\alpha t)}{\alpha} \right)^2 \rho(E_f)\, 2\hbar\, d\alpha.
\end{equation*}

整理得
\begin{equation}
P_{\text{tot}}(t) = \frac{2}{\hbar} \int_{-\infty}^{\infty} |\langle f| V |i\rangle|^2 \left( \frac{\sin(\alpha t)}{\alpha} \right)^2 \rho(E_f)\, d\alpha.
\end{equation}

利用关键积分恒等式:
\[
\int_{-\infty}^{\infty} \left( \frac{\sin(\alpha t)}{\alpha} \right)^2 d\alpha = \pi t,
\]

并且注意到当 $t \to \infty$ 时,
\begin{equation*}
	\lim_{t\to\infty}\frac{1}{t}\left(\frac{\sin(\alpha t)}{\alpha}\right)^2=\pi\delta(\alpha)
\end{equation*}

在长时间极限下,可将 $\rho(E_f)$ 和 $|\langle f| V |i\rangle|^2$ 在 $\alpha = 0$(即 $E_f = E_i$)附近视为常数,提出积分外:

\[
P_{\text{tot}}(t) \approx \frac{2}{\hbar} |\langle f| V |i\rangle|^2 \rho(E_f) \bigg|_{E_f = E_i} \int_{-\infty}^{\infty} \left( \frac{\sin(\alpha t)}{\alpha} \right)^2 d\alpha
\]

于是,跃迁速率(单位时间的跃迁概率)为
\[
R_{i \to f} = \lim_{t \to \infty} \frac{P_{\text{tot}}(t)}{t}
= \frac{2\pi}{\hbar} |\langle f| V |i\rangle|^2 \rho(E_f) \Big|_{E_f = E_i}.
\]

这就是\textbf{费米黄金法则}(Fermi's Golden Rule):
\begin{equation}
	\boxed{
		R_{i \to f} = \frac{2\pi}{\hbar} \, |\langle f| V |i\rangle|^2 \, \rho(E_f) \Big|_{E_f = E_i}
	}
\end{equation}

该公式表明:从初态 $|i\rangle$ 到能量匹配的连续末态的跃迁速率,正比于微扰矩阵元的模平方和末态在能量守恒处的态密度。

可能有人会担忧:在长时间极限下,跃迁概率似乎趋于发散——那么如何证明微扰理论的适用性?对于$\omega_{fi}\neq\omega$的跃迁,"长时间"极限的条件为$t\gg1/(\omega_{fi}-\omega)$,而该时间仍可能远短于依赖于矩阵元的平均跃迁时间。事实上,费米黄金规则应用于原子系统时,与实验结果吻合得极好。

▷说明:黄金规则的另一种推导:当光照射原子时,完整的周期性势并非在原子的时间尺度上突然施加,而是在多个周期(原子周期和光周期)内逐渐建立。假设$V(t)=e^{\varepsilon t}Ve^{-i\omega t}$(其中$\varepsilon$非常小),即势场在过去被极其缓慢地开启,且我们关注的时间远小于$1/\varepsilon$。此时可将初始时间取为$-\infty$,即:
\begin{align*}
	c_f^{(1)}(t)&=-\frac{i}{\hbar}\int_{-\infty}^t \matrixel{f}{V}{i}e^{i(\omega_{fi}-\omega-i\varepsilon)t'}dt'\\
	&=-\frac{1}{\hbar}\frac{e^{i(\omega_{fi}-\omega-i\varepsilon)t}}{\omega_{fi}-\omega-i\varepsilon}\matrixel{f}{V}{i}
\end{align*}

因此
\[|c_f(t)|^2=\frac{1}{\hbar^2}\frac{e^{2\varepsilon t}}{(\omega_{fi}-\omega)^2+\varepsilon^2}|\matrixel{f}{V}{i}|^2\]

对跃迁速率\[\frac{d}{dt}|c_f^{(1)}(t)|^2\]

利用恒等式\[\lim_{\varepsilon\to0}\frac{2\varepsilon}{(\omega_{fi}-\omega)^2+\varepsilon^2}\to2\pi\delta(\omega_{fi}-\omega)\]

可得到费米黄金法则

从黄金规则的表达式(\ref{12.5})可以看出,要发生跃迁并满足能量守恒,需满足以下条件之一:

(a) 末态需分布在连续的能量范围内,以匹配固定微扰频率$\omega$对应的$\Delta E=\hbar\omega$

(b) 微扰需覆盖足够宽的频率谱,使得固定$\Delta E=\hbar\omega$的离散跃迁成为可能。

对于两个离散态,由于$|V_{fi}|^2=|V_{if}|^2$,可得半经典结果$P_{i\to f}=P_{f\to i}$——这是细致平衡原理的一种表述。

\subsection{简谐微扰:二阶跃迁}
尽管一阶微扰理论通常足以描述跃迁概率,但在某些情况下,一阶矩阵元$\matrixel{f}{V}{i}$可能因对称性(如宇称对称性或某种选择定则等)而严格为零,而其他矩阵元不为零。此时,跃迁可通过间接路径实现,需采用二阶微扰理论(式 \ref{12.4})估算跃迁概率:
\begin{equation}
	c_f^{(2)}(t)=-\frac{1}{\hbar^2}\sum_m \int_{t_0}^t dt'\int_{t_0}^{t'} dt'' e^{i\omega_{fm}t'+i\omega_{mi}t''} V_{fm}(t')V_{mi}(t'')
\end{equation}
若假设简谐势微扰被缓慢开启 ($V(t)=e^{\varepsilon t}Ve^{-i\omega t}$,初始时间$t_0\to-\infty$), 则:
\begin{equation}
	c_f^{(2)}(t)=-\frac{1}{\hbar^2}\sum_m \matrixel{f}{V}{m}\matrixel{m}{V}{i} \int_{-\infty}^t dt'\int_{-\infty}^{t'} dt'' e^{i(\omega_{fm}-\omega-i\varepsilon)t'}e^{i(\omega_{mi}-\omega-i\varepsilon)t''}
\end{equation}
积分求解较为直接,结果为:
\begin{equation}
	c_f^{(2)}=-\frac{1}{\hbar^2}e^{i(\omega_{fi}-2\omega)t}\frac{e^{2\varepsilon t}}{\omega_{fi}-2\omega-2i\varepsilon}\sum_m \frac{\matrixel{f}{V}{m}\matrixel{m}{V}{i}}{\omega_m-\omega_i-\omega-i\varepsilon}
\end{equation}
遵循前文的讨论,跃迁速率为:
\begin{equation}
	\frac{d}{dt}|c_f^{(2)}(t)|^2=\frac{2\pi}{\hbar^4}\left|\sum_m \frac{\matrixel{f}{V}{m}\matrixel{m}{V}{i}}{\omega_m-\omega_i-\omega-i\varepsilon}\right|^2\delta(\omega_{fi}-2\omega)
\end{equation}
该跃迁过程中,系统从简谐微扰中获得能量$2\hbar\omega$,即跃迁中吸收了两个"光子":第一个光子将系统激发至中间能级$\omega_m$(该能级寿命极短,能量不明确——事实上,向虚态的跃迁无需满足能量守恒,仅初始态与末态间需满足能量守恒)。当然,若任意态的原子暴露于单色光中,还可能发生其他二阶过程,如发射两个光子、吸收一个光子并发射一个光子(两种顺序均可)等。

\newpage
	\section{量子计算基本理论}
量子计算利用了量子力学的两个主要特性:叠加与纠缠。

\subsection{量子叠加}
量子叠加是指一个量子系统可以同时存在于多个状态或位置。这是量子计算并行处理能力的基础。

单个量子位的状态可以表示为二维向量空间 $\mathbb{C}^2$ 中的单位列向量:
\[
\ket{\psi} = \alpha\ket{0} + \beta\ket{1}
\]
其中 $\alpha,\beta \in \mathbb{C}$,且满足 $|\alpha|^2 + |\beta|^2 = 1$。

\subsection{量子纠缠}
量子纠缠是指两个或多个量子粒子之间存在的一种极强的关联。无论这些粒子相距多远,它们的状态都能瞬间完全一致地改变,多个量子位组成的联合状态需要用考虑了量子位间相互作用(即纠缠)的新向量空间来描述。

\subsection{张量积空间}
多量子位量子系统可以由多个已知的量子子系统构成。设第 $i$ 个子系统的状态空间为可分离的希尔伯特空间 $H_i$,那么多量子位系统的状态空间就是张量积空间:
\[
H = H_1 \otimes H_2 \otimes \cdots \otimes H_n
\]

多量子位系统的基态由单量子位的基态通过向量的张量积构成:
\[
\ket{\psi_1} \otimes \ket{\psi_2} \otimes \cdots \otimes \ket{\psi_n}
\]

\subsection{量子寄存器}
量子寄存器由若干量子位组成,寄存器的大小由量子位的数量决定。$m$ 个量子位可以同时表示 $2^m$ 个状态。

例如,4量子位寄存器可以由16种状态的叠加来表示:
\[
\ket{\psi} = c_0\ket{0000} + c_1\ket{0001} + \cdots + c_{15}\ket{1111}
\]

\section{量子纠缠的数学描述}
\subsection{可分离态与纠缠态}
如果一个多粒子系统的态可以写成单粒子态的直积形式:
\[
\ket{\psi} = \ket{\psi_1} \otimes \ket{\psi_2}
\]
则称这个态为可分离态。否则,称为纠缠态。

\subsection{纠缠态的例子}
考虑两个量子比特的贝尔态:
\[
\ket{\Phi^+} = \frac{1}{\sqrt{2}}(\ket{00} + \ket{11})
\]
这个态不能分解为两个单量子比特态的直积,因此是纠缠态。

\subsection{纠缠的数学特性}
对于一般的两粒子态:
\[
\ket{\Psi} = r\ket{a_0b_0} + u\ket{a_0b_1} + s\ket{a_1b_0} + t\ket{a_1b_1}
\]

当 $rt \neq us$ 时,这个态就是纠缠态。纠缠态的一个重要特性是:对其中一个粒子的测量会立即影响另一个粒子的状态。

\subsection{量子计算的优势}
由于量子寄存器中的每个状态可以同时处于所有可能的状态,量子计算具有很强的并行处理能力。这种并行性使得量子算法在某些问题上比经典算法有指数级的加速。

\paragraph*{量子并行性}
量子并行性允许量子计算机同时处理多个计算路径:
\[
U(\alpha\ket{x} + \beta\ket{y}) = \alpha U\ket{x} + \beta U\ket{y}
\]

这种特性使得量子算法能够同时探索多个解空间,从而在某些问题上实现指数加速。

\end{document}
